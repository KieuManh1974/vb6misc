AUTHOR'S NOTE:
Over the past twenty years, I have written a series of books about my apprenticeship with a Mexican Yaqui Indian sorcerer, don Juan Matus. I have explained in those books that he taught me sorcery but not as we understand sorcery in the context of our daily world: the use of supernatural powers over others, or the calling of spirits through charms, spells, or rituals to produce supernatural effects. For don Juan, sorcery was the act of embodying some specialized theoretical and practical premises about the nature and role of perception in molding the universe around us.finity." I remarked, at the time he said it, that the metaphor had no meaning to me.
"Let's then do away with metaphors," he conceded. "Let's say that dreaming is the sorcerers' practical way of putting ordinary dreams to use.

"But how can ordinary dreams be put to use?" I asked.

"We always get tricked by words," he said. "In my own case, my teacher attempted to describe dreaming to me by saying that it is the way sorcerers say good night to the world. He was, of course, tailoring his description to fit my mentality. I'm doing the same with you.
I also entered into the sion of our gratitude to him and our commitment to his quest.