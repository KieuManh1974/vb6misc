********************************************************Author: Carlos CastanedaTitle: The Fire from WithinOriginal copyright year: 1984Genre: FictionComments:Source:Date of e-text:Prepared by:********************************************************

CARLOSCASTANEDA

THE FIREFROMWITHIN

PUBLISHED BY POCKET BOOKS NEW YORK



Copyright � 1984 by Carlos CastanedaCover artwork copyright � 1985 Robert Giusti

Something was grabbing the edge of the mirror, as iffrom the inside of the glass, as if the glass surfacewere an open window and something or somebodywere just climbing through it.

Don Juan and I fought desperately; the loud thrash-ing continued unremittingly like an enormous fish inour bare hands. A strange shape was actually tryingto climb up through it. . .

I vacillated a second and the mirror flew out of myhands.

"Grab it! Grab it!" Don Juan yelled. . .

"A VISION OF THE SORCERER'S WORLD THAT ISFULL OF MIND-SPINNING IMPLICATIONS IN THECASTANEDA TRADITION."

	?United Press International

"HIS STORIES OF INITIATION INTO THE WORLDOF MAGIC AND SORCERY. . . CAN BE BOTHMOCKING AND TERRIFYING. . . . THE FIRE FROMWITHIN WILL FASCINATE YOU."

	?The Nashville Tennessean

"ONE CAN'T EXAGGERATE THE SIGNIFICANCEOF WHAT CASTANEDA HAS DONE."

	?The New York Times

Each of Carlos Castaneda's books is a brilliantand tantalizing burst of illumination intothe depths of our deepest mysteries, like asudden flash of light, like a burst oflightning over the desert at night, which shows usa world that is both alien and totallyfamiliar?the landscape of our dreams.

THE FIRE FROM WITHIN is the author's mostbrilliant, thought-provoking and unusualbook, one in which Castaneda, under the tutelageof don Juan and his "disciples," at lastconstructs, from the teachings of don Juan andhis own experiences, a stunning portraitof the "sorcerer's world" that is crystal-clear anddizzying in its implications.

"It's impossible to view the world in quitethe same way after readingTHE FIRE FROM WITHIN."

	?Chicago Tribune



I WANT TO EXPRESS MY ADMIRATION AND GRAT-ITUDE TO A MASTERFUL TEACHER, H. Y. L., FORHELPING ME RESTORE MY ENERGY, AND FORTEACHING ME AN ALTERNATE WAY TO PLENI-TUDE AND WELL-BEING.

Contents

FOREWORD 1. The New Seers 2. Petty Tyrants 3. The Eagle's Emanations 4. The Glow of Awareness 5. The First Attention 6. Inorganic Beings 7. The Assemblage Point 8. The Position of the Assemblage Point 9. The Shift Below10. Great Bands of Emanations11. Stalking, Intent, and the Dreaming Position12. The Nagual Julian13. The Earth's Boost14. The Rolling Force15. The Death Defiers16. The Mold of Man17. The Journey of the Dreaming Body18. Breaking the Barrier of PerceptionEPILOGUE



Foreword

I have written extensive descriptive accounts of myapprentice relationship with a Mexican Indian sor-cerer, don Juan Matus. Due to the foreignness of theconcepts and practices don Juan wanted me to under-stand and internalize, I have had no other choice butto render his teachings in the form of a narrative, anarrative of what happened, as it happened.

The organization of don Juan's instruction waspredicated on the idea that man has two types ofawareness. He labeled them the right side and the leftside. He described the first as the state of normalawareness necessary for everyday life. The second,he said, was the mysterious side of man, the state ofawareness needed to function as sorcerer and seer.Don Juan divided his instruction, accordingly, intoteachings for the right side and teachings for the leftside.

He conducted his teachings for the right side whenI was in my state of normal awareness, and I havedescribed those teachings in all my accounts. In mystate of normal awareness don Juan told me that hewas a sorcerer. He even introduced me to anothersorcerer, don Genaro Flores, and because of the na-ture of our association, I logically concluded that theyhad taken me as their apprentice.

That apprenticeship ended with an incomprehensi-ble act that both don Juan and don Genaro led me toperform. They made me jump from the top of a flatmountain into an abyss.

I have described in one of my accounts what tookplace on that mountaintop. The last drama of donJuan's teachings for the right side was played there bydon Juan himself; don Genaro; two apprentices, Pa-blito and Nestor; and me. Pablito, Nestor, and Ijumped from that mountaintop into an abyss.

For years afterward I thought that just my total trustin don Juan and don Genaro had been sufficient toobliterate all my rational fears on facing actual anni-hilation. I know now that it wasn't so; I know that thesecret was in don Juan's teachings for the left side,and that it took tremendous discipline and persever-ance for don Juan, don Genaro, and their companionsto conduct those teachings.

It has taken me nearly ten years to recollect whatexactly took place in his teachings for the left side thatled me to be so willing to perform such an incompre-hensible act: jumping into an abyss.

It was in his teachings for the left side that don Juanlet on what he, don Genaro, and their companionswere really doing to me. and who they were. Theywere not teaching me sorcery, but how to master threeaspects of an ancient knowledge they possessed:awareness, stalking, and intent. And they were notsorcerers; they were seers. And don Juan was not onlya seer, but also a nagual.

Don Juan had already explained to me, in his teach-ings for the right side, a great deal about the nagualand about seeing. I had understood seeing to be thecapacity of human beings to enlarge their perceptualfield until they are capable of assessing not only theouter appearances but the essence of everything. Hehad also explained that seers see man as a field ofenergy, which looks like a luminous egg. The majorityof people, he said, have their fields of energy dividedinto two parts. A few men and women have four orsometimes three parts. Because these people are moreresilient than the average man, they can become na-guals after learning to see.

In his teachings for the left side, don Juan explainedto me the intricacies of seeing and of being a nagual.To be a nagual, he said, is something more complexand far-reaching than being merely a more resilientman who has learned to see. To be a nagual entailsbeing a leader, being a teacher and a guide.

As a nagual, don Juan was the leader of a group ofseers known as the nagual's party, which was com-posed of eight female seers, Cecilia, Delia, Herme-linda, Carmela. Nelida, Florinda, Zuleica, and Zoila;three male seers, Vicente, Silvio Manuel, and Genaro;and four couriers or messengers, Emilito, John Tuma,Marta, and Teresa.

In addition to leading the nagual's party, don Juanalso taught and guided a group of apprentice seersknown as the new nagual's party. It consisted of fouryoung men, Pablito, Nestor, Eligio, and Benigno,along with five women, Soledad, la Gorda, Lidia, Jo-sefina, and Rosa. I was the nominal leader of the newnagual's party together with the nagual woman Carol.

In order for don Juan to impart to me his teachingsfor the left side it was necessary for me to enter into aunique state of perceptual clarity known as heightenedawareness. Throughout the years of my associationwith him, he had me repeatedly shift into such a stateby means of a blow that he delivered with the palm ofhis hand on my upper back.

Don Juan explained that in a state of heightenedawareness apprentices can behave almost as naturallyas in everyday life, but can bring their minds to focuson anything with uncommon force and clarity. Yet, aninherent quality of heightened awareness is that it isnot susceptible to normal recall. What transpires insuch a state becomes part of the apprentice's every-day awareness only through a staggering effort of re-covery.

My interaction with the nagual's party was an ex-ample of this difficulty of recall. With the exception ofdon Genaro, I had contact with them only when I wasin a state of heightened awareness; hence in my nor-mal everyday life I could not remember them, noteven as vague characters in dreams. The manner inwhich I met with them every time was almost a ritual.I would drive to don Genaro's house in a small townin the southern part of Mexico. Don Juan would joinus immediately and the three of us would then getbusy with don Juan's teachings for the right side. Afterthat, don Juan would make me change levels of aware-ness and then we would drive to a larger, nearby townwhere he and the other fifteen seers were living.

Every time I entered into heightened awareness Icould not cease marveling at the difference betweenmy two sides. I always felt as if a veil had been liftedfrom my eyes, as if I had been partially blind beforeand now I could see. The freedom, the sheer joy thatused to possess me on those occasions cannot be com-pared with anything else I have ever experienced. Yetat the same time, there was a frightening feeling ofsadness and longing that went hand in hand with thatfreedom and joy. Don Juan had told me that there isno completeness without sadness and longing, forwithout them there is no sobriety, no kindness. Wis-dom without kindness, he said, and knowledge with-out sobriety are useless.

The organization of his teachings for the left sidealso required that don Juan, together with some of hisfellow seers, explain to me the three facets of theirknowledge: the mastery of awareness, the mastery ofstalking, and the mastery of intent.

This work deals with the mastery of awareness,which is part of his total set of teachings for the leftside; the set he used in order to prepare me for per-forming the astonishing act of jumping into an abyss.

Due to the fact that the experiences I narrate heretook place in heightened awareness, they cannot havethe texture of daily life. They are lacking in worldlycontext, although I have tried my best to supply itwithout fictionalizing it. In heightened awareness oneis minimally conscious of the surroundings, becauseone's total concentration is taken by the details of theaction at hand.

In this case the action at hand was, naturally, theelucidation of the mastery of awareness. Don Juanunderstood the mastery of awareness as being themodern-day version of an extremely old tradition,which he called the tradition of the ancient Toltecseers.

Although he felt that he was inextricably linked tothat old tradition, he considered himself to be one ofthe seers of a new cycle. When I asked him once whatwas the essential character of the seers of the newcycle, he said that they are the warriors of total free-dom, that they are such masters of awareness, stalk-ing, and intent that they are not caught by death, likethe rest of mortal men, but choose the moment andthe way of their departure from this world. At thatmoment they are consumed by a fire from within andvanish from the face of the earth, free, as if they hadnever existed.



THE FIREFROMWITHIN

1The New Seers

I had arrived in the city of Oaxaca in southern Mexicoon my way to the mountains to look for don Juan. Onmy way out of town in the early morning, I had thegood sense to drive by the main square, and there Ifound him sitting on his favorite bench, as if waitingfor me to go by.

I joined him. He told me that he was in the city onbusiness, that he was staying at a local boardinghouse,and that I was welcome to stay with him because hehad to remain in town for two more days. We talkedfor a while about my activities and problems in theacademic world.

As was customary with him, he suddenly hit me onmy back when I least expected it, and the blow shiftedme into a state of heightened awareness.

We sat in silence for a very long time. I anxiouslywaited for him to begin talking, yet when he did, hecaught me by surprise.

"Ages before the Spaniards came to Mexico," hesaid, "there were extraordinary Toltec seers, men ca-pable of inconceivable deeds. They were the last linkin a chain of knowledge that extended over thousandsof years.

"The Toltec seers were extraordinary men?pow-erful sorcerers, somber, driven men who unraveledmysteries and possessed secret knowledge that theyused to influence and victimize people by fixating theawareness of their victims on whatever they chose."

He stopped talking and looked at me intently. I feltthat he was waiting for me to ask a question, but I didnot know what to ask.

"I have to emphasize an important fact," he contin-ued, "the fact that those sorcerers knew how to fixatethe awareness of their victims. You didn't pick up onthat. When I mentioned it, it didn't mean anything toyou. That's not surprising. One of the hardest thingsto acknowledge is that awareness can be manipu-lated."

I felt confused. I knew that he was leading me to-ward something. I felt a familiar apprehension?thesame feeling I had whenever he began a new round ofhis teachings.

I told him how I felt. He smiled vaguely. Usuallywhen he smiled he exuded happiness; this time he wasdefinitely preoccupied. He seemed to consider for amoment whether or not to go on talking. He stared atme intently again, slowly moving his gaze over theentire length of my body. Then, apparently satisfied,he nodded and said that I was ready for my final ex-ercise, something that all warriors go through beforeconsidering themselves fit to be on their own. I wasmore mystified than ever.

"We are going to be talking about awareness," hecontinued. "The Toltec seers knew the art of handlingawareness. As a matter of fact, they were the suprememasters of that art. When I say that they knew how tofixate the awareness of their victims, I mean that theirsecret knowledge and secret practices allowed themto pry open the mystery of being aware. Enough oftheir practices have survived to this day, but fortu-nately in a modified form. I say fortunately becausethose activities, as I will explain, did not lead the an-cient Toltec seers to freedom, but to their doom.""Do you know those practices yourself?" I asked."Why, certainly," he replied. "There is no way forus not to know those techniques, but that doesn'tmean that we practice them ourselves. We have otherviews. We belong to a new cycle."

"But you don't consider yourself a sorcerer, donJuan, do you?" I asked.

"No, I don't," he said. "I am a warrior who sees.In fact, all of us are los nuevos videntes?the newseers. The old seers were the sorcerers.

"For the average man," he continued, "sorcery isa negative business, but it is fascinating all the same.That's why I encouraged you, in your normal aware-ness, to think of us as sorcerers. It's advisable to doso. It serves to attract interest. But for us to be sor-cerers would be like entering a dead-end street."

I wanted to know what he meant by that, but herefused to talk about it. He said that he would elabo-rate on the subject as he proceeded with his explana-tion of awareness.

I asked him then about the origin of the Toltecs'knowledge.

"The way the Toltecs first started on the path ofknowledge was by eating power plants," he replied."Whether prompted by curiosity, or hunger, or error,they ate them. Once the power plants had producedtheir effects on them, it was only a matter of timebefore some of them began to analyze their experi-ences. In my opinion, the first men on the path ofknowledge were very daring, but very mistaken."

"Isn't all this a conjecture on your part, don Juan?"

"No, this is no conjecture of mine. I am a seer, andwhen I focus my seeing on that time I know every-thing that took place."

"Can you see the details of things of the past?" Iasked.

"Seeing is a peculiar feeling of knowing," he re-plied, "of knowing something without a shadow ofdoubt. In this case, I know what those men did, notonly because of my seeing, but because we are soclosely bound together."

Don Juan explained then that his use of the term"Toltec" did not correspond to what I understood itto mean. To me it meant a culture, the Toltec Empire.To him, the term "Toltec" meant "man of knowl-edge."

He said that in the time he was referring to, centu-ries or perhaps even millennia before the Spanish Con-quest, all such men of knowledge lived within a vastgeographical area, north and south of the valley ofMexico, and were employed in specific lines of work:curing, bewitching, storytelling, dancing, being an or-acle, preparing food and drink. Those lines of workfostered specific wisdom, wisdom that distinguishedthem from average men. These Toltecs, moreover,were also people who fitted into the structure of every-day life, very much as doctors, artists, teachers,priests, and merchants in our own time do. They prac-ticed their professions under the strict control of or-ganized brotherhoods and became proficient andinfluential, to such an extent that they even dominatedgroups of people who lived outside the Toltecs' geo-graphical regions.

Don Juan said that after some of these men hadfinally learned to see?after centuries of dealing withpower plants?the most enterprising of them thenbegan to teach other men of knowledge how to see.And that was the beginning of their end. As timepassed, the number of seers increased, but their ob-session with what they saw, which filled them withreverence and fear, became so intense that theyceased to be men of knowledge. They became extraor-dinarily proficient in seeing and could exert great con-trol over the strange worlds they were witnessing. Butit was to no avail. Seeing had undermined theirstrength and forced them to be obsessed with whatthey saw.

"There were seers, however, who escaped thatfate," don Juan continued, "great men who, in spiteof their seeing, never ceased to be men of knowledge.Some of them endeavored to use seeing positively andto teach it to their fellow men. I'm convinced thatunder their direction, the populations of entire citieswent into other worlds and never came back.

"But the seers who could only see were fiascos, andwhen the land where they lived was invaded by a con-quering people they were as defenseless as everyoneelse.

"Those conquerors," he went on, "took over theToltec world?they appropriated everything?butthey never learned to see."'

"Why do you think they never learned to see?" Iasked.

"Because they copied the procedures of the Toltecseers without having the Toltecs' inner knowledge. Tothis day there are scores of sorcerers all over Mexico,descendants of those conquerors, who follow the Tol-tec ways but don't know what they're doing, or whatthey're talking about, because they're not seers."

"Who were those conquerors, don Juan?"

"Other Indians," he said. "When the Spaniardscame, the old seers had been gone for centuries, butthere was a new breed of seers who were starting tosecure their place in a new cycle."

"What do you mean. a new breed of seers?"

"After the world of the first Toltecs was destroyed,the surviving seers retreated and began a serious ex-amination of their practices. The first thing they didwas to establish stalking, dreaming, and intent as thekey procedures and to deemphasize the use of powerplants; perhaps that gives us a hint as to what reallyhappened to them with power plants.

"The new cycle was just beginning to take holdwhen the Spanish conquerors swept the land. Fortu-nately, by that time the new seers were thoroughlyprepared to face that danger. They were already con-summate practitioners of the art of stalking."

Don Juan said that the subsequent centuries of sub-jugation provided for these new seers the ideal circum-stances in which to perfect their skills. Oddly enough,it was the extreme rigor and coercion of that periodthat gave them the impetus to refine their new princi-ples. And, owing to the fact that they never divulgedtheir activities, they were left alone to map their find-ings.

"Were there a great many new seers during theConquest?" I asked.

"At the beginning there were. Near the end therewere only a handful. The rest had been extermi-nated."

"What about in our day, don Juan?" I asked.

"There are a few. They are scattered all over, youunderstand."

"Do you know them?" I asked.

"Such a simple question is the hardest one to an-swer," he replied. "There are some we know verywell. But they are not exactly like us because theyhave concentrated on other specific aspects of knowl-edge, such as dancing, curing, bewitching, talking, in-stead of what the new seers recommend, stalking,dreaming, and intent. Those who are exactly like uswould not cross our path. The seers who lived duringthe Conquest set it up that way so as to avoid beingexterminated in the confrontation with the Spaniards.Each of those seers founded a lineage. And not all ofthem had descendants, so the lines are few."

"Do you know any who are exactly like us?" Iasked.

"A few," he replied laconically.

I asked him then to give me all the information hecould, for I was vitally interested in the topic; to me itwas of crucial importance to know names and ad-dresses for purposes of validation and corroboration.

Don Juan did not seem inclined to oblige me. "Thenew seers went through that bit of corroboration," hesaid. "Half of them left their bones in the corroborat-ing room. So now they are solitary birds. Let's leaveit that way. All we can talk about is our line. Aboutthat, you and I can say as much as we please."

He explained that all the lines of seers were startedat the same time and in the same fashion. Around theend of the sixteenth century every nagual deliberatelyisolated himself and his group of seers from any overtcontact with other seers. The consequence of thatdrastic segregation, he said, was the formation of theindividual lineages. Our lineage consisted of fourteennaguals and one hundred and twenty-six seers, hesaid. Some of those fourteen naguals had as few asseven seers with them. others had eleven, and someup to fifteen.

He told me that his teacher?or his benefactor, ashe called him?was the nagual Julian, and the one whocame before Julian was the nagual Ellas. I asked himif he knew the names of all fourteen naguals. Henamed and enumerated them for me, so I could learnwho they were. He also said that he had personallyknown the fifteen seers who formed his benefactor'sgroup and that he had also known his benefactor'steacher, the nagual Ellas, and the eleven seers of hisparty.

Don Juan assured me that our line was quite excep-tional, because it underwent a drastic change in theyear 1723 as a result of an outside influence that cameto bear on us and inexorably altered our course. Hedid not want to discuss the event itself at the moment,but he said that a new beginning is counted from thattime; and that the eight naguals who have ruled theline since then are considered intrinsically differentfrom the six who preceded them.

Don Juan must have had business to take care ofthe next day, for I did not see him until around noon.in the meantime, three of his apprentices had come totown, Pablito, Nestor, and la Gorda. They were shop-ping for tools and materials for Pablito's carpentrybusiness. I accompanied them and helped them tocomplete all their errands. Then all of us went back tothe boardinghouse.

All four of us were sitting around talking when donJuan came into my room. He announced that we wereleaving after lunch, but that before we went to eat hestill had something to discuss with me, in private. Hewanted the two of us to take a stroll around the mainsquare and then all of us would meet at a restaurant.

Pablito and Nestor stood up and said that they hadsome errands to run before meeting us. La Gordaseemed very displeased.

"What are you going to talk about?" she blurtedout, but quickly realized her mistake and giggled.

Don Juan gave her a strange look but did not sayanything.

Encouraged by his silence, la Gorda proposed thatwe take her along. She assured us that she would notbother us in the least.

"I'm sure you won't bother us," don Juan said toher, "but I really don't want you to hear anything ofwhat I have to say to him."

La Gorda's anger was very obvious. She blushedand, as don Juan and I walked out of the room, herentire face clouded with anxiety and tension, becom-ing instantly distorted. Her mouth was open and herlips were dry.

La Gorda's mood made me very apprehensive. I feltan actual discomfort. I didn't say anything, but donJuan seemed to notice my feelings.

"You should thank la Gorda day and night," he saidall of a sudden. "She's helping you destroy your self-importance. She's the petty tyrant in your life, but youstill haven't caught on to that."

We strolled around the plaza until all my nervous-ness had vanished. Then we sat down on his favoritebench again.

"The ancient seers were very fortunate indeed,"don Juan began, "because they had plenty of time tolearn marvelous things. Let me tell you, they knewwonders that we can't even imagine today."

"Who taught them all that?" I asked.

"They learned everything by themselves throughseeing,"' he replied. "Most of the things we know inour lineage were figured out by (hem. The new seerscorrected the mistakes of the old seers, but the basisof what we know and do is lost in Toltec time."

He explained. One of the simplest and yet most im-portant findings, from the point of view of instruction,he said, is the knowledge that man has two types ofawareness. The old seers called them the right and theleft side of man.

"The old seers figured out," he went on, "that thebest way to teach their knowledge was to make theirapprentices shift to their left side, to a state ofheightened awareness. Real learning takes placethere.

"Very young children were given to the old seers asapprentices," don Juan continued, "so that theywouldn't know any other way of life. Those children,in turn, when they came of age took other children asapprentices. Imagine the things they must have uncov-ered in their shifts to the left and to the right, aftercenturies of that kind of concentration."

I remarked how disconcerting those shifts were tome. He said that my experience was similar to hisown. His benefactor, the nagual Julian, had created aprofound schism in him, by making him shift back andforth from one type of awareness to the other. He saidthat the clarity and freedom he experienced inheightened awareness were in total contrast to the ra-tionalizations, the defenses, the anger, and the fear ofhis normal state of awareness.

The old seers used to create this polarity to suit theirown particular purposes; with it, they forced their ap-prentices to achieve the concentration needed to learnsorcery techniques. But the new seers, he said, use itto lead their apprentices to the conviction that thereare unrealized possibilities in man.

"The best effort of the new seers," don Juan contin-ued, "is their explanation of the mystery of aware-ness. They condensed it all into some concepts andactions which are taught while the apprentices are inheightened awareness."

He said that the value of the new seers' method ofteaching is that it takes advantage of the fact that noone can remember anything that happens while beingin a state of heightened awareness. This inability toremember sets up an almost insurmountable barrierfor warriors, who have to recollect all the instructiongiven to them if they are to go on. Only after years ofstruggle and discipline can warriors recollect their in-struction. By then the concepts and the proceduresthat were taught to them have been internalized andhave thus acquired the force the new seers meantthem to have.

2Petty Tyrants

Don Juan did not discuss the mastery of awarenesswith me until months later. We were at that time inthe house where the nagual's party lived.

"Let's go for a walk," don Juan said to me, placinghis hand on my shoulder. "Or better yet, let's go tothe town's square, where there are a lot of people, andsit down and talk."

I was surprised when he spoke to me, as I had beenin the house for a couple of days then and he had notsaid so much as hello.

As don Juan and I were leaving the house, la Gordaintercepted us and demanded that we take her along.She seemed determined not to take no for an answer.Don Juan in a very stern voice told her that he had todiscuss something in private with me.

"You're going to talk about me," la Gorda said, hertone and gestures betraying both suspicion and annoy-ance.

"You're right," don Juan replied dryly. He movedpast her without turning to look at her.

I followed him, and we walked in silence to thetown's square. When we sat down I asked him whaton earth we would find to discuss about la Gorda. Iwas still smarting from her look of menace when weleft the house.

"We have nothing to discuss about la Gorda or any-body else," he said. "I told her that just to provokeher enormous self-importance. And it worked. She isfurious with us. If I know her, by now she will havetalked to herself long enough to have built up her con-fidence and her righteous indignation at having beenrefused and made to look like a fool. I wouldn't besurprised if she barges in on us here, at the parkbench."

"If we're not going to talk about la Gorda, what arewe going to discuss?" I asked.

"We're going to continue the discussion we startedin Oaxaca," he replied. "To understand the explana-tion of awareness will require your utmost effort andyour willingness to shift back and forth between levelsof awareness. While we are involved in our discussionI will demand your total concentration and patience."

Half-complaining, I told him that he had made mefeel very uncomfortable by refusing to talk to me forthe past two days. He looked at me and arched hisbrows. A smile played on his lips and vanished. I re-alized that he was letting me know I was no betterthan la Gorda.

"I was provoking your self-importance," he saidwith a frown. "Self-importance is our greatest enemy.Think about it?what weakens us is feeling offendedby the deeds and misdeeds of our fellow men. Ourself-importance requires that we spend most of ourlives offended by someone.

"The new seers recommended that every effortshould be made to eradicate self-importance from thelives of warriors. I have followed that recommenda-tion, and much of my endeavors with you has beengeared to show you that without self-importance weare invulnerable."

As I listened his eyes suddenly became very shiny.I was thinking to myself that he seemed to be on theverge of laughter and there was no reason for it whenI was startled by an abrupt, painful slap on the rightside of my face.

I jumped up from the bench. La Gorda was standingbehind me, her hand still raised. Her face was flushedwith anger.

"Now you can say what you like about me and withmore justification," she shouted. "If you have any-thing to say, however, say it to my face!"

Her outburst appeared to have exhausted her, be-cause she sat down on the cement and began to weep.Don Juan was transfixed with inexpressible glee. I wasfrozen with sheer fury. La Gorda glared at me andthen turned to don Juan and meekly told him that wehad no right to criticize her.

Don Juan laughed so hard he doubled over almostto the ground. He couldn't even speak. He tried twoor three times to say something to me, then finally gotup and walked away, his body still shaking withspasms of laughter.

I was about to run after him, still glowering at laGorda?at that moment I found her despicable ?when something extraordinary happened to me. I re-alized what don Juan had found so hilarious. La Gordaand I were horrendously alike. Our self-importancewas monumental. My surprise and fury at beingslapped were just like la Gorda's feelings of anger andsuspicion. Don Juan was right. The burden of self-importance is a terrible encumbrance.

I ran after him then, elated, the tears flowing downmy cheeks. I caught up with him and told him what Ihad realized. His eyes were shining with mischievous-ness and delight.

"What should I do about la Gorda?" I asked.

"Nothing," he replied. "Realizations are alwayspersonal."

He changed the subject and said that the omenswere telling us to continue our discussion back at hishouse, either in a large room with comfortable chairsor in the back patio, which had a roofed corridoraround it. He said that whenever he conducted hisexplanation inside the house those two areas would beoff limits to everyone else.

We went back to the house. Don Juan told everyonewhat la Gorda had done. The delight all the seersshowed in taunting her made la Gorda's position ex-tremely uncomfortable.

"Self-importance can't be fought with niceties,"don Juan commented when I expressed my concernabout la Gorda.

He then asked everyone to leave the room. We satdown and don Juan began his explanations.

He said that seers, old and new, are divided intotwo categories. The first one is made up of those whoare willing to exercise self-restraint and can channeltheir activities toward pragmatic goals, which wouldbenefit other seers and man in general. The other cat-egory consists of those who don't care about self-re-straint or about any pragmatic goals. It is theconsensus among seers that the latter have failed toresolve the problem of self-importance.

"Self-importance is not something simple andnaive," he explained. "On the one hand, it is the coreof everything that is good in us, and on the other hand,the core of everything that is rotten. To get rid of theself-importance that is rotten requires a masterpieceof strategy. Seers, through the ages, have given thehighest praise to those who have accomplished it."

I complained that the idea of eradicating self-impor-tance, although very appealing to me at times, wasreally incomprehensible; I told him that I found hisdirectives for getting rid of it so vague I could notfollow them.

"I've said to you many times," he said, "that inorder to follow the path of knowledge one has to bevery imaginative. You see, in the path of knowledgenothing is as clear as we'd like it to be."

My discomfort made me argue that his admonitionsabout self-importance reminded me of Catholic die-turns. After a lifetime of being told about the evils ofsin, I had become callous.

"Warriors fight self-importance as a matter of strat-egy, not principle," he replied. "Your mistake is tounderstand what I say in terms of morality."

"I see you as a highly moral man, don Juan," Iinsisted.

"You've noticed my impeccability, that's all," hesaid.

"Impeccability, as well as getting rid of self-impor-tance, is too vague a concept to be of any value tome," I remarked.

Don Juan choked with laughter, and I challengedhim to explain impeccability.

"Impeccability is nothing else but the proper use ofenergy," he said. "My statements have no inkling ofmorality. I've saved energy and that makes me impec-cable. To understand this, you have to save enoughenergy yourself."

We were quiet for a long time. I wanted to thinkabout what he had said. Suddenly, he started talkingagain.

"Warriors take strategic inventories," he said."They list everything they do. Then they decidewhich of those things can be changed in order to allowthemselves a respite, in terms of expending their en-ergy."

I argued that their list would have to include every-thing under the sun. He patiently answered that thestrategic inventory he was talking about covered onlybehavioral patterns that were not essential to our sur-vival and well-being.

I jumped at the opportunity to point out that sur-vival and well-being were categories that could be in-terpreted in endless ways, hence, there was no way ofagreeing what was or was not essential to survival andwell-being.

As I kept on talking I began to lose momentum.Finally, I stopped because I realized the futility of myarguments.

Don Juan said then that in the strategic inventoriesof warriors, self-importance figures as the activity thatconsumes the greatest amount of energy, hence, theireffort to eradicate it.

"One of the first concerns of warriors is to free thatenergy in order to face the unknown with it," don Juanwent on. "The action of rechanneling that energy i?impeccability."

He said that the most effective strategy was workedout by the seers of the Conquest, the unquestionablemasters of stalking. It consists of six elements thatinterplay with one another. Five of them are called theattributes of warriorship: control, discipline, forbear-ance, timing, and will. They pertain to the world of thewarrior who is fighting to lose self-importance. Thesixth element, which is perhaps the most important ofall, pertains to the outside world and is called the pettytyrant.

He looked at me as if silently asking me whether ornot I had understood.

"I'm really mystified," I said. "You keep on sayingthat la Gorda is the petty tyrant of my life. Just whatis a petty tyrant?"

"A petty tyrant is a tormentor," he replied. "Some-one who either holds the power of life and death overwarriors or simply annoys them to distraction."

Don Juan had a beaming smile as he spoke to me.He said that the new seers developed their own clas-sification of petty tyrants; although the concept is oneof their most serious and important findings, the newseers had a sense of humor about it. He assured methat there was a tinge of malicious humor in every oneof their classifications, because humor was the onlymeans of counteracting the compulsion of humanawareness to take inventories and to make cumber-some classifications.

The new seers, in accordance with their practice,saw fit to head their classification with the primalsource of energy, the one and only ruler in the uni-verse, and they called it simply the tyrant. The rest ofthe despots and authoritarians were found to be, nat-urally, infinitely below the category of tyrant. Com-pared to the source of everything, the most fearsome,tyrannical men are buffoons; consequently, they wereclassified as petty tyrants, pinches tiranos.

He said that there were two subclasses of minorpetty tyrants. The first subclass consisted of the pettytyrants who persecute and inflict misery but withoutactually causing anybody's death. They were calledlittle petty tyrants, pinches tiranitos. The second con-sisted of the petty tyrants who are only exasperatingand bothersome to no end. They were called small-frypetty tyrants, repinches tiranitos, or teensy-weensypetty tyrants, pinches tiranitos chiquititos.

I thought his classifications were ludicrous. I wassure that he was improvising the Spanish terms. Iasked him if that was so.

"Not at all," he replied with an amused expression."The new seers were great ones for classifications.Genaro is doubtless one of the greatest; if you'd ob-serve him carefully, you'd realize exactly how the newseers feel about their classifications."

He laughed uproariously at my confusion when Iasked him if he was pulling my leg.

"I wouldn't dream of doing that," he said, smiling."Genaro may do that, but not I, especially when Iknow how you feel about classifications. It's just thatthe new seers were terribly irreverent."

He added that the little petty tyrants are furtherdivided into four categories. One that torments withbrutality and violence. Another that does it by creat-ing unbearable apprehension through deviousness.Another which oppresses with sadness. And the last,which torments by making warriors rage.

"La Gorda is in a class of her own," he added."She is an acting, small-fry petty tyrant. She annoysyou to pieces and makes you rage. She even slaps you.With all that she is teaching you detachment."

"That's not possible!" I protested.

"You haven't yet put together all the ingredients ofthe new seers' strategy," he said. "Once you do that,you'll know how efficient and clever is the device ofusing a petty tyrant. I would certainly say that thestrategy not only gets rid of self-importance; it alsoprepares warriors for the final realization that impec-cability is the only thing that counts in the path ofknowledge."

He said that what the new seers had in mind was adeadly maneuver in which the petty tyrant is like amountain peak and the attributes of warriorship arelike climbers who meet at the summit.

"Usually, only four attributes are played," he wenton. "The fifth, will, is always saved for an ultimateconfrontation, when warriors are facing the firingsquad, so to speak."

"Why is it done that way?"

"Because wilt belongs to another sphere, the un-known. The other four belong to the known, exactlywhere the petty tyrants are lodged. In fact, what turnshuman beings into petty tyrants is precisely the obses-sive manipulation of the known."

Don Juan explained that the interplay of all the fiveattributes of warriorship is done only by seers who arealso impeccable warriors and have mastery over will.Such an interplay is a supreme maneuver that cannotbe performed on the daily human stage.

"Four attributes are all that is needed to deal withthe worst of petty tyrants," he continued. "Provided,of course, that a petty tyrant has been found. As Isaid, the petty tyrant is the outside element, the onewe cannot control and the element that is perhaps themost important of them all. My benefactor used to saythat the warrior who stumbles on a petty tyrant is alucky one. He meant that you're fortunate if you comeupon one in your path, because if you don't, you haveto go out and look for one."

He explained that one of the greatest accomplish-ments of the seers of the Conquest was a construct hecalled the three-phase progression. By understandingthe nature of man, they were able to reach the incon-testable conclusion that if seers can hold their own infacing petty tyrants, they can certainly face the un-known with impunity, and then they can even standthe presence of the unknowable.

"The average man's reaction is to think that theorder of that statement should be reversed," he wenton. "A seer who can hold his own in the face of theunknown can certainly face petty tyrants. But that'snot so. What destroyed the superb seers of ancienttimes was that assumption. We know better now. Weknow that nothing can temper the spirit of a warrioras much as the challenge of dealing with impossiblepeople in positions of power. Only under those condi-tions can warriors acquire the sobriety and serenity tostand the pressure of the unknowable."

I vociferously disagreed with him. I told him that inmy opinion tyrants can only render their victims help-less or make them as brutal as they themselves are. Ipointed out that countless studies had been done onthe effects of physical and psychological torture onsuch victims.

"The difference is in something you just said," heretorted. "They are victims, not warriors. Once I feltjust as you do. I'll tell you what made me change, butfirst let's go back again to what I said about the Con-quest. The seers of that time couldn't have found abetter ground. The Spaniards were the petty tyrantswho tested the seers' skills to the limit; after dealingwith the conquerors, the seers were capable of facinganything. They were the lucky ones. At that time therewere petty tyrants everywhere.

"After all those marvelous years of abundancethings changed a great deal. Petty tyrants never againhad that scope; it was only during those times thattheir authority was unlimited. The perfect ingredientfor the making of a superb seer is a petty tyrant withunlimited prerogatives.

"In our times, unfortunately, seers have to go toextremes to find a worthy one. Most of the time theyhave to be satisfied with very small fry."

"Did you find a petty tyrant yourself, don Juan?"

"I was lucky. A king-size one found me. At thetime, though, I felt like you; I couldn't consider myselffortunate."

Don Juan said that his ordeal began a few weeksbefore he met his benefactor. He was barely twentyyears old at the time. He had gotten a job at a sugarmill working as a laborer. He had always been verystrong, so it was easy for him to get jobs that requiredmuscle. One day when he was moving some heavysacks of sugar a woman came by. She was very welldressed and seemed to be a woman of means. She wasperhaps in her fifties, don Juan said, and very domi-neering. She looked at don Juan and then spoke to theforeman and left. Don Juan was then approached bythe foreman, who told him that for a fee he wouldrecommend him for a job in the boss's house. DonJuan told the man that he had no money. The foremansmiled and said not to worry because he would haveplenty on payday. He patted don Juan's back and as-sured him it was a great honor to work for the boss.

Don Juan said that being a lowly ignorant Indianliving hand-to-mouth, not only did he believe everyword, he thought a good fairy had touched him. Hepromised to pay the foreman anything he wished. Theforeman named a large sum, which had to be paid ininstallments.

Immediately thereafter the foreman himself tookdon Juan to the house, which was quite a distancefrom the town, and left him there with another fore-man, a huge, somber, ugly man who asked a lot ofquestions. He wanted to know about don Juan's fam-ily. Don Juan answered that he didn't have any. Theman was so pleased that he even smiled through hisrotten teeth.

He promised don Juan that they would pay himplenty, and that he would even be in a position to savemoney, because he didn't have to spend any, for hewas going to live and eat in the house.

The way the man laughed was terrifying. Don Juanknew that he had to escape immediately. He ran forthe gate, but the man cut in front of him with a re-volver in his hand. He cocked it and rammed it intodon Juan's stomach. "You're here to work yourself tothe bone," he said. "And don't you forget it." Heshoved don Juan around with a billy club. Then hetook him to the side of the house and, after observingthat he worked his men every day from sunrise tosunset without a break, he put don Juan to work dig-ging out two enormous tree stumps. He also told donJuan that if he ever tried to escape or went to theauthorities he would shoot him dead?and that if donJuan should ever get away, he would swear in courtthat don Juan had tried to murder the boss. "You'llwork here until you die," he said. "Another Indianwill get your job then, just as you're taking a deadIndian's place."

Don Juan said that the house looked like a fortress,with armed men with machetes everywhere. So he gotbusy working and tried not to think about his predica-ment. At the end of the day, the man came back andkicked him all the way to the kitchen, because he didnot like the defiant look in don Juan's eyes. He threat-ened to cut the tendons of don Juan's arms if he didn'tobey him.

In the kitchen an old woman brought food, but donJuan was so upset and afraid that he couldn't eat. Theold woman advised him to eat as much as he could.He had to be strong, she said, because his work wouldnever end. She warned him that the man who had heldhis job had died just a day earlier. He was too weak towork and had fallen from a second-story window.

Don Juan said that he worked at the boss's place forthree weeks and that the man bullied him every mo-ment of every day. He made him work under the mostdangerous conditions, doing the heaviest work imag-inable, under the constant threat of his knife, gun, orbilly club. He sent him daily to the stables to clean thestalls while the nervous stallions were in them. At thebeginning of every day don Juan thought it would behis last one on earth. And surviving meant only thathe had to go through the same hell again the next day.

What precipitated the end was don Juan's requestto have some time off. The pretext was that he neededto go to town to pay the foreman of the sugar mill themoney that he owed him. The other foreman retortedthat don Juan could not stop working, not even for aminute, because he was in debt up to his ears just forthe privilege of working there.

Don Juan knew that he was done for. He understoodthe man's maneuvers. Both he and the other foremanwere in cahoots to get lowly Indians from the mill,work them to death, and divide their salaries. Thatrealization angered him so intensely that he ranthrough the kitchen screaming and got inside the mainhouse. The foreman and the other workers werecaught totally by surprise. He ran out the front doorand almost got away, but the foreman caught up withhim on the road and shot him in the chest. He left himfor dead.

Don Juan said that it was not his destiny to die; hisbenefactor found him there and tended him until hegot well.

"When I told my benefactor the whole story," donJuan said, "he could hardly contain his excitement.That foreman is really a prize, ' my benefactor said.'He is too good to be wasted. Someday you must goback to that house. '

"He raved about my luck in finding a one-in-a-mil-lion petty tyrant with almost unlimited power. Ithought the old man was nuts. It was years before Ifully understood what he was talking about."

"That is one of the most horrible stories I have everheard," I said. "Did you really go back to thathouse?"

"I certainly did, three years later. My benefactorwas right. A petty tyrant like that one was one in amillion and couldn't be wasted."

"How did you manage to go back?"

"My benefactor developed a strategy using the fourattributes of warriorship: control, discipline, forbear-ance, and timing."

Don Juan said that his benefactor, in explaining tohim what he had to do to profit from facing that ogreof a man, also told him what the new seers consideredto be the four steps on the path of knowledge. The firststep is the decision to become apprentices. After theapprentices change their views about themselves andthe world they take the second step and become war-riors, which is to say, beings capable of the utmostdiscipline and control over themselves. The third step,after acquiring forbearance and timing, is to becomemen of knowledge. When men of knowledge learn tosee they have taken the fourth step and have becomeseers.

His benefactor stressed the fact that don Juan hadbeen on the path of knowledge long enough to haveacquired a minimum of the first two attributes: controland discipline. Don Juan emphasized that both ofthese attributes refer to an inner state. A warrior isself-oriented, not in a selfish way, but in the sense ofa total and continuous examination of the self.

"At that time, I was barred from the other two at-tributes," don Juan went on. "Forbearance and tim-ing are not quite an inner state. They are in the domainof the man of knowledge. My benefactor showed themto me through his strategy."

"Does this mean that you couldn't have faced thepetty tyrant by yourself?" I asked.

"I'm sure that I could have done it myself, althoughI have always doubted that I would have carried it offwith flair and joyfulness. My benefactor was simplyenjoying the encounter by directing it. The idea ofusing a petty tyrant is not only for perfecting the war-rior's spirit, but also for enjoyment and happiness."

"How could anyone enjoy the monster you de-scribed?"

"He was nothing in comparison to the real monstersthat the new seers faced during the Conquest. By allindications those seers enjoyed themselves blue deal-ing with them. They proved that even the worst ty-rants can bring delight, provided, of course, that oneis a warrior."

Don Juan explained that the mistake average menmake in confronting petty tyrants is not to have astrategy to fall back on; the fatal flaw is that averagemen take themselves too seriously; their actions andfeelings, as well as those of the petty tyrants, are all-important. Warriors, on the other hand, not only havea well-thought-out strategy, but are free from self-im-portance. What restrains their self-importance is thatthey have understood that reality is an interpretationwe make. That knowledge was the definitive advan-tage that the new seers had over the simple-mindedSpaniards.

He said that he became convinced he could defeatthe foreman using only the single realization that pettytyrants take themselves with deadly seriousness whilewarriors do not.

Following his benefactor's strategic plan, therefore,don Juan got a job in the same sugar mill as before.Nobody remembered that he had worked there in thepast; peons came to that sugar mill and left it withoutleaving a trace.

His benefactor's strategy specified that don Juanhad to be solicitous of whoever came to look for an-other victim. As it happened, the same woman cameand spotted him, as she had done years ago. This timehe was physically even stronger than before.

The same routine took place. The strategy, how-ever, called for refusing payment to the foreman fromthe outset. The man had never been turned down andwas taken aback. He threatened to fire don Juan fromthe job. Don Juan threatened him back, saying that hewould go directly to the lady's house and see her. DonJuan knew that the woman, who was the wife of theowner of the mill, did not know what the two foremenwere up to. He told the foreman that he knew whereshe lived, because he had worked in the surroundingfields cutting sugar cane. The man began to haggle,and don Juan demanded money from him before hewould accept going to the lady's house. The foremangave in and handed him a few bills. Don Juan wasperfectly aware that the foreman's acquiescence wasjust a ruse to get him to go to the house.

"He himself once again took me to the house," donJuan said. "It was an old hacienda owned by the peo-ple of the sugar mill?rich men who either knew whatwas going on and didn't care, or were too indifferenteven to notice.

"As soon as we got there, I ran into the house tolook for the lady. I found her and dropped to my kneesand kissed her hand to thank her. The two foremenwere livid.

"The foreman at the house followed the same pat-tern as before. But I had the proper equipment to dealwith him; I had control, discipline, forbearance, andtiming. It turned out as my benefactor had planned it.My control made me fulfill the man's most asininedemands. What usually exhausts us in a situation likethat is the wear and tear on our self-importance. Anyman who has an iota of pride is ripped apart by beingmade to feel worthless.

"I gladly did everything he asked of me. I was joyfuland strong. And I didn't give a fig about my pride ormy fear. I was there as an impeccable warrior. To tunethe spirit when someone is trampling on you is calledcontrol."

Don Juan explained that his benefactor's strategyrequired that instead of feeling sorry for himself as hehad done before, he immediately go to work mappingthe man's strong points, his weaknesses, his quirks ofbehavior.

He found that the foreman's strongest points werehis violent nature and his daring. He had shot donJuan in broad daylight and in sight of scores of onlook-ers. His great weakness was that he liked his job anddid not want to endanger it. Under no circumstancescould he attempt to kill don Juan inside the compoundin the daytime. His other weakness was that he was afamily man. He had a wife and children who lived in ashack near the house.

"To gather all this information while they are beating you up is called discipline," don Juan said. "Theman was a regular fiend. He had no saving grace. According to the new seers, a perfect petty tyrant has noredeeming feature."

Don Juan said that the other two attributes of war-riorship, forbearance and timing, which he did not yethave, had been automatically included in his benefac-tor's strategy. Forbearance is to wait patiently?norush, no anxiety?a simple, joyful holding back ofwhat is due.

"I groveled daily," don Juan continued, "some-times crying under the man's whip. And yet I washappy. My benefactor's strategy was what made mego from day to day without hating the man's guts. Iwas a warrior. I knew that I was waiting and I knewwhat I was waiting for. Right there is the great joy ofwarriorship."

He added that his benefactor's strategy called for asystematic harassment of the man by taking coverwith a higher order, just as the seers of the new cyclehad done during the Conquest by shielding themselveswith the Catholic church. A lowly priest was some-times more powerful than a nobleman.

Don Juan's shield was the lady who got him the job.He kneeled in front of her and called her a saint everytime he saw her. He begged her to give him the me-dallion of her patron saint so he could pray to him forher health and well-being.

"She gave me one," don Juan went on, "and thatrattled the foreman to pieces. And when I got the ser-vants to pray at night he nearly had a heart attack. Ithink he decided then to kill me. He couldn't afford tolet me go on.

"As a countermeasure I organized a rosary amongall the servants of the house. The lady thought I hadthe makings of a most pious man.

"I didn't sleep soundly after that, nor did I sleep inmy bed. I climbed to the roof every night. From thereI saw the man twice looking for me in the middle ofthe night with murder in his eyes.

"Daily he shoved me into the stallions' stalls hopingthat I would be crushed to death, but I had a plank ofheavy boards that I braced against one of the cornersand protected myself behind it. The man never knewbecause he was nauseated by the horses?another ofhis weaknesses, the deadliest of all, as things turnedout."

Don Juan said that timing is the quality that governsthe release of all that is held back. Control, discipline,and forbearance are like a dam behind which every-thing is pooled. Timing is the gate in the dam.

The man knew only violence, with which he terror-ized. If his violence was neutralized he was renderednearly helpless. Don Juan knew that the man wouldnot dare to kill him in view of the house, so one day,in the presence of the other workers but in sight of hislady as well, don Juan insulted the man. He called hima coward, who was mortally afraid of the boss's wife.

His benefactor's strategy had called for being on thealert for a moment like that and using it to turn thetables on the petty tyrant. Unexpected things alwayshappen that way. The lowest of the slaves suddenlymakes fun of the tyrant, taunts him, makes him feelridiculous in front of significant witnesses, and thenrushes away without giving the tyrant time to retaliate.

"A moment later, the man went crazy with rage,but I was already solicitously kneeling in front of thelady," he continued.

Don Juan said that when the lady went inside thehouse, the man and his friends called him to the back,allegedly to do some work. The man was very pale,white with anger. From the sound of his voice donJuan knew what the man was really planning to do.Don Juan pretended to acquiesce, but instead of head-ing for the back, he ran for the stables. He trusted thatthe horses would make such a racket the ownerswould come out to see what was wrong. He knew thatthe man would not dare shoot him. That would havebeen too noisy and the man's fear of endangering hisjob was too overpowering. Don Juan also knew thatthe man would not go where the horses were?that is,unless he had been pushed beyond his endurance.

"I jumped inside the stall of the wildest stallion,"don Juan said, "and the petty tyrant, blinded by rage,took out his knife and jumped in after me. I wentinstantly behind my planks. The horse kicked himonce and it was all over.

"I had spent six months in that house and in thatperiod of time I had exercised the four attributes ofwarriorship. Thanks to them, I had succeeded. Notonce had I felt sorry for myself or wept in impotence.I had been joyful and serene. My control and disci-pline were as keen as they'd ever been, and I had hada firsthand view of what forbearance and timing didfor impeccable warriors. And I had not once wishedthe man to die.

"My benefactor explained something very interest-ing. Forbearance means holding back with the spiritsomething that the warrior knows is rightfully due. Itdoesn't mean that a warrior goes around plotting to doanybody mischief, or planning to settle past scores.Forbearance is something independent. As long as thewarrior has control, discipline, and timing, forbear-ance assures giving whatever is due to whoever de-serves it."

"Do petty tyrants sometimes win, and destroy thewarrior facing them?" I asked.

"Of course. There was a time when warriors diedlike flies at the beginning of the Conquest. Their rankswere decimated. The petty tyrants could put anyoneto death, simply acting on a whim. Under that kind ofpressure seers reached sublime states."

Don Juan said that that was the time when the sur-viving seers had to exert themselves to the limit to findnew ways.

"The new seers used petty tyrants," don Juan said,staring at me fixedly, "not only to get rid of their self-importance, but to accomplish the very sophisticatedmaneuver of moving themselves out of this world.You'll understand that maneuver as we keep on dis-cussing the mastery of awareness."

I explained to don Juan that what I had wanted toknow was whether, in the present, in our times, thepetty tyrants he had called small fry could ever defeata warrior.

"All the time," he replied. "The consequencesaren't as dire as those in the remote past. Today itgoes without saying that warriors always have achance to recuperate or to retrieve and come backlater. But there is another side to this problem. To bedefeated by a small-fry petty tyrant is not deadly, butdevastating. The degree of mortality, in a figurativesense, is almost as high. By that I mean that warriorswho succumb to a small-fry petty tyrant are obliter-ated by their own sense of failure and unworthiness.That spells high mortality to me."

"How do you measure defeat?"

"Anyone who joins the petty tyrant is defeated. Toact in anger, without control and discipline, to haveno forbearance, is to be defeated."

"What happens after warriors are defeated?"

"They either regroup themselves or they abandonthe quest for knowledge and join the ranks of the pettytyrants for life."

3The Eagle'sEmanations

The next day, don Juan and I went for a walk alongthe road to the city of Oaxaca. The road was desertedat that hour. It was 2: 00 p. m.

As we strolled leisurely, don Juan suddenly beganto talk. He said that our discussion about the pettytyrants had been merely an introduction to the topicof awareness. I remarked that it had opened a newview for me. He asked me to explain what I meant.

I told him that it had to do with an argument we hadhad some years before about the Yaqui Indians. In thecourse of his teachings for the right side, he had triedto tell me about the advantages that the Yaquis couldfind in being oppressed. I had passionately argued thatthere were no possible advantages in the wretchedconditions in which they lived. And I had told him thatI could not understand how, being a Yaqui himself, hedid not react against such a flagrant injustice.

He had listened attentively. Then, when I was surehe was going to defend his point, he agreed that theconditions of the Yaqui Indians were indeedwretched. But he pointed out that it was useless tosingle out the Yaquis when life conditions of man ingeneral were horrendous.

"Don't just feel sorry for the poor Yaqui Indians,"he had said. "Feel sorry for mankind. In the case ofthe Yaqui Indians, I can even say they're the luckyones. They are oppressed, and because of that, someof them may come out triumphant in the end. But theoppressors, the petty tyrants that tread upon them,they don't have a chance in hell."

I had immediately answered him with a barrage ofpolitical slogans. I had not understood his point at all.He again tried to explain to me the concept of pettytyrants, but the whole idea bypassed me. It was onlynow that everything fit into place.

"Nothing has fit into place yet," he said, laughingat what I had told him. "Tomorrow, when you are inyour normal state of awareness, you won't even re-member what you've realized now."

I felt utterly depressed, for I knew he was right.

"What's going to happen to you is what happenedto me," he continued. "My benefactor, the nagualJulian, made me realize in heightened awareness whatyou have realized yourself about petty tyrants. And Iended up, in my daily life, changing my opinions with-out knowing why.

"I had always been oppressed, so I had real venomtoward my oppressors, imagine my surprise when Ifound myself seeking the company of petty tyrants. Ithought I had lost my mind."

We came to a place, on the side of the road, wheresome large boulders were half buried by an old land-slide; don Juan headed for them and sat down on a flatrock. He signaled me to sit down, facing him. Andthen without further preliminaries, he started his ex-planation of the mastery of awareness.

He said that there were a series of truths that seers,old and new, had discovered about awareness, andthat such truths had been arranged in a specific se-quence for purposes of comprehension.

He explained that the mastery of awareness con-sisted in internalizing the total sequence of suchtruths. The first truth, he said, was that our familiaritywith the world we perceive compels us to believe thatwe are surrounded by objects, existing by themselvesand as themselves, just as we perceive them, whereas,in fact, there is no world of objects, but a universe ofthe Eagle's emanations.

He told me then that before he could explain theEagle's emanations, he had to talk about the known,the unknown, and the unknowable. Most of the truthsabout awareness were discovered by the old seers, hesaid. But the order in which they were arranged hadbeen worked out by the new seers. And without thatorder those truths were nearly incomprehensible.

He said that not to seek order was one of the greatmistakes that the ancient seers made. A deadly con-sequence of that mistake was their assumption thatthe unknown and the unknowable are the same thing.It was up to the new seers to correct that error. Theyset up boundaries and defined the unknown as some-thing that is veiled from man, shrouded perhaps by aterrifying context, but which, nonetheless, is withinman's reach. The unknown becomes the known at agiven time. The unknowable, on the other hand, is theindescribable, the unthinkable, the unrealizable. It issomething that will never be known to us, and yet it isthere, dazzling and at the same time horrifying in itsvastness.

"How can seers make the distinction between thetwo?" I asked.

"There is a simple rule of thumb," he said. "In theface of the unknown, man is adventurous. It is a qual-ity of the unknown to give us a sense of hope andhappiness. Man feels robust, exhilarated. Even theapprehension that it arouses is very fulfilling. The newseers saw that man is at his best in the face of theunknown."

He said that whenever what is taken to be the un-known turns out to be the unknowable the results aredisastrous. Seers feel drained, confused. A terribleoppression takes possession of them. Their bodieslose tone, their reasoning and sobriety wander awayaimlessly, for the unknowable has no energizing ef-fects whatsoever. It is not within human reach; there-fore, it should not be intruded upon foolishly or evenprudently. The new seers realized that they had to beprepared to pay exorbitant prices for the faintest con-tact with it.

Don Juan explained that the new seers had had for-midable barriers of tradition to overcome. At the timewhen the new cycle began, none of them knew forcertain which procedures of their immense traditionwere the right ones and which were not. Obviously,something had gone wrong with the ancient seers, butthe new seers did not know what. They began by as-suming that everything their predecessors had donewas erroneous. Those ancient seers had been the mas-ters of conjecture. They had, for one thing, assumedthat their proficiency in seeing was a safeguard. Theythought that they were untouchable?that is, until theinvaders smashed them, and put most of them to hor-rendous deaths. The ancient seers had no protectionwhatsoever, despite their total certainty that theywere invulnerable.

The new seers did not waste their time in specula-tions about what went wrong. Instead, they began tomap the unknown in order to separate it from the un-knowable.

"How did they map the unknown, don Juan?" Iasked.

"Through the controlled use of seeing," he replied.

I said that what I had meant to ask was, what wasentailed in mapping the unknown?

He answered that mapping the unknown meansmaking it available to our perception. By steadilypracticing seeing, the new seers found that the un-known and the known are really on the same footing,because both are within the reach of human percep-tion. Seers, in fact, can leave the known at a givenmoment and enter into the unknown.

Whatever is beyond our capacity to perceive is theunknowable. And the distinction between it and theknowable is crucial. Confusing the two would putseers in a most precarious position whenever they areconfronted with the unknowable.

"When this happened to the ancient seers," donJuan went on, "they thought their procedures hadgone haywire. It never occurred to them that most ofwhat's out there is beyond our comprehension. It wasa terrifying error of judgment on their part, for whichthey paid dearly."

"What happened after the distinction between theunknown and the unknowable was realized?" I asked.

"The new cycle began," he replied. "That distinc-tion is the frontier between the old and the new.Everything that the new seers have done stems fromunderstanding that distinction."

Don Juan said that seeing was the crucial elementin both the destruction of the ancient seers' world andin the reconstruction of the new view. It was throughseeing that the new seers discovered certain undeni-able facts, which they used to arrive at certain conclu-sions, revolutionary to them, about the nature of manand the world. These conclusions, which made thenew cycle possible, were the truths about awarenesshe was explaining to me.

Don Juan asked me to accompany him to the centerof town for a stroll around the square. On our way, webegan to talk about machines and delicate instru-ments. He said that instruments are extensions of oursenses, and I maintained that there are instrumentsthat are not in that category, because they performfunctions that we are not physiologically capable ofperforming.

"Our senses are capable of everything," he as-serted.

"I can tell you offhand that there are instrumentsthat can detect radio waves that come from outerspace," I said. "Our senses cannot detect radiowaves."

"I have a different idea," he said. "I think oursenses can detect everything we are surrounded by."

"What about the case of ultrasonic sounds?" I in-sisted. "We don't have the organic equipment to hearthem."

"It is the seers' conviction that we've tapped a verysmall portion of ourselves," he replied.

He immersed himself in thought for a while as if hewere trying to decide what to say next. Then hesmiled.

"The first truth about awareness, as I have alreadytold you," he began, "is that the world out there isnot really as we think it is. We think it is a world ofobjects and it's not."

He paused as if to measure the effect of his words.I told him that I agreed with his premise, becauseeverything could be reduced to being a field of energy.He said that I was merely intuiting a truth, and that toreason it out was not to verify it. He was not inter-ested in my agreement or disagreement, he said, butin my attempt to comprehend what was involved inthat truth.

"You cannot witness fields of energy," he went on."Not as an average man, that is. Now, if you wereable to see them, you would be a seer, in which caseyou would be explaining the truths about awareness.Do you understand what I mean?"

He went on to say that conclusions arrived atthrough reasoning had very little or no influence inaltering the course of our lives. Hence, the countlessexamples of people who have the clearest convictionsand yet act diametrically against them time and timeagain; and have as the only explanation for their be-havior the idea that to err is human.

"The first truth is that the world is as it looks andyet it isn't," he went on. "It's not as solid and real asour perception has been led to believe, but it isn't amirage either. The world is not an illusion, as it hasbeen said to be; it's real on the one hand, and unrealon the other. Pay close attention to this, for it must beunderstood, not just accepted. We perceive. This is ahard fact. But what we perceive is not a fact of thesame kind, because we learn what to perceive.

"Something out there is affecting our senses. Thisis the part that is real. The unreal part is what oursenses tell us is there. Take a mountain, for instance.Our senses tell us that it is an object. It has size, color,form. We even have categories of mountains, and theyare downright accurate. Nothing wrong with that; theflaw is simply that it has never occurred to us that oursenses play only a superficial role. Our senses per-ceive the way they do because a specific feature of ourawareness forces them to do so."

I began to agree with him again, but not because Iwanted to, for I had not quite understood his point.Rather, I was reacting to a threatening situation. Hemade me stop.

"I've used the term 'the world, ' " don Juan wenton, "to mean everything that surrounds us. I have abetter term, of course, but it would be quite incompre-hensible to you. Seers say that we think there is aworld of objects out there only because of our aware-ness. But what's really out there are the Eagle's ema-nations, fluid, forever in motion, and yet unchanged,eternal."

He stopped me with a gesture of his hand just as Iwas about to ask him what the Eagle's emanationswere. He explained that one of the most dramaticlegacies the old seers had left us was their discoverythat the reason for the existence of all sentient beingsis to enhance awareness. Don Juan called it a colossaldiscovery.

In a half-serious tone he asked me if I knew of abetter answer to the question that has always hauntedman: the reason for our existence. I immediately tooka defensive position and began to argue about themeaninglessness of the question because it cannot belogically answered. I told him that in order to discussthat subject we would have to talk about religious be-liefs and turn it all into a matter of faith.

"The old seers were not just talking about faith,"he said. "They were not as practical as the new seers,but they were practical enough to know what theywere seeing. What I was trying to point out to youwith that question, which has rattled you so badly, isthat our rationality alone cannot come up with an an-swer about the reason for our existence. Every time ittries, the answer turns into a matter of beliefs. The oldseers took another road, and they did find an answerwhich doesn't involve faith alone."

He said that the old seers, risking untold dangers,actually saw the indescribable force which is thesource of all sentient beings. They called it the Eagle,because in the few glimpses that they could sustain,they saw it as something that resembled a black-and-white eagle of infinite size.

They saw that it is the Eagle who bestows aware-ness. The Eagle creates sentient beings so that theywill live and enrich the awareness it gives them withlife. They also saw that it is the Eagle who devoursthat same enriched awareness after making sentientbeings relinquish it at the moment of death.

"For the old seers," don Juan went on, "to say thatthe reason for existence is to enhance awareness isnot a matter of faith or deduction. They saw it.

"They saw that the awareness of sentient beingsflies away at the moment of death and floats like aluminous cotton puff right into the Eagle's beak to beconsumed. For the old seers that was the evidencethat sentient beings live only to enrich the awarenessthat is the Eagle's food."

Don Juan's elucidation was interrupted because hehad to leave on a short business trip. Nestor drovehim to Oaxaca. As I saw them off, I remembered thatat the beginning of my association with don Juan,every time he mentioned a business trip I thought hewas employing a euphemism for something else. Ieventually realized that he meant what he said. When-ever such a trip was about to take place, he would puton one of his many immaculately tailored three-piecesuits and would look like anything but the old Indian Iknew. I had commented to him about the sophistica-tion of his metamorphosis.

"A nagual is someone flexible enough to be any-thing," he had said. "To be a nagual, among otherthings, means to have no points to defend. Rememberthis?we'll come back to it over and over."

We had come back to it over and over, in everypossible way; he did indeed seem to have no points todefend, but during his absence in Oaxaca I was givento just a shadow of doubt. Suddenly I realized that anagual did have one point to defend?the descriptionof the Eagle and what it does required, in my opinion,a passionate defense.

I tried to pose that question to some of don Juan'scompanions, but they eluded my probings. They toldme that I was in quarantine from that kind of discus-sion until don Juan had finished his explanation.

The moment he returned, we sat down to talk and Iasked him about it.

"Those truths are not something to defend passion-ately," he replied. "If you think that I'm trying todefend them, you are mistaken. Those truths were puttogether for the delight and enlightenment of warriors,not to engage any proprietary sentiments. When I toldyou that a nagual has no points to defend, I meant,among other things, that a nagual has no obsessions."

I told him that I was not following his teachings, forI had become obsessed with his description of theEagle and what it does. I remarked over and overabout the awesomeness of such an idea.

"It is not just an idea," he said. "It is a fact. And adamn scary one if you ask me. The new seers werenot simply playing with ideas."

"But what kind of a force would the Eagle be?"

"I wouldn't know how to answer that. The Eagle isas real for the seers as gravity and time are for you,and just as abstract and incomprehensible."

"Wait a minute, don Juan. Those are abstract con-cepts, but they do refer to real phenomena that can becorroborated. There are whole disciplines dedicatedto that."

"The Eagle and its emanations are equally corrob-oratable," don Juan retorted. "And the discipline ofthe new seers is dedicated to doing just that."

I asked him to explain what the Eagle's emanationsare.

He said that the Eagle's emanations are an immut-able thing-in-itself, which engulfs everything that ex-ists, the knowable and the unknowable.

"There is no way to describe in words what theEagle's emanations really are," don Juan continued."A seer must witness them."

"Have you witnessed them yourself, don Juan?"

"Of course I have, and yet I can't tell you what theyare. They are a presence, almost a mass of sorts, apressure that creates a dazzling sensation. One cancatch only a glimpse of them, as one can catch only aglimpse of the Eagle itself."

"Would you say, don Juan, that the Eagle is thesource of the emanations?"

"It goes without saying that the Eagle is the sourceof its emanations."

"I meant to ask if that is so visually."

"There is nothing visual about the Eagle. The entirebody of a seer senses the Eagle. There is something inall of us that can make us witness with our entirebody. Seers explain the act of seeing the Eagle in verysimple terms: because man is composed of the Eagle'semanations, man need only revert back to his compo-nents. The problem arises with man's awareness; it ishis awareness that becomes entangled and confused.At the crucial moment when it should be a simple caseof the emanations acknowledging themselves, man'sawareness is compelled to interpret. The result is avision of the Eagle and the Eagle's emanations. Butthere is no Eagle and no Eagle's emanations. What isout there is something that no living creature cangrasp."

I asked him if the source of the emanations wascalled the Eagle because eagles in general have impor-tant attributes.

"This is simply the case of something unknowablevaguely resembling something known," he replied."On account of that, there have certainly been at-tempts to imbue eagles with attributes they don'thave. But that always happens when impressionablepeople learn to perform acts that require great sobri-ety. Seers come in all sizes and shapes."

"Do you mean to say that there are different kindsof seers?"

"No. I mean that there are scores of imbeciles whobecome seers. Seers are human beings full of foibles,or rather, human beings full of foibles are capable ofbecoming seers. Just as in the case of miserable peoplewho become superb scientists.

"The characteristic of miserable seers is that theyare willing to forget the wonder of the world. Theybecome overwhelmed by the fact that they see andbelieve that it's their genius that counts. A seer mustbe a paragon in order to override the nearly invinciblelaxness of our human condition. More important thanseeing itself is what seers do with what they see."

"What do you mean by that, don Juan?"

"Look at what some seers have done to us. We arestuck with their vision of an Eagle that rules us anddevours us at the moment of our death."

He said that there is a definite laxness in that ver-sion, and that personally he did not appreciate the ideaof something devouring us. For him, it would be moreaccurate to say that there is a force that attracts ourconsciousness, much as a magnet attracts iron shav-ings. At the moment of dying, all of our being disinte-grates under the attraction of that immense force.

That such an event was interpreted as the Eagledevouring us he found grotesque, because it turns anindescribable act into something as mundane as eat-ing.

"I'm a very average man," I said. "The descriptionof an Eagle that devours us had a great impact onme."

"The real impact can't be measured until the mo-ment when you see it yourself," he said. "But youmust bear in mind that our flaws remain with us evenafter we become seers. So when you see that force,you may very well agree with the lax seers who calledit the Eagle, as I did myself. On the other hand, youmay not. You may resist the temptation to ascribehuman attributes to what is incomprehensible, and ac-tually improvise a new name for it, a more accurateone."

"Seers who see the Eagle's emanations often callthem commands," don Juan said. "I wouldn't mindcalling them commands myself if I hadn't got used tocalling them emanations. It was a reaction to my ben-efactor's preference; for him they were commands. Ithought that term was more in keeping with his force-ful personality than with mine. I wanted somethingimpersonal. 'Commands' sounds too human to me,but that's what they really are, commands."

Don Juan said that to see the Eagle's emanations isto court disaster. The new seers soon discovered thetremendous difficulties involved, and only after greattribulations in trying to map the unknown and separateit from the unknowable did they realize that every-thing is made out of the Eagle's emanations. Only asmall portion of those emanations is within reach ofhuman awareness, and that small portion is still fur-ther reduced, to a minute fraction, by the constraintsof our daily lives. That minute fraction of the Eagle'semanations is the known; the small portion within pos-sible reach of human awareness is the unknown, andthe incalculable rest is the unknowable.

He went on to say that the new seers, being prag-matically oriented, became immediately cognizant ofthe compelling power of the emanations. They real-ized that all living creatures are forced to employ theEagle's emanations without ever knowing what theyare. They also realized that organisms are constructedto grasp a certain range of those emanations and thatevery species has a definite range. The emanationsexert great pressure on organisms, and through thatpressure organisms construct their perceivable world.

"In our case, as human beings," don Juan said, "weemploy those emanations and interpret them as real-ity. But what man senses is such a small portion of theEagle's emanations that it's ridiculous to put muchstock in our perceptions, and yet it isn't possible forus to disregard our perceptions. The new seers foundthis out the hard way?after courting tremendous dan-gers."

Don Juan was sitting where he usually sat in thelarge room. Ordinarily there was no furniture in thatroom?people sat on mats on the floor?but Carol,the nagual woman, had managed to furnish it withvery comfortable armchairs for the sessions when sheand I took turns reading to him from the works ofSpanish-speaking poets.

"I want you to be very aware of what we aredoing," he said as soon as I sat down. "We are dis-cussing the mastery of awareness. The truths we'rediscussing are the principles of that mastery."

He added that in his teachings for the right side hehad demonstrated those principles to my normalawareness with the help of one of his seer compan-ions, Genaro, and that Genaro had played around withmy awareness with all the humor and irreverence forwhich the new seers were known.

"Genaro is the one who should be here telling youabout the Eagle," he said, "except that his versionsare too irreverent. He thinks that the seers who calledthat force the Eagle were either very stupid or weremaking a grand joke, because eagles not only lay eggs,they also lay turds."

Don Juan laughed and said that he found Genaro'scomments so appropriate that he couldn't resist laugh-ter. He added that if the new seers had been the onesto describe the Eagle the description would certainlyhave been made half in fun.

I told don Juan that on one level I took the Eagle asa poetic image, and as such it delighted me, but onanother level I took it literally, and that terrified me.

"One of the greatest forces in the lives of warriorsis fear," he said. "It spurs them to learn."

He reminded me that the description of the Eaglecame from the ancient seers. The new seers werethrough with description, comparison, and conjectureof any sort. They wanted to get directly to the sourceof things and consequently risked unlimited danger toget to it. They did see the Eagle's emanations. Butthey never tampered with the description of the Eagle.They felt that it took too much energy to see theEagle, and that the ancient seers had already paidheavily for their scant glimpse of the unknowable.

"How did the old seers come around to describingthe Eagle?" I asked.

"They needed a minimal set of guidelines about theunknowable for purposes of instruction," he replied."They resolved it with a sketchy description of theforce that rules all there is, but not of its emanations,because the emanations cannot be rendered at all in alanguage of comparisons. Individual seers may feelthe urge to make comments about certain emanations,but that will remain personal, in other words, there isno pat version of the emanations, as there is of theEagle."

"The new seers seem to have been very abstract,"I commented. "They sound like modern-day philoso-phers."

"No. The new seers were terribly practical men,"he replied. "They weren't involved in concocting ra-tional theories."

He said that the ancient seers were the ones whowere the abstract thinkers. They built monumental ed-ifices of abstractions proper to them and their time.And just like the modern-day philosophers, they werenot at all in control of their concatenations. The newseers, on the other hand, imbued with practicality,were able to see a flux of emanations and to see howman and other living beings utilize them to constructtheir perceivable world.

"How are those emanations utilized by man, donJuan?"

"It's so simple it sounds idiotic. For a seer, men areluminous beings. Our luminosity is made up of thatportion of the Eagle's emanations which is encased inour egglike cocoon. That particular portion, that hand-ful of emanations that is encased, is what makes usmen. To perceive is to match the emanations con-tained inside our cocoon with those that are outside.

"Seers can see, for instance, the emanations insideany living creature and can tell which of the outsideemanations would match them."

"Are the emanations like beams of light?" I asked.

"No. Not at all. That would be too simple. They aresomething indescribable. And yet, my personal com-ment would be to say that they are like filaments oflight. What's incomprehensible to normal awarenessis that the filaments are aware. I can't tell you whatthat means, because I don't know what I am saying.All I can tell you with my personal comments is thatthe filaments are aware of themselves, alive and vi-brating, that there are so many of them that numbershave no meaning and that each of them is an eternityin itself."

4The Glowof Awareness

Don Juan, don Genaro, and I had just returned fromgathering plants in the surrounding mountains. Wewere at don Genaro's house, sitting around the table,when don Juan made me change levels of awareness.Don Genaro had been staring at me and began tochuckle. He remarked how odd he thought it was thatI had two completely different standards for dealingwith the two sides of awareness. My relation with himwas the most obvious example. On my right side, hewas the respected and feared sorcerer don Genaro, aman whose incomprehensible acts delighted me and atthe same time filled me with mortal terror. On my leftside, he was plain Genaro, or Genarito, with no donattached to his name, a charming and kind seer whoseacts were thoroughly comprehensible and coherentwith what I myself did or tried to do.

I agreed with him and added that on my left side,the man whose mere presence made me shake like aleaf was Silvio Manuel, the most mysterious of donJuan's companions. I also said that don Juan, being atrue nagual, transcended arbitrary standards and wasrespected and admired by me in both states.

"But is he feared?" Genaro asked in a quiveringvoice.

"Very feared," don Juan interjected in a falsettovoice.

We all laughed, but don Juan and Genaro laughedwith such abandon that I immediately suspected theyknew something they were holding back.

Don Juan was reading me like a book. He explainedthat in the intermediate stage, before one enters fullyinto the left-side awareness, one is capable of tremen-dous concentration, but one is also susceptible toevery conceivable influence. I was being influencedby suspicion.

"La Gorda is always in this stage," he said. "Shelearns beautifully, but she's a royal pain in the neck.She can't help being driven by anything that comesher way, including, of corse, very good things, likekeen concentration."

Don Juan explained that the new seers discoveredthat the transition period is the time when the deepestlearning takes place, and that it is also the time whenwarriors must be supervised and explanations must begiven to them so they can evaluate them properly. Ifno explanations are given to them before they enterinto the left side, they will be great sorcerers but poorseers, as the ancient Toltecs were.

Female warriors in particular fall prey to the lure ofthe left side, he said. They are so nimble that they cango into the left side with no effort, often too soon fortheir own good.

After a long silence, Genaro fell asleep. Don Juanbegan to speak. He said that the new seers had had toinvent a number of terms in order to explain the sec-ond truth about awareness. His benefactor hadchanged some of those terms to suit himself, and hehimself had done the same, guided by the seers' beliefthat it does not make any difference what terms areused as long as the truths have been verified by seeing.

I was curious to know what terms he had changed,but I didn't know quite how to word my question. Hetook it that I was doubting his right or his ability tochange them and explained that if the terms we pro-pose originate in our reason they can only communi-cate the mundane agreement of everyday life. Whenseers propose a term, on the other hand, it is never afigure of speech because it stems from seeing and em-braces everything that seers can attain.

I asked him why he had changed the terms.

"It's a nagual's duty always to look for better waysto explain," he replied. "Time changes everything,and every new nagual has to incorporate new words,new ideas, to describe his seeing. '"

"Do you mean that a nagual takes ideas from theworld of every day life?" I asked.

"No. I mean that a nagual talks about seeing in evernew ways," he said. "For instance, as the new na-gual, you'd have to say that awareness gives rise toperception. You'd be saying the same thing my bene-factor said, but in a different way."

"What do the new seers say perception is, donJuan?"

"They say that perception is a condition of align-ment; the emanations inside the cocoon becomealigned with those outside that fit them. Alignment iswhat allows awareness to be cultivated by every livingcreature. Seers make these statements because theysee living creatures as they really are: luminous beingsthat look like bubbles of whitish light."

I asked him how the emanations inside the cocoonfit those outside so as to accomplish perception.

"The emanations inside and the emanations out-side," he said, "are the same filaments of light. Sen-tient beings are minute bubbles made out of thosefilaments, microscopic points of light, attached to theinfinite emanations."

He went on to explain that the luminosity of livingbeings is made by the particular portion of the Eagle'semanations they happen to have inside their luminouscocoons. When seers see perception, they witnessthat the luminosity of the Eagle's emanations outsidethose creatures' cocoons brightens the luminosity ofthe emanations inside their cocoons. The outside lu-minosity attracts the inside one; it traps it, so to speak,and fixes it. That fixation is the awareness of everyspecific being.

Seers can also see how the emanations outside thecocoon exert a particular pressure on the portion ofemanations inside. This pressure determines the de-gree of awareness that every living being has.

I asked him to clarify how the Eagle's emanationsoutside the cocoon exert pressure on those inside.

"The Eagle's emanations are more than filaments oflight," he replied. "Each one of them is a source ofboundless energy. Think of it this way: since some ofthe emanations outside the cocoon are the same as theemanations inside, their energies are like a continuouspressure. But the cocoon isolates the emanations thatare inside its web and thereby directs the pressure.

"I've mentioned to you that the old seers were mas-ters of the art of handling awareness," he went on."What I can add now is that they were the masters ofthat art because they learned to manipulate the struc-ture of man's cocoon. I've said to you that they unrav-eled the mystery of being aware. By that I meant thatthey saw and realized that awareness is a glow in thecocoon of living beings. They rightly called it the glowof awareness."

He explained that the old seers saw that man'sawareness is a glow of amber luminosity more intensethan the rest of the cocoon. That glow is on a narrow,vertical band on the extreme right side of the cocoon,running along its entire length. The mastery of the oldseers was to move that glow, to make it spread fromits original setting on the surface of the cocoon inwardacross its width.

He stopped talking and looked at Genaro, who wasstill sound asleep.

"Genaro doesn't give a fig about explanations," hesaid. "He's a doer. My benefactor pushed him con-stantly to face insoluble problems. So he entered intothe left side proper and never had a chance to ponderand wonder."

"Is it better to be that way, don Juan?"

"It depends. For him, it's perfect. For you and forme, it wouldn't be satisfactory, because in one way oranother we are called upon to explain. Genaro or mybenefactor are more like the old than the new seers:they can control and do what they want with the glowof awareness."

He stood up from the mat where we were sitting andstretched his arms and legs. I pressed him to continuetalking. He smiled and said that I needed to rest, thatmy concentration was waning.

There was a knock at the door. I woke up. It wasdark. For a moment I could not remember where Iwas. There was something in me that was far away, asif part of me were still asleep, yet I was fully awake.Enough moonlight came through the open window sothat I could see.

I saw don Genaro get up and go to the door. I real-ized then that I was at his house. Don Juan was soundasleep on a mat on the floor. I had the distinct impres-sion that the three of us had fallen asleep after return-ing dead tired from a trip to the mountains.

Don Genaro lit his kerosene lantern. I followed himinto the kitchen. Someone had brought him a pot ofhot stew and a stack of tortillas.

"Who brought you food?" I asked him. "Do youhave a woman around here that cooks for you?"

Don Juan had come into the kitchen. Both of themlooked at me, smiling. For some reason their smileswere terrifying to me. I was about to scream in terror,in fact, when don Juan hit me on the back and mademe shift into a state of heightened awareness. I real-ized then that perhaps during my sleep, or as I wokeup, I had drifted back to everyday awareness.

The sensation I experienced then, once I was backin heightened awareness, was a mixture of relief andanger and the most acute sadness. I was relieved thatI was myself again, for I had come to regard thoseincomprehensible states as being my true self. Therewas one simple reason for that?in those states I feltcomplete; nothing was missing from me. The angerand the sadness were a reaction to impotence. I wasmore aware than ever of the limitations of my being.

I asked don Juan to explain to me how it was pos-sible for me to do what I was doing. In states ofheightened awareness I could look back and remem-ber everything about myself; I could give an accountof everything I had done in either state; I could evenremember my incapacity to recollect. But once I hadreturned to my normal, everyday level of awareness Icould not recall anything I had done in heightenedawareness, even if my life depended on it.

"Hold it, hold it there," he said. "You haven't re-membered anything yet. Heightened awareness isonly an intermediate state. There is infinitely morebeyond that, and you have been there many, manytimes. Right now you can't remember, even if yourlife depends on it."

He was right. I had no idea what he was talkingabout. I pleaded for an explanation.

"The explanation is coming," he said. "It's a slowprocess, but we'll get to it. It is slow because I am justlike you: I like to understand. I am the opposite of mybenefactor, who was not given to explaining. For himthere was only action. He used to put us squarelyagainst incomprehensible problems and let us resolvethem for ourselves. Some of us never did resolve any-thing, and we ended up very much in the same boatwith the old seers: all action and no real knowledge."

"Are those memories trapped in my mind?" Iasked.

"No. That would make it too simple," he replied."The actions of seers are more complex than dividinga man into mind and body. You have forgotten whatyou've done, or what you've witnessed, because whenyou were performing what you've forgotten you wereseeing."

I asked don Juan to reinterpret what he had justsaid.

Patiently, he explained that everything I had forgot-ten had taken place in states in which my everydayawareness had been enhanced, intensified, a conditionthat meant that other areas of my total being wereused.

"Whatever you've forgotten is trapped in thoseareas of your total being," he said. "To be using thoseother areas is to see."

"I'm more confused than ever, don Juan," I said.

"I don't blame you," he said. "Seeing is to lay barethe core of everything, to witness the unknown and toglimpse into the unknowable. As such, it doesn't bringone solace. Seers ordinarily go to pieces on finding outthat existence is incomprehensibly complex and thatour normal awareness maligns it with its limitations."

He reiterated that my concentration had to be total,that to understand was of crucial importance, that thenew seers placed the highest value on deep, unemo-tional realizations.

"For instance, the other day," he went on, "whenyou understood about la Gorda's and your self-impor-tance, you didn't understand anything really. You hadan emotional outburst, that was all. I say this becausethe next day you were back on your high horse of self-importance as if you never had realized anything.

"The same thing happened to the old seers. Theywere given to emotional reactions. But when the timecame for them to understand what they had seen, theycouldn't do it. To understand one needs sobriety, notemotionality. Beware of those who weep with reali-zation, for they have realized nothing.

"There are untold dangers in the path of knowledgefor those without sober understanding," he continued."I am outlining the order in which the new seers ar-ranged the truths about awareness, so it will serve youas a map. a map that you have to corroborate withyour seeing, but not with your eyes."

There was a long pause. He stared at me. He wasdefinitely waiting for me to ask him a question.

"Everybody falls prey to the mistake that seeing isdone with the eyes," he continued. "But don't besurprised that after so many years you haven't real-ized yet that seeing is not a matter of the eyes. It'squite normal to make that mistake."

"What is seeing, then?" I asked.

He replied that seeing is alignment. And I remindedhim that he had said that perception is alignment. Heexplained then that the alignment of emanations usedroutinely is the perception of the day-to-day world,but the alignment of emanations that are never usedordinarily is seeing. When such an alignment occursone sees. Seeing, therefore, being produced by align-ment out of the ordinary, cannot be something onecould merely look at. He said that in spite of the factthat I had seen countless times, it had not occurred tome to disregard my eyes. I had succumbed to the wayseeing is labeled and described.

"When seers see, something explains everything asthe new alignment takes place," he continued. "It's avoice that tells them in their ear what's what. If thatvoice is not present, what the seer is engaged in isn'tseeing. "

After a moment's pause, he continued explainingthe voice of seeing. He said that it was equally falla-cious to say that seeing was hearing, because it wasinfinitely more than that, but that seers had opted forusing sound as a gauge of a new alignment.

He called the voice of seeing a most mysteriousinexplicable thing. "My personal conclusion is thatthe voice of seeing belongs only to man," he said. "Itmay happen because talking is something that no oneelse besides man does. The old seers believed it wasthe voice of an overpowering entity intimately relatedto mankind, a protector of man. The new seers foundout that that entity, which they called the mold ofman, doesn't have a voice. The voice of seeing for thenew seers is something quite Incomprehensible; theysay it's the glow of awareness playing on the Eagle'semanations as a harpist plays on a harp."

He refused to explain it any further, arguing thatlater on, as he proceeded with his explanation, every-thing would become clear to me.

My concentration had been so total while don Juanspoke that I actually did not remember sitting down atthe table to eat. When don Juan stopped talking, Inoticed that his plate of stew was nearly finished.

Genaro was staring at me with a beaming smile. Myplate was in front of me on the table, and it too wasempty. There was only a tiny residue of stew left in it,as if I had just finished eating. I did not remembereating it at all, but neither did I remember walking tothe table or sitting down.

"Did you like the stew?" Genaro asked me andlooked away.

I said I did, because I did not want to admit that Iwas having problems recollecting.

"It had too much chile for my taste," Genaro said."You never eat hot food yourself, so I'm sort of wor-ried about what it will do to you. You shouldn't haveeaten two servings. I suppose you're a little more pig-gish when you're in heightened awareness, eh?"

I admitted that he was probably right. He handedme a large pitcher of water to quench my thirst andsoothe my throat. When I eagerly drank all of it, bothof them broke into howling laughter.

Suddenly, I realized what was going on. My reali-zation was physical. It was a flash of yellowish lightthat hit me as if a match had been struck right betweenmy eyes. I knew then that Genaro was joking. I hadnot eaten. I had been so absorbed in don Juan's expla-nation that I had forgotten about everything else. Theplate in front of me was Genaro's.

After dinner don Juan went on with his explanationabout the glow of awareness. Genaro sat by me, listen-ing as if he had never heard the explanation before.

Don Juan said that the pressure that the emanationsoutside the cocoon, which are called emanations atlarge, exert on the emanations inside the cocoon is thesame in all sentient beings. Yet the results of that pres-sure are vastly different among them, because theircocoons react to that pressure in every conceivableway. There are. however, degrees of uniformitywithin certain boundaries.

"Now," he went on, "when seers see that the pres-sure of the emanations at large bears down on theemanations inside, which are always in motion, andmakes them stop moving, they know that the luminousbeing at that moment is fixated by awareness.

"To say that the emanations at large bear down onthose inside the cocoon and make them stop movingmeans that seers see something indescribable, themeaning of which they know without a shadow ofdoubt. It means that the voice of seeing has told themthat the emanations inside the cocoon are completelyat rest and match some of those which are outside."

He said that seers maintain, naturally, that aware-ness always comes from outside ourselves, that thereal mystery is not inside us. Since by nature the em-anations at large are made to fixate what is inside thecocoon, the trick of awareness is to let the fixatingemanations merge with what is inside us. Seers be-lieve that if we let that happen we become what wereally are?fluid, forever in motion, eternal.

There was a long pause. Don Juan's eyes had anintense shine. They seemed to be looking at me froma great depth. I had the feeling that each of his eyeswas an independent point of brilliance. For an instanthe appeared to be struggling against an invisible force,a fire from within that intended to consume him. Itpassed and he went on talking.

"The degree of awareness of every individual sen-tient being," he continued, "depends on the degree towhich it is capable of letting the pressure of the ema-nations at large carry it."

After a long interruption, don Juan continued ex-plaining. He said that seers saw that from the momentof conception awareness is enhanced, enriched, by theprocess of being alive. He said that seers saw, forinstance, that the awareness of an individual insect orthat of an individual man grows from the moment ofconception in astoundingly different ways, but withequal consistency.

"Is it from the moment of conception or from themoment of birth that awareness develops?" I asked.

"Awareness develops from the moment of concep-tion," he replied. "I have always told you that sexualenergy is something of ultimate importance and that ithas to be controlled and used with great care. But youhave always resented what I said, because youthought I was speaking of control in terms of morality;I always meant it in terms of saving and rechannelingenergy."

Don Juan looked at Genaro. Genaro nodded hishead in approval.

"Genaro is going to tell you what our benefactor,the nagual Julian, used to say about saving and re-channeling sexual energy," don Juan said to me.

"The nagual Julian used to say that to have sex is amatter of energy," Genaro began. "For instance, henever had any problems having sex, because he hadbushels of energy. But he took one look at me andprescribed right away that my peter was just forpeeing. He told me that I didn't have enough energyto have sex. He said that my parents were too boredand too tired when they made me; he said that I wasthe result of very boring sex, cojida aburrida. I wasborn like that, bored and tired. The nagual Julian rec-ommended that people like me should never have sex;this way we can store the little energy we have.

"He said the same thing to Silvio Manuel and toEmilito. He saw that the others had enough energy.They were not the result of bored sex. He told themthat they could do anything they wanted with theirsexual energy, but he recommended that they controlthemselves and understand the Eagle's command thatsex is for bestowing the glow of awareness. We allsaid we had understood.

"One day, without any warning at all, he openedthe curtain of the other world with the help of his ownbenefactor, the nagual Ellas, and pushed all of us in-side, with no hesitation whatsoever. All of us, exceptSilvio Manuel, nearly died in there. We had no energyto withstand the impact of the other world. None ofus, except Silvio Manuel, had followed the nagual'srecommendation."

"What is the curtain of the other world?" I askeddon Juan.

"What Genaro said?it is a curtain," don Juan re-plied. "But you're getting off the subject. You alwaysdo. We're talking about the Eagle's command aboutsex. It is the Eagle's command that sexual energy beused for creating life. Through sexual energy, theeagle bestows awareness. So when sentient beings areengaged in sexual intercourse, the emanations insidetheir cocoons do their best to bestow awareness to thenew sentient being they are creating."

He said that during the sexual act, the emanationsencased inside the cocoon of both partners undergo aprofound agitation, the culminating point of which is amerging, a fusing of two pieces of the glow of aware-ness, one from each partner, that separate from theircocoons.

"Sexual intercourse is always a bestowal of aware-ness even though the bestowal may not be consoli-dated," he went on. "The emanations inside thecocoon of human beings don't know of intercourse forfun."

Genaro leaned over toward me from his chair acrossthe table and talked to me in a low voice, shaking hishead for emphasis.

"The nagual is telling you the truth," he said andwinked at me. "Those emanations really don'tknow."

Don Juan fought not to laugh and added that thefallacy of man is to act with total disregard for themystery of existence and to believe that such a sub-lime act of bestowing life and awareness is merely aphysical drive that one can twist at will.

Genaro made obscene sexual gestures, twisting hispelvis around, on and on. Don Juan nodded and saidthat that was exactly what he meant. Genaro thankedhim for acknowledging his one and only contributionto the explanation of awareness.

Both of them laughed like idiots, saying that if I hadknown how serious their benefactor was about theexplanation of awareness, I would be laughing withthem.

I earnestly asked don Juan what all this meant foran average man in the day-to-day world.

"You mean what Genaro is doing?" he asked me inmock seriousness.

Their glee was always contagious. It took a longtime for them to calm down. Their level of energy wasso high that next to them, I seemed old and decrepit.

"I really don't know," don Juan finally answeredme. "All I know is what it means to warriors. Theyknow that the only real energy we possess is a life-bestowing sexual energy. This knowledge makes thempermanently conscious of their responsibility.

"If warriors want to have enough energy to see,they must become misers with their sexual energy.That was the lesson the nagual Julian gave us. Hepushed us into the unknown, and we all nearly died.Since everyone of us wanted to see, we, of course,abstained from wasting our glow of awareness."

I had heard him voice that belief before. Every timehe did, we got into an argument. I always felt com-pelled to protest and raise objections to what I thoughtwas a puritanical attitude toward sex.

I again raised my objections. Both of them laughedto tears.

"What can be done with man's natural sensuality?"I asked don Juan.

"Nothing," he replied. "There is nothing wrongwith man's sensuality, it's man's ignorance of and dis-regard for his magical nature that is wrong. It's a mis-take to waste recklessly the life-bestowing force of sexand not have children, but it's also a mistake not toknow that in having children one taxes the glow ofawareness."

"How do seers know that having children taxes theglow of awareness?" I asked.

"They see that on having a child, the parents' glowof awareness diminishes and the child's increases. Insome supersensitive, frail parents, the glow of aware-ness almost disappears. As children enhance theirawareness, a big dark spot develops in the luminouscocoon of the parents, on the very place from whichthe glow was taken away. It is usually on the midsec-tion of the cocoon. Sometimes those spots can evenbe seen superimposed on the body itself."

I asked him if there was anything that could be doneto give people a more balanced understanding of theglow of awareness.

"Nothing," he said. "At least, there is nothing thatseers can do. Seers aim to be free, to be unbiasedwitnesses incapable of passing judgment; otherwisethey would have to assume the responsibility forbringing about a more adjusted cycle. No one can dothat. The new cycle, if it is to come, must come ofitself."

5The First Attention

The following day we ate breakfast at dawn, then donJuan made me shift levels of awareness.

"Today, let's go to an original setting," don Juansaid to Genaro.

"By all means," Genaro said gravely. He glancedat me and then added in a low voice, as if not wantingme to overhear him, "Does he have to. . . perhapsit's too much. . ."

In a matter of seconds my fear and suspicion esca-lated to unbearable heights. I was sweating and pant-ing. Don Juan came to my side and, with anexpression of almost uncontrollable amusement, as-sured me that Genaro was just entertaining himself atmy expense, and that we were going to a place wherethe original seers had lived thousands of years ago.

As don Juan was speaking to me, I happened toglance at Genaro. He slowly shook his head from sideto side. It was an almost imperceptible gesture, as ifhe were letting me know that don Juan was not tellingthe truth. I went into a state of nervous frenzy, nearhysteria?and stopped only when Genaro burst intolaughter.

I marveled how easily my emotional states couldescalate to nearly unmanageable heights or drop tonothing.

Don Juan, Genaro, and I left Genaro's house in theearly morning and traveled a short distance into thesurrounding eroded hills. Presently we stopped andsat down on top of an enormous flat rock, on a gradualslope, in a corn field that seemed to have been recentlyharvested.

"This is the original setting," don Juan said to me."We'll come back here a couple more times, duringthe course of my explanation."

"Very weird things happen here at night," Genarosaid. "The nagual Julian actually caught an ally here.Or rather, the ally ..."

Don Juan made a noticeable gesture with his eye-brows and Genaro stopped in midsentence. He smiledat me.

"It's too early in the day for scary stories," Genarosaid. "Let's wait until dark."

He stood up and began creeping all around the rock,tiptoeing with his spine arched backward.

"What was he saying about your benefactor'scatching an ally here?" I asked don Juan.

He did not answer right away. He was ecstatic,watching Genaro's antics.

"He was referring to some sophisticated use ofawareness," he finally replied, still staring at Genaro.

Genaro completed a circle around the rock andcame back and sat down by me. He was panting heav-ily, almost wheezing, out of breath.

Don Juan seemed fascinated by what Genaro haddone. Again I had the feeling that they were amusingthemselves at my expense, that both of them were upto something I knew nothing about.

Suddenly, don Juan began his explanation. Hisvoice soothed me. He said that after much toiling,seers arrived at the conclusion that the consciousnessof adult human beings, matured by the process ofgrowth, can no longer be called awareness, because ithas been modified into something more intense andcomplex, which seers call attention.

"How do seers know that man's awareness is beingcultivated and that it grows?" I asked.

He said that at a given time in the growth of humanbeings a band of the emanations inside their cocoonsbecomes very bright; as human beings accumulate ex-perience, it begins to glow. In some instances, theglow of this band of emanations increases so dramati-cally that it fuses with the emanations from the out-side. Seers, witnessing an enhancement of this kind,had to surmise that awareness is the raw material andattention the end product of maturation.

"How do seers describe attention?" I asked.

"They say that attention is the harnessing and en-hancing of awareness through the process of beingalive," he replied.

He said that the danger of definitions is that theysimplify matters to make them understandable; in thiscase, in defining attention, one runs the risk of trans-forming a magical, miraculous accomplishment intosomething commonplace. Attention is man's greatestsingle accomplishment. It develops from raw animalawareness until it covers the entire gamut of humanalternatives. Seers perfect it even further until it cov-ers the whole scope of human possibilities.

I wanted to know if there was a special significanceto alternatives and possibilities in the seers' view.

Don Juan replied that human alternatives are every-thing we are capable of choosing as persons. Theyhave to do with the level of our day-to-day range, theknown; and owing to that fact, they are quite limitedin number and scope. Human possibilities belong tothe unknown. They are not what we are capable ofchoosing but what we are capable of attaining. He saidthat an example of human alternatives is our choice tobelieve that the human body is an object among ob-jects. An example of human possibilities is the seers'achievement in viewing man as an egglike luminousbeing. With the body as an object one tackles theknown, with the body as a luminous egg one tacklesthe unknown; human possibilities have, therefore,nearly an inexhaustible scope.

"Seers say that there are three types of attention,"don Juan went on. "When they say that, they mean itjust for human beings, not for all the sentient beingsin existence. But the three are not just types of atten-tion, they are rather three levels of attainment. Theyare the first, second, and third attention, each of theman independent domain, complete in itself."

He explained that the first attention in man is animalawareness, which has been developed, through theprocess of experience, into a complex, intricate, andextremely fragile faculty that takes care of the day-to-day world in all its innumerable aspects, in otherwords, everything that one can think about is part ofthe first attention.

"The first attention is everything we are as averagemen," he continued. "By virtue of such an absoluterule over our lives, the first attention is the most valu-able asset that the average man has. Perhaps it is evenour only asset.

"Taking into account its true value, the new seersstarted a rigorous examination of the first attentionthrough seeing. Their findings molded their total out-look and the outlook of all their descendants, eventhough most of them do not understand what thoseseers really saw."

He emphatically warned me that the conclusions ofthe new seers' rigorous examination had very little todo with reason or rationality, because in order to ex-amine and explain the first attention, one must see it.Only seers can do that. But to examine what seers seein the first attention is essential. It allows the firstattention the only opportunity it will ever have to re-alize its own workings.

"In terms of what seers see, the first attention is theglow of awareness developed to an ultra shine," hecontinued. "But it is a glow fixed on the surface of thecocoon, so to speak. It is a glow that covers theknown.

"The second attention, on the other hand, is a morecomplex and specialized state of the glow of aware-ness. It has to do with the unknown. It comes aboutwhen unused emanations inside man's cocoon are uti-lized.

"The reason I called the second attention special-ized is that in order to utilize those unused emana-tions, one needs uncommon, elaborate tactics thatrequire supreme discipline and concentration."

He said that he had told me before, when he wasteaching me the art of dreaming, that the concentra-tion needed to be aware that one is having a dream isthe forerunner of the second attention. That concen-tration is a form of consciousness that is not in thesame category as the consciousness needed to dealwith the daily world.

He said that the second attention is also called theleft-side awareness; and it is the vastest field that onecan imagine, so vast in fact that it seems limitless.

"I wouldn't stray into it for anything in this world,"he went on. "It is a quagmire so complex and bizarrethat sober seers go into it only under the strictest con-ditions.

"The great difficulty is that the entrance into thesecond attention is utterly easy and its lure nearly ir-resistible."

He said that the old seers, being the masters ofawareness, applied their expertise to their own glowsof awareness and made them expand to inconceivablelimits. They actually aimed at lighting up all the ema-nations inside their cocoons, one band at a time. Theysucceeded, but oddly enough the accomplishment oflighting up one band at a time was instrumental in theirbecoming imprisoned in the quagmire of the secondattention.

"The new seers corrected that error," he contin-ued, "and let the mastery of awareness develop to itsnatural end, which is to extend the glow of awarenessbeyond the bounds of the luminous cocoon in one sin-gle stroke.

"The third attention is attained when the glow ofawareness turns into the fire from within: a glow thatkindles not one band at a time but all the Eagle's em-anations inside man's cocoon."

Don Juan expressed his awe for the new seers' de-liberate effort to attain the third attention while theyare alive and conscious of their individuality.

He did not consider it worthwhile to discussthe random cases of men and other sentient beingswho enter into the unknown and the unknowablewithout being aware of it; he referred to this asthe Eagle's gift. He asserted that for the new seersto enter into the third attention is also a gift, buthas a different meaning, it is more like a reward for anattainment.

He added that at the moment of dying all humanbeings enter into the unknowable and some of themdo attain the third attention, but altogether too brieflyand only to purify the food for the Eagle.

"The supreme accomplishment of human beings,"he said, "is to attain that level of attention while re-taining the life-force, without becoming a disembodiedawareness moving like a flicker of light up to the Ea-gle's beak to be devoured."

While listening to don Juan's explanation I hadagain completely lost sight of everything that sur-rounded me. Genaro apparently had gotten up and leftus, and was nowhere in sight. Strangely, I found my-self crouching on the rock, with don Juan squatting byme holding me down by gently pushing on my shoul-ders. I reclined on the rock and closed my eyes. Therewas a soft breeze blowing from the west.

"Don't fall asleep," don Juan said. "Not for anyreason should you fall asleep on this rock."

I sat up. Don Juan was staring at me.

"Just relax," he went on. "Let the internal dialoguedie out."

All my concentration was involved in followingwhat he was saying when I got a jolt of fright. I didnot know what it was at first; I thought I was goingthrough another attack of distrust. But then it struckme, like a bolt, that it was very late in the afternoon.What I had thought was an hour's conversation hadconsumed an entire day.

I jumped up, fully aware of the incongruity, al-though I could not conceive what had happened to me.I felt a strange sensation that made my body want torun. Don Juan jumped me, restraining me forcefully.We fell to the soft ground, and he held me there withan iron grip. I had had no idea that don Juan was sostrong.

My body shook violently. My arms flew everywhich way as they shook. I was having something likea seizure. Yet some part of me was detached to thepoint of becoming fascinated with watching my bodyvibrate, twist, and shake.

The spasms finally died out and don Juan let go ofme. He was panting with the exertion. He recom-mended that we climb back up on the rock and sitthere until I was all right.

I could not help pressing him with my usual ques-tion: What had happened to me? He answered that ashe talked to me I had pushed beyond a certain limitand had entered very deeply into the left side. He andGenaro had followed me in there. And then I hadrushed out in the same fashion I had rushed in.

"I caught you right on time," he said. "Otherwiseyou would have gone straight out to your normalself."

I was totally confused. He explained that the threeof us had been playing with awareness. I must havegotten scared and run out on them.

"Genaro is the master of awareness," don Juanwent on. "Silvio Manuel is the master of wilt. The twoof them were mercilessly pushed into the unknown.My benefactor did to them what his benefactor did tohim. Genaro and Silvio Manuel are very much like theold seers in some respects. They know what they cando, but they don't care to know how they do it. Today,Genaro seized the opportunity to push your glow ofawareness and we all ended up in the weird confinesof the unknown."

I begged him to tell me what had happened in theunknown.

"You'll have to remember that yourself," a voicesaid just by my ear.

I was so convinced that it was the voice of seeingthat it did not frighten me at all. I did not even obeythe impulse to turn around.

"I am the voice of seeing and I tell you that you area peckerhead," the voice said again and chuckled.

I turned around. Genaro was sitting behind me. Iwas so surprised that I laughed perhaps a bit morehysterically than they did.

"It's getting dark now," Genaro said to me. "As Ipromised you earlier today, we are going to have aball here."

Don Juan intervened and said that we should stopfor the day, because I was the kind of nincompoopwho could die offright.

"Nah, he's all right," Genaro said, patting me onthe shoulder.

"You'd better ask him," don Juan said to Genaro."He himself will tell you that he's that kind of nincom-poop."

"Are you really that kind of nincompoop?" Genaroasked me with a frown.

I didn't answer him. And that made them rollaround laughing. Genaro rolled all the way to theground.

"He's caught," Genaro said to don Juan, referringto me, after don Juan had swiftly jumped down andhelped him to stand up. "He'll never say he's a nin-compoop. He's too self-important for that, but he'sshivering in his pants with fear of what might happenbecause he didn't confess he's a nincompoop."

Watching them laugh, I was convinced that onlyIndians could laugh with such joyfulness. But I alsobecame convinced that there was a mile-wide streakof maliciousness in them. They were poking fun at anon-Indian.

Don Juan immediately caught my feelings.

"Don't let your self-importance run rampant," hesaid. "You're not special by any standards. None ofus are, Indians and non-Indians. The nagual Julian andhis benefactor added years of enjoyment to their liveslaughing at us."

Genaro nimbly climbed back onto the rock andcame to my side.

"If I were you. I'd feel so frigging embarrassed I'dcry," he said to me. "Cry, cry. Have a good cry andyou'll feel better."

To my utter amazement I began to weep softly.Then I got so angry that I roared with fury. Only thenI felt better.

Don Juan patted my back gently. He said that usu-ally anger is very sobering, or sometimes fear is, orhumor. My violent nature made me respond only toanger.

He added that a sudden shift in the glow of aware-ness makes us weak. They had been trying to rein-force me, to bolster me. Apparently, Genaro hadsucceeded by making me rage.

It was twilight by then. Suddenly Genaro pointed toa flicker in midair at eye level, in the twilight it ap-peared to be a large moth flying around the placewhere we sat.

"Be very gentle with your exaggerated nature,"don Juan said to me. "Don't be eager. Just let Genaroguide you. Don't take your eyes from that spot."

The flickering point was definitely a moth. I couldclearly distinguish all its features. I followed its con-voluted, tired flight, until I could see every speck ofdust on its wings.

Something got me out of my total absorption. Isensed a flurry of soundless noise, if that could bepossible, just behind me. I turned around and caughtsight of an entire row of people on the other edge ofthe rock, an edge that was a bit higher than the one onwhich we were sitting. I supposed that the people wholived nearby must have gotten suspicious of us hang-ing around all day and had climbed onto the rock in-tending to harm us. I knew about their intentionsinstantly.

Don Juan and Genaro slid down from the rock andtold me to hurry down. We left immediately withoutturning back to see if the men were following us. DonJuan and Genaro refused to talk while we walked backto Genaro's house. Don Juan even made me hush witha fierce grunt, putting his finger to his lips. Genaro didnot come into the house, but kept on walking as donJuan dragged me inside.

"Who were those people, don Juan?" I asked him,when the two of us were safely inside the house andhe had lit the lantern.

"They were not people," he replied.

"Come on, don Juan, don't mystify me," I said."They were men; I saw them with my own eyes."

"Of course, you saw them with your own eyes," heretorted, "but that doesn't say anything. Your eyesmisled you. Those were not people and they were fol-lowing you. Genaro had to draw them away fromyou."

"What were they, then, if not people?"

"Oh, there is the mystery," he said. "It's a mysteryof awareness and it can't be solved rationally by talk-ing about it. The mystery can only be witnessed."

"Let me witness it then." I said.

"But you already have, twice in one day," he said."You don't remember now. You will, however, whenyou rekindle the emanations that were glowing whenyou witnessed the mystery of awareness i'm referringto. In the meantime, let's go back to our explanationof awareness."

He reiterated that awareness begins with the per-manent pressure that the emanations at large exert onthe ones trapped inside the cocoon. This pressure pro-duces the first act of consciousness; it stops the mo-tion of the trapped emanations, which are fighting tobreak the cocoon, fighting to die.

"For a seer, the truth is that all living beings arestruggling to die," he went on. "What stops death isawareness."

Don Juan said that the new seers were profoundlydisturbed by the fact that awareness forestalls deathand at the same time induces it by being food for theEagle. Since they could not explain it, for there is norational way to understand existence, seers realizedthat their knowledge is composed of contradictorypropositions.

"Why did they develop a system of contradic-tions?" I asked.

"They didn't develop anything," he said. "Theyfound unquestionable truths by means of their seeing.Those truths are arranged in terms of supposedly bla-tant contradictions, that's all.

"For example, seers have to be methodical, rationalbeings, paragons of sobriety, and at the same timethey must shy away from all of those qualities in orderto be completely free and open to the wonders andmysteries of existence."

His example left me baffled, but not to the extreme.I understood what he meant. He himself had spon-sored my rationality only to crush it and demand atotal absence of it. I told him how I understood hispoint.

"Only a feeling of supreme sobriety can bridge thecontradictions," he said.

"Could you say, don Juan, that art is that bridge?"

"You may call the bridge between contradictionsanything you want?art, affection, sobriety, love, oreven kindness."

Don Juan continued his explanation and said that inexamining the first attention, the new seers realizedthat all organic beings, except man, quiet down theiragitated trapped emanations so that those emanationscan align themselves with their matching ones outside.Human beings do not do that; instead, their first atten-tion lakes an inventory of the Eagle's emanations in-side their cocoons.

"What is an inventory, don Juan?" I asked.

"Human beings take notice of the emanations theyhave inside their cocoons," he replied. "No othercreatures do that. The moment the pressure from theemanations at large fixates the emanations inside, thefirst attention begins to watch itself. It notes every-thing about itself, or at least it tries to, in whateveraberrant ways it can. This is the process seers calltaking an inventory.

"I don't mean to say that human beings choose totake an inventory, or that they can refuse to take it.To take an inventory is the Eagle's command. What issubject to volition, however, is the manner in whichthe command is obeyed."

He said that although he disliked calling the ema-nations commands, that is what they are: commandsthat no one can disobey. Yet the way out of obeyingthe commands is in obeying them.

"In the case of the inventory of the first attention,"he went on, "seers take it, for they can't disobey. Butonce they have taken it they throw it away. The Eagledoesn't command us to worship our inventory; it com-mands us to take it, that's all."

"How do seers see that man takes an inventory?" Iasked.

"The emanations inside the cocoon of man are notquieted down for purposes of matching them withthose outside," he replied. "This is evident afterseeing what other creatures do. On quieting down,some of them actually merge themselves with the em-anations at large and move with them. Seers can see,for instance, the light of the scarabs' emanations ex-panding to great size.

"But human beings quiet down their emanationsand then reflect on them. The emanations focus onthemselves."

He said that human beings carry the command oftaking an inventory to its logical extreme and disre-gard everything else. Once they are deeply involvedin the inventory, two things may happen. They mayignore the impulses of the emanations at large, or theymay use them in a very specialized way.

The end result of ignoring those impulses after tak-ing an inventory is a unique state known as reason.The result of using every impulse in a specialized wayis known as self-absorption.

Human reason appears to a seer as an unusuallyhomogeneous dull glow that rarely if ever responds tothe constant pressure from the emanations at large?a glow that makes the egglike shell become tougher,but more brittle.

Don Juan remarked that reason in the human spe-cies should be bountiful, but that in actuality it is veryrare. The majority of human beings turn to self-ab-sorption.

He asserted that the awareness of all living beingshas a degree of self-reflection in order for them tointeract. But none except man's first attention hassuch a degree of self-absorption. Contrary to men ofreason, who ignore the impulse of the emanations atlarge, the self-absorbed individuals use every impulseand turn them all into a force to stir the trapped ema-nations inside their cocoons.

Observing all this, seers arrived at a practical con-clusion. They saw that men of reason are bound tolive longer, because by disregarding the impulse of theemanations at large, they quiet down the natural agi-tation inside their cocoons. The self-absorbed individ-uals, on the other hand, by using the impulse of theemanations at large to create more agitation, shortentheir lives.

"What do seers see when they gaze at self-absorbedhuman beings?" I asked.

"They see them as intermittent bursts of white light,followed by long pauses of dullness," he said.

Don Juan stopped talking. I had no more questionsto ask, or perhaps I was too tired to ask about any-thing. There was a loud bang that made me jump. Thefront door flew open and Genaro came in, out ofbreath. He slumped on the mat. He was actually cov-ered with perspiration.

"I was explaining about the first attention," donJuan said to him.

"The first attention works only with the known,"Genaro said. "it isn't worth two plugged nickels withthe unknown."

"That is not quite right," don Juan retorted. "Thefirst attention works very well with the unknown. Itblocks it; it denies it so fiercely that in the end, theunknown doesn't exist for the first attention.

"Taking an inventory makes us invulnerable. Thatis why the inventory came into existence in the firstplace."

"What are you talking about?" I asked don Juan.

He didn't reply. He looked at Genaro as if waitingfor an answer.

"But if I open the door," Genaro said, "would thefirst attention be capable of dealing with what willcome in?"

"Yours and mine wouldn't, but his will," don Juansaid, pointing at me. "Let's try it."

"Even though he's in heightened awareness?" Ge-naro asked don Juan.

"That won't make any difference," don Juan an-swered.

Genaro got up and went to the front door and threwit open. He instantly jumped back. A gust of cold windcame in. Don Juan came to my side, and so did Ge-naro. Both of them looked at me in amazement.

I wanted to close the front door. The cold was mak-ing me uncomfortable. But as I moved toward thedoor, don Juan and Genaro jumped in front of me andshielded me.

"Do you notice anything in the room?" Genaroasked me.

"No, I don't," I said, and I really meant it.

Except for the cold wind pouring in through theopen door, there was nothing to notice in there.

"Weird creatures came in when I opened the door,"he said. "Don't you notice anything?"

There was something in his voice that told me hewas not joking this time.

The three of us, with both of them flanking me,walked out of the house. Don Juan picked up the ker-osene lantern, and Genaro locked the front door. Wegot inside the car, through the passenger's side. Theypushed me in first. And then we drove to don Juan'shouse in the next town.

6Inorganic Beings

The next day I repeatedly asked don Juan to explainour hasty departure from Genaro's house. He refusedeven to mention the incident. Genaro was no helpeither. Every time I asked him he winked at me, grin-ning like a fool.

In the afternoon, don Juan came to the back patioof his house, where I was talking with his apprentices.As if on cue, all the young apprentices left instantly.

Don Juan took me by the arm, and we began to walkalong the corridor. He did not say anything; for awhile we just strolled around, very much as if we werein the public square.

Don Juan stopped walking and turned to me. Hecircled me, looking over my entire body. I knew thathe was seeing me. I felt a strange fatigue, a laziness Ihad not felt until his eyes swept over me. He began totalk all of a sudden.

"The reason Genaro and I didn't want to focus onwhat happened last night," he said, "was that you hadbeen very frightened during the time you were in theunknown. Genaro pushed you, and things happenedto you in there."

"What things, don Juan?"

"Things that are still difficult if not impossible toexplain to you now," he said. "You don't haveenough surplus energy to enter into the unknown andmake sense of it. When the new seers arranged theorder of the truths about awareness, they saw that thefirst attention consumes all the glow of awareness thathuman beings have, and not an iota of energy is leftfree. That's your problem now. So, the new seers pro-posed that warriors, since they have to enter into theunknown, have to save their energy. But where arethey going to get energy, if all of it is taken? They'llget it, the new seers say, from eradicating unnecessaryhabits."

He stopped talking and solicited questions. I askedhim what eradicating unnecessary habits did to theglow of awareness.

He replied that it detaches awareness from self-re-flection and allows it the freedom to focus on some-thing else.

"The unknown is forever present," he continued,"but it is outside the possibility of our normal aware-ness. The unknown is the superfluous part of the av-erage man. And it is superfluous because the averageman doesn't have enough free energy to grasp it.

"After all the time you've spent in the warrior'spath, you have enough free energy to grasp the un-known, but not enough energy to understand it or evento remember it."

He explained that at the site of the flat rock, I hadentered very deeply into the unknown. But I indulgedin my exaggerated nature and became terrified, whichwas about the worst thing anyone can do. So I hadrushed out of the left side, like a bat out of hell; unfor-tunately, taking a legion of strange things with me.

I told don Juan that he was not getting to the point,that he should come out and tell me exactly what hemeant by a legion of strange things.

He took me by the arm and continued strollingaround with me.

"In explaining awareness," he said, "I am presum-ably fitting everything or nearly everything into place.Let's talk a little bit about the old seers. Genaro, asI've told you, is very much like them."

He led me then to the big room. We sat down thereand he began his elucidation.

"The new seers were simply terrified by the knowl-edge that the old seers had accumulated over theyears," don Juan said. "It's understandable. The newseers knew that that knowledge leads only to totaldestruction. Yet they were also fascinated by it?es-pecially by the practices."

"How did the new seers know about those prac-tices?" I asked.

"They are the legacy of the old Toltecs," he said."The new seers learn about them as they go along.They hardly ever use them, but the practices are thereas part of their knowledge."

"What kind of practices are they, don Juan?"

"They are very obscure formulas, incantations,lengthy procedures that have to do with the handlingof a very mysterious force. At least it was mysteriousto the ancient Toltecs, who masked it and made itmore horrifying than it really is."

"What is that mysterious force?" I asked.

"It's a force that is present throughout everythingthere is," he said. "The old seers never attempted tounravel the mystery of the force that made them cre-ate their secret practices; they simply accepted it assomething sacred. But the new seers took a close lookand called it wilt, the will of the Eagle's emanations,or intent."'

Don Juan went on explaining that the ancient Tol-tecs had divided their secret knowledge into five setsof two categories each: the earth and the dark regions,fire and water, the above and the below, the loud andthe silent, the moving and the stationary. He specu-lated that there must have been thousands of differenttechniques, which became more and more intricate astime passed.

"The secret knowledge of the earth," he went on,"had to do with everything that stands on the ground.There were particular sets of movements, words, un-guents, potions that were applied to people, animals,insects, trees, small plants, rocks, soil.

"These were techniques that made the old seersinto horrid beings. And their secret knowledge of theearth was employed either to groom or to destroy any-thing that stands on the ground.

"The counterpart of the earth was what they knewas the dark regions. These practices were by far themost dangerous. They dealt with entities without or-ganic life. Living creatures that are present on theearth and populate it together with all organic beings.

"Doubtlessly, one of the most worthwhile findingsof the ancient seers, especially for them, was the dis-covery that organic life is not the only form of lifepresent on this earth."

I did not quite comprehend what he had said. Iwaited for him to clarify his statements.

"Organic beings are not the only creatures that havelife," he said and paused again as if to allow me timeto think his statements over.

I countered with a long argument about the defini-tion of life and being alive. I talked about reproduc-tion, metabolism, and growth, the processes thatdistinguish live organisms from inanimate things.

"You're drawing from the organic," he said. "Butthat's only one instance. You shouldn't draw all youhave to say from one category alone."

"But how else can it be?" I asked.

"For seers, to be alive means to be aware," hereplied. "For the average man, to be aware means tobe an organism. This is where seers are different. Forthem, to be aware means that the emanations thatcause awareness are encased inside a receptacle.

"Organic living beings have a cocoon that enclosesthe emanations. But there are other creatures whosereceptacles don't look like a cocoon to a seer. Yetthey have the emanations of awareness in them andcharacteristics of life other than reproduction and me-tabolism."

"Such as what, don Juan?"

"Such as emotional dependency, sadness, joy,wrath, and so forth and so on. And I forgot the bestyet, love; a kind of love man can't even conceive."

"Are you serious, don Juan?" I asked in earnest.

"Inanimately serious," he answered with a deadpanexpression and then broke into laughter.

"If we take as our clue what seers see," he contin-ued, "life is indeed extraordinary."

"If those beings are alive, why don't they makethemselves known to man?" I asked.

"They do, all the time. And not only to seers butalso to the average man. The problem is that all theenergy available is consumed by the first attention.Man's inventory not only takes it all, but it also tough-ens the cocoon to the point of making it inflexible.Under those circumstances there is no possible inter-action."

He reminded me of the countless times, in thecourse of my apprenticeship with him, when I had hada firsthand view of inorganic beings. I retorted that Ihad explained away nearly every one of those in-stances. I had even formulated the hypothesis that histeachings, through the use of hallucinogenic plants,were geared to force an agreement, on the part of theapprentice, about a primitive interpretation of theworld. I told him that I had not formally called it prim-itive interpretation but in anthropological terms I hadlabeled it a "world view more proper to hunting andgathering societies."

Don Juan laughed until he was out of breath.

"I really don't know whether you're worse in yournormal state of awareness or in a heightened one," hesaid. "In your normal state you're not suspicious, butboringly reasonable. I think I like you best when youare way inside the left side, in spite of the fact thatyou are terribly afraid of everything, as you were yes-terday."

Before I had time to say anything at all, he statedthat he was pitting what the old seers did against theaccomplishments of the new seers, as a sort of coun-terpoint, with which he intended to give me a moreinclusive view of the odds I was up against.

He continued then with his elucidation of the prac-tices of the old seers. He said that another of theirgreat findings had to do with the next category of se-cret knowledge: fire and water. They discovered thatflames have a most peculiar quality; they can transportman bodily, just as water does.

Don Juan called it a brilliant discovery. I remarkedthat there are basic laws of physics that would provethat to be impossible. He asked me to wait until hehad explained everything before drawing any conclu-sions. He remarked that I had to check my excessiverationality, because it constantly affected my states ofheightened awareness. It was not a case of reactingevery which way to external influences, but of suc-cumbing to my own devices.

He went on explaining that the ancient Toltecs, al-though they obviously saw, did not understand whatthey saw. They merely used their findings withoutbothering to relate them to a larger picture. In the caseof their category of fire and water, they divided fireinto heat and flame, and water into wetness and fluid-ity. They correlated heat and wetness and called themlesser properties. They considered flames and fluidityto be higher, magical properties, and they used themas a means for bodily transportation to the realm ofnonorganic life. Between their knowledge of that kindof life and their fire and water practices, the ancientseers became bogged down in a quagmire with no wayout.

Don Juan assured me that the new seers agreed thatthe discovery of nonorganic living beings was indeedextraordinary, but not in the way the old seers be-lieved it to be. To find themselves in a one-to-onerelation with another kind of life gave the ancient seersa false feeling of invulnerability, which spelled theirdoom.

I wanted him to explain the fire and water tech-niques in greater detail. He said that the old seers'knowledge was as intricate as it was useless and thathe was only going to outline it.

Then he summarized the practices of the above andthe below. The above dealt with secret knowledgeabout wind, rain, sheets of lightning, clouds, thunder,daylight, and the sun. The knowledge of the below hadto do with fog, water of underground springs, swamps,lightning bolts, earthquakes, the night, moonlight, andthe moon.

The loud and the silent were a category of secretknowledge that had to do with the manipulation ofsound and quiet. The moving and the stationary werepractices concerned with mysterious aspects of mo-tion and motionlessness.

I asked him if he could give me an example of anyof the techniques he had outlined. He replied that hehad already given me dozens of demonstrations overthe years. I insisted that I had rationally explainedaway everything he had done to me.

He did not answer. He seemed to be either angry atme for asking questions or seriously involved insearching for a good example. After a while he smiledand said that he had visualized the proper example.

"The technique I have in mind has to be put inaction in the shallow depths of a stream," he said."There is one near Genaro's house."

"What will I have to do?"

"You'll have to get a medium-size mirror."

I was surprised at his request. I remarked that theancient Toltecs did not know about mirrors.

"They didn't," he admitted, smiling. "This is mybenefactor's addition to the technique. All the ancientseers needed was a reflecting surface."

He explained that the technique consisted of sub-merging a shiny surface into the shallow water of astream. The surface could be any flat object that hadsome capacity to reflect images.

"I want you to construct a solid frame made ofsheet metal for a medium-size mirror," he said. "ithas to be waterproof, so you must seal it with tar. Youmust make it yourself with your own hands. When youhave made it, bring it over and we'll proceed."

"What's going to happen, don Juan?"

"Don't be apprehensive. You yourself have askedme to give you an example of an ancient Toltec prac-tice. I asked the same thing of my benefactor. I thinkeverybody asks for one at a certain moment. My ben-efactor said that he did the same thing himself. Hisbenefactor, the nagual Ellas, gave him an example;my benefactor in turn gave the same one to me, andnow I am going to give it to you.

"At the time my benefactor gave me the example Ididn't know how he did it. I know now. Someday youyourself will also know how the technique works; youwill understand what's behind all this."

I thought that don Juan wanted me to go back hometo Los Angeles and construct the frame for the mirrorthere. I commented that it would be impossible for meto remember the task if I did not remain in heightenedawareness.

"There are two things out of kilter with your com-ment," he said. "One is that there is no way for youto remain in heightened awareness, because you won'tbe able to function unless I or Genaro or any of thewarriors in the nagual's party nurse you every minuteof the day, as I do now. The other is that Mexico isnot the moon. There are hardware stores here. Wecan go to Oaxaca and buy anything you need."

We drove to the city the next day and I bought allthe pieces for the frame. I assembled it myself in amechanic's shop for a minimal fee. Don Juan told meto put it in the trunk of my car. He did not so much asglance at it.

We drove back to Genaro's house in the late after-noon and arrived there in the early morning. I lookedfor Genaro. He was not there. The house seemed de-serted.

"Why does Genaro keep this house?" I asked donJuan. "He lives with you, doesn't he?"

Don Juan did not answer. He gave me a strange lookand went to light the kerosene lantern. I was alone inthe room in total darkness. I felt a great tiredness thatI attributed to the long, tortuous drive up the moun-tains. I wanted to lie down. In the darkness, I couldnot see where Genaro had put the mats. I stumbledover a pile of them. And then I knew why Genaro keptthat house; he took care of the male apprentices Pa-blito, Nestor, and Benigno, who lived there when theywere in their state of normal awareness.

I felt exhilarated; I was no longer tired. Don Juancame in with a lantern. I told him about my realiza-tion, but he said that it did not matter, that I wouldnot remember it for too long.

He asked me to show him the mirror. He seemedpleased and remarked about its being light yet solid.He noticed that I had used metal screws to affix analuminum frame to a piece of sheet metal that I hadused as a backing for a mirror eighteen inches long byfourteen inches wide.

"I made a wooden frame for my mirror," he said."This looks much better than mine. My frame was toocumbersome and at the same time frail.

"Let me explain what we're going to do," he con-tinued after he had finished examining the mirror. "Orperhaps I should say, what we're going to attempt todo. The two of us together are going to place thismirror on the surface of the stream near the house. Itis wide enough and shallow enough to serve our pur-poses.

"The idea is to let the fluidity of the water exertpressure on us and transport us away."

Before I could make any remarks or ask any ques-tions, he reminded me that in the past I had utilizedthe water of a similar stream and accomplished ex-traordinary feats of perception. He was referring tothe aftereffects of ingesting hallucinogenic plants,which I had experienced various times while beingsubmerged in the irrigation ditch behind his house innorthern Mexico.

"Save any questions until I explain to you what theseers knew about awareness," he said. "Then you'llunderstand everything we're doing in a different light.But first let's go on with our procedure."

We walked to the nearby stream, and he selected aplace with flat, exposed rocks. He said that there thewater was shallow enough for our purposes.

"What do you expect to happen?" I asked in themidst of a gripping apprehension.

"I don't know. All I know is what we are going toattempt. We will hold the mirror very carefully, butvery firmly. We will gently place it on the surface ofthe water and then let it submerge. We will then holdit on the bottom. I've checked it. There is enough siltthere to allow us to dig our fingers underneath themirror to hold it firmly."

He asked me to squat on a flat rock above the sur-face in the middle of the gentle stream and made mehold the mirror with both hands, almost at the cornerson one side. He squatted facing me and held the mirrorthe same way I did. We let the mirror sink and thenwe held it by plunging our arms in the water almost toour elbows.

He commanded me to empty myself of thoughts andstare at the surface of the mirror. He repeated overand over that the trick was not to think at all. I lookedintently into the mirror. The gentle current mildly dis-arranged the reflection of don Juan's face and mine.After a few minutes of steady gazing into the mirror itseemed to me that gradually the image of his face andmine became much clearer. And the mirror grew insize until it was at least a yard square. The currentseemed to have stopped, and the mirror looked asclear as if it were placed on top of the water. Evenmore odd was the crispness of our reflections, it wasas if my face had been magnified, not in size but infocus. I could see the pores in the skin of my forehead.

Don Juan gently whispered not to stare at my eyesor his, but to let my gaze wander around without fo-cusing on any part of our reflections.

"Gaze fixedly without staring!" he repeatedly or-dered in a forceful whisper.

I did what he said without stopping to ponder aboutthe seeming contradiction. At that moment somethinginside me was caught in that mirror and the contradic-tion actually made sense. "It is possible to gazefixedly without staring," I thought, and the instantthat thought was formulated another head appearednext to don Juan's and mine. It was on the lower sideof the mirror, to my left.

My whole body trembled. Don Juan whispered tocalm down and not show fear or surprise. He againcommanded me to gaze without staring at the new-comer. I had to make an unimaginable effort not togasp and release the mirror. My body was shakingfrom head to toe. Don Juan whispered again to gethold of myself. He nudged me repeatedly with hisshoulder.

Slowly I got my fear under control. I gazed at thethird head and gradually realized that it was not ahuman head, or an animal head either. In fact, it wasnot a head at all. It was a shape that had no innermobility. As the thought occurred to me, I instantlyrealized that I was not thinking it myself. The realiza-tion was not a thought either. I had a moment oftremendous anxiety and then something incompre-hensible became known to me. The thoughts were avoice in my ear!

"I am seeing!" I yelled in English, but there was nosound. "Yes, you're seeing," the voice in my ear saidin Spanish.

I felt that I was encased in a force greater thanmyself. I was not in pain or even anguished. I feltnothing. I knew beyond the shadow of a doubt, be-cause the voice was telling me so, that I could notbreak the grip of that force by an act of will orstrength. I knew I was dying. I lifted my eyes auto-matically to look at don Juan, and at the instant oureyes met the force let go of me. I was free. Don Juanwas smiling at me as if he knew exactly what I hadgone through.

I realized that I was standing up. Don Juan washolding the mirror edgewise to let the water drip off.

We walked back to the house in silence.

"The ancient Toltecs were simply mesmerized bytheir findings," don Juan said.

"I can understand why," I said.

"So can I," don Juan retorted.

The force that had enveloped me had been so pow-erful as to incapacitate me for speech, even forthought, for hours afterward. It had frozen me with atotal lack of volition. And I had thawed out only bytiny degrees.

"Without any deliberate intervention on our part,"don Juan continued, "this ancient Toltec techniquehas been divided into two parts for you. The first wasjust enough to familiarize you with what takes place.In the second, we will try to accomplish what the oldseers pursued."

"What really took place out there, don Juan?" Iasked.

"There are two versions. I'll give you the old seers'version first. They thought that the reflecting surfaceof a shiny object submerged in water enlarges thepower of the water. What they used to do was gazeinto bodies of water, and the reflecting surface servedthem as an aid to accelerate the process. They be-lieved that our eyes are the keys to entering into theunknown; by gazing into water, they were allowingthe eyes to open the way."

Don Juan said that the old seers observed that thewetness of water only dampens or soaks, but that thefluidity of water moves. It runs, they surmised, insearch of other levels underneath us. They believedthat water had been given to us not only for life, butalso as a link, a road to the other levels below.

"Are there many levels below?" I asked.

"The ancient seers counted seven levels," he re-plied.

"Do you know them yourself, don Juan?"

"I am a seer of the new cycle, and consequently Ihave a different view," he said. "I am just showingyou what the old seers did and I'm telling you whatthey believed."

He asserted that just because he had different viewsdid not mean the old seers' practices were invalid;their interpretations were wrong, but their truths hadpractical value for them. In the instance of the waterpractices, they were convinced that it was humanlypossible to be transported bodily by the fluidity ofwater anywhere between this lev-el of ours and theother seven levels below; or to be transported in es-sence anywhere on this level, along the watercourseof a river in either direction. They used, accordingly,running water to be transported on this level of oursand the water of deep lakes or that of waterholes to betransported to the depths.

"What they pursued with the technique I'm show-ing you was twofold," he went on. "On the one handthey used the fluidity of the water to be transported tothe first level below. On the other, they used it to havea face-to-face meeting with a living being from thatfirst level. The headlike shape in the mirror was oneof those creatures that came to look us over."

"So, they really exist!" I exclaimed.

"They certainly do," he retorted.

He said that ancient seers were damaged by theiraberrant insistence on staying glued to their proce-dures, but that whatever they found was valid. Theyfound out that the surest way to meet one of thosecreatures is through a body of water. The size of thebody of water is not relevant; an ocean or a pondserves the same purpose. He had chosen a smallstream because he hated to get wet. We could havegotten the same results in a lake or a large river.

"The other life comes to find out what's going onwhen human beings call," he continued. "That Toltectechnique is like a knock on their door. The old seerssaid the shiny surface on the bottom of the waterserved as a bait and a window. So humans and thosecreatures meet at a window."

"Is that what happened to me there?" I asked.

"The old seers would've said that you were beingpulled by the power of the water and the power of thefirst level, plus the magnetic influence of the creatureat the window."

"But I heard a voice in my ear saying that I wasdying," I said.

"The voice was right. You were dying, and youwould have if I hadn't been there. That is the dangerof practicing the Toltecs' techniques. They are ex-tremely effective but most of the time they aredeadly."

I told him that I was ashamed to confess that I wasterrified. Seeing that shape in the mirror and havingthe sensation of an enveloping force around me hadproved too much for me the day before.

"I don't want to alarm you," he said, "but nothinghas happened to you yet. If what happened to me isgoing to be the guideline of what will happen to you,you'd better prepare yourself for the shock of yourlife. It's better to shake in your boots now than to dieof fright tomorrow."

My fear was so terrifying that I couldn't even voicethe questions that came to my mind. I had a hard limeswallowing. Don Juan laughed until he was coughing.His face got purple. When I got my voice back, everyone of my questions prompted another attack ofcoughing laughter.

"You have no idea how funny this all is to me," hefinally said. "I'm not laughing at you. It's just thesituation. My benefactor made me go through thesame motions, and looking at you I can't help seeingmyself."

I told him that I felt sick to my stomach. He saidthat that was fine, that it was natural to be scared, andthat to control fear was wrong and senseless. The an-cient seers got trapped by suppressing their terrorwhen they should have been scared out of their wits.Since they did not want to stop their pursuits or aban-don their comforting constructs they controlled theirfear instead.

"What else are we going to do with the mirror?" Iasked.

"That mirror is going to be used for a face-to-facemeeting between you and that creature you only gazedat yesterday."

"What happens in a face-to-face meeting?"

"What happens is that one form of life, the humanform, meets another form of life. The old seers saidthat in this case, it is a creature from the first level ofthe fluidity of water."

He explained that the ancient seers surmised thatthe seven levels below ours were levels of the fluidityof water. For them a spring had untold significance,because they thought that in such a case the fluidity ofwater is reversed and goes from the depth to the sur-face. They took that to be the means whereby crea-tures from other levels, these other forms of life, cometo our plane to peer at us, to observe us.

"In this respect those old seers were not mistaken,"he went on. "They hit the nail right on the head. Enti-ties that the new seers call allies do appear aroundwaterholes."

"Was the creature in the mirror an ally?" I asked.

"Of course. But not one that can be utilized. Thetradition of the allies, which I have acquainted youwith in the past, comes directly from the ancient seers.They did wonders with allies, but nothing they did wasworth anything when the real enemy came along: theirfellow men."

"Since those creatures are allies, they must be verydangerous," I said.

"As dangerous as we men are, no more, no less."

"Can they kill us?"

"Not directly, but they certainly can frighten us todeath. They can cross the boundaries themselves, orthey can just come to the window. As you may haverealized by now, the ancient Toltecs didn't stop at thewindow, either. They found weird ways to go beyondit."

The second stage of the technique proceeded verymuch as had the first except that it took perhaps twiceas long for me to relax and stop my internal turmoil.When that was done, the reflection of don Juan's faceand mine became instantly clear. I gazed from his re-flection to mine for perhaps an hour. I expected theally to appear any moment, but nothing happened. Myneck hurt. My back was stiff and my legs were numb.I wanted to kneel on the rock to relieve the pain in mylower back. Don Juan whispered that the moment theally showed its shape my discomfort would vanish.

He was absolutely right. The shock of witnessing around shape appear on the edge of the mirror dispelledevery discomfort of mine.

"What do we do now?" I whispered.

"Relax and don't focus your gaze on anything, noteven for an instant," he replied. "Watch everythingthat appears in the mirror. Gaze without staring."

I obeyed him. I glanced at everything within theframe of the mirror. There was a peculiar buzzing inmy ears. Don Juan whispered that I should move myeyes in a clockwise direction if I felt that I was beingenveloped by an unusual force; but under no circum-stances, he stressed, should I lift my head to look athim.

After a moment I noticed that the mirror was reflect-ing more than the reflection of our faces and the roundshape. Its surface had become dark. Spots of an in-tense violet light appeared. They grew large. Therewere also spots of jet blackness. Then it turned intosomething like a flat picture of a cloudy sky at night,in the moonlight. Suddenly, the whole surface cameinto focus, as if it were a moving picture. The newsight was a three-dimensional, breathtaking view ofthe depths.

I knew that it was absolutely impossible for me tofight off the tremendous attraction of that sight. Itbegan to pull me in.

Don Juan whispered forcefully that I should roll myeyes for dear life. The movement brought immediaterelief. I could again distinguish our reflections and thatof the ally. Then the ally disappeared and reappearedagain on the other end of the mirror.

Don Juan commanded me to grip the mirror with allmy might. He warned me to be calm and not makeany sudden movements.

"What's going to happen?" I whispered.

"The ally will try to come out," he replied.

As soon as he had said that I felt a powerful tug.Something jerked my arms. The tug was from under-neath the mirror. It was like a suction force that cre-ated a uniform pressure all around the frame.

"Hold the mirror tightly but don't break it," donJuan ordered. "Fight the suction. Don't let the allysink the mirror too deep."

The force pulling down on us was enormous. I feltthat my fingers were going to break or be crushedagainst the rocks on the bottom. Don Juan and I bothlost our balance at one point and had to step downfrom the flat rocks into the stream. The water wasquite shallow, but the thrashing of the ally's forcearound the frame of the mirror was as frightening as ifwe had been in a large river. The water around ourfeet was being swirled around madly, but the imagesin the mirror were undisturbed.

"Watch out!" don Juan yelled. "Here it comes!"

The tugging changed into a thrust from underneath.Something was grabbing the edge of the mirror; notthe outer edge of the frame where we were holding it,but from the inside of the glass. It was as if the glasssurface were indeed an open window and somethingor somebody were just climbing through it.

Don Juan and I fought desperately either to push themirror down when it was being thrust up or pull it upwhen it was being tugged downward. In a stooped-over position we slowly moved downstream from theoriginal spot. The water was deeper and the bottomwas covered with slippery rocks.

"Let's lift the mirror out of the water and shake himloose," don Juan said in a harsh voice.

The loud thrashing continued unremittingly. It wasas if we had caught an enormous fish with our barehands and it was swimming around wildly.

It occurred to me that the mirror was in essence ahatch. A strange shape was actually trying to climb upthrough it. It was leaning on the edge of the hatch witha mighty weight and was big enough to displace thereflection of don Juan's face and mine. I could not seeus anymore. I could only distinguish a mass trying topush itself up.

The mirror was not resting on the bottom anymore.My fingers were not compressed against the rocks.The mirror was in mid-depth, held by the opposingforces of the ally's tugs and ours. Don Juan said hewas going to extend his hands underneath the mirrorand that I should very quickly grab them in order tohave a better leverage to lift the mirror with our fore-arms. When he let go it tilled to his side. I quicklyreached for his hands but there was nothing under-neath. I vacillated a second too long and the mirrorflew out of my hands.

"Grab it! Grab it!" don Juan yelled.

I caught the mirror just as it was going to land onthe rocks. I lifted it out of the water, but not quicklyenough. The water seemed to be like glue. As I pulledthe mirror out, I also pulled a portion of a heavy rub-bery substance that simply pulled the mirror out of myhands and back into the water.

Don Juan, displaying extraordinary nimbleness,caught the mirror and lifted it up edgewise without anydifficulty.

Never in my life had I had such an attack of melan-choly. It was a sadness that had no precise foundation;I associated it with the memory of the depths I hadseen in the mirror. It was a mixture of pure longing forthose depths plus an absolute fear of their chilling sol-itude.

Don Juan remarked that in the life of warriors it wasextremely natural to be sad for no overt reason. Seerssay that the luminous egg, as a field of energy, sensesits final destination whenever the boundaries of theknown are broken. A mere glimpse of the eternityoutside the cocoon is enough to disrupt the cozinessof our inventory. The resulting melancholy is some-times so intense that it can bring about death.

He said that the best way to get rid of melancholy isto make fun of it. He commented in a mocking tonethat my first attention was doing everything to restorethe order that had been disrupted by my contact withthe ally. Since there was no way of restoring it byrational means, my first attention was doing it by fo-cusing all its power on sadness.

I told him that the fact remained the melancholy wasreal. Indulging in it, moping around, being gloomy,were not part of the feeling of aloneness that I had feltupon remembering those depths.

"Something is finally getting through to you," hesaid. "You're right. There is nothing more lonely thaneternity. And nothing is more cozy for us than to be ahuman being. This indeed is another contradiction?how can man keep the bonds of his humanness andstill venture gladly and purposefully into the absoluteloneliness of eternity? Whenever you resolve this rid-dle, you'll be ready for the definitive journey."

I knew then with total certainty the reason for mysadness. It was a recurrent feeling with me, one that Iwould always forget until I again realized the samething: the puniness of humanity against the immensityof that thing-in-itself which I had seen reflected in themirror.

"Human beings are truly nothing, don Juan," Isaid.

"I know exactly what you're thinking," he said."Sure, we're nothing, but that's exactly what makesit the ultimate challenge, that we nothings could ac-tually face the loneliness of eternity."

He abruptly changed the subject, leaving me withmy mouth open, my next question unsaid. He beganto discuss our bout with the ally. He said that first ofall, the struggle with the ally had been no joke. It hadnot really been a matter of life or death, but it had notbeen a picnic either.

"I chose that technique," he went on, "because mybenefactor showed it to me. When I asked him to giveme an example of the old seers' techniques, he nearlysplit a gut laughing; my request reminded him so muchof his own experience. His benefactor, the nagualElias, had also given him a harsh demonstration of thesame technique."

Don Juan said that as he had made the frame for hismirror out of wood, he should have asked me to dothe same, but he wanted to know what would happenif the frame was sturdier than his or his benefactor's.Both of their frames broke, and both times the allycame out.

He explained that during his own bout the allyripped the frame apart. He and his benefactor wereleft holding two pieces of wood while the mirror sankand the ally climbed out of it.

His benefactor knew what kind of trouble to expect.In the reflection of mirrors, allies are not reallyfrightening because one sees only a shape, a mass ofsorts. But when they are out, besides being truly fear-some-looking things, they are a pain in the neck. Heremarked that once the allies get out of their level it isvery difficult for them to go back. The same prevailsfor man. If seers venture into a level of those crea-tures, chances are they are never heard of again.

"My mirror was shattered with the ally's force," hesaid. "There was no more window and the allycouldn't go back, so it came after me. It actually ranafter me, rolling on itself. I scrambled on all fours attop speed, screaming with terror. I went up and downhills like a possessed man. The ally was inches awayfrom me the whole time."

Don Juan said that his benefactor ran after him, buthe was too old and could not move fast enough; hehad the good sense, however, to tell don Juan to back-track, and in that way was able to take measures toget rid of the ally. He shouted that he was going tobuild a fire and that don Juan should run in circlesuntil everything was ready. He went ahead to gatherdry branches while don Juan ran around a hill, drivenmad with fear.

Don Juan confessed that the thought had occurredto him, as he ran around in circles, that his benefactorwas actually enjoying the whole thing. He knew thathis benefactor was a warrior capable of finding delightin any conceivable situation. Why not also in this one?For a moment he got so angry at his benefactor thatthe ally stopped chasing him, and don Juan, in nouncertain terms, accused his benefactor of malice. Hisbenefactor didn't answer, but made a gesture of gen-uine horror as he looked past don Juan at the ally,which was looming over the two of them. Don Juanforgot his anger and began running around in circlesagain.

"My benefactor was indeed a devilish old man,"don Juan said, laughing. "He had learned to laughinternally. It wouldn't show on his face, so he couldpretend to be weeping or raging when he was reallylaughing. That day, as the ally chased me in circles,my benefactor stood there and defended himself frommy accusations. I only heard bits of his long speechevery time I ran by him. When he was through withthat, I heard bits of another long explanation: that hehad to gather a great deal of wood, that the ally wasbig, that the fire had to be as big as the ally itself, thatthe maneuver might not work.

"Only my maddening fear kept me going. Finally hemust have realized that I was about to drop dead fromexhaustion; he built the fire and with the flames heshielded me from the ally."

Don Juan said that they stayed by the fire for theentire night. The worst time for him was when hisbenefactor had to go away to look for more drybranches and left him alone. He was so afraid that hepromised to God that he was going to leave the pathof knowledge and become a farmer.

"In the morning, after I had exhausted all my en-ergy, the ally managed to shove me into the fire, and Iwas badly burned," don Juan added.

"What happened to the ally?" I asked.

"My benefactor never told me what happened toit," he replied. "But I have the feeling that it is stillrunning around aimlessly, trying to find its way back."

"And what happened to your promise to God?"

"My benefactor said not to worry, that it had beena good promise, but that I didn't know yet that thereis no one to hear such promises, because there is noGod. All there is is the Eagle's emanations, and thereis no way to make promises to them."

"What would have happened if the ally had caughtyou?" I asked.

"I might have died of fright," he said. "If I hadknown what was entailed in being caught I would'velet it catch me. At that time I was a reckless man.Once an ally catches you, you either have a heartattack and die or you wrestle with it. Then after amoment of thrashing around in sham ferocity, the al-ly's energy wanes. There is nothing that an ally can doto us, or vice versa. We are separated by an abyss.

"The ancient seers believed that at the moment theally's energy dwindles the ally surrenders its power toman. Power, my eye! The old seers had allies comingout of their ears and their allies' power didn't mean athing."

Don Juan explained that once again it had been upto the new seers to straighten out this confusion. Theyhad found that the only thing that counts is impecca-bility, that is, freed energy. There were indeed someamong the ancient seers who were saved by their al-lies, but that had had nothing to do with the allies'power to fend off anything; rather, it was the impec-cability of the men that had permitted them to use theenergy of those other forms of life.

The new seers also found out the most importantthing yet about the allies: what makes them useless orusable to man. Useless allies, of which there are stag-gering numbers, are those that have emanations insidethem for which we have no match inside ourselves.They are so different from us as to be thoroughly un-usable. Other allies, which are remarkably few innumber, are akin to us, meaning that they possessoccasional emanations that match ours.

"How is that kind utilized by man?" I asked.

"We should use another word instead of 'utilize, ' "he replied. "I'd say that what takes place betweenseers and allies of this kind is a fair exchange of en-ergy."

"How does the exchange take place?" I asked.

"Through their matching emanations," he said."Those emanations are, naturally, on the left-sideawareness of man; the side that the average man neveruses. For this reason, allies are totally barred from theworld of the right-side awareness, or the side of ratio-nality."

He said that the matching emanations give both acommon ground. Then, with familiarity, a deeper linkis established, which allows both forms of life toprofit. Seers seek the allies' ethereal quality; theymake fabulous scouts and guardians. Allies seek thegreater energy field of man, and with it they can evenmaterialize themselves.

He assured me that experienced seers play thoseshared emanations until they bring them into totalfocus; the exchange lakes place at that time. The an-cient seers did not understand this process, and theydeveloped complex techniques of gazing in order todescend into the depths that I had seen in the mirror.

"The old seers had a very elaborate tool to helpthem in their descent," he went on. "It was a rope ofspecial twine that they tied around their waist. It hada soft butt soaked in resin which fitted into the navelitself, like a plug. The seers had an assistant or a num-ber of them who held them by the rope while theywere lost in their gazing. Naturally, to gaze directlyinto the reflection of a deep, clear pond or lake isinfinitely more overwhelming and dangerous thanwhat we did with the mirror."

"But did they actually descend bodily?" I asked.

"You'd be surprised what men are capable of, es-pecially if they control awareness," he replied. "Theold seers were aberrant. In their excursions to thedepths they found marvels. It was routine for them toencounter allies.

"Of course, by now you realize that to say thedepths is a figure of speech. There are no depths, thereis only the handling of awareness. Yet the old seersnever made that realization."

I told don Juan that from what he had said about hisexperience with the ally, plus my own subjectiveimpression on feeling the ally's thrashing force in thewater, I had concluded that allies are very aggressive.

"Not really," he said. "It is not that they don't haveenough energy to be aggressive, but rather that theyhave a different kind of energy. They are more like anelectric current. Organic beings are more like heatwaves."

"But why did it chase you for such a long time?" Iasked.

"That's no mystery," he said. "They are attractedto emotions. Animal fear is what attracts them themost; it releases the kind of energy that suits them.The emanations inside them are rallied by animal fear.Since my fear was relentless the ally went after it, orrather, my fear hooked the ally and didn't let it go."

He said that it was the old seers who found out thatallies enjoy animal fear more than anything else. Theyeven went to the extreme of purposely feeding it totheir allies by actually scaring people to death. Theold seers were convinced that the allies had humanfeelings, but the new seers saw it differently. Theysaw that allies are attracted to the energy released byemotions; love is equally effective, as well as hatred,or sadness.

Don Juan added that if he had felt love for that ally,the ally would have come after him anyway, althoughthe chase would have had a different mood. I askedhim whether the ally would have stopped going afterhim if he had controlled his fear. He answered thatcontrolling fear was a trick of the old seers. Theylearned to control it to the point of being able to parcelit out. They hooked their allies with their own fear andby gradually doling it out. like food, they actually heldthe allies in bondage.

"Those old seers were terrifying men," don Juancontinued. "I shouldn't use the past tense?they areterrifying even today. Their bid is to dominate, to mas-ter everybody and everything."

"Even today, don Juan?" I asked, trying to get himto explain further.

He changed the subject by commenting that I hadmissed the opportunity of being really scared beyondmeasure. He said that doubtless the way I had sealedthe frame of the mirror with tar had prevented thewater from seeping behind the glass. He counted thatas the deciding factor that had kept the ally fromsmashing the mirror.

"Too bad," he said. "You might even have likedthat ally. By the way, it was not the same one thatcame the day before. The second one was perfectlyakin to you."

"Don't you have some allies yourself, don Juan?" Iasked.

"As you know, I have my benefactor's allies," hesaid. "I can't say that I have the same feeling for themthat my benefactor did. He was a serene but thor-oughly passionate man, who lavishly gave awayeverything he possessed, including his energy. Heloved his allies. To him it was no sweat to allow theallies to use his energy and materialize themselves.There was one in particular that could even take agrotesque human form."

Don Juan went on to say that since he was not par-tial to allies, he had never given me a real taste ofthem, as his benefactor had done to him while he wasstill recovering from the wound in his chest. It allbegan with the thought that his benefactor was astrange man. Having barely escaped from the clutchesof the petty tyrant, don Juan suspected that he hadfallen into another trap. His intention was to wait afew days to get his strength back and then run awaywhen the old man was not home. But the old man musthave read his thoughts, because one day, in a confi-dential tone, he whispered to don Juan that he oughtto get well as quickly as possible so that the two ofthem could escape from his captor and tormentor.Then, shaking with fear and impotence, the old manflung the door open and a monstrous fish-faced mancame into the room, as if he had been listening behindthe door. He was a grayish-green, had only one hugeunblinking eye, and was as big as a door. Don Juansaid that he was so surprised and terrified that hepassed out, and it took him years to get out from underthe spell of that fright.

"Are your allies useful to you, don Juan?" I asked.

"That's a very difficult thing to decide," he said.

"In some way, I love the allies my benefactor gaveme. They are capable of giving back inconceivableaffection. But they are incomprehensible to me. Theywere given to me for companionship in case I am everstranded alone in that immensity that is the Eagle'semanations."

7The Assemblage Point

Don Juan discontinued his explanation of the masteryof awareness for several months after my bout withthe allies. One day he started it again. A strange eventtriggered it.

Don Juan was in northern Mexico. It was late after-noon. I had just arrived at the house he kept there,and he immediately had me shift into heightenedawareness. And I had instantly remembered that donJuan always came back to Sonora as means of re-newal. He had explained that a nagual, being a leaderwho has tremendous responsibilities, has to have aphysical point of reference, a place where an amena-ble confluence of energies occurs. The Sonoran desertwas such a place for him.

On entering into heightened awareness, I had no-ticed that there was another person hiding in the semi-darkness inside the house. I asked don Juan if Genarowas with him. He replied that he was alone, that whatI had noticed was one of his allies, the one thatguarded the house.

Don Juan then made a strange gesture. He con-torted his face as if he were surprised or terrified. Andinstantly the frightening shape of a strange man ap-peared at the door of the room where we were. Thepresence of the strange man scared me so much that Iactually felt dizzy. And before I could recuperate frommy fright, the man lurched at me with a chilling feroc-ity. As he grabbed my forearms, I felt ajolt of some-thing quite like a discharge of an electric current.

I was speechless, caught in a terror I could not dis-pel. Don Juan was smiling at me. I mumbled andgroaned, trying to voice a plea for help, while I felt aneven greater jolt.

The man tightened his grip and tried to throw mebackward on the ground. Don Juan, with no hurry inhis voice, urged me to pull myself together and notfight my fear, but roll with it. "Be afraid without beingterrified," he said. Don Juan came to my side and,without intervening in my struggle, whispered in myear that I should put all my concentration on the mid-point of my body.

Over the years, he had insisted that I measure mybody to the hundredth of an inch and establish itsexact midpoint, lengthwise as well as in width. He hadalways said that such a point is a true center of energyin all of us.

As soon as I had focused my attention on that mid-point, the man let go of me. At that instant I becameaware that what I had thought was a human being wassomething that only looked like one. The moment itlost its human shape to me, the ally became an amor-phous blob of opaque light. It moved away. I wentafter it, moved by a great force that made me followthat opaque light.

Don Juan stopped me. He gently walked me to theporch of his house and made me sit down on a sturdycrate he used as a bench.

I was terribly disturbed by the experience, but evenmore disturbed by the fact that my paralyzing fear haddisappeared so fast and so completely.

I commented on my abrupt change of mood. DonJuan said that there was nothing strange about myvolatile change, and that fear did not exist as soon asthe glow of awareness moved beyond a certain thresh-old inside man's cocoon.

He then began his explanation. He briefly outlinedthe truths about awareness he had discussed: thatthere is no objective world, but only a universe ofenergy fields which seers call the Eagle's emanations.That human beings are made of the Eagle's emana-tions and are in essence bubbles of luminescent en-ergy; each of us is wrapped in a cocoon that enclosesa small portion of these emanations. That awarenessis achieved by the constant pressure that the emana-tions outside our cocoons, which are called emana-tions at large, exert on those inside our cocoons. Thatawareness gives rise to perception, which happenswhen the emanations inside our cocoons align them-selves with the corresponding emanations at large.

"The next truth is that perception takes place," hewent on, "because there is in each of us an agentcalled the assemblage point that selects internal andexternal emanations for alignment. The particularalignment that we perceive as the world is the productof the specific spot where our assemblage point is lo-cated on our cocoon."

He repeated this several times, allowing me time tograsp it. Then he said that in order to corroborate thetruths about awareness, I needed energy.

"I've mentioned to you," he continued, "that deal-ing with petty tyrants helps seers accomplish a sophis-ticated maneuver: that maneuver is to move theirassemblage points."

He said that for me to have perceived an ally meantthat I had moved my assemblage point away from itscustomary position. In other words, my glow ofawareness had moved beyond a certain threshold, alsoerasing my fear. And all this had happened because Ihad enough surplus energy.

Later that night, after we had returned from a tripinto the surrounding mountains, which had been partof his teachings for the right side, don Juan had meshift again into heightened awareness and then contin-ued his explanation. He told me that in order to dis-cuss the nature of the assemblage point, he had to startwith a discussion of the first attention.

He said that the new seers looked into the unnoticedways in which the first attention functions, and as theytried to explain them to others, they devised an orderfor the truths about awareness. He assured me thatnot every seer is given to explaining. For instance, hisbenefactor, the nagual Julian, could not have caredless about explanations. But the nagual Julian's bene-factor, the nagual Elias, whom don Juan was fortunateenough to meet, did care. Between the nagual Elias'sdetailed, lengthy explanations, the nagual Julian'sscanty ones, and his own personal seeing, don Juancame to understand and to corroborate those truths.

Don Juan explained that in order for our first atten-tion to bring into focus the world that we perceive, ithas to emphasize certain emanations selected from thenarrow band of emanations where man's awareness islocated. The discarded emanations are still within ourreach but remain dormant, unknown to us for the du-ration of our lives.

The new seers call the emphasized emanations theright side, normal awareness, the tonal, this world, theknown, the first attention. The average man calls itreality, rationality, common sense.

The emphasized emanations compose a large por-tion of man's band of awareness, but a very smallpiece of the total spectrum of emanations present in-side the cocoon of man. The disregarded emanationswithin man's band are thought of as a sort of preambleto the unknown, the unknown proper consisting of thebulk of emanations which are not part of the humanband and which are never emphasized. Seers call themthe left-side awareness, the nagual, the other world,the unknown, the second attention.

"This process of emphasizing certain emanations,"don Juan went on, "was discovered and practiced bythe old seers. They realized that a nagual man or anagual woman, by the fact that they have extrastrength, can push the emphasis away from the usualemanations and make it shift to neighboring ones.That push is known as the nagual's blow."

Don Juan said that the shift was utilized by the oldseers in practical ways to keep their apprentices inbondage. With that blow they made their apprenticesenter into a state of heightened, keenest, most impres-sionable awareness; while they were helplessly pli-able, the old seers taught them aberrant techniquesthat made the apprentices into sinister men, just liketheir teachers.

The new seers employ the same technique, but in-stead of using it for sordid purposes, they use it toguide their apprentices to learn about man's possibili-ties.

Don Juan explained that the nagual's blow has to bedelivered on a precise spot, on the assemblage point,which varies minutely from person to person. Also,the blow has to be delivered by a nagual who sees. Heassured me that it is equally useless to have thestrength of a nagual and not see, as it is to see and nothave the strength of a nagual, in either case the resultsare just blows. A seer could strike on the precise spotover and over without the strength to move aware-ness. and a non-seeing nagual would not be able tostrike the precise spot.

He also said that the old seers discovered that theassemblage point is not in the physical body, but inthe luminous shell, in the cocoon itself. The nagualidentifies that spot by its intense luminosity andpushes it, rather than striking it. The force of the pushcreates a dent in the cocoon and it is felt like a blowto the right shoulder blade, a blow that knocks all theair out of the lungs.

"Are there different types of dents?" I asked.

"There are only two types," he responded. "One isa concavity and the other is a crevice; each has adistinct effect. The concavity is a temporary featureand produces a temporary shift?but the crevice is aprofound and permanent feature of the cocoon andproduces a permanent shift."

He explained that usually a luminous cocoonhardened by self-reflection is not affected at all by thenagual's blow. Sometimes, however, the cocoon ofman is very pliable and the smallest force creates abowl-like dent ranging in size from a small depressionto one that is a third the size of the total cocoon; or itcreates a crevice that may run across the width of theegglike shell, or along its length, making the cocoonlook as if it has curled in on itself.

Some luminous shells, after being dented, go backto their original shape instantly. Others remain dentedfor hours or even days at a time, but they revert backby themselves. Still others get a firm, impervious dentthat requires another blow from the nagual on a bor-dering area to restore the original shape of the lumi-nous cocoon. And a few never lose their indentationonce they get it. No matter how many blows they getfrom a nagual they never revert back to their egglikeshapes.

Don Juan further said that the dent acts on the firstattention bydisplacing the glow of awareness. Thedent presses the emanations inside the luminous shell,and the seers witness how the first attention shifts itsemphasis under the force of that pressure. The dent,by displacing the Eagle's emanations inside the co-coon, makes the glow of awareness fall on other ema-nations from areas that are ordinarily inaccessible tothe first attention.

I asked him if the glow of awareness is seen only onthe surface of the luminous cocoon. He did not answerme right away. He seemed to immerse himself inthought. After perhaps ten minutes he answered myquestion; he said that normally the glow of awarenessis seen on the surface of the cocoon of all sentientbeings. After man develops attention, however, theglow of awareness acquires depth. In other words, itis transmitted from the surface of the cocoon to quitea number of emanations inside the cocoon.

"The old seers knew what they were doing whenthey handled awareness," he went on. "They realizedthat by creating a dent in the cocoon of man, theycould force the glow of awareness, since it is alreadyglowing on the emanations inside the cocoon, tospread to other neighboring ones."

'You make it all sound as if it's a physical affair,"I said. "How can dents be made in something that isjust aglow?"

"In some inexplicable way, it is a matter of a glowthat creates a dent in another glow," he replied."Your flaw is to remain glued to the inventory of rea-son. Reason doesn't deal with man as energy. Reasondeals with instruments that create energy, but it hasnever seriously occurred to reason that we are betterthan instruments: we are organisms that create en-ergy. We are a bubble of energy. It isn't farfetched,then, that a bubble of energy would make a dent inanother bubble of energy."

He said that the glow of awareness created by thedent should rightfully be called temporary heightenedattention, because it emphasizes emanations that areso proximal to the habitual ones that the change isminimal, yet the shift produces a greater capacity tounderstand and to concentrate and, above all, agreater capacity to forget. Seers knew exactly how touse this upshift in the scale of quality. They saw thatonly the emanations surrounding those we use dailysuddenly become bright after the nagual's blow. Themore distant ones remain unmoved, which meant tothem that while being in a state of heightened atten-tion, human beings could work as if they were in theworld of everyday life. The need of a nagual man anda nagual woman became paramount to them, becausethat state lasts only for as long as the depression re-mains, after which the experiences are immediatelyforgotten.

"Why does one have to forget?" I asked.

"Because the emanations that account for greaterclarity cease to be emphasized once warriors are outof heightened awareness," he replied. "Without thatemphasis whatever they experience or witness van-ishes."

Don Juan said that one of the tasks the new seershad devised for their students was to force them toremember, that is, to reemphasize by themselves, at alater time, those emanations used during states ofheightened awareness.

He reminded me that Genaro was always recom-mending to me that I learn to write with the tip of myfinger instead of a pencil so as not to accumulatenotes. Don Juan said that what Genaro had actuallymeant was that while I was in states of heightenedawareness I should utilize some unused emanationsfor storage of dialogue and experience, and somedayrecall it all by reemphasizing the emanations that wereused.

He went on to explain that a state of heightenedawareness is seen not only as a glow that goes deeperinside the egglike shape of human beings, but also asa more intense glow on the surface of the cocoon. Yetit is nothing in comparison to the glow produced by astate of total awareness, which is seen as a burst ofincandescence in the entire luminous egg. It is an ex-plosion of light of such a magnitude that the bounda-ries of the shell are diffused and the inside emanationsextend themselves beyond anything imaginable.

"Are those special cases, don Juan?"

"Certainly. They happen only to seers. No othermen or any other living creatures brighten up like that.Seers who deliberately attain total awareness are asight to behold. That is the moment when they burnfrom within. The fire from within consumes them. Andin full awareness they fuse themselves to the emana-tions at large, and glide into eternity."

After a few days in Sonora I drove don Juan backto the town in the southern part of Mexico where heand his party of warriors lived.

The next day was hot and hazy. I felt lazy and some-how annoyed. In midafternoon, there was a most un-pleasant quietude in that town. Don Juan and I weresitting on the comfortable chairs in the big room. I toldhim that life in rural Mexico was not my cup of tea. Idisliked the feeling I had that the silence of that townwas forced. The only noise I ever heard was the soundof children's voices yelling in the distance. I was neverable to find out whether they were playing or yellingin pain.

"When you're here, you're always in a state ofheightened awareness," don Juan said. "That makesa great difference. But no matter what, you should begetting used to living in a town like this. Someday youwill live in one."

"Why should I have to live in a town like this, donJuan?"

"I've explained to you that the new seers aim to befree. And freedom has the most devastating implica-tions. Among them is the implication that warriorsmust purposely seek change. Your predilection is tolive the way you do. You stimulate your reason byrunning through your inventory and pitting it againstyour friends' inventories. Those maneuvers leave youvery little time to examine yourself and your fate. Youwill have to give up all that. Likewise, if all you knewwere the dead calm of this town, you'd have to seek,sooner or later, the other side of the coin."

"Is that what you're doing here, don Juan?"

"Our case is a little bit different, because we are atthe end of our trail. We are not seeking anything.What all of us do here is something comprehensibleonly to a warrior. We go from day to day doing noth-ing. We are waiting. I will not tire of repeating this:we know that we are waiting and we know what weare waiting for. We are waiting for freedom!

"And now that you know that," he added with agrin, "let's get back to our discussion of awareness."

Usually, when we were in that room we were neverinterrupted by anyone and don Juan would always de-cide on the length of our discussions. But this timethere was a polite knock on the door and Genarowalked in and sat down. I had not seen Genaro sincethe day after we had run out of his house in a greathurry. I embraced him.

"Genaro has something to tell you," don Juan said."I've told you that he is the master of awareness.Now I can tell you what all that means. He can makethe assemblage point move deeper into the luminousegg after that point has been jolted out of its positionby the nagual's blow."

He explained that Genaro had pushed my assem-blage point countless times after I had attainedheightened awareness. The day we had gone to thegigantic flat rock to talk, Genaro had made my assem-blage point move dramatically into the left side?sodramatically, in fact, that it had been a bit dangerous.

Don Juan stopped talking and seemed to be readyto give Genaro the spotlight. He nodded as if to signalGenaro to say something. Genaro stood up and cameto my side.

"Flame is very important," he said softly. "Do youremember that day when I made you look at the re-flection of the sunlight on a piece of quartz, when wewere sitting on that big flat rock?"

When Genaro mentioned it I remembered. On thatday just after don Juan had stopped talking, Genarohad pointed to the refraction of light as it went througha piece of polished quartz that he had taken out of hispocket and placed on the flat rock. The shine of thequartz had immediately caught my attention. The nextthing I knew, I was crouching on the flat rock as donJuan stood by with a worried look on his face.

I was about to tell Genaro what I had rememberedwhen he began to talk. He put his mouth to my earand pointed to one of the two gasoline lamps in theroom.

"Look at the flame," he said. "There is no heat init. It's pure flame. Pure flame can take you to thedepths of the unknown."

As he talked, I began to feel a strange pressure; itwas a physical heaviness. My ears were buzzing; myeyes teared to the point that I could hardly make outthe shape of the furniture. My vision seemed to betotally out of focus. Although my eyes were open, Icould not see the intense light of the gasoline lamps.Everything around me was dark. There were streaksof chartreuse phosphorescence that illuminated dark,moving clouds. Then, as abruptly as it had fadedaway, my eyesight returned.

I could not make out where I was. I seemed to befloating like a balloon. I was alone. I had a pang ofterror, and my reason rushed in to construct an expla-nation that made sense to me at that moment: Genarohad hypnotized me, using the flame of the gasolinelamp. I felt almost satisfied. I quietly floated, tryingnot to worry; I thought that a way to avoid worryingwas to concentrate on the stages that I would have togo through to wake up.

The first thing I noticed was that I was not myself.I could not really look at anything because I had noth-ing to look with. When I tried to examine my body Irealized that I could only be aware and yet it was as ifI were looking down into infinite space. There wereportentous clouds of brilliant light and masses ofblackness; both were in motion. I clearly saw a rippleof amber glow that was coming at me, like an enor-mous, slow ocean wave. I knew then that I was like abuoy floating in space and that the wave was going toovertake me and carry me. I accepted it as unavoid-able. But just before it hit me something thoroughlyunexpected happened?a wind blew me out of thewave's path.

The force of that wind carried me with tremendousspeed. I went through an immense tunnel of intensecolored lights. My vision blurred completely and thenI felt that I was waking up, that I had been having adream, a hypnotic dream brought about by Genaro, inthe next instant I was back in the room with don Juanand Genaro.

I slept most of the following day. In the late after-noon, don Juan and I again sat down to talk. Genarohad been with me earlier but had refused to commenton my experience.

"Genaro again pushed your assemblage point lastnight," don Juan said. "But perhaps the shove wastoo forceful."

I eagerly told don Juan the content of my vision. Hesmiled, obviously bored.

"Your assemblage point moved away from its nor-mal position," he said. "And that made you perceiveemanations that are not ordinarily perceived. Soundslike nothing, doesn't it? And yet it is a supreme ac-complishment that the new seers strive to elucidate."

He explained that human beings repeatedly choosethe same emanations for perceiving because of tworeasons. First, and most important, because we havebeen taught that those emanations are perceivable,and second because our assemblage points select andprepare those emanations for being used.

"Every living being has an assemblage point," hewent on, "which selects emanations for emphasis.Seers can see whether sentient beings share the sameview of the world, by seeing if the emanations theirassemblage points have selected are the same."

He affirmed that one of the most important break-throughs for the new seers was to find that the spotwhere that point is located on the cocoon of all livingcreatures is not a permanent feature, but is establishedon that specific spot by habit. Hence the tremendousstress the new seers put on new actions, on new prac-ticalities. They want desperately to arrive at newusages, new habits.

"The nagual's blow is of great importance," hewent on, "because it makes that point move. It altersits location. Sometimes it even creates a permanentcrevice there. The assemblage point is totally dis-lodged, and awareness changes dramatically. Butwhat is a matter of even greater importance is theproper understanding of the truths about awareness inorder to realize that that point can be moved fromwithin. The unfortunate truth is that human beingsalways lose by default. They simply don't know abouttheir possibilities."

"How can one accomplish that change fromwithin?" I asked.

"The new seers say that realization is the tech-nique," he said. "They say that, first of all, one mustbecome aware that the world we perceive is the resultof our assemblage points' being located on a specificspot on the cocoon. Once that is understood, the as-semblage point can move almost at will, as a conse-quence of new habits."

I did not quite understand what he meant by habits.I asked him to clarify his point.

"The assemblage point of man appears around adefinite area of the cocoon, because the Eagle com-mands it," he said. "But the precise spot is deter-mined by habit, by repetitious acts. First we learn thatit can be placed there and then we ourselves commandit to be there. Our command becomes the Eagle'scommand and that point is fixated at that spot. Con-sider this very carefully; our command becomes theEagle's command. The old seers paid dearly for thatfinding. We'll come back to that later on."

He stated once again that the old seers had concen-trated exclusively on developing thousands of themost complex techniques of sorcery. He added thatwhat they never realized was that their intricate de-vices, as bizarre as they were, had no other value thanbeing the means to break the fixation of their assem-blage points and make them move.

I asked him to explain what he had said.

"I've mentioned to you that sorcery is somethinglike entering a dead-end street," he replied. "What Imeant was that sorcery practices have no intrinsicvalue. Their worth is indirect, for their real function isto make the assemblage point shift by making the firstattention release its control on that point.

"The new seers realized the true role those sorcerypractices played and decided to go directly into theprocess of making their assemblage points shift,avoiding all the other nonsense of rituals and incanta-tions. Yet rituals and incantations are indeed neces-sary at one time in every warrior's life. I personallyhave initiated you in all kinds of sorcery procedures,but only for purposes of luring your first attentionaway from the power of self-absorption, which keepsyour assemblage point rigidly fixed."

He added that the obsessive entanglement of thefirst attention in self-absorption or reason is a power-ful binding force, and that ritual behavior, because itis repetitive, forces the first attention to free someenergy from watching the inventory, as a consequenceof which the assemblage point loses its rigidity.

"What happens to the persons whose assemblagepoints lose rigidity?" I asked.

"If they're not warriors, they think they're losingtheir minds," he said, smiling. "Just as you thoughtyou were going crazy at one time. If they're warriors,they know they've gone crazy, but they patiently wait.You see, to be healthy and sane means that the assem-blage point is immovable. When it shifts, it literallymeans that one is deranged."

He said that two options are opened to warriorswhose assemblage points have shifted. One is to ac-knowledge being ill and to behave in deranged ways,reacting emotionally to the strange worlds that theirshifts force them to witness; the other is to remainimpassive, untouched, knowing that the assemblagepoint always returns to its original position.

"What if the assemblage point doesn't return to itsoriginal position?" I asked.

"Then those people are lost," he said. "They areeither incurably crazy, because their assemblagepoints could never assemble the world as we know it,or they are peerless seers who have begun their move-ment toward the unknown."

"What determines whether it is one or the other?"

"Energy! Impeccability! Impeccable warriors don'tlose their marbles. They remain untouched. I've saidto you many times that impeccable warriors may seehorrifying worlds and yet the next moment they aretelling a joke, laughing with their friends or withstrangers."

I said to him then what I had told him many timesbefore, that what made me think I was ill was a seriesof disruptive sensorial experiences that I had had asaftereffects of ingesting hallucinogenic plants. I wentthrough states of total space and time discordance,very annoying lapses of mental concentration, actualvisions or hallucinations of places and people I wouldbe staring at as if they really existed. I could not helpthinking that I was losing my mind.

"By all ordinary measures, you were indeed losingyour mind," he said, "but in the seers' view, if youhad lost it, you wouldn't have lost much. The mind,for a seer, is nothing but the self-reflection of the in-ventory of man. If you lose that self-reflection, butdon't lose your underpinnings, you actually live aninfinitely stronger life than if you had kept it."

He remarked that my flaw was my emotional reac-tion, which prevented me from realizing that the odd-ity of my sensorial experiences was determined by thedepth to which my assemblage point had moved intoman's band of emanations.

I told him that I couldn't understand what he wasexplaining because the configuration that he had calledman's band of emanations was something incompre-hensible to me. I had pictured it to be like a ribbonplaced on the surface of a ball.

He said that calling it a band was misleading, andthat he was going to use an analogy to illustrate whathe meant. He explained that the luminous shape ofman is like a ball of jack cheese with a thick disk ofdarker cheese injected into it. He looked at me andchuckled. He knew that I did not like cheese.

He made a diagram on a small blackboard. He drewan egglike shape and divided it in four longitudinalsections, saying that he would immediately erase thedivision lines because he had drawn them only to giveme an idea where the band was located in the cocoonof man. He then drew a thick band at the line betweenthe first and second sections and erased the divisionlines. He explained that the band was like a disk ofcheddar cheese that had been inserted into the ball ofjack cheese.

"Now if that ball of jack cheese were transparent,"he went on, "you would have the perfect replica ofman's cocoon. The cheddar cheese goes all the wayinside the ball of jack cheese. It's a disk that goes fromthe surface on one side to the surface on the otherside.

"The assemblage point of man is located high up,three-fourths of the way toward the top of the egg onthe surface of the cocoon. When a nagual presses onthat point of intense luminosity, the point moves intothe disk of the cheddar cheese. Heightened awarenesscomes about when the intense glow of the assemblagepoint lights up dormant emanations way inside thedisk of cheddar cheese. To see the glow of the assem-blage point moving inside that disk gives the feelingthat it is shifting toward the left on the surface of thecocoon."

He repeated his analogy three or four times, but Idid not understand it and he had to explain it further.He said that the transparency of the luminous egg cre-ates the impression of a movement toward the left,when in fact every movement of the assemblage pointis in depth, into the center of the luminous egg alongthe thickness of man's band.

I remarked that what he was saying made it soundas if seers would be using their eyes when they see theassemblage point move.

"Man is not the unknowable," he said. "Man's lu-minosity can be seen almost as if one were using theeyes alone."

He further explained that the old seers had seen themovement of the assemblage point but it never oc-curred to them that it was a movement in depth; in-stead they followed their seeing and coined the phrase"shift to the left," which the new seers retained al-though they knew that it was erroneous to call it ashift to the left.

He also said that in the course of my activity withhim he had made my assemblage point move countlesstimes, as was the case at that very moment. Since theshift of the assemblage point was always in depth, Ihad never lost my sense of identity, in spite of the factthat I was always using emanations I had never usedbefore.

"When the nagual pushes that point," he went on,"the point ends up any which way along man's band,but it absolutely doesn't matter where, because wher-ever it ends up is always virgin ground.

"The grand test that the new seers developed fortheir warrior-apprentices is to retrace the journey thattheir assemblage points took under the influence of thenagual. This retracing, when it is completed, is calledregaining the totality of oneself."

He went on to say that the contention of the newseers is that in the course of our growth, once the glowof awareness focuses on man's band of emanationsand selects some of them for emphasis, it enters intoa vicious circle. The more it emphasizes certain ema-nations, the more stable the assemblage point gets tobe. This is equivalent to saying that our commandbecomes the Eagle's command. It goes without sayingthat when our awareness develops into first attentionthe command is so strong that to break that circle andmake the assemblage point shift is a genuine triumph.

Don Juan said that the assemblage point is also re-sponsible for making the first attention perceive interms of clusters. An example of a cluster of emana-tions that receive emphasis together is the humanbody as we perceive it. Another part of our totalbeing, our luminous cocoon, never receives emphasisand is relegated to oblivion; for the effect of the as-semblage point is not only to make us perceive clus-ters of emanations, but also to make us disregardemanations.

When I pressed hard for an explanation of clusteringhe replied that the assemblage point radiates a glowthat groups together bundles of encased emanations.These bundles then become aligned, as bundles, withthe emanations at large. Clustering is carried out evenwhen seers deal with the emanations that are neverused. Whenever they are emphasized, we perceivethem just as we perceive the clusters of the first atten-tion.

"One of the greatest moments the new seers had,"he continued, "was when they found out that the un-known is merely the emanations discarded by the firstattention, it's a huge affair, but an affair, mind you,where clustering can be done. The unknowable, onthe other hand, is an eternity where our assemblagepoint has no way of clustering anything."

He explained that the assemblage point is like aluminous magnet that picks emanations and groupsthem together wherever it moves within the bounds ofman's band of emanations. This discovery was theglory of the new seers, for it put the unknown in a newlight. The new seers noticed that some of the obses-sive visions of seers, the ones that were almost impos-sible to conceive, coincided with a shift of theassemblage point to the region of man's band which isdiametrically opposed to where it is ordinarily located.

"Those were visions of the dark side of man," heasserted.

"Why do you call it the dark side of man?" I asked.

"Because it is somber and foreboding," he said."It's not only the unknown, but the who-cares-to-know-it."

"How about the emanations that are inside the co-coon but out of the bounds of man's band?" I asked."Can they be perceived?"

"Yes, but in really indescribable ways," he said."They're not the human unknown, as is the case withthe unused emanations in the band of man, but thenearly immeasurable unknown where human traits donot figure at all. It is really an area of such an over-powering vastness that the best of seers would be hardput to describe it."

I insisted once more that it seemed to me that themystery is obviously within us.

"The mystery is outside us," he said, "Inside us wehave only emanations trying to break the cocoon. Andthis fact aberrates us, one way or another, whetherwe're average men or warriors. Only the new seersget around this. They struggle to see. And by meansof the shifts of their assemblage points, they get torealize that the mystery is perceiving. Not so muchwhat we perceive, but what makes us perceive.

"I've mentioned to you that the new seers believethat our senses are capable of detecting anything.They believe this because they see that the position ofthe assemblage point is what dictates what our sensesperceive.

"If the assemblage point aligns emanations insidethe cocoon in a position different from its normal onethe human senses perceive in inconceivable ways."

8The Position ofthe Assemblage Point

The next time don Juan resumed his explanation ofthe mastery of awareness we were again in his housein southern Mexico. That house was actually ownedby all the members of the nagual's party, but SilvioManuel officiated as the owner and everyone openlyreferred to it as Silvio Manuel's house, although I, forsome inexplicable reason, had gotten used to calling itdon Juan's house.

Don Juan, Genaro, and I had returned to the housefrom a trip to the mountains. That day, as we relaxedafter the long drive and ate a late lunch, I asked donJuan the reason for the curious deception. He assuredme that no deception was involved, and that to call itSilvio Manuel's house was an exercise in the art ofstalking to be performed by all the members of thenagual's party under any circumstances, even in theprivacy of their own thoughts. For any of them toinsist on thinking about the house in any other termswas tantamount to denying their links to the nagual'sparty.

I protested that he had never told me that. I did notwant to cause any dissension with my habits.

"Don't worry about it," he said, smiling at me andpatting my back. "You can call this house whateveryou want. The nagual has authority. The nagualwoman, for instance, calls it the house of shadows."

Our conversation was interrupted, and I did not seehim until he sent for me to come to the back patio acouple of hours later.

He and Genaro were strolling around at the far endof the corridor; I could see them moving their handsin what seemed to be an animated conversation.

It was a clear sunny day. The midafternoon sunshone directly on some of the flower pots that hungfrom the eaves of the roof around the corridor andprojected their shadows on the north and east walls ofthe patio. The combination of intense yellow sunlight,the massive black shadows of the pots, and the lovely,delicate, bare shadows of the frail flowering plants thatgrew in them was stunning. Someone with a keen eyefor balance and order had pruned those plants to cre-ate such an exquisite effect.

"The nagual woman has done that," don Juan saidas if reading my thoughts. "She gazes at these shad-ows in the afternoons."

The thought of her gazing at shadows in the after-noons had a swift and devastating effect on me. Theintense yellow light of that hour, the quietness of thattown, and the affection that I felt for the nagualwoman conjured up for me in one instant all the soli-tude of the warriors' endless path.

Don Juan had defined the scope of that path whenhe said to me that the new seers are the warriors oftotal freedom, that their only search is the ultimateliberation that comes when they attain total aware-ness. I understood with unimpaired clarity, as I lookedat those haunting shadows on the wall, what it meantto the nagual woman when she said that to read poemsout loud was the only release that her spirit had.

I remember that the day before she had read some-thing to me, there in the patio, but I had not quiteunderstood her urgency, her longing. It was a poemby Juan Ramon Jimenez, "Hora Inmensa," which shetold me synthesized for her the solitude of warriorswho live to escape to total freedom.

Only a bell and a bird break the stillness. . .It seems that the two talk with the setting sun.Golden colored silence, the afternoon is made ofcrystals.A roving purity sways the cool trees,and beyond all that,a transparent river dreams that trampling overpearlsit breaks looseand flows into infinity.

Don Juan and Genaro came to my side and lookedat me with an expression of surprise.

"What are we really doing, don Juan?" I asked. "Isit possible that warriors are only preparing themselvesfor death?"

"No way," he said, gently patting my shoulder."Warriors prepare themselves to be aware, and fullawareness comes to them only when there is no moreself-importance left in them. Only when they are noth-ing do they become everything."

We were quiet for a moment. Then don Juan askedme if I was in the throes of self-pity. I did not answerbecause I was not sure.

"You're not sorry that you're here, are you?" donJuan asked with a faint smile.

"He's certainly not," Genaro assured him. Then heseemed to have a moment of doubt. He scratched hishead, then looked at me and arched his brows."Maybe you are," he said. "Are you?"

"He's certainly not," don Juan assured Genaro thistime. He went through the same gestures of scratchinghis head and arching his brows. "Maybe you are," hesaid. "Are you?"

"He's certainly not!" Genaro boomed, and both ofthem exploded into uncontrolled laughter.

When they had calmed down, don Juan said thatself-importance is the motivating force for every at-tack of melancholy. He added that warriors are enti-tled to have profound states of sadness, but thatsadness is there only to make them laugh.

"Genaro has something to show you which is moreexciting than all the self-pity you can muster up," donJuan continued, "it has to do with the position of theassemblage point."

Genaro immediately began to walk around the cor-ridor, arching his back and lifting his thighs to hischest.

"The nagual Julian showed him how to walk thatway," don Juan said in a whisper, "it's called the gaitof power. Genaro knows several gaits of power.Watch him fixedly."

Genaro's movements were indeed mesmeric. Ifound myself following his gait, first with my eyes andthen irresistibly with my feet. I imitated his gait. Wewalked once around the patio and stopped.

While walking, I had noticed the extraordinary lu-cidity that each step brought to me. When we stopped,I was in a state of keen alertness. I could hear everysound; I could detect every change in the light or inthe shadows around me. I became enthralled with afeeling of urgency, of impending action. I felt extraor-dinarily aggressive, muscular, daring. At that momentI saw an enormous span of flat land in front of me;right behind me I saw a forest. Huge trees were linedup as straight as a wall. The forest was dark and green;the plain was sunny and yellow.

My breathing was deep and strangely accelerated,but not in an abnormal way. Yet it was the rhythm ofmy breathing that was forcing me to trot on the spot.I wanted to take off running, or rather my bodywanted to, but just as I was taking off somethingstopped me.

Don Juan and Genaro were suddenly by my side.We walked down the corridor with Genaro to myright. He nudged me with his shoulder. I felt theweight of his body on me. He gently shoved me to theleft and we angled off straight for the east wall of thepatio. For a moment I had the weird impression thatwe were going to go through the wall, and I evenbraced myself for the impact, but we stopped right infront of the wall.

While my face was still against the wall, they bothexamined me with great care. I knew what they weresearching for; they wanted to make sure that I hadshifted my assemblage point. I knew that I had be-cause my mood had changed. They obviously knew ittoo. They gently took me by the arms and walked insilence with me to the other side of the corridor, to adark passageway, a narrow hall that connected thepatio with the rest of the house. We stopped there.Don Juan and Genaro moved a few feet away fromme.

I was left facing the side of the house that was indark shadows. I looked into an empty dark room. Ihad a sense of physical weariness. I felt languid, indif-ferent, and yet I experienced a sense of spiritualstrength. I realized then that I had lost something.There was no strength in my body. I could hardlystand. My legs finally gave in and I sat down and thenI lay down on my side. While I lay there, I had themost wonderful, fulfilling thoughts of love for God, forgoodness.

Then all at once I was in front of the main altar of achurch. The bas-reliefs covered with gold leaf glitteredwith the light of thousands of candles. I saw the darkfigures of men and women carrying an enormous cru-cifix mounted on a huge palanquin. I moved out oftheir way and stepped outside the church. I saw amultitude of people, a sea of candles, coming towardme. I felt elated. I ran to join them. I was moved byprofound love. I wanted to be with them, to pray tothe Lord. I was only a few feet away from the mass ofpeople when something swished me away.

The next instant, I was with don Juan and Genaro.They flanked me as we walked lazily around the patio.

While we were having lunch the next day, don Juansaid that Genaro had pushed my assemblage pointwith his gait of power, and that he had been able to dothat because I had been in a state of inner silence. Heexplained that the articulation point of everythingseers do is something he had talked about since theday we met: stopping the internal dialogue. Hestressed over and over that the internal dialogue iswhat keeps the assemblage point fixed to its originalposition.

"Once silence is attained, everything is possible,"he said.

I told him I was very conscious of the fact that ingeneral I had stopped talking to myself, but did notknow how I had done it. If asked to explain the pro-cedure I would not know what to say.

"The explanation is simplicity itself," he said."You willed it, and thus you set a new intent, a newcommand. Then your command became the Eagle'scommand.

"This is one of the most extraordinary things thatthe new seers found out: that our command can be-come the Eagle's command. The internal dialoguestops in the same way it begins: by an act of will. Afterall, we are forced to start talking to ourselves by thosewho teach us. As they teach us, they engage their will

and we engage ours, both without knowing it. As welearn to talk to ourselves, we learn to handle will. Wewill ourselves to talk to ourselves. The way to stoptalking to ourselves is to use exactly the same method:we must will it, we must inlend it."

We were silent for a few minutes. I asked him towhom he was referring when he said that we hadteachers who taught us to talk to ourselves.

"I was talking about what happens to human beingswhen they are infants," he replied, "a time when theyare taught by everyone around them to repeat an end-less dialogue about themselves. The dialogue becomesinternalized, and that force alone keeps the assem-blage point fixed.

"The new seers say that infants have hundreds ofteachers who teach them exactly where to place theirassemblage point."

He said that seers see that infants have no fixedassemblage point at first. Their encased emanationsare in a state of great turmoil, and their assemblagepoints shift everywhere in the band of man, givingchildren a great capacity to focus on emanations thatlater will be thoroughly disregarded. Then as theygrow, the older humans around them, through theirconsiderable power over them, force the children'sassemblage points to become more steady by meansof an increasingly complex internal dialogue. The in-ternal dialogue is a process that constantly strengthensthe position of the assemblage point, because that po-sition is an arbitrary one and needs steady reinforce-ment.

"The fact of the matter is that many children see,"he went on. "Most of those who see are considered tobe oddballs and every effort is made to correct them,to make them solidify the position of their assemblagepoints."

"But would it be possible to encourage children tokeep their assemblage points more fluid?" I asked.

"Only if they live among the new seers," he said."Otherwise they would get entrapped, as the old seersdid, in the intricacies of the silent side of man. And,believe me, that's worse than being caught in theclutches of rationality."

Don Juan went on to express his profound admira-tion for the human capacity to impart order to thechaos of the Eagle's emanations. He maintained thatevery one of us, in his own right, is a masterful magi-cian and that our magic is to keep our assemblagepoint unwaveringly fixed.

"The force of the emanations at large," he went on,"makes our assemblage point select certain emana-tions and cluster them for alignment and perception.That's the command of the Eagle, but all the meaningthat we give to what we perceive is our command, ourgift of magic."

He said that in the light of what he had explained,what Genaro had made me do the day before wassomething extraordinarily complex and yet very sim-ple. It was complex because it required a tremendousdiscipline on everybody's part; it required that the in-ternal dialogue be stopped, that a state of heightenedawareness be reached, and that someone walk awaywith one's assemblage point. The explanation behindall these complex procedures was very simple; thenew seers say that since the exact position of the as-semblage point is an arbitrary position chosen for usby our ancestors, it can move with a relatively smalleffort; once it moves, it forces new alignments of em-anations, thus new perceptions.

"I used to give you power plants in order to makeyour assemblage point move," don Juan continued."Power plants have that effect; but hunger, tiredness,fever, and other things like that can have a similareffect. The flaw of the average man is that he thinksthe result of a shift is purely mental. It isn't, as youyourself can attest."

He explained that my assemblage point had shiftedscores of times in the past, just as it had shifted theday before, and that most of the time the worlds it hadassembled had been so close to the world of everydaylife as to be virtually phantom worlds. He emphati-cally added that visions of that kind are automaticallyrejected by the new seers.

"Those visions are the product of man's inven-tory," he went on. "They are of no value for warriorsin search of total freedom, because they are producedby a lateral shift of the assemblage point."

He stopped talking and looked at me. I knew thatby "lateral shift" he had meant a shift of the pointfrom one side to the other along the width of man'sband of emanations instead of a shift in depth. I askedhim if I was right.

"That's exactly what I meant," he said. "On bothedges of man's band of emanations there is a strangestorage of refuse, an incalculable pile of human junk.It's a very morbid, sinister storehouse. It had greatvalue for the old seers but not for us.

"One of the easiest things one can do is to fall intoit. Yesterday Genaro and I wanted to give you a quickexample of that lateral shift; that was why we walkedyour assemblage point, but any person can reach thatstorehouse by simply stopping his internal dialogue. Ifthe shift is minimal, the results are explained as fan-tasies of the mind. If the shift is considerable, theresults are called hallucinations."

I asked him to explain the act of walking the assem-blage point. He said that once warriors have attainedinner silence by stopping their internal dialogue, thesound of the gait of power, more than the sight of it,is what traps their assemblage points. The rhythm ofmuffled steps instantly catches the alignment force ofthe emanations inside the cocoon, which has been dis-connected by inner silence.

"That force hooks itself immediately to the edges ofthe band," he went on. "On the right edge we findendless visions of physical activity, violence, killing,sensuality. On the left edge we find spirituality, reli-gion, God. Genaro and I walked your assemblagepoint to both edges, so as to give you a complete viewof that human junk pile."

Don Juan restated, as if on second thought, that oneof the most mysterious aspects of the seers' knowl-edge is the incredible effects of inner silence. He saidthat once inner silence is attained, the bonds that tiethe assemblage point to the particular spot where it isplaced begin to break and the assemblage point is freeto move.

He said that the movement ordinarily is toward theleft, that such a directional preference is a natural re-action of most human beings, but that there are seerswho can direct that movement to positions below thecustomary spot where the point is located. The newseers call that shift "the shift below."

"Seers also suffer accidental shifts below," he wenton. "The assemblage point doesn't remain there long,and that's fortunate, because that is the place of thebeast. To go below is counter to our interest, althoughthe easiest thing to do."

Don Juan also said that among the many errors ofjudgment the old seers had committed, one of the mostgrievous was moving their assemblage points to theimmeasurable area below, which made them expertsat adopting animal forms. They chose different ani-mals as their point of reference and called those ani-mals their nagual. They believed that by moving theirassemblage points to specific spots they would acquirethe characteristics of the animal of their choice, itsstrength or wisdom or cunning or agility or ferocity.

Don Juan assured me that there are many dreadfulexamples of such practices even among the seers ofour day. The relative facility with which the assem-blage point of man moves toward any lower positionposes a great temptation to seers, especially to thosewhose inclination leans toward that end. It is the dutyof a nagual, therefore, to test his warriors.

He told me then that he had put me to the test bymoving my assemblage point to a position below,while I was under the influence of a power plant. Hethen guided my assemblage point until I could isolatethe crows' band of emanations, which resulted in mychanging into a crow.

I again asked don Juan the question I had asked himdozens of times. I wanted to know whether I had phys-ically turned into a crow or had merely thought andfelt like one. He explained that a shift of the assem-blage point to the area below always results in a totaltransformation. He added that if the assemblage pointmoves beyond a crucial threshold, the world vanishes;it ceases to be what it is to us at man's level.

He conceded that my transformation was indeedhorrifying by any standards. My reaction to that ex-perience proved to him that I had no leanings towardthat direction. Had it not been that way, I would havehad to employ enormous energy in order to fight off atendency to remain in that area below, which someseers find most comfortable.

He further said that an unwitting downshift occursperiodically to every seer, but that such a downshiftbecomes less and less frequent as their assemblagepoints move farther toward the left. Every time it oc-curs, however, the power of a seer undergoing it di-minishes considerably. It is a drawback that takestime and great effort to correct.

"Those lapses make seers extremely morose andnarrow-minded," he continued, "and in certain cases,extremely rational."

"How can seers avoid those downshifts?" I asked.

"It all depends on the warrior," he said. "Some ofthem are naturally inclined to indulge in their quirks?yourself, for instance. They are the ones who are hardhit. For those like you, I recommend a twenty-four-hour vigil of everything they do. Disciplined men orwomen are less prone to that kind of shift; for those Iwould recommend a twenty-three-hour vigil."

He looked at me with shiny eyes and laughed.

"Female seers have downshifts more often thanmales," he said. "But they are also capable of bounc-ing out of that position with no effort at all. whilemales linger dangerously in it."

He also said that women seers have an extraordi-nary capacity to make their assemblage points hold onto any position in the area below. Men cannot. Menhave sobriety and purpose, but very little talent; thatis the reason why a nagual must have eight womenseers in his party. Women give the impulse to crossthe immeasurable vastness of the unknown. Togetherwith that natural capacity, or as a consequence of it,women have a most fierce intensity. They can, there-fore, reproduce an animal form with flare, ease, and amatchless ferocity.

"If you think about scary things," he continued,"about something unnamable lurking in the darkness,you're thinking, without knowing it, about a womanseer holding a position in the immeasurable areabelow. True horror lies right there. If you ever find anaberrant woman seer, run for the hills!"

I asked him whether other organisms were capableof shifting their assemblage points.

"Their points can shift," he said, "but the shift isnot a voluntary thing with them."

"Is the assemblage point of other organisms alsotrained to appear where it does?" I asked.

"Every newborn organism is trained, one way oranother," he replied. "We may not understand howtheir training is done?after all, we don't even under-stand how it is done to us?but seers see that thenewborn are coaxed to do what their kind does. That'sexactly what happens to human infants: seers see theirassemblage points shifting every which way and thenthey see how the presence of adults fastens each pointto one spot. The same happens to every other organ-ism."

Don Juan seemed to reflect for a moment and thenadded that there was indeed one unique effect thatman's assemblage point has. He pointed to a tree out-side.

"When we, as serious adult human beings, look ata tree," he said, "our assemblage points align an infi-nite number of emanations and achieve a miracle. Ourassemblage points make us perceive a cluster of ema-nations that we call tree."

He explained that the assemblage point not onlyeffects the alignment needed for perception, but alsoobliterates the alignment of certain emanations inorder to arrive at a greater refinement of perception, askimming, a tricky human construct with no parallel.

He said that the new seers had observed that onlyhuman beings were capable of further clustering theclusters of emanations. He used the Spanish word forskimming, desnate, to describe the act of collectingthe most palatable cream off the top of a container ofboiled milk after it cools. Likewise, in terms of per-ception, man's assemblage point takes some part ofthe emanations already selected for alignment andmakes a more palatable construct with it.

"The skimmings of men," don Juan continued, "aremore real than what other creatures perceive. That isour pitfall. They are so real to us that we forget wehave constructed them by commanding our assem-blage points to appear where they do. We forget theyare real to us only because it is our command to per-ceive them as real. We have the power to skim the topoff the alignments, but we don't have the power toprotect ourselves from our own commands. That hasto be learned. To give our skimmings a free hand, aswe do, is an error of judgment for which we pay asdearly as the old seers paid for theirs."

9The Shift Below

Don Juan and Genaro made their yearly trip to thenorthern part of Mexico, to the Sonoran desert, tolook for medicinal plants. One of the seers of the na-gual's party, Vicente Medrano, the herbalist amongthem, used those plants to make medicines.

I had joined don Juan and Genaro in Sonora, at thelast stage of their journey, just in time to drive themsouth, back to their home.

The day before we started on our drive, don Juanabruptly continued his explanation of the mastery ofawareness. We were resting in the shade of some tallbushes in the foothills of the mountains. It was lateafternoon, almost dark. Each of us carried a large bur-lap sack filled with plants. As soon as we had put themdown, Genaro lay down on the ground and fell asleep,using his folded jacket as a pillow.

Don Juan spoke to me in a low voice, as if he didn'twant to wake up Genaro. He said that by now he hadexplained most of the truths about awareness, and thatthere was only one truth left to discuss. The last truth,he assured me, was the best of the old seers' findings,although they never knew that themselves. Its tremen-dous value was only recognized, ages later, by thenew seers.

"I've explained to you that man has an assemblagepoint," he went on, "and that that assemblage pointaligns emanations for perception. We've also dis-cussed that that point moves from its fixed position.Now, the last truth is that once that assemblage pointmoves beyond a certain limit, it can assemble worldsentirely different from the world we know."

Still in a whisper, he said that certain geographicalareas not only help that precarious movement of theassemblage point, but also select specific directionsfor that movement. For instance, the Sonoran deserthelps the assemblage point move downward from itscustomary position, to the place of the beast.

"That's why there are true sorcerers in Sonora,"he continued. "Especially sorceresses. You alreadyknow one, la Catalina. In the past, I have arrangedbouts between the two of you. I wanted to make yourassemblage point shift, and la Catalina, with her sor-cery antics, jolted it loose."

Don Juan explained that the chilling experiences Ihad had with la Catalina had been part of a prear-ranged agreement between the two of them.

"What would you think if we invited her to joinus?" Genaro asked me in a loud voice, as he sat up.

The abruptness of his question and the strangesound of his voice plunged me into instant terror.

Don Juan laughed and shook me by the arms. Heassured me that there was no need for alarm. He saidthat la Catalina was like a cousin or an aunt to us. Shewas part of our world, although she did not quite fol-low our quests. She was infinitely closer to the ancientseers.

Genaro smiled and winked at me.

"I understand that you've got hot pants for her,"he said to me. "She herself confessed to me that everytime you have had a confrontation with her, thegreater your fright, the hotter your pants."

Don Juan and Genaro laughed to near hysteria.

I had to admit that somehow I had always found laCatalina to be a very scary but at the same time anextremely appealing woman. What impressed me themost about her was her exuding energy.

"She has so much energy saved," don Juan com-mented, "that you didn't have to be in heightenedawareness for her to move your assemblage point allthe way to the depths of the left side."

Don Juan said again that la Catalina was veryclosely related to us, because she belonged to the na-gual Julian's party. He explained that usually the na-gual and all the members of his party leave the worldtogether, but that there are instances when they leaveeither in smaller groups or one by one. The nagualJulian and his party were an example of the latter.Although he had left the world nearly forty years ago,la Catalina was still here.

He reminded me about something he mentioned tome before, that the nagual Julian's party consisted ofa group of three thoroughly inconsequential men andeight superb women. Don Juan had always maintainedthat such a disparity was one of the reasons why themembers of the nagual Julian's party left the worldone by one.

He said that la Catalina had been attached to one ofthe superb women seers of the nagual Julian's party,who taught her extraordinary maneuvers to shift herassemblage point to the area below. That seer was oneof the last to leave the world. She lived to an ex-tremely old age, and since both she and la Catalinawere originally from Sonora, they returned, in her ad-vanced years, to the desert and lived together until theseer left the world. In the years they spent together,la Catalina became her most dedicated helper and dis-ciple, a disciple who was willing to learn the extrava-gant ways the old seers knew to make the assemblagepoint shift.

I asked don Juan if la Catalina's knowledge wasinherently different from his own.

"We are exactly the same," he replied. "She'smore like Silvio Manuel or Genaro; she is really thefemale version of them, but, of course, being a womanshe's infinitely more aggressive and dangerous thanboth of them."

Genaro assented with a nod of his head. "Infinitelymore," he said and winked again.

"Is she attached to your party?" I asked don Juan.

"I said that she's like a cousin or an aunt to us," hereplied. "I meant she belongs to the older generation,although she's younger than all of us. She is the lastof that group. She is rarely in contact with us. Shedoesn't quite like us. We are too stiff for her, becauseshe's used to the nagual Julian's touch. She prefersthe high adventure of the unknown to the quest forfreedom."

"What is the difference between the two?" I askeddon Juan.

"In the last part of my explanation of the truthsabout awareness," he replied, "we are going to dis-cuss that difference slowly and thoroughly. What'simportant for you to know. at this moment, is thatyou're jealously guarding weird secrets in your left-side awareness; that is why la Catalina and you likeeach other."

I insisted again that it was not that I liked her, it wasrather that I admired her great strength.

Don Juan and Genaro laughed and patted me as ifthey knew something I did not.

"She likes you because she knows what you'relike," Genaro said and smacked his lips. "She knewthe nagual Julian very well."

Both of them gave me a long look that made me feelembarrassed.

"What are you driving at?" I asked Genaro in abelligerent tone.

He grinned at me and moved his eyebrows up anddown in a comical gesture. But he kept quiet.

Don Juan spoke and broke the silence.

"There are very strange points in common betweenthe nagual Julian and you," he said. "Genaro is justtrying to figure out if you're aware of it."

I asked both of them how on earth I would be awareof something so farfetched.

"La Catalina thinks you are," Genaro said. "Shesays so because she knew the nagual Julian better thanany of us here."

I commented that I couldn't believe that she knewthe nagual Julian, since he had left the world nearlyforty years ago.

"La Catalina is no spring chicken," Genaro said."She just looks young; that's part of her knowledge.Just as it was part of the nagual Julian's knowledge.You've seen her only when she looks young. If yousee her when she looks old, she'll scare the livingdaylights out of you."

"What la Catalina does," don Juan interrupted,"can be explained only in terms of the three master-ies: the mastery of awareness, the mastery of stalking,and the mastery of intent.

"But today, we are going to examine what she doesonly in light of the last truth about awareness: thetruth that says that the assemblage point can assembleworlds different from our own after it moves from itsoriginal position."

Don Juan signaled me to get up. Genaro also stoodup. I automatically grabbed the burlap sack filled withmedicinal plants. Genaro stopped me as I was aboutto put it on my shoulders.

"Leave the sack alone," he said, smiling. "Wehave to take a little hike up the hill and meet la Cata-lina."

"Where is she?" I asked.

"Up there," Genaro said, pointing to the top of asmall hill. "If you stare with your eyes half-closed,you'll see her as a very dark spot against the greenshrubbery."

I strained to see the dark spot, but I couldn't seeanything.

"Why don't you walk up there?" don Juan sug-gested to me.

I felt dizzy and sick to my stomach. Don Juan urgedme with a movement of his hand to go up, but I didn'tdare move. Finally, Genaro took me by the arm andboth of us climbed toward the top of the hill. When wegot there, I realized that don Juan had come up rightbehind us. The three of us reached the top at the sametime.

Don Juan very calmly began to talk to Genaro. Heasked him if he remembered the many times the na-gual Julian was about to choke both of them to death,because they indulged in their fears.

Genaro turned to me and assured me that the nagualJulian had been a ruthless teacher. He and his ownteacher, the nagual Elias, who was still in the worldthen, used to push everyone's assemblage points be-yond a crucial limit and let them fend for themselves.

"I once told you that the nagual Julian recom-mended us not to waste our sexual energy," Genarowent on. "He meant that for the assemblage point toshift, one needs energy. If one doesn't have it, thenagual's blow is not the blow of freedom, but the blowof death."

"Without enough energy," don Juan said, "theforce of alignment is crushing. You have to have en-ergy to sustain the pressure of alignments which nevertake place under ordinary circumstances."

Genaro said that the nagual Julian was an inspiringteacher. He always found ways to teach and at thesame time entertain himself. One of his favorite teach-ing devices was to catch them unawares once or twice,in their normal awareness, and make their assemblagepoints shift. From then on, all he had to do to havetheir undivided attention was to threaten them with anunexpected nagual's blow.

"The nagual Julian was really an unforgettableman," don Juan said. "He had a great touch withpeople. He would do the worst things in the world,but done by him they were great. Done by anyoneelse, they would have been crude and callous.

"The nagual Ellas, on the other hand, had no touch,but he was indeed a great, great teacher."

"The nagual Elias was very much like the nagualJuan Matus," Genaro said to me. "They got alongvery fine. And the nagual Elias taught him everythingwithout ever raising his voice, or playing tricks onhim.

"But the nagual Julian was quite different," Genarowent on, giving me a friendly shove. "I'd say that hejealously guarded strange secrets in his left side, justlike you. Wouldn't you say so?" he asked don Juan.

Don Juan did not answer, but nodded affirmatively.He seemed to be holding back his laughter.

"He had a playful nature," don Juan said, and bothof them broke into a great laughter.

The fact that they were obviously alluding to some-thing they knew made me feel even more threatened.

Don Juan matter-of-factly said that they were refer-ring to the bizarre sorcery techniques that the nagualJulian had learned in the course of his life. Genaroadded that the nagual Julian had a unique teacher be-sides the nagual Elias. A teacher who had liked himimmensely and had taught him novel and complexways of moving his assemblage point. As a result ofthis, the nagual Julian was extraordinarily eccentric inhis behavior.

"Who was that teacher, don Juan?" I asked.

Don Juan and Genaro looked at each other and gig-gled like two children.

"That is a very tough question to answer," donJuan replied. "All I can say is that he was the teacherthat deviated the course of our line. He taught usmany things, good and bad, but among the worst, hetaught us what the old seers did. So, some of us gottrapped. The nagual Julian was one of them, and so isla Catalina. We only hope that you won't followthem."

I immediately began to protest. Don Juan inter-rupted me. He said that I did not know what I wasprotesting.

As don Juan spoke, I became terribly angry withhim and Genaro. Suddenly, I was raging, yelling atthem at the top of my voice. My reaction was so outof tone with me that it scared me. It was as if I weresomeone else. I stopped and looked at them for help.

Genaro had his hands on don Juan's shoulders as ifhe needed support. Both of them were laughing un-controllably.

I became so despondent I was nearly in tears. DonJuan came to my side. He reassuringly put his handon my shoulder. He said that the Sonoran desert, forreasons incomprehensible to him, fostered definitebelligerence in man or any other organism.

"People may say that it's because the air is too dryhere," he continued, "or because it's too hot. Seerswould say that there is a particular confluence of theEagle's emanations here, which, as I've already said,helps the assemblage point to shift below.

"Be that as it may, warriors are in the world to trainthemselves to be unbiased witnesses, so as to under-stand the mystery of ourselves and relish the exulta-tion of finding what we really are. This is the highestof the new seers' goals. And not every warrior attainsit. We believe that the nagual Julian didn't attain it.He was waylaid, and so was la Catalina."

He further said that to be a peerless nagual, one hasto love freedom, and one has to have supreme detach-ment. He explained that what makes the warrior'spath so very dangerous is that it is the opposite of thelife situation of modern man. He said that modern manhas left the realm of the unknown and the mysterious,and has settled down in the realm of the functional.He has turned his back to the world of the forebodingand the exulting and has welcomed the world of bore-dom.

"To be given a chance to go back again to the mys-tery of the world," don Juan continued, "is sometimestoo much for warriors, and they succumb; they arewaylaid by what I've called the high adventure of theunknown. They forget the quest for freedom; they for-get to be unbiased witnesses. They sink into the un-known and love it."

"And you think i'm like that, don't you?" I askeddon Juan.

"We don't think, we know," Genaro replied. "Andla Catalina knows better than anyone else."

"Why would she know it?" I demanded.

"Because she's like you," Genaro replied, pro-nouncing his words with a comical intonation.

I was about to get into a heated argument againwhen don Juan interrupted me.

"There's no need to get so worked up," he said tome. "You are what you are. The fight for freedom isharder for some. You are one of them.

"In order to be unbiased witnesses," he went on,"we begin by understanding that the fixation or themovement of the assemblage point is all there is to usand the world we witness, whatever that world mightbe.

"The new seers say that when we were taught totalk to ourselves, we were taught the means to dullourselves in order to keep the assemblage point fixedon one spot."

Genaro clapped his hands noisily and let out a pierc-ing whistle that imitated the whistle of a footballcoach.

"Let's get that assemblage point moving!" heyelled. "Up, up, up! Move, move, move!"

We were all still laughing when the bushes by myright side were suddenly stirred. Don Juan and Genaroimmediately sat down with the left leg tucked underthe seat. The right leg, with the knee up, was like ashield in front of them. Don Juan signaled me to dothe same. He raised his brows and made a gesture ofresignation at the corner of his mouth.

"Sorcerers have their own quirks," he said in awhisper. "When the assemblage point moves to theregions below its normal position, the vision of sorcer-ers becomes limited. If they see you standing, they'llattack you."

"The nagual Julian kept me once for two days inthis warrior's position," Genaro whispered to me. "Ieven had to urinate while I sat in this position."

"And defecate," don Juan added.

"Right," Genaro said. And then he whispered tome, as if on second thought, "I hope you did yourkaka earlier. If your bowels aren't empty when la Cat-alina shows up, you'll shit in your pants, unless I showyou how to take them off. If you have to shit in thisposition, you've got to get your pants off."

He began to show me how to maneuver out of mytrousers. He did it in a most serious and concernedmanner. All my concentration was focused on hismovements. It was only when I had gotten out of mypants that I became aware that don Juan was roaringwith laughter. I realized that Genaro was again pokingfun at me. I was about to stand up to put on my pants,when don Juan stopped me. He was laughing so hardthat he could hardly articulate his words. He told meto stay put, that Genaro did things only half in fun,and that la Catalina was really there behind thebushes.

His tone of urgency, in the midst of laughter, got tome. I froze on the spot. A moment later a rustle in thebushes sent me into such a panic that I forgot aboutmy pants. I looked at Genaro. He was again wearinghis pants. He shrugged his shoulders.

"I'm sorry," he whispered. "I didn't have time toshow you how to put them back on without gettingup."

I did not have time to get angry or to join them intheir mirth. Suddenly, right in front of me, the bushesseparated and a most horrendous creature came out.It was so outlandish I was no longer afraid. I wasspellbound. Whatever was in front of me was not ahuman being; it was something not even remotely re-sembling one. It was more like a reptile. Or a bulkygrotesque insect. Or even a hairy, ultimately repulsivebird. Its body was dark and had coarse reddish hair. Icould not see any legs, just the ugly enormous head.The nose was flat and the nostrils were two enormouslateral holes. It had something like a beak with teeth.Horrifying as that thing was, its eyes were magnifi-cent. They were like two mesmeric pools of inconceiv-able clarity. They had knowledge. They were nothuman eyes, or bird eyes, or any kind of eyes I hadever seen.

The creature moved toward my left, rustling thebushes. As I moved my head to follow it, I noticedthat don Juan and Genaro seemed to be as spellboundby its presence as I was. It occurred to me that theyhad never seen anything like that either.

In an instant, the creature had moved completelyout of sight. But a moment later there was a growl andits gigantic shape again loomed in front of us.

I was fascinated and at the same time worried bythe fact that I was not in the least afraid of that gro-tesque creature. It was as if my early panic had beenexperienced by someone else.

I felt, at one moment, that I was beginning to standup. Against my volition, my legs straightened up andI found myself standing up, facing the creature. Ivaguely felt that I was taking off my jacket, my shirt,and my shoes. Then I was naked. The muscles of mylegs tensed with a tremendously powerful contraction.I jumped up and down with colossal agility, and thenthe creature and I raced toward some ineffable green-ness in the distance.

The creature raced ahead of me, coiling on itself,like a serpent. But then I caught up with it. As wespeeded together, I became aware of something I al-ready knew?the creature was really la Catalina. Allof a sudden, la Catalina, in the flesh, was next to me.We moved effortlessly. It was as if we were station-ary, only posed in a bodily gesture of movement andspeed, while the scenery around us was being moved,giving the impression of enormous acceleration.

Our racing stopped as suddenly as it had started,and then I was alone with la Catalina in a differentworld. There was not a single recognizable feature init. There was an intense glare and heat coming fromwhat seemed to be the ground, a ground covered withhuge rocks. Or at least they seemed to be rocks. Theyhad the color of sandstone, but they had no weight;they were like chunks of sponge tissue. I could sendthem hurling around by only leaning on them.

I became so fascinated with my strength that I wasoblivious to anything else. I had assessed, in whateverway, that the chunks of seemingly weightless materialopposed resistance to me. It was my superior strengththat sent them hurling around.

I tried to grab them with my hands, and I realizedthat my entire body had changed. La Catalina waslooking at me. She was again the grotesque creatureshe had been before, and so was I. I could not seemyself, but I knew that both of us were exactly alike.

An indescribable joy possessed me, as if joy weresome force that came from outside me. La Catalinaand I cavorted, and twisted, and played until I had nomore thoughts, or feelings, or human awareness in anydegree. Yet, I was definitely aware. My awarenesswas a vague knowledge that gave me confidence; itwas a limitless trust, a physical certainty of my exis-tence, not in the sense of a human feeling of individ-uality, but in the sense of a presence that waseverything.

Then, everything came again into human focus allat once. La Catalina was holding my hand. We werewalking on the desert floor among the desert shrubs. Ihad the immediate and painful realization that the des-ert rocks and hard clumps of dirt were horribly inju-rious to my bare feet.

We came to a spot clear of vegetation. Don Juanand Genaro were there. I sat down and put on myclothes.

My experience with la Catalina delayed our tripback to the south of Mexico. It had unhinged me insome indescribable way. In my normal state of aware-ness, I became disassociated. It was as if I had lost apoint of reference. I had become despondent. I tolddon Juan that I had even lost my desire to live.

We were sitting around in the ramada of don Juan'shouse. My car was loaded with sacks and we wereready to leave, but my feeling of despair got the bestof me and I began to weep.

Don Juan and Genaro laughed until their eyes weretearing. The more desperate I felt, the greater wastheir enjoyment. Finally, don Juan had me shift intoheightened awareness and explained that their laugh-ter was not unkindness on their part, or the result of aweird sense of humor, but the genuine expression ofhappiness at seeing me advance in the path of knowl-edge.

"I'll tell you what the nagual Julian used to say tous when we got to where you are," don Juan went on."That way, you'll know that you're not alone. What'shappening to you happens to anyone who storesenough energy to catch a glimpse of the unknown."

He said that the nagual Julian used to tell them thatthey had been evicted from the homes where they hadlived all their lives. A result of having saved energyhad been the disruption of their cozy but utterly limit-ing and boring nest in the world of everyday life. Theirdepression, the nagual Julian told them, was not somuch the sadness of having lost their nest, but theannoyance of having to look for new quarters.

"The new quarters," don Juan went on, "are not ascozy. But they are infinitely more roomy.

"Your eviction notice came in the form of a greatdepression, a loss of the desire to live, just as it hap-pened to us. When you told us that you didn't want tolive, we couldn't help laughing."

"What's going to happen to me now?" I asked.

"Using the vernacular, you got to get another pad,"don Juan replied.

Don Juan and Genaro again entered into a state ofgreat euphoria. Every one of their statements and re-marks made them laugh hysterically.

"It's all very simple," don Juan said. "Your newlevel of energy will create a new spot to house yourassemblage point. And the warriors' dialogue youcarry on with us every time we get together will solid-ify that new position."

Genaro adopted a serious look and in a boomingvoice he asked me, "Did you shit today?"

He urged me with a movement of his head to an-swer. "Did you, did you?" he demanded. "Let's getgoing with our warriors' dialogue."

When their laughter had subsided, Genaro said thatI had to be aware of a drawback, the fact that fromtime to time the assemblage point returns to its origi-nal position. He told me that in his own case, thenormal position of his assemblage point had forcedhim to see people as threatening and often terrifyingbeings. To his utter amazement, one day he realizedthat he had changed. He was considerably more dar-ing and had successfully dealt with a situation thatwould have ordinarily thrown him into chaos and fear.

"I found myself making love," Genaro continued,and he winked at me. "Usually I was afraid to deathof women. But one day I found myself in bed with amost ferocious woman, it was so unlike me that whenI realized what I was doing I nearly had a heart attack.The jolt made my assemblage point return to its mis-erable normal position and I had to run out of thehouse, shaking like a scared rabbit.

"You'd better watch out for the recoil of the assem-blage point," Genaro added, and they were laughingagain.

"The position of the assemblage point on man's co-coon," don Juan explained, "is maintained by the in-ternal dialogue, and because of that, it is a flimsyposition at best. That's why men and women lose theirminds so easily, especially those whose internal dia-logue is repetitious, boring, and without any depth.

"The new seers say that the more resilient humanbeings are those whose internal dialogue is more fluidand varied."

He said that the position of the warrior's assem-blage point is infinitely stronger, because as soon asthe assemblage point begins to move in the cocoon, itcreates a dimple in the luminosity, a dimple thathouses the assemblage point from then on.

"That's the reason why we can't say that warriorslose their minds," don Juan went on. "If they loseanything, they lose their dimple."

Don Juan and Genaro found that statement so hilar-ious that they rolled on the floor laughing.

I asked don Juan to explain my experience with laCatalina. And both of them again howled with laugh-ter.

"Women are definitely more bizarre than men,"don Juan finally said. "The fact that they have anextra opening between their legs makes them fall preyto strange influences. Strange, powerful forces pos-sess them through that opening. That's the only way Ican understand their quirks."

He kept silent for a while, and I asked what hemeant by that.

"La Catalina came to us as a giant worm," he re-plied.

Don Juan's expression when he said that, and Ge-naro's explosion of laughter, took me into sheer mirth.I laughed until I was nearly sick.

Don Juan said that la Catalina's skill was so extraor-dinary that she could do anything she wanted in therealm of the beast. Her unparalleled display had beenmotivated by her affinity with me. The final result ofall that, he said, was that la Catalina pulled my assem-blage point with her.

"What did you two do as worms?" Genaro askedand slapped me on the back.

Don Juan seemed to be close to choking with laugh-ter.

"That's why I've said that women are more bizarrethan men," he commented at last.

"I don't agree with you," Genaro said to don Juan."The nagual Julian didn't have an extra hole betweenhis legs and he was more weird than la Catalina. Ibelieve she learned the worm bit from him. He usedto do that to her."

Don Juan jumped up and down, like a child who istrying to keep from wetting his pants.

When he had regained a measure of calm, don Juansaid that the nagual Julian had a knack for creatingand exploiting the most bizarre situations. He alsosaid that la Catalina had given me a superb exampleof the shift below. She had let me see her as the beingwhose form she had adopted by moving her assem-blage point, and she had then helped me move mine tothe same position that gave her her monstrous appear-ance.

"The other teacher that the nagual Julian had," donJuan went on, "taught him how to get to specific spotsin that immensity of the area below. None of us couldfollow him there, but all the members of his party did,especially la Catalina and the woman seer who taughther."

Don Juan further said that a shift below entailed aview, not of another world proper, but of our sameworld of everyday life seen from a different perspec-tive. He added that in order for me to see anotherworld I had to perceive another great band of the Ea-gle's emanations.

He then brought his explanation to an end. He saidthat he had no time to elaborate on the subject of thegreat bands of emanations, because we had to be onour way. I wanted to stay a bit longer and keep ontalking, but he argued that he would need a good dealof time to explain that topic and I would need freshconcentration.

10Great Bandsof Emanations

Days later, in his house in southern Mexico, don Juancontinued with his explanation. He took me to the bigroom. It was early evening. The room was in dark-ness. I wanted to light the gasoline lanterns, but donJuan would not let me. He said that I had to let thesound of his voice move my assemblage point so thatit would glow on the emanations of total concentrationand total recall.

He then told me that we were going to talk aboutthe great bands of emanations. He called it anotherkey discovery that the old seers made, but that, intheir aberration, they relegated to oblivion until it wasrescued by the new seers.

"The Eagle's emanations are always grouped inclusters," he went on. "The old seers called thoseclusters the great bands of emanations. They aren'treally bands, but the name stuck.

"For instance, there is an immeasurable cluster thatproduces organic beings. The emanations of that or-ganic band have a sort of fluffiness. They are transpar-ent and have a unique light of their own, a peculiarenergy. They are aware, they jump. That's the reasonwhy all organic beings are filled with a peculiar con-suming energy. The other bands are darker, less fluffy.Some of them have no light at all, but a quality ofopaqueness."

"Do you mean, don Juan, that all organic beingshave the same kind of emanations inside their co-coons?" I asked.

"No. I don't mean that. It isn't really that simple,although organic beings belong to the same greatband. Think of it as an enormously wide band of lu-minous filaments, luminous strings with no end. Or-ganic beings are bubbles that grow around a group ofluminous filaments. Imagine that in this band of or-ganic life some bubbles are formed around the lumi-nous filaments in the center of the band, others areformed close to the edges; the band is wide enough toaccommodate every kind of organic being with roomto spare. In such an arrangement, bubbles that areclose to the edges of the band miss altogether the em-anations that are in the center of the band, which areshared only by bubbles that are aligned with the cen-ter. By the same token, bubbles in the center miss theemanations from the edges.

"As you can understand, organic beings share theemanations of one band; yet seers see that within thatorganic band beings are as different as they can be."

"Are there many of these great bands?" I asked.

"As many as infinity itself," he replied. "Seershave found out, however, that in the earth there areonly forty-eight such bands."

"What is the meaning of that, don Juan?"

"For seers it means that there are forty-eight typesof organizations on the earth, forty-eight types of clus-ters or structures. Organic life is one of them."

"Does that mean that there are forty-seven types ofinorganic life?"

"No, not at all. The old seers counted seven bandsthat produced inorganic bubbles of awareness. Inother words, there are forty bands that produce bub-bles without awareness; those are bands that generateonly organization.

"Think of the great bands as being like trees. All ofthem bear fruit; they produce containers filled withemanations; yet only eight of those trees bear ediblefruit, that is, bubbles of awareness. Seven have sourfruit, but edible nonetheless, and one has the mostjuicy, luscious fruit there is."

He laughed and said that in his analogy he had takenthe point of view of the Eagle, for whom the mostdelectable morsels are the organic bubbles of aware-ness.

"What makes those eight bands produce aware-ness?" I asked.

"The Eagle bestows awareness through its emana-tions," he replied.

His answer made me argue with him. I told him thatto say that the Eagle bestows awareness through itsemanations is like what a religious man would sayabout God, that God bestows life through love. It doesnot mean anything.

"The two statements are not made from the samepoint of view," he patiently said. "And yet I thinkthey mean the same thing. The difference is that seerssee how the Eagle bestows awareness through its em-anations and religious men don't see how God bestowslife through his love."

He said that the way the Eagle bestows awarenessis by means of three giant bundles of emanations thatrun through eight great bands. These bundles are quitepeculiar, because they make seers feel a hue. Onebundle gives the feeling of being beige-pink, some-thing like the glow of pink-colored street lamps; an-other gives the feeling of being peach, like buff neonlights; and the third bundle gives the feeling of beingamber, like clear honey.

"So, it is a matter of seeing a hue when seers seethat the Eagle bestows awareness through its emana-tions," he went on. "Religious men don't see God'slove, but if they would see it, they would know that itis either pink, peach, or amber.

"Man, for example, is attached to the amber bun-dle, but so are other beings."

I wanted to know which beings shared those ema-nations with man.

"Details like that you will have to find out for your-self through your own seeing,"' he said. "There is nopoint in my telling you which ones; you will only bemaking another inventory. Suffice it to say that findingthat out for yourself will be one of the most excitingthings you'll ever do."

"Do the pink and peach bundles also show inman?" I asked.

"Never. Those bundles belong to other livingbeings," he replied.

I was about to ask a question, but with a forcefulmovement of his hand, he signaled me to stop. Hethen became immersed in thought. We were envel-oped in complete silence for a long time.

"I've told you that the glow of awareness in manhas different colors." he finally said. "What I didn'ttell you then, because we hadn't gotten to that pointyet, was that they are not colors but casts of amber."

He said that the amber bundle of awareness has aninfinitude of subtle variants, which always denote dif-ferences in quality of awareness. Pink and pale-greenamber are the most common casts. Blue amber ismore unusual, but pure amber is by far the most rare.

"What determines the particular casts of amber?"

"Seers say that the amount of energy that one savesand stores determines the cast. Countless numbers ofwarriors have begun with an ordinary pink amber castand have finished with the purest of all ambers. Ge-naro and Silvio Manuel are examples of that."

"What forms of life belong to the pink and the peachbundles of awareness?" I asked.

"The three bundles with all their casts crisscrossthe eight bands," he replied. "In the organic band, thepink bundle belongs mainly to plants, the peach bandbelongs to insects, and the amber band belongs to manand other animals.

"The same situation is prevalent in the inorganicbands. The three bundles of awareness produce spe-cific kinds of inorganic beings in each of the sevengreat bands."

I asked him to elaborate on the kinds of inorganicbeings that existed.

"That is another thing that you must see for your-self," he said. "The seven bands and what they pro-duce are indeed inaccessible to human reason, but notto human seeing."

I told him that I could not quite grasp his explana-tion of the great bands, because his description hadforced me to imagine them as independent bundles ofstrings, or even as flat bands, like conveyor belts.

He explained that the great bands are neither flatnor round, but indescribably clustered together, like apile of hay, which is held together in midair by theforce of the hand that pitched it. Thus, there is noorder to the emanations; to say that there is a centralpart or that there are edges is misleading, but neces-sary to understanding.

Continuing, he explained that inorganic beings pro-duced by the seven other bands of awareness are char-acterized by having a container that has no motion; itis rather a formless receptacle with a low degree ofluminosity. It does not look like the cocoon of organicbeings. It lacks the tautness, the inflated quality thatmakes organic beings look like luminous balls burstingwith energy.

Don Juan said that the only similarity between in-organic and organic beings is that all of them have theawareness-bestowing pink or peach or amber emana-tions.

"Those emanations, under certain circumstances,"he continued, "make possible the most fascinatingcommunication between the beings of those eightgreat bands."

He said that usually the organic beings, with theirgreater fields of energy, are the initiators of commu-nication with inorganic beings, but a subtle and so-phisticated follow-up is always the province of theinorganic beings. Once the barrier is broken, inorganicbeings change and become what seers call allies. Fromthat moment inorganic beings can anticipate the seer'smost subtle thoughts or moods or fears.

"The old seers became mesmerized by such devo-tion from their allies," he went on. "Stories are thatthe old seers could make their allies do anything theywanted. That was one of the reasons they believed intheir own invulnerability. They got fooled by theirself-importance. The allies have power only if the seerwho sees them is the paragon of impeccability; andthose old seers just weren't."

"Are there as many inorganic beings as there areliving organisms?" I asked.

He said that inorganic beings are not as plentiful asorganic ones, but that this is offset by the greater num-ber of bands of inorganic awareness. Also, the differ-ences among the inorganic beings themselves aremore vast than the differences among organisms, be-cause organisms belong to only one band while inor-ganic beings belong to seven bands.

"Besides, inorganic beings live infinitely longerthan organisms," he continued. "This matter is whatprompted the old seers to concentrate their seeing onthe allies, for reasons I will tell you about later on."

He said that the old seers also came to realize thatit is the high energy of organisms and the subsequenthigh development of their awareness that make themdelectable morsels for the Eagle. In the old seers'view, gluttony was the reason the Eagle produced asmany organisms as possible.

He explained next that the product of the other fortygreat bands is not awareness at all, but a configurationof inanimate energy. The old seers chose to call what-ever is produced by those bands, vessels. Whilecocoons and containers are fields of energeticawareness, which accounts for their independentluminosity, vessels are rigid receptacles that hold em-anations without being fields of energetic awareness.Their luminosity comes only from the energy of theencased emanations.

"You must bear in mind that everything on theearth is encased," he continued. "Whatever we per-ceive is made up of portions of cocoons or vesselswith emanations. Ordinarily, we don't perceive thecontainers of inorganic beings at all."

He looked at me, waiting for a sign of comprehen-sion. When he realized I was not going to oblige him,he continued explaining.

"The total world is made of the forty-eight bands,"he said. "The world that our assemblage point assem-bles for our normal perception is made up of twobands; one is the organic band, the other is a band thathas only structure, but no awareness. The other forty-six great bands are not part of the world we normallyperceive."

He paused again for pertinent questions. I had none.

"There are other complete worlds that our assem-blage points can assemble," he went on. "The oldseers counted seven such worlds, one for each bandof awareness. I'll add that two of those worlds, be-sides the world of everyday life, are easy to assemble;the other five are something else."

When we again sat down to talk, don Juan immedi-ately began to talk about my experience with la Cata-lina. He said that a shift of the assemblage point to thearea below its customary position allows the seer adetailed and narrow view of the world we know. Sodetailed is that view that it seems to be an entirelydifferent world. It is a mesmerizing view that has atremendous appeal, especially for those seers whohave an adventurous but somehow indolent and lazyspirit.

"The change of perspective is very pleasant," donJuan went on. "Minimal effort is required, and theresults are staggering. If a seer is driven by quick gain,there is no better maneuver than the shift below. Theonly problem is that in those positions of the assem-blage point, seers are plagued by death, which hap-pens even more brutally and more quickly than inman's position.

"The nagual Julian thought it was a great place forcavorting, but that's all."

He said that a true change of worlds happens onlywhen the assemblage point moves into man's band,deep enough to reach a crucial threshold, at whichstage the assemblage point can use another of thegreat bands.

"How does it use it?" I asked.

He shrugged his shoulders. "It's a matter of en-ergy," he said. "The force of alignment hooks anotherband, provided that the seer has enough energy. Ournormal energy allows our assemblage points to use theforce of alignment of one great band of emanations.And we perceive the world we know. But if we havea surplus of energy, we can use the force of alignmentof other great bands, and consequently we perceiveother worlds."

Don Juan abruptly changed the subject and beganto talk about plants.

"This may seem like an oddity to you," he said,"but trees, for instance, are closer to man than ants.I've told you that trees and man can develop a greatrelationship; that's so because they share emana-tions."

"How big are their cocoons?" I asked.

"The cocoon of a giant tree is not much larger thanthe tree itself. The interesting part is that some tinyplants have a cocoon almost as big as a man's bodyand three times its width. Those are power plants.They share the largest amount of emanations withman, not the emanations of awareness, but other em-anations in general.

"Another thing unique about plants is that their lu-minosities have different casts. They are pinkish ingeneral, because their awareness is pink. Poisonousplants are a pale yellow pink and medicinal plants area bright violet pink. The only ones that are white pinkare power plants; some are murky white, others arebrilliant white.

"But the real difference between plants and otherorganic beings is the location of their assemblagepoints. Plants have it on the lower part of their co-coon, while other organic beings have it on the upperpart of their cocoon."

"What about the inorganic beings?" I asked."Where do they have their assemblage points?"

"Some have it on the lower part of their contain-ers," he said. "Those are thoroughly alien to man, butakin to plants. Others have it anywhere on the upperpart of their containers. Those are close to man andother organic creatures."

He added that the old seers were convinced thatplants have the most intense communication with in-organic beings. They believed that the lower the as-semblage point, the easier for plants to break thebarrier of perception; very large trees and very smallplants have their assemblage points extremely low intheir cocoon. Because of this, a great number of theold seers' sorcery techniques were means to harnessthe awareness of trees and small plants in order to usethem as guides to descend to what they called thedeepest levels of the dark regions.

"You understand, of course," don Juan went on,"that when they thought they were descending to thedepths, they were, in fact, pushing their assemblagepoints to assemble other perceivable worlds withthose seven great bands.

"They taxed their awareness to the limit and assem-bled worlds with five great bands that are accessibleto seers only if they undergo a dangerous transforma-tion."

"But did the old seers succeed in assembling thoseworlds?" I asked.

"They did," he said. "In their aberration they be-lieved it was worth their while to break all the barriersof perception, even if they had to become trees to dothat."

11Stalking, Intent andthe Dreaming Position

The next day, in the early evening again, don Juancame to the room where I was talking with Genaro.He took me by the arm and walked me through thehouse to the back patio. It was already fairly dark. Westarted to walk around in the corridor that encircledthe patio.

As we walked, don Juan told me that he wanted towarn me once again that it is very easy in the path ofknowledge to get lost in intricacies and morbidity. Hesaid that seers are up against great enemies that candestroy their purpose, muddle their aims, and makethem weak; enemies created by the warriors' path it-self together with the sense of indolence, laziness, andself-importance that are integral parts of the dailyworld.

He remarked that the mistakes the ancient seersmade as a result of indolence, laziness, and self-im-portance were so enormous and so grave that the newseers had no option but to scorn and reject their owntradition.

"The most important thing the new seers needed,"don Juan continued, "was practical steps in order tomake their assemblage points shift. Since they hadnone, they began by developing a keen interest inseeing the glow of awareness, and as a result theyworked out three sets of techniques that became theircornerstone."

Don Juan said that with these three sets, the newseers accomplished a most extraordinary and difficultfeat. They succeeded in systematically making the as-semblage point shift away from its customary posi-tion. He acknowledged that the old seers had alsoaccomplished that feat, but by means of capricious,idiosyncratic maneuvers.

He explained that what the new seers saw in theglow of awareness resulted in the sequence in whichthey arranged the old seers' truths about awareness.This is known as the mastery of awareness. From that,they developed the three sets of techniques. The firstis the mastery of stalking, the second is the masteryof intent, and the third is the mastery of dreaming. Hemaintained that he had taught me these three sets fromthe very first day we met.

He told me that he had taught me the mastery ofawareness in two ways, just as the new seers recom-mend. In his teachings for the right side, which he haddone in normal awareness, he accomplished twogoals: he taught me the warriors' way, and he loos-ened my assemblage point from its original position.In his teachings for the left side, which he had done inheightened awareness, he also accomplished twogoals: he had made my assemblage point shift to asmany positions as I was capable of sustaining, and hehad given me a long series of explanations.

Don Juan stopped talking and stared at me fixedly.There was an awkward silence; then he started to talkabout stalking. He said that it had very humble andfortuitous origins. It started from an observation thenew seers made that when warriors steadily behave inways not customary for them, the unused emanationsinside their cocoons begin to glow. And their assem-blage points shift in a mild, harmonious, barely notice-able fashion.

Stimulated by this observation, the new seers beganto practice the systematic control of their behavior.They called this practice the art of stalking. Don Juanremarked that the name, although objectionable, wasappropriate, because stalking entailed a specific kindof behavior with people, behavior that could be cate-gorized as surreptitious.

The new seers, armed with this technique, tackledthe known in a sober and fruitful way. By continualpractice, they made their assemblage points movesteadily.

"Stalking is one of the two greatest accomplish-ments of the new seers." he said. "The new seersdecided that it should be taught to a modern-day na-gual when his assemblage point has moved quite deepinto the left side. The reason for this decision is that anagual must learn the principles of stalking withoutthe encumbrance of the human inventory. After all,the nagual is the leader of a group, and to lead themhe has to act quickly without first having to thinkabout it.

"Other warriors can learn stalking in their normalawareness, although it is advisable that they do it inheightened awareness?not so much because of thevalue of heightened awareness, but because it imbuesstalking with a mystery that it doesn't really have;stalking is merely behavior with people."

He said that I could now understand that shiftingthe assemblage point was the reason why the newseers placed such a high value on the interaction withpetty tyrants. Petty tyrants forced seers to use theprinciples of stalking and, in doing so, helped seers tomove their assemblage points.

I asked him if the old seers knew anything at allabout the principles of stalking.

'"Stalking belongs exclusively to the new seers," hesaid, smiling. "They are the only seers who had todeal with people. The old ones were so wrapped up intheir sense of power that they didn't even know thatpeople existed, until people started clobbering themon the head. But you already know all this."

Don Juan said next that the mastery of intent to-gether with the mastery of stalking are the new seers'two masterpieces, which mark the arrival of the mod-ern-day seers. He explained that in their efforts to gainan advantage over their oppressors the new seers pur-sued every possibility. They knew that <heir prede-cessors had accomplished extraordinary feats bymanipulating a mysterious and miraculous force,which they could only describe as power. The newseers had very little information about that force, sothey were obliged to examine it systematically throughseeing. Their efforts were amply rewarded when theydiscovered that the energy of alignment is that force.

They began by seeing how the glow of awarenessincreases in size and intensity as the emanations insidethe cocoon are aligned with the emanations at large.They used that observation as a springboard, just asthey had done with stalking, and went on to develop acomplex series of techniques to handle that alignmentof emanations.

At first they referred to those techniques as the mas-tery of alignment. Then they realized that what wasinvolved was much more than alignment; what wasinvolved was the energy that comes out of the align-ment of emanations. They called that energy will.

Will became the second basis. The new seers under-stood it as a blind, impersonal, ceaseless burst of en-ergy that makes us behave in the ways we do. Willaccounts for our perception of the world of ordinaryaffairs, and indirectly, through the force of that per-ception, it accounts for the placement of the assem-blage point in its customary position.

Don Juan said that the new seers examined how theperception of the world of everyday life takes placeand saw the effects of will. They saw that alignment isceaselessly renewed in order to imbue perception withcontinuity. To renew alignment every time with thefreshness that it needs to make up a living world, theburst of energy that comes out of those very align-ments is automatically rerouted to reinforce somechoice alignments.

This new observation served the new seers as an-other springboard that helped them reach the thirdbasis of the set. They called it intent, and they de-scribed it as the purposeful guiding of will, the energyof alignment.

"Silvio Manuel, Genaro, and Vicente were pushedby the nagual Julian to learn those three aspects of theseers' knowledge," he went on. "Genaro is the masterof handling awareness, Vicente is the master of stalk-ing, and Silvio Manuel is the master of intent.

"We are now doing a final explanation of the mas-tery of awareness; this is why Genaro is helping you."

Don Juan talked to the female apprentices for a longtime. The women listened with serious expressions ontheir faces. I felt sure he was giving them detailedinstructions about difficult procedures, judging fromthe women's fierce concentration.

I had been barred from their meeting, but I hadwatched them as they talked in the front room of Ge-naro's house. I sat at the kitchen table, waiting untilthey were through.

Then the women got up to leave, but before theydid, they came to the kitchen with don Juan. He satdown facing me while the women talked to me withawkward formality. They actually embraced me. Allof them were unusually friendly, even talkative. Theysaid that they were going to join the male apprentices,who had gone with Genaro hours earlier. Genaro wasgoing to show all of them his dreaming body.

As soon as the women left, don Juan quite abruptlyresumed his explanation. He said that as time passedand the new seers established their practices, they re-alized that under the prevailing conditions of life,stalking only moved the assemblage points minimally.For maximum effect, stalking needed an ideal setting;it needed petty tyrants in positions of great authorityand power. It became increasingly difficult for the newseers to place themselves in such situations; the taskof improvising them or seeking them out became anunbearable burden.

The new seers deemed it imperative to see the Ea-gle's emanations in order to find a more suitable wayto move the assemblage point. As they tried to see theemanations they were faced with a very serious prob-lem. They found out that there is no way to see themwithout running a mortal risk, and yet they had to seethem. That was the time when they used the old seers'technique of dreaming as a shield to protect them-selves from the deadly blow of the Eagle's emana-tions. And in doing so, they realized that dreamingwas in itself the most effective way to move the as-semblage point.

"One of the strictest commands of the new seers,"don Juan continued, "was that warriors have to learndreaming while they are in their normal state ofawareness. Following that command, I began teachingyou dreaming almost from the first day we met."

"Why do the new seers command that dreaming hasto be taught in normal awareness?" I asked.

"Because dreaming is so dangerous and dreamersso vulnerable," he said. "It is dangerous because ithas inconceivable power; it makes dreamers vulnera-ble because it leaves them at the mercy of the incom-prehensible force of alignment.

"The new seers realized that in our normal state ofawareness, we have countless defenses that can safe-guard us against the force of unused emanations thatsuddenly become aligned in dreaming."

Don Juan explained that dreaming, like stalking,began with a simple observation. The old seers be-came aware that in dreams the assemblage point shiftsslightly to the left side in a most natural manner. Thatpoint indeed relaxes when man sleeps and all kinds ofunused emanations begin to glow.

The old seers became immediately intrigued withthat observation and began to work with that naturalshift until they were able to control it. They called thatcontrol dreaming, or the art of handling the dreamingbody.

He remarked that there is hardly a way of describingthe immensity of their knowledge about dreaming.Very little of it, however, was of any use to the newseers. So when the time of reconstruction came, thenew seers took for themselves only the bare essentialsof dreaming to aid them in seeing the Eagle's emana-tions and to help them move their assemblage points.

He said that seers, old and new, understand dream-ing as being the control of the natural shift that theassemblage point undergoes in sleep. He stressed thatto control that shift does not mean in any way to directit, but to keep the assemblage point fixed at the posi-tion where it naturally moves in sleep, a most difficultmaneuver that took the old seers enormous effort andconcentration to accomplish.

Don Juan explained that dreamers have to strike avery subtle balance, for dreams cannot be interferedwith, nor can they be commanded by the consciouseffort of the dreamer, and yet the shift of the assem-blage point must obey the dreamer's command?acontradiction that cannot be rationalized but must beresolved in practice.

After observing dreamers while they slept, the oldseers hit upon the solution of letting dreams followtheir natural course. They had seen that in somedreams, the assemblage point of the dreamer woulddrift considerably deeper into the left side than inother dreams. This observation posed to them thequestion of whether the content of the dream makesthe assemblage point move, or the movement of theassemblage point by itself produces the content of thedream by activating unused emanations.

They soon realized that the shifting of the assem-blage point into the left side is what produces dreams.The farther the movement, the more vivid and bizarrethe dream. Inevitably, they attempted to commandtheir dreams, aiming to make their assemblage pointsmove deeply into the left side. Upon trying it, theydiscovered that when dreams are consciously or semi-consciously manipulated, the assemblage point imme-diately returns to its usual place. Since what theywanted was for that point to move, they reached theunavoidable conclusion that interfering with dreamswas interfering with the natural shift of the assemblagepoint.

Don Juan said that from there the old seers went onto develop their astounding knowledge on the subject?a knowledge which had a tremendous bearing onwhat the new seers aspired to do with dreaming, butwas of little use to them in its original form.

He told me that thus far I had understood dreamingas being the control of dreams, and that every one ofthe exercises he had given me to perform, such asfinding my hands in my dreams, was not, although itmight seem to be, aimed at teaching me to commandmy dreams. Those exercises were designed to keepmy assemblage point fixed at the place where it hadmoved in my sleep. It is here that the dreamers haveto strike a subtle balance. All they can direct is thefixation of their assemblage points. Seers are like fish-ermen equipped with a line that casts itself whereverit may; the only thing they can do is keep the lineanchored at the place where it sinks.

"Wherever the assemblage point moves in dreamsis called the dreaming position,"' he went on. "Theold seers became so expert at keeping their dreamingposition that they were even able to wake up whiletheir assemblage points were anchored there.

"The old seers called that state the dreaming body,because they controlled it to the extreme of creating atemporary new body every time they woke up at anew dreaming position.

"I have to make it clear to you that dreaming has aterrible drawback," he went on. "It belongs to the oldseers. It's tainted with their mood. I've been verycareful in guiding you through it, but still there is noway to make sure."

"What are you warning me about, don Juan?" Iasked.

"I'm warning you about the pitfalls of dreaming,which are truly stupendous," he replied. "In dream-ing, there is really no way of directing the movementof the assemblage point; the only thing that dictatesthat shift is the inner strength or weakness of dream-ers. Right there we have the first pitfall."

He said that at first the new seers were hesitant touse dreaming. It was their belief that dreaming, in-stead of fortifying, made warriors weak, compulsive,capricious. The old seers were all like that. In orderto offset the nefarious effect of dreaming, since theyhad no other option but to use it, the new seers devel-oped a complex and rich system of behavior called thewarriors' way, or the warriors' path.

With that system, the new seers fortified themselvesand acquired the internal strength they needed toguide the shift of the assemblage point in dreams. DonJuan stressed that the strength that he was talkingabout was not conviction alone. No one could havehad stronger convictions than the old seers, and yetthey were weak to the core. Internal strength meant asense of equanimity, almost of indifference, a feelingof being at ease, but, above all, it meant a natural andprofound bent for examination, for understanding.The new seers called all these traits of character so-briety.

"The conviction that the new seers have," he con-tinued, "is that a life of impeccability by itself leadsunavoidably to a sense of sobriety, and this in turnleads to the movement of the assemblage point.

"I've said that the new seers believed that the as-semblage point can be moved from within. They wentone step further and maintained that impeccable menneed no one to guide them, that by themselves,through saving their energy, they can do everythingthat seers do. All they need is a minimal chance, justto be cognizant of the possibilities that seers have un-raveled."

I told him that we were back in the same positionwe had been in in my normal state of awareness. I wasstill convinced that impeccability or saving energy wassomething so vague that it could be interpreted byanyone in whatever whimsical way he wanted.

I wanted to say more to build my argument, but astrange feeling overtook me. It was an actual physicalsensation that I was rushing through something. Andthen I rebuffed my own argument. I knew without anydoubt whatsoever that don Juan was right. All that isrequired is impeccability, energy, and that begins witha single act that has to be deliberate, precise, andsustained. If that act is repeated long enough, one ac-quires a sense of unbending intent, which can be ap-plied to anything else. If that is accomplished the roadis clear. One thing will lead to another until the warriorrealizes his full potential.

When I told don Juan what I had just realized, helaughed with apparent delight and exclaimed that thiswas indeed a godsent example of the strength that hewas talking about. He explained that my assemblagepoint had shifted, and that it had been moved by so-briety to a position that fostered understanding. Itcould have as well been moved by capriciousness to aposition that only enhances self-importance, as hadbeen the case many times before.

"Let's talk now about the dreaming body."' he wenton. "The old seers concentrated all their efforts onexploring and exploiting the dreaming body. And theysucceeded in using it as a more practical body, whichis tantamount to saying they recreated themselves inincreasingly weird ways."

Don Juan maintained that it is common knowledgeamong the new seers that flocks of the old sorcerersnever came back after waking up at a dreaming posi-tion of their liking. He said that chances are they alldied in those inconceivable worlds, or they may stillbe alive today in who knows what kind of contortedshape or manner.

He stopped and looked at me and broke into a greatlaugh.

"You're dying to ask me what the old seers did withthe dreaming body, aren't you?" he asked, and urgedme with a movement of his chin to ask the question.

Don Juan stated that Genaro, being the indisputablemaster of awareness, had shown me the dreamingbody many times while I was in a state of normalawareness. The effect that Genaro was after with hisdemonstrations was to make my assemblage pointmove, not from a position of heightened awareness,but from its normal setting.

Don Juan told me then, as if he were letting a secretbe known, that Genaro was waiting for us in somefields near the house to show me his dreaming body.He repeated over and over that I was now in the per-fect state of awareness to see and understand what thedreaming body really is. Then he had me get up, andwe walked through the front room to reach the doorto the outside. As I was about to open the door, Inoticed that someone was lying on the pile of floormats that the apprentices used as beds. I thought thatone of the apprentices must have returned to thehouse while don Juan and I were talking in thekitchen.

I went up to him, and then I realized that it wasGenaro. He was sound asleep, snoring peacefully,lying face down.

"Wake him up," don Juan said to me. "We've gotto be going. He must be dead tired."

I gently shook Genaro. He slowly turned around,made the sounds of someone waking up from a deepslumber. He stretched his arms, and then he openedhis eyes. I screamed involuntarily and jumped back.

Genaro's eyes were not human eyes at all. Theywere two points of intense amber light. The jolt of myfright had been so intense that I became dizzy. DonJuan tapped my back and restored my equilibrium.

Genaro stood up and smiled at me. His featureswere rigid. He moved as if he were drunk or physicallyimpaired. He walked by me and headed directly forthe wall. I winced at the imminent crash, but he wentthrough the wall as if it were not there at all. He cameback into the room through the kitchen doorway. Andthen, as I looked in true horror, Genaro walked on thewalls, with his body parallel to the ground, and on theceiling, with his head upside down.

I fell backwards as I tried to follow his movements.From that position I didn't see Genaro anymore; in-stead I was looking at a blob of light that moved onthe ceiling above me and on the walls, circling theroom. It was as if someone with a giant flashlight wasshining the beam on the ceiling and the walls. Thebeam of light was finally turned off. It disappearedfrom view by vanishing against a wall.

Don Juan remarked that my animal fright was al-ways out of measure, that I had to struggle to bring itunder control, but that all in all, I had behaved verywell. I had seen Genaro's dreaming body as it reallyis, a blob of light.

I asked him how he was so sure I had done that. Hereplied that he had seen my assemblage point firstmove toward its normal setting in order to compensatefor my fright, then move deeper into the left, beyondthe point where there are no doubts.

"At that position there is only one thing one cansee: blobs of energy," he went on. "But fromheightened awareness to that other point deeper intothe left side, it is only a short hop. The real feat is tomake the assemblage point shift from its normal set-ting to the point of no doubt."

He added that we still had an appointment with Ge-naro's dreaming body in the fields around the house,while I was in normal awareness.

When we were back in Silvio Manuel's house, donJuan said that Genaro's proficiency with the dreamingbody was a very minor affair compared with what theold seers did with it.

"You'll see that very soon," he said with an omi-nous tone, then laughed.

I questioned him about it with mounting fear, andthat only evoked more laughter. He finally stoppedand said that he was going to talk about the way thenew seers got to the dreaming body and the way theyused it.

"The old seers were after a perfect replica of thebody," he continued, "and they nearly succeeded ingetting one. The only thing they never could copy wasthe eyes. Instead of eyes, the dreaming body has justthe glow of awareness. You never realized that before,when Genaro used to show you his dreaming body.

"The new seers could not care less about a perfectreplica of the body; in fact, they are not even inter-ested in copying the body at all. But they have keptjust the name dreaming body to mean a feeling, asurge of energy that is transported by the movementof the assemblage point to any place in this world, orto any place in the seven worlds available to man."

Don Juan then outlined the procedure for getting tothe dreaming body. He said that it starts with an initialact, which by the fact of being sustained breeds un-bending intent. Unbending intent leads to internal si-lence, and internal silence to the inner strength neededto make the assemblage point shift in dreams to suit-able positions.

He called this sequence the groundwork. The devel-opment of control comes after the groundwork hasbeen completed; it consists of systematically main-taining the dreaming position by doggedly holding onto the vision of the dream. Steady practice results in agreat facility to hold new dreaming positions with newdreams, not so much because one gains deliberatecontrol with practice, but because every time this con-trol is exercised the inner strength gets fortified. For-tified inner strength in turn makes the assemblagepoint shift into dreaming positions, which are moreand more suitable to fostering sobriety; in otherwords, dreams by themselves become more and moremanageable, even orderly.

"The development of dreamers is indirect," hewent on. "That's why the new seers believed we cando dreaming by ourselves, alone. Since dreaming usesa natural, built-in shift of the assemblage point, weshould need no one to help us.

"What we badly need is sobriety, and no one cangive it to us or help us get it except ourselves. Withoutit, the shift of the assemblage point is chaotic, as ourordinary dreams are chaotic.

"So, all in all, the procedure to get to the dreamingbody is impeccability in our daily life."

Don Juan explained that once sobriety is acquiredand the dreaming positions become increasinglystronger, the next step is to wake up at any dreamingposition. He remarked that the maneuver, althoughmade to sound simple, was really a very complex af-fair?so complex that it requires not only sobriety butall the attributes of warriorship as well, especially in-tent.

I asked him how intent helps seers wake up at adreaming position. He replied that intent, being themost sophisticated control of the force of alignment,is what maintains, through the dreamer's sobriety, thealignment of whatever emanations have been lit up bythe movement of the assemblage point.

Don Juan said that there is one more formidablepitfall of dreaming: the very strength of the dreamingbody. For example, it is very easy for the dreamingbody to gaze at the Eagle's emanations uninterrupt-edly for long periods of time, but it is also very easyin the end for the dreaming body to be totally con-sumed by them. Seers who gazed at the Eagle's ema-nations without their dreaming bodies died, and thosewho gazed at them with their dreaming bodies burnedwith the fire from within. The new seers solved theproblem by seeing in teams. While one seer gazed atthe emanations, others stood by ready to end theseeing.

"How did the new seers see in teams?" I asked.

"They dreamed together, '" he replied. "As youyourself know, it's perfectly possible for a group ofseers to activate the same unused emanations. And inthis case also, there are no known steps, it just hap-pens; there is no technique to follow."

He added that in dreaming together, something inus takes the lead and suddenly we find ourselves shar-ing the same view with other dreamers. What happensis that our human condition makes us focus the glowof awareness automatically on the same emanationsthat other human beings are using; we adjust the po-sition of our assemblage points to fit the others aroundus. We do that on the right side, in our ordinary per-ception, and we also do it on the left side, whiledreaming together.

12The Nagual Julian

There was a strange excitement in the house. All theseers of don Juan's party seemed to be so elated thatthey were actually absentminded, a thing that I hadnever witnessed before. Their usual high level of en-ergy appeared to have increased. I became very ap-prehensive. I asked don Juan about it. He took me tothe back patio. We walked in silence for a moment.He said that the time was getting closer for all of themto leave. He was pressing his explanation in order tofinish it in time.

"How do you know that you are closer to leaving?"I asked.

"It is an internal knowledge," he said. "You'llknow it someday yourself. You see, the nagual Julianmade my assemblage point shift countless times, justas I have made yours shift. Then he left me the task ofrealigning all those emanations which he had helpedme align through these shifts. That is the task thatevery nagual is left to do.

"At any rate, the job of realigning all those emana-tions paves the way for the peculiar maneuver of light-ing up all the emanations inside the cocoon. I havenearly done that. I am about to reach my maximum.Since I am the nagual, once I do light up all the ema-nations inside my cocoon we will all be gone in aninstant."

I felt I should be sad and weep, but something in mewas so overjoyed to hear that the nagual Juan Matuswas about to be free that I jumped and yelled withsheer delight. I knew that sooner or later I wouldreach another state of awareness and I would weepwith sadness. But that day I was filled with happinessand optimism.

I told don Juan how I felt. He laughed and pattedmy back.

"Remember what I've told you," he said. "Don'tcount on emotional realizations. Let your assemblagepoint move first, then years later have the realiza-tion."

We walked to the big room and sat down to talk.Don Juan hesitated for a moment. He looked out ofthe window. From my chair I could see the patio. Itwas early afternoon; a cloudy day. It looked like rain.Thunderhead clouds were moving in from the west. Iliked cloudy days. Don Juan did not. He seemed rest-less as he tried to find a more comfortable sitting po-sition.

Don Juan began his elucidation by commenting thatthe difficulty in remembering what takes place inheightened awareness is due to the infinitude of posi-tions that the assemblage point can adopt after beingloosened from its normal setting. Facility in remem-bering everything that takes place in normal aware-ness, on the other hand, has to do with the fixity ofthe assemblage point on one spot, the spot where itnormally sets.

He told me that he commiserated with me. He sug-gested that I accept the difficulty of recollecting andacknowledge that I might fail in my task and never beable to realign all the emanations that he had helpedme align.

"Think of it this way," he said, smiling. "You maynever be able to remember this very conversation thatwe are having now, which at this moment seems toyou so commonplace, so taken for granted.

"This indeed is the mystery of awareness. Humanbeings reek of that mystery; we reek of darkness, ofthings which are inexplicable. To regard ourselves inany other terms is madness. So don't demean the mys-tery of man in you by feeling sorry for yourself or bytrying to rationalize it. Demean the stupidity of manin you by understanding it. But don't apologize foreither; both are needed.

"One of the great maneuvers of stalkers is to pit themystery against the stupidity in each of us."

He explained that stalking practices are not some-thing one can rejoice in; in fact, they are downrightobjectionable. Knowing this, the new seers realizethat it would be against everybody's interest to dis-cuss or practice the principles of stalking in normalawareness.

I pointed out to him an incongruity. He had saidthat there is no way for warriors to act in the worldwhile they are in heightened awareness, and he hadalso said that stalking is simply behaving with peoplein specific ways. The two statements contradictedeach other.

"By not teaching it in normal awareness I was re-ferring only to teaching it to a nagual," he said. "Thepurpose of stalking is twofold: first, to move the as-semblage point as steadily and safely as possible, andnothing can do the job as well as stalking: second, toimprint its principles at such a deep level that thehuman inventory is bypassed, as is the natural reac-tion of refusing and judging something that may beoffensive to reason."

I told him that I sincerely doubted I could judge orrefuse anything like that. He laughed and said that Icould not be an exception, that I would react likeeveryone else once I heard about the deeds of a masterstalker, such as his benefactor, the nagual Julian.

"I am not exaggerating when I tell you that the na-gual Julian was the most extraordinary stalker I haveever met," don Juan said. "You have already heardabout his stalking skills from everybody else. But I'venever told you what he did to me."

I wanted to make it clear to him that I had not heardanything about the nagual Julian from anyone, but justbefore I voiced my protest a strange feeling of uncer-tainty swept over me. Don Juan seemed to know in-stantly what I was feeling. He chuckled with delight.

"You can't remember, because will is not availableto you yet," he said. "You need a life of impeccabilityand a great surplus of energy, and then will mightrelease those memories.

"I am going to tell you the story of how the nagualJulian behaved with me when I first met him. If youjudge him and find his behavior objectionable whileyou are in heightened awareness, think of how re-volted you might be with him in normal awareness."

I protested that he was setting me up. He assuredme that all he wanted to do with his story was toillustrate the manner in which stalkers operate and thereasons why they do it.

"The nagual Julian was the last of the old-timestalkers," he went on. "He was a stalker not so muchbecause of the circumstances of his life but becausethat was the bent of his character."

Don Juan explained that the new seers saw thatthere are two main groups of human beings: those whocare about others and those who do not. In betweenthese two extremes they saw an endless mixture of thetwo. The nagual Julian belonged to the category ofmen who do not care; don Juan classified himself asbelonging to the opposite category.

"But didn't you tell me that the nagual Julian wasgenerous, that he would give you the shirt off hisback?" I asked.

"He certainly was," don Juan replied. "Not onlywas he generous; he was also utterly charming, win-ning. He was always deeply and sincerely interestedin everybody around him. He was kind and open andgave away everything he had to anyone who neededit, or to anyone he happened to like. He was in turnloved by everyone, because being a master stalker, heconveyed to them his true feelings: he didn't give aplugged nickel for any of them."

I did not say anything, but don Juan was aware ofmy sense of disbelief or even distress at what he wassaying. He chuckled and shook his head from side toside.

"That's stalking," he said. "You see, I haven'teven begun my story of the nagual Julian and you arealready annoyed."

He exploded into a giant laugh as I tried to explainwhat I was feeling.

"The nagual Julian didn't care about anyone," hecontinued. "That's why he could help people. And hedid; he gave them the shirt off his back, because hedidn't give a fig about them."

"Do you mean, don Juan, that the only ones whohelp their fellow men are those who don't give a damnabout them?" I asked, truly miffed.

"That's what stalkers say," he said with a beamingsmile. "The nagual Julian, for instance, was a fabu-lous curer. He helped thousands and thousands ofpeople, but he never took credit for it. He let peoplebelieve that a woman seer of his party was the curer.

"Now, if he had been a man who cared for his fel-low men, he would've demanded acknowledgment.Those who care for others care for themselves anddemand recognition where recognition is due."

Don Juan said that he, since he belonged to thecategory of those who care for their fellow men, hadnever helped anyone: he felt awkward with generos-ity; he could not even conceive being loved as thenagual Julian was, and he would certainly feel stupidgiving anyone the shirt off his back.

"I care so much for my fellow man," he continued,"that I don't do anything for him. I wouldn't knowwhat to do. And I would always have the naggingsense that I was imposing my will on him with mygifts.

"Naturally, I have overcome all these feelings withthe warriors' way. Any warrior can be successful withpeople, as the nagual Julian was, provided he moveshis assemblage point to a position where it is immater-ial whether people like him, dislike him, or ignore him.But that's not the same."

Don Juan said that when he first became aware ofthe stalkers' principles, as I was then doing, he was asdistressed as he could be. The nagual Elias, who wasvery much like don Juan, explained to him that stalk-ers like the nagual Julian are natural leaders of people.They can help people do anything.

"The nagual Elias said that these warriors can helppeople to get cured," don Juan went on, "or they canhelp them to get ill. They can help them to find happi-ness or they can help them to find sorrow. I suggestedto the nagual Elias that instead of saying that thesewarriors help people, we should say that they affectpeople. He said that they don't just affect people, butthat they actively herd them around."

Don Juan chuckled and looked at me fixedly. Therewas a mischievous glint in his eyes.

"Strange, isn't it?" he asked. "The way stalkersarranged what they see about people?"

Then don Juan started his story about the nagualJulian. He said that the nagual Julian spent many,many years waiting for an apprentice nagual. He stum-bled on don Juan one day while returning home aftera short visit with acquaintances in a nearby village.He was, in fact, thinking about an apprentice nagualas he walked on the road when he heard a loud gun-shot and saw people scrambling in every direction. Heran with them into the bushes by the side of the roadand only came out from his hiding place at the sight ofa group of people gathered around someone wounded,lying on the ground.

The wounded person was, of course, don Juan, whohad been shot by the tyrannical foreman. The nagualJulian saw instantly that don Juan was a special manwhose cocoon was divided into four sections insteadof two; he also realized that don Juan was badlywounded. He knew that he had no time to waste. Hiswish had been fulfilled, but he had to work fast, beforeanyone sensed what was going on. He held his headand cried, "They've shot my son!"

He was traveling with one of the female seers of hisparty, a husky Indian woman, who always officiatedpublicly as his mean shrewish wife. They were an ex-cellent team of stalkers. He cued the woman seer, andshe also started weeping and wailing for their son, whowas unconscious and bleeding to death. The nagualJulian begged the onlookers not to call the authoritiesbut rather to help him move his son to his house in thecity, which was some distance away. He offeredmoney to some strong young men if they would carryhis wounded, dying son.

The men carried don Juan to the nagual Julian'shouse. The nagual was very generous with them andpaid them handsomely. The men were so touched bythe grieving couple, who had cried all the way to thehouse, that they refused to take the money, but thenagual Julian insisted that they take it to give his sonluck.

For a few days, don Juan did not know what to thinkabout the kind couple who had taken him into theirhome. He said that to him, the nagual Julian appearedas an almost senile old man. He was not an Indian,but was married to a young, irascible, fat Indian wife,who was as physically strong as she was ill-tempered.Don Juan thought that she was definitely a curer, judg-ing by the way she treated his wound and by the quan-tities of medicinal plants stashed away in the roomwhere they had put him.

The woman also dominated the old man and madehim tend to don Juan's wound every day. They hadmade a bed for don Juan out of a thick floor mat, andthe old man had a terrible time kneeling down to reachhim. Don Juan had to fight not to laugh at the comicalsight of the frail old man trying his best to bend hisknees. Don Juan said that while the old man washedhis wound, he would mumble incessantly; he had avacant look in his eyes; his hands shook, and his bodytrembled from head to toe.

When he was down on his knees, he could never getup by himself. He would call his wife, yelling in araspy voice, filled with contained anger. The wifewould come into the room and both of them would getinto a horrible argument. Often she would walk out,leaving the old man to get up by himself.

Don Juan assured me that he had never felt so sorryfor anyone as he felt for that poor, kind old man. Manytimes he wanted to rise and help him up, but he couldhardly move himself. Once the old man spent half anhour cursing and yelling, as he puffed and crawled likea slug, before he dragged himself to the door and pain-fully lifted himself up to a standing position.

He explained to don Juan that his poor health wasdue to advanced age, broken bones that had notmended properly, and rheumatism. Don Juan said thatthe old man raised his eyes toward heaven and con-fessed to don Juan that he was the most wretched manon earth; he had come to the curer for help and hadended up marrying her and becoming a slave.

"I asked the old man why he didn't leave," donJuan continued. "The old man's eyes widened withfear. He choked on his own saliva trying to hush meand then he went rigid and fell down like a log on thefloor, next to my bed, trying to make me stop talking.'You don't know what you're saying; you don't knowwhat you're saying. Nobody can run away from thisplace, ' the old man kept on repeating with a wildexpression in his eyes.

"And I believed him. I was convinced that he wasmore miserable, more wretched than I had ever beenmyself. And with every day that passed I becamemore and more uncomfortable in that house. The foodwas great and the woman was always out curing peo-ple, so I was left with the old man. We talked a lotabout my life. I liked to talk to him. I told him that Ihad no money to pay him for his kindness, but that Iwould do anything to help him. He told me that he wasbeyond help, that he was ready to die, but that if Ireally meant what I said, he would appreciate it if Iwould marry his wife after he died.

"Right then I knew the old man was nuts. And rightthen I also knew that I had to run away as soon aspossible."

Don Juan said that when he was well enough to walkaround unaided, his benefactor gave him a chillingdemonstration of his ability as a stalker. Without anywarning or preamble he put don Juan face to face withan inorganic living being. Sensing that don Juan wasplanning to run away, he seized the opportunity toscare him with an ally that was somehow able to looklike a monstrous man.

"The sight of that ally nearly drove me insane," donJuan continued. "I couldn't believe my eyes, and yetthe monster was right in front of me. And the frail oldman was next to me whimpering and begging the mon-ster to spare his life. You see, my benefactor was likethe old seers; he could dole out his fear, a piece at atime, and the ally was reacting to it. I didn't knowthat. All I could see with my very own eyes was ahorrendous creature advancing on us, ready to tear usapart, limb from limb.

"The moment the ally lurched onto us, hissing likea serpent, I passed out cold. When I came to mysenses again, the old man told me that he had made adeal with the creature."

He explained to don Juan that the man had agreedto let both of them live, provided don Juan enter theman's service. Don Juan apprehensively asked whatwas involved in the service. The old man replied thatit would be slavery, but pointed out that don Juan'slife had nearly ended a few days back when he hadbeen shot. Had not he and his wife come along to stopthe bleeding, don Juan would surely have died, sothere was really very little to bargain with, or to bar-gain for. The monstrous man knew that and had himover a barrel. The old man told don Juan to stop vac-illating and accept the deal, because if he refused, themonstrous man, who was listening behind the door,would burst in and kill them both on the spot and bedone with it.

"I had enough nerve to ask the frail old man, whowas shaking like a leaf, how the man would kill us,"don Juan went on. "He said that the monster plannedto break all the bones in our bodies, starting with ourfeet, as we screamed in unspeakable agony, and thatit would take at least five days for us to die.

"I accepted that man's conditions instantly. The oldman, with tears in his eyes, congratulated me and saidthat the deal wasn't really that bad. We were going tobe more prisoners than slaves of the monstrous man,but we would eat at least twice a day; and since wehad life, we could work for our freedom; we couldplot, connive, and fight our way out of that hell."

Don Juan smiled and then broke into laughter. Hehad known beforehand how I would feel about thenagual Julian.

"I told you you'd be upset," he said.

"I really don't understand, don Juan," I said."What was the point of putting on such an elaboratemasquerade?"

"The point is very simple," he said, still smiling."This is another method of teaching, a very good one.it requires tremendous imagination and tremendouscontrol on the part of the teacher. My method ofteaching is closer to what you consider teaching. Itrequires a tremendous amount of words. I go to theextremes of talking. The nagual Julian went to theextremes of stalking."

Don Juan said that there were two methods ofteaching among the seers. He was familiar with bothof them. He preferred the one that called for explain-ing everything and letting the other person know thecourse of action beforehand. It was a system that fos-tered freedom, choice, and understanding. His bene-factor's method, on the other hand, was morecoercive and did not allow for choice or understand-ing. Its great advantage was that it forced warriors tolive the seers' concepts directly with no intermediaryelucidation.

Don Juan explained that everything his benefactordid to him was a masterpiece of strategy. Every oneof the nagual Julian's words and actions was deliber-ately selected to cause a particular effect. His art wasto provide his words and actions with the most suit-able context, so that they would have the necessaryimpact.

"That's the stalkers' method," don Juan went on."It fosters not understanding but total realization. Forinstance, it took me a lifetime to understand what hehad done to me by making me face the ally, althoughI realized all that without any explanation as I livedthat experience.

"I've told you that Genaro, for example, doesn'tunderstand what he does, but his realization of whathe is doing is as keen as it can be. That's because hisassemblage point was moved by the stalkers'method."

He said that if the assemblage point is forced out ofits customary setting by the method of explainingeverything, as in my case, there is always the need forsomeone else not only to help in the actual dislodgingof the assemblage point, but in dispensing the expla-nations of what is going on. But if the assemblagepoint is moved by the stalkers' method, as in his owncase, or Genaro's, there is only a need for the initialcatalytic act that yanks the point from its location.

Don Juan said that when the nagual Julian made himface the monstrous-looking ally his assemblage pointmoved under the impact of fear. So intense a fright asthat caused by the confrontation, coupled with hisweak physical condition, was ideal for dislodging hisassemblage point.

In order to offset the injurious effects of fright, itsimpact had to be cushioned, but not minimized. Ex-plaining what was happening would have minimizedfear. What the nagual Julian wanted was to make surethat he could use that initial catalytic fright as manytimes as he needed it, but he also wanted to make surethat he could cushion its devastating impact; that wasthe reason for his masquerade. The more elaborateand dramatic his stories were, the greater their cush-ioning effect. If he, himself, seemed to be in the sameboat with don Juan, the fright would not be as intenseas if don Juan were alone.

"With his penchant for drama," don Juan went on,"my benefactor was able to move my assemblagepoint enough to imbue me right away with an over-powering feeling for the two basic qualities of war-riors: sustained effort and unbending intent. I knewthat in order to be free again someday, I would haveto work in an orderly and steady fashion and in coop-eration with the frail old man, who in my opinionneeded my help as much as I needed his. I knew be-yond a shadow of a doubt that that was what I wantedto do more than anything else in life."

I did not get to talk to don Juan again until two dayslater. We were in Oaxaca, strolling in the main square,in the early morning. There were children walking toschool, people going to church, a few men sitting onthe benches, and taxi drivers waiting for tourists fromthe main hotel.

"It goes without saying that the most difficult thingin the warriors' path is to make the assemblage pointmove," don Juan said. "That movement is the com-pletion of the warriors' quest. To go on from there isanother quest; it is the seers' quest proper."

He repeated that in the warriors' way, the shift ofthe assemblage point is everything. The old seers ab-solutely failed to realize this truth. They thought themovement of the point was like a marker that deter-mined their positions on a scale of worth. They neverconceived that it was that very position which deter-mined what they perceived.

"The stalkers' method," don Juan went on, "in thehands of a master stalker like the nagual Julian, ac-counts for stupendous shifts of the assemblage point.These are very solid changes; you see, by buttressingthe apprentice, the stalker-teacher gets the appren-tice's full cooperation and full participation. To getanybody's full cooperation and full participation isabout the most important outcome of the stalkers'method; and the nagual Julian was the best at gettingboth of them."

Don Juan said that there was no way for him todescribe the turmoil that he went through as he foundout, little by little, about the richness and the complex-ity of the nagual Julian's personality and life. As longas don Juan faced a scared, frail old man who seemedhelpless, he was fairly at ease, comfortable. But oneday, soon after they had made the deal with what donJuan thought of as a monstrous-looking man, his com-fort was shot to pieces when the nagual Julian gavedon Juan another unnerving demonstration of hisstalking skills.

Although don Juan was quite well by then, the na-gual Julian still slept in the same room with him inorder to nurse him. When he woke up that day, heannounced to don Juan that their captor was gone fora couple of days, which meant that he did not have toact like an old man. He confided to don Juan that heonly pretended to be old in order to fool the mon-strous-looking man.

Without giving don Juan time to think, he jumpedup from his mat with incredible agility; he bent overand dunked his head in a pot of water and kept it therefor a while. When he straightened up, his hair was jetblack, the gray hair had washed away, and don Juanwas looking at a man he had never seen before, a manperhaps in his late thirties. He flexed his muscles,breathed deeply, and stretched every part of his bodyas if he had been too long inside a constricting cage.

"When I saw the nagual Julian as a young man, Ithought that he was indeed the devil," don Juan wenton. "I closed my eyes and knew that my end was near.The nagual Julian laughed until he was crying."

Don Juan said that the nagual Julian then put him atease by making him shift back and forth between theright side and the left side awareness.

"For two days the young man pranced around thehouse," don Juan continued. "He told me storiesabout his life and jokes that sent me reeling around theroom with laughter. But what was even more astound-ing was the way his wife had changed. She was ac-tually thin and beautiful. I thought she was acompletely different woman. I raved about how com-plete her change was and how beautiful she looked.The young man said that when their captor was awayshe was actually another woman."

Don Juan laughed and said that his devilish benefac-tor was telling the truth. The woman was really an-other seer of the nagual's party.

Don Juan asked the young man why they pretendedto be what they were not. The young man looked atdon Juan, his eyes filled with tears, and said that themysteries of the world are indeed unfathomable. Heand his young wife had been caught by inexplicableforces and had to protect themselves with that pre-tense. The reason why he carried on the way he did,as a feeble old man, was that their captor was alwayspeeking in through cracks in the doors. He begged donJuan to forgive him for having fooled him.

Don Juan asked who that monstrous-looking manwas. With a deep sigh, the young man confessed thathe could not even guess. He told don Juan that al-though he himself was an educated man, a famousactor from the theater in Mexico City, he was at a lossfor explanations. All he knew was that he had come tobe treated for the consumption that he had sufferedfrom for many years. He was near death when hisrelatives brought him to meet the curer. She helpedhim to get well, and he fell madly in love with thebeautiful young Indian and married her. His planswere to take her to the capital so they could get richwith her curing ability.

Before they started on the trip to Mexico City, shewarned him that they had to disguise themselves inorder to escape a sorcerer. She explained to him thather mother had also been a curer, and had been taughtcuring by that master sorcerer, who had demandedthat she, the daughter, stay with him for life. Theyoung man said that he had refused to ask his wifeabout that relationship. He only wanted to free her, sohe disguised himself as an old man and disguised heras a fat woman.

Their story did not end happily. The horrible mancaught them and kept them as prisoners. They did notdare to take off their disguise in front of that nightmar-ish man, and in his presence they carried on as if theyhated each other; but in reality, they pined for eachother and lived only for the short times when that manwas away.

Don Juan said that the young man embraced himand told him that the room where don Juan was sleep-ing was the only safe place in the house. Would heplease go out and be on the lockout while he madelove to his wife?

"The house shook with their passion," don Juanwent on, "while I sat by the door feeling guilty forlistening and scared to death that the man would comeback any minute. And sure enough, I heard him com-ing into the house. I banged on the door, and whenthey didn't answer, I walked in. The young womanwas asleep naked and the young man was nowhere insight. I had never seen a beautiful naked woman in mylife. I was still very weak. I heard the monstrous manrattling outside. My embarrassment and my fear wereso great that I passed out."

The story about the nagual Julian's doings annoyedme no end. I told don Juan that I had failed to under-stand the value of the nagual Julian's stalking skills.Don Juan listened to me without making a single com-ment and let me ramble on and on.

When we finally sat down on a bench, I was verytired. I did not know what to say when he asked mewhy his account of the nagual Julian's method ofteaching had upset me so much.

"I can't shake off the feeling that he was a pranks-ter," I finally said.

"Pranksters don't teach anything deliberately withtheir pranks," don Juan retorted. "The nagual Julianplayed dramas, magical dramas that required a move-ment of the assemblage point."

"He seems like a very selfish person to me," I in-sisted.

"He seems like that to you because you are judg-ing," he replied. "You are being a moralist. I wentthrough all that myself. If you feel the way you do onhearing about the nagual Julian, think of the way Imust have felt myself living in his house for years. Ijudged him, I feared him, and I envied him, in thatorder.

"I also loved him, but my envy was greater than mylove. I envied his ease, his mysterious capacity to beyoung or old at will; I envied his flair and above all hisinfluence on whoever happened to be around. It woulddrive me up the walls to hear him engage people in themost interesting conversation. He always had some-thing to say; I never did, and I always felt incompe-tent, left out."

Don Juan's revelations made me feel ill at ease. Iwished that he would change the subject, for I did notwant to hear that he was like me. In my opinion, hewas indeed unequaled. He obviously knew how I felt.He laughed and patted my back.

"What I am trying to do with the story of my envy,"he went on, "is to point out to you something of greatimportance, that the position of the assemblage pointdictates how we behave and how we feel.

"My great flaw at that time was that I could notunderstand this principle. I was raw. I lived throughself-importance, just as you do, because that waswhere my assemblage point was lodged. You see, Ihadn't learned yet that the way to move that point isto establish new habits, to will it to move. When it didmove, it was as if I had just discovered that the onlyway to deal with peerless warriors like my benefactoris not to have self-importance, so that one can cele-brate them unbiasedly."

He said that realizations are of two kinds. One isjust pep talk, great outbursts of emotion and nothingmore. The other is the product of a shift of the assem-blage point; it is not coupled with an emotional out-burst but with action. The emotional realizations comeyears later after warriors have solidified, by usage, thenew position of their assemblage points.

"The nagual Julian tirelessly guided all of us to thatkind of shift," don Juan went on. "He got from all ofus total cooperation and total participation in his big-ger-than-life dramas. For instance, with his drama ofthe young man and his wife and their captor he hadmy undivided attention and concern. To me the storyof the old man who was young was very consistent. Ihad seen the monstrous-looking man with my veryown eyes, which meant that the young man got myundying affiliation."

Don Juan said that the nagual Julian was a magician,a conjurer who could handle the force of will to adegree that would be incomprehensible to the averageman. His dramas included magical characters sum-moned by the force of intent, like the inorganic beingthat could adopt a grotesque human form.

"The nagual Julian's power was so impeccable,"don Juan went on, "that he could force anyone's as-semblage point to shift and align emanations thatwould make him perceive whatever the nagual Julianwanted. For example, he could look very old or veryyoung for his age, depending on what he wanted toaccomplish. And all anyone who knew the nagualcould say about his age was that it fluctuated. Duringthe thirty-two years that I knew him he was at timesnot much older than you are now, and at other timeshe was so wretchedly old that he could not evenwalk."

Don Juan said that under his benefactor's guidancehis assemblage point moved unnoticeably and yet pro-foundly. For instance, out of nowhere one day he re-alized that he had a fear that on the one hand made nosense to him at all, and on the other made all the sensein the world.

"My fear was that through stupidity I would losemy chance to be free and I would repeat my father'slife.

"There was nothing wrong with my father's life,mind you. He lived and died no better and no worsethan most men; the important point is that my assem-blage point had moved and I realized one day that myfather's life and death hadn't amounted to a hill ofbeans, either to others or to himself.

"My benefactor told me that my father and motherhad lived and died just to have me, and that their ownparents had done the same for them. He said that war-riors were different in that they shift their assemblagepoints enough to realize the tremendous price that hasbeen paid for their lives. This shift gives them therespect and awe that their parents never felt for life ingeneral, or for being alive in particular."

Don Juan said that not only was the nagual Juliansuccessful in guiding his apprentices to move their as-semblage points, but that he enjoyed himself tremen-dously while doing it.

"He certainly entertained himself immensely withme," don Juan went on. "When the other seers of myparty began to come, years later, even I looked for-ward to the preposterous situations that he createdand developed with each one of them.

"When the nagual Julian left the world, delight wentaway with him and never came back. Genaro delightsus sometimes, but no one can take the nagual Julian'splace. His dramas were always bigger than life. I as-sure you we didn't know what enjoyment was until wesaw what he did when some of those dramas backfiredon him."

Don Juan rose from his favorite bench. He turnedto me. His eyes were brilliant and peaceful.

"If you are ever so dumb as to fail in your task,"he said, "you must have at least enough energy tomove your assemblage point in order to come to thisbench. Sit down here for an instant, free of thoughtsand desires; I will try to come here from wherever Iam and collect you. I promise you that I will try."

He then broke into a great laugh, as if the scope ofhis promise was too ludicrous to be believed.

"These words should be said in the late afternoon,"he said, still laughing. "Never in the morning. Themorning makes one feel optimistic and such wordslose their meaning."

13The Earth's Boost

"Let's walk on the road to Oaxaca," don Juan said tome. "Genaro is waiting for us somewhere along theway."

His request took me by surprise. I had been waitingall day for him to continue his explanation. We left hishouse and walked in silence through the town to theunpaved highway. We walked leisurely for a longtime. Suddenly don Juan began to talk.

"I've been telling you all along about the great find-ings that the old seers made," he said. "Just as theyfound out that organic life is not the only life presenton earth, they also discovered that the earth itself is aliving being."

He waited a moment before continuing. He smiledat me as if inviting me to make a comment. I could notthink of anything to say.

"The old seers saw that the earth has a cocoon," hewent on. "They saw that there is a ball encasing theearth, a luminous cocoon that entraps the Eagle's em-anations. The earth is a gigantic sentient being sub-jected to the same forces we are."

He explained that the old seers, on discovering this,became immediately interested in the practical uses ofthat knowledge. The result of their interest was thatthe most elaborate categories of their sorcery had todo with the earth. They considered the earth to be theultimate source of everything we are.

Don Juan reaffirmed that the old seers were not mis-taken in this respect, because the earth is indeed ourultimate source.

He didn't say anything else until we met Genaroabout a mile up the road. He was waiting for us, sittingon a rock by the side of the road.

He greeted me with great warmth. He said to methat we should climb up to the top of some small rug-ged mountains covered with hardy vegetation.

"The three of us are going to sit against a rock,"don Juan said to me, "and look at the sunlight as it isreflected on the eastern mountains. When the sun goesdown behind the western peaks, the earth may let yousee alignment."

When we reached the top of a mountain, we satdown, as don Juan had said, with our backs against arock. Don Juan made me sit in between the two ofthem.

I asked him what he was planning to do. His crypticstatements and his long silences were ominous. I feltterribly apprehensive.

He didn't answer me. He kept on talking as if I hadnot spoken at all.

"it was the old seers who, on discovering that per-ception is alignment," he said, "stumbled onto some-thing monumental. The sad part is that theiraberrations again kept them from knowing what theyhad accomplished."

He pointed at the mountain range east of the smallvalley where the town is located.

"There is enough glitter in those mountains to joltyour assemblage point," he said to me. "Just beforethe sun goes down behind the western peaks, you willhave a few moments to catch all the glitter you need.The magic key that opens the earth's doors is made ofinternal silence plus anything that shines."

"What exactly should I do, don Juan?" I asked.

Both of them examined me. I thought I saw in theireyes a mixture of curiosity and revulsion.

"Just cut off the internal dialogue," don Juan saidto me.

I had an intense pang of anxiety and doubt; I had noconfidence that I could do it at will. After an initialmoment of nagging frustration, I resigned my self justto relax.

I looked around. I noticed that we were high enoughto look down into the long, narrow valley. More thanhalf of it was in the late-afternoon shadows. The sunwas still shining on the foothills of the eastern rangeof mountains, on the other side of the valley; the sun-light made the eroded mountains look ocher, while themore distant bluish peaks had acquired a purple tone.

"You do realize that you've done this before, don'tyou?" don Juan said to me in a whisper.

I told him that I had not realized anything.

"We've sat here before on other occasions," heinsisted, "but that doesn't matter, because this occa-sion is the one that will count.

"Today, with the help of Genaro, you are going tofind the key that unlocks everything. You won't beable to use it as yet, but you'll know what it is andwhere it is. Seers pay the heaviest prices to know that.You, yourself, have been paying your dues all theseyears."

He explained that what he called the key to every-thing was the firsthand knowledge that the earth is asentient being and as such can give warriors a tremen-dous boost; it is an impulse that comes from theawareness of the earth itself at the instant in which theemanations inside warriors' cocoons are aligned withthe appropriate emanations inside the earth's cocoon.Since both the earth and man are sentient beings, theiremanations coincide, or rather, the earth has all theemanations present in man and all the emanations thatare present in all sentient beings, organic and inor-ganic for that matter. When a moment of alignmenttakes place, sentient beings use that alignment in alimited way and perceive their world. Warriors canuse that alignment either to perceive, like everyoneelse, or as a boost that allows them to enter unimagin-able worlds.

"I've been waiting for you to ask me the only mean-ingful question you can ask, but you never ask it," hecontinued. "You are hooked on asking about whetherthe mystery of it all is inside us. You came closeenough, though.

"The unknown is not really inside the cocoon ofman in the emanations untouched by awareness, andyet it is there, in a manner of speaking. This is thepoint you haven't understood. When I told you thatwe can assemble seven worlds besides the one weknow, you took it as being an internal affair, becauseyour total bias is to believe that you are only imaginingeverything you do with us. Therefore, you have neverasked me where the unknown really is. For years Ihave circled with my hand to point to everythingaround us and I have told you that the unknown isthere. You never made the connection."

Genaro began to laugh, then coughed and stood up."He still hasn't made the connection," he said to donJuan.

I admitted to them that if there was a connection tobe made, I had failed to make it.

Don Juan restated over and over that the portion ofemanations inside man's cocoon is in there only forawareness, and that awareness is matching that por-tion of emanations with the same portion of emana-tions at large. They are called emanations at largebecause they are immense; and to say that outsideman's cocoon is the unknowable is to say that withinthe earth's cocoon is the unknowable. However, in-side the earth's cocoon is also the unknown, and in-side man's cocoon the unknown is the emanationsuntouched by awareness. When the glow of awarenesstouches them, they become active and can be alignedwith the corresponding emanations at large. Once thathappens the unknown is perceived and becomes theknown.

"I'm too dumb, don Juan. You have to break it intosmaller pieces for me," I said.

"Genaro is going to break it up for you," don Juanretorted.

Genaro stood up and started doing the same gait ofpower that he had done before, when he circled anenormous flat rock in a corn field by his house, whiledon Juan had watched in fascination. This time donJuan whispered in my ear that I should try to hearGenaro's movements, especially the movements of histhighs as they went up against his chest every time hestepped.

I followed Genaro's movements with my eyes. In afew seconds I felt that some part of me had gottentrapped in Genaro's legs. The movement of his thighswould not let me go. I felt as if I were walking withhim. I was even out of breath. Then I realized that Iwas actually following Genaro. I was in fact walkingwith him, away from the place where we had beensitting.

I did not see don Juan, just Genaro walking aheadof me in the same strange manner. We walked forhours and hours. My fatigue was so intense that I gota terrible headache, and suddenly I got sick. Genarostopped walking and came to my side. There was anintense glare around us, and the light was reflected inGenaro's features. His eyes glowed.

"Don't look at Genaro!" a voice ordered me in myear. "Look around!"

I obeyed. I thought I was in hell! The shock ofseeing the surroundings was so great that I screamedin terror, but there was no sound to my voice. Aroundme was the most vivid picture of all the descriptionsof hell in my Catholic upbringing. I was seeing a red-dish world, hot and oppressive, dark and cavernous,with no sky, no light but the malignant reflections ofreddish lights that kept on moving around us, at greatspeed.

Genaro started to walk again, and something pulledme with him. The force that was making me followGenaro also kept me from looking around. My aware-ness was glued to Genaro's movements.

I saw Genaro plop down as if he were utterly ex-hausted. The instant he touched the ground andstretched himself to rest, something was released inme and I was able again to look around. Don Juan waswatching me inquisitively. I was standing up facinghim. We were at the same place where we had satdown, a wide rocky ledge on top of a small mountain.Genaro was panting and wheezing, and so was I. I wascovered with perspiration. My hair was dripping wet.My clothes were soaked, as if I had been dunked in ariver.

"My God, what's going on!" I exclaimed in utterseriousness and concern.

The exclamation sounded so silly that don Juan andGenaro started to laugh.

"We're trying to make you understand alignment,"Genaro said.

Don Juan gently helped me to sit down. He sat byme.

"Do you remember what happened?" he asked me.

I told him that I did and he insisted that I tell himexactly what I had seen. His request was incongruouswith what he had told me, that the only value of myexperiences was the movement of my assemblagepoint and not the content of my visions.

He explained that Genaro had tried to help me be-fore in very much the same fashion as he had justdone, but that I could never remember anything. Hesaid that Genaro had guided my assemblage point thistime, as he had done before, to assemble a world withanother of the great bands of emanations.

There was a long silence. I was numb, shocked, yetmy awareness was as keen as it had ever been. Ithought I had finally understood what alignment was.Something inside me, which I had been activatingwithout knowing how, gave me the certainty that I hadcomprehended a great truth.

"I think you're beginning to gather your own mo-mentum," don Juan said to me. "Let's go home.You've had enough for one day."

"Oh, come on," Genaro said. "He's stronger thana bull. He's got to be pushed a little further."

"No!" don Juan said emphatically. "We've got tosave his strength. He's only got so much of it."

Genaro insisted that we stay. He looked at me andwinked.

"Look," he said to me, pointing to the easternrange of mountains. "The sun has hardly moved aninch over those mountains and yet you plodded in hellfor hours and hours. Don't you find that overwhelm-ing?"

"Don't scare him unnecessarily!" don Juan pro-tested almost vehemently.

It was then that I saw their maneuvers. At that mo-ment the voice of seeing told me that don Juan andGenaro were a team of superb stalkers playing withme. It was don Juan who always pushed me beyondmy limits, but he always let Genaro be the heavy. Thatday at Genaro's house, when I reached a dangerousstate of hysterical fright as Genaro questioned donJuan whether I should be pushed, and don Juan as-sured me that Genaro was enjoying himself at my ex-pense, Genaro was actually worrying about me.

My seeing was so shocking to me that I began tolaugh. Both don Juan and Genaro looked at me withsurprise. Then don Juan seemed to realize at oncewhat was going through my mind. He told Genaro,and both of them laughed like children.

"You're coming of age," don Juan said to me."Right on time; you're neither too stupid nor toobright. Just like me. You're not like me in your aber-rations. There you are more like the nagual Julian,except that he was brilliant."

He stood up and stretched his back. He looked atme with the most piercing, ferocious eyes I had everseen. I stood up.

"A nagual never lets anyone know that he is incharge," he said to me. "A nagual comes and goeswithout leaving a trace. That freedom is what makeshim a nagual."

His eyes glared for an instant, and then they werecovered by a cloud of mellowness, kindness, human-ness, and they were again don Juan's eyes.

I could hardly keep my balance. I was swooninghelplessly. Genaro jumped to my side and helped meto sit down. Both of them sat down flanking me.

"You are going to catch a boost from the earth,"don Juan said to me in one ear.

"Think about the nagual's eyes," Genaro said tome in the other.

"The boost will come at the moment you see theglitter on the top of that mountain," don Juan said andpointed to the highest peak on the eastern range.

"You'll never see the nagual's eyes again," Genarowhispered.

"Go with the boost wherever it takes you," donJuan said.

"If you think of the nagual's eyes, you'll realize thatthere are two sides to a coin," Genaro whispered.

I wanted to think about what both of them weresaying, but my thoughts did not obey me. Somethingwas pressing down on me. I felt I was shrinking. I hada sensation of nausea. I saw the evening shadows ad-vancing rapidly up the sides of those eastern moun-tains. I had the feeling that I was running after them.

"Here we go," Genaro said in my ear.

"Watch the big peak, watch the glitter," don Juansaid in my other ear.

There was indeed a point of intense brilliance wheredon Juan had pointed, on the highest peak of therange. I watched the last ray of sunlight being reflectedon it. I felt a hole in the pit of my stomach, just as if Iwere on a roller coaster.

I felt, rather than heard, a faraway earthquake rum-ble which abruptly overtook me. The seismic waveswere so loud and so enormous that they lost all mean-ing for me. I was an insignificant microbe beingtwisted and twirled.

The motion slowed down by degrees. There was onemore jolt before everything came to a halt. I tried tolook around. I had no point of reference. I seemed tobe planted, like a tree. Above me there was a white,shiny, inconceivably big dome. Its presence made mefeel elated. I flew toward it, or rather I was ejectedlike a projectile. I had the sensation of being comfort-able, nurtured, secure; the closer I got to the dome,the more intense those feelings became. They finallyoverwhelmed me and I lost all sense of myself.

The next thing I knew, I was rocking slowly in theair like a leaf that falls. I felt exhausted. A suctionforce started to pull me. I went through a dark holeand then I was with don Juan and Genaro.

The next day don Juan, Genaro, and I went to Oax-aca. While don Juan and I strolled around the mainsquare, in the later afternoon, he suddenly started totalk about what we had done the day before. He askedme if I had understood what he was referring to whenhe said that the old seers had stumbled onto somethingmonumental.

I told him that I did, but that I couldn't explain it inwords.

"And what do you think was the main thing wewanted you to find out on top of that mountain?" heasked.

"Alignment," a voice said in my ear, at the sametime I said it myself.

I turned around in a reflex action and bumped intoGenaro, who was just behind me, walking in mytracks. The speed of my movement startled him. Hebroke into a giggle and then embraced me.

We sat down. Don Juan said that there were veryfew things that he could say about the boost I hadgotten from the earth, that warriors are always alonein such cases, and true realizations come much later,after years of struggle.

I told don Juan that my problem in understandingwas magnified by the fact that he and Genaro weredoing all the work. I was simply a passive subject whocould only react to their maneuvers. I could not forthe life of me initiate any action, because I did notknow what a proper action should be, nor did I knowhow to initiate it.

"That's precisely the point," don Juan said. "Youare not supposed to know yet. You are going to be leftbehind, by yourself, to reorganize on your own every-thing we are doing to you now. This is the task everynagual has to face.

"The nagual Julian did the same thing to me, muchmore ruthlessly than the way we do it to you. He knewwhat he was doing; he was a brilliant nagual who wasable to reorganize in a few years everything the nagualEllas had taught him. He did, in no time at all, some-thing that would take a lifetime for you or for me. Thedifference was that all the nagual Julian ever neededwas a slight insinuation; his awareness would take itfrom there and open the only door there is."

"What do you mean, don Juan, by the only doorthere is?"

"I mean that when man's assemblage point movesbeyond a crucial limit, the results are always the samefor every man. The techniques to make it move maybe as different as they can be, but the results are al-ways the same, meaning that the assemblage pointassembles other worlds, aided by the boost from theearth."

"Is the boost from the earth the same for everyman, don Juan?"

"Of course. The difficulty for the average man isthe internal dialogue. Only when a state of total si-lence is attained can one use the boost. You will cor-roborate that truth the day you try to use that boostby yourself."

"I wouldn't recommend that you try it," Genarosaid sincerely. "It takes years to become an impecca-ble warrior. In order to withstand the impact of theearth's boost you must be better than you are now."

"The speed of that boost will dissolve everythingabout you," don Juan said. "Under its impact we be-come nothing. Speed and the sense of individual exis-tence don't go together. Yesterday on the mountain,Genaro and I supported you and served as your an-chors; otherwise you wouldn't have returned. You'dbe like some men who purposely used that boost andwent into the unknown and are still roaming in someincomprehensible immensity."

I wanted him to elaborate on that, but he refused.He changed the subject abruptly.

"There's one thing you haven't understood yetabout the earth's being a sentient being," he said."And Genaro, this awful Genaro, wants to push youuntil you understand."

Both of them laughed. Genaro playfully shoved meand winked at me as he mouthed the words, "I amawful."

"Genaro is a terrible taskmaster, mean and ruth-less," don Juan continued. "He doesn't give a hootabout your fears and pushes you mercilessly. If itwasn't for me. . ."

He was a perfect picture of a good, thoughtful oldgentleman. He lowered his eyes and sighed. The twoof them broke into roaring laughter.

When they had quieted down, don Juan said thatGenaro wanted to show me what I had not understoodyet, that the supreme awareness of the earth is whatmakes it possible for us to change into other greatbands of emanations.

"We living beings are perceivers," he said. "Andwe perceive because some emanations inside man'scocoon become aligned with some emanations out-side. Alignment, therefore, is the secret passageway,and the earth's boost is the key.

"Genaro wants you to watch the moment of align-ment. Watch him!"

Genaro stood up like a showman and took a bow,then showed us that he had nothing up his sleeves orinside the legs of his pants. He took his shoes off andshook them to show that there was nothing concealedthere either.

Don Juan was laughing with total abandon. Genaromoved his hands up and down. The movement createdan immediate fixation in me. I sensed that the three ofus suddenly got up and walked away from the square,the two of them flanking me.

As we continued walking, I lost my peripheral vi-sion. I did not distinguish any more houses or streets.I did not notice any mountains or any vegetationeither. At one moment I realized that I had lost sightof don Juan and Genaro; instead I saw two luminousbundles moving up and down beside me.

I felt an instantaneous panic, which I immediatelycontrolled. I had the unusual but well-known sensa-tion that I was myself and yet I was not. I was aware,however, of everything around me by means of astrange and at the same time most familiar capacity.The sight of the world came to me all at once. All ofme saw; the entirety of what I in my normal con-sciousness call my body was capable of sensing as ifit were an enormous eye that detected everything.What I first detected, after seeing the two blobs oflight, was a sharp violet-purple world made out ofsomething that looked like colored panels and cano-pies. Flat, screenlike panels of irregular concentriccircles were everywhere.

I felt a great pressure all over me, and then I hearda voice in my ear. I was seeing. The voice said thatthe pressure was due to the act of moving. I was mov-ing together with don Juan and Genaro. I felt a faintjolt, as if I had broken a paper barrier, and I foundmyself facing a luminescent world. Light radiatedfrom everyplace, but without being glaring. It was asif the sun were about to erupt from behind some whitediaphanous clouds. I was looking down into the sourceof light. It was a beautiful sight. There were no land-masses, just fluffy white clouds and light. And wewere walking on the clouds.

Then something imprisoned me again. I moved atthe same pace as the two blobs of light by my sides.Gradually they began to lose their brilliance, then be-came opaque, and finally they were don Juan and Ge-naro. We were walking on a deserted side street awayfrom the main square. Then we turned back.

"Genaro just helped you to align your emanationswith those emanations at large that belong to anotherband," don Juan said to me. "Alignment has to be avery peaceful, unnoticeable act. No flying away, nogreat fuss."

He said that the sobriety needed to let the assem-blage point assemble other worlds is something thatcannot be improvised. Sobriety has to mature and be-come a force in itself before warriors can break thebarrier of perception with impunity.

We were coming closer to the main square. Genarohad not said a word. He walked in silence, as if ab-sorbed in thought. Just before we came into thesquare, don Juan said that Genaro wanted to show meone more thing: that the position of the assemblagepoint is everything, and that the world it makes usperceive is so real that it does not leave room foranything except realness.

"Genaro will let his assemblage point assemble an-other world just for your benefit," don Juan said tome. "And then you'll realize that as he perceives it,the force of his perception will leave room for nothingelse."

Genaro walked ahead of us, and don Juan orderedme to roll my eyes in a counterclockwise directionwhile I looked at Genaro, to avoid being dragged withhim. I obeyed him. Genaro was five or six feet awayfrom me. Suddenly his shape became diffuse and inone instant he was gone like a puff of air.

I thought of the science fiction movies I had seenand wondered whether we are subliminally aware ofour possibilities.

"Genaro is separated from us at this moment by theforce of perception," don Juan said quietly. "Whenthe assemblage point assembles a world, that world istotal. This is the marvel that the old seers stumbledupon and never realized what it was: the awareness ofthe earth can give us a boost to align other great bandsof emanations, and the force of that new alignmentmakes the world vanish.

"Every time the old seers made a new alignmentthey believed they had descended to the depths'or ascended to the heavens above. They never knewthat the world disappears like a puff of air when a newtotal alignment makes us perceive another totalworld."

14The Rolling Force

Don Juan was about to start his explanation of themastery of awareness, but he changed his mind andstood up. We had been sitting in the big room, observ-ing a moment of quiet.

"I want you to try seeing the Eagle's emanations,"he said. "For that you must first move your assem-blage point until you see the cocoon of man."

We walked from the house to the center of town.We sat down on art empty, worn park bench in frontof the church, it was early afternoon; a sunny, windyday with lots of people milling around.

He repeated, as if he were trying to drill it into me,that alignment is a unique force because it either helpsthe assemblage point shift, or it keeps it glued to itscustomary position. The aspect of alignment thatkeeps the point stationary, he said, is will; and theaspect that makes it shift is intent. He remarked thatone of the most haunting mysteries is how will, theimpersonal force of alignment, changes into intent, thepersonalized force, which is at the service of eachindividual.

"The strangest part of this mystery is that thechange is so easy to accomplish," he went on. "Butwhat is not so easy is to convince ourselves that it ispossible. There, right there, is our safety catch. Wehave to be convinced. And none of us wants to be."

He told me then that I was in my keenest state ofawareness, and that it was possible for me to infendmy assemblage point to shift deeper into my left side,to a dreaming position. He said that warriors shouldnever attempt seeing unless they are aided by dream-ing. I argued that to fall asleep in public was not oneof my fortes. He clarified his statement, saying that tomove the assemblage point away from its natural set-ting and to keep it fixed at a new location is to beasleep; with practice, seers learn to be asleep and yetbehave as if nothing is happening to them.

After a moment's pause he added that for purposesof seeing the cocoon of man, one has to gaze at peoplefrom behind, as they walk away. It is useless to gazeat people face to face, because the front of the egglikecocoon of man has a protective shield, which seerscall the front plate, it is an almost impregnable, un-yielding shield that protects us throughout our livesagainst the onslaught of a peculiar force that stemsfrom the emanations themselves.

He also told me not to be surprised if my body wasstiff, as though it were frozen; he said that I was goingto feel very much like someone standing in the middleof a room looking at the street through a window, andthat speed was of the essence, as people were going tomove extremely fast by my seeing window. He toldme then to relax my muscles, shut off my internaldialogue, and let my assemblage point drift awayunder the spell of inner silence. He urged me to smackmyself gently but firmly on my right side, between myhipbone and my ribcage.

I did that three times and I was sound asleep. It wasa most peculiar state of sleep. My body was dormant,but I was perfectly aware of everything that was tak-ing place. I could hear don Juan talking to me and Icould follow every one of his statements as if I wereawake, yet I could not move my body at all.

Don Juan said that a man was going to walk by myseeing window and that I should try to see him. Iunsuccessfully attempted to move my head and then ashiny egglike shape appeared, it was resplendent. Iwas awed by the sight and before I could recover frommy surprise, it was gone. It floated away, bobbing upand down.

Everything had been so sudden and fast that it mademe feel frustrated and impatient. I felt that I was be-ginning to wake up. Don Juan talked to me again andurged me to relax. He said that I had no right and notime to be impatient. Suddenly, another luminousbeing appeared and moved away. It seemed to bemade of a white fluorescent shag.

Don Juan whispered in my ear that if I wanted to,my eyes were capable of slowing down everythingthey focused on. Then he warned me that another manwas coming. I realized at that instant that there weretwo voices. The one I had just heard was the sameone that had admonished me to be patient. That wasdon Juan's. The other, the one that told me to use myeyes to slow down movement, was the voice of seeing.

That afternoon, I saw ten luminous beings in slowmotion. The voice of seeing guided me to witness inthem everything don Juan had told me about the glowof awareness. There was a vertical band with astronger amber glow on the right side of those egglikeluminous creatures, perhaps one-tenth of the total vol-ume of the cocoon. The voice said that that was man'sband of awareness. The voice pointed out a dot onman's band, a dot with an intense shine; it was highon the oblong shapes, almost on the crest of them, onthe surface of the cocoon; the voice said that it wasthe assemblage point.

When I saw each luminous creature in profile, fromthe point of view of its body, its egglike shape was likea gigantic asymmetrical yoyo that was standing edge-wise, or like an almost round pot that was resting onits side with its lid on. The part that looked like a lidwas the front plate; it was perhaps one-fifth the thick-ness of the total cocoon.

I would have gone on seeing those creatures, butdon Juan said that I should now gaze at people face toface and sustain my gaze until I had broken the barrierand I was seeing the emanations.

I followed his command. Almost instantaneously, Isaw a most brilliant array of live, compelling fibers oflight. It was a dazzling sight that immediately shat-tered my balance. I fell down on the cement walk onmy side. From there, I saw the compelling fibers oflight multiply themselves. They burst open and myri-ads of other fibers came out of them. But the fibers,compelling as they were, somehow did not interferewith my ordinary view. There were scores of peoplegoing into church. I was no longer seeing them. Therewere quite a few women and men just around thebench. I wanted to focus my eyes on them, but insteadI noticed how one of those fibers of light bulged sud-denly. It became like a ball of fire that was perhapsseven feet in diameter, it rolled on me. My first im-pulse was to roll out of its way. Before I could evenmove a muscle the ball had hit me. I felt it as clearlyas if someone had punched me gently in the stomach.An instant later another ball of fire hit me, this timewith considerably more strength, and then don Juanwhacked me really hard on the cheek with his openhand. I jumped up involuntarily and lost sight of thefibers of light and the balloons that were hitting me.

Don Juan said that I had successfully endured myfirst brief encounter with the Eagle's emanations, butthat a couple of shoves from the tumbler had danger-ously opened up my gap. He added that the balls thathad hit me were called the rolling force, or the tum-bler.

We had returned to his house, although I did notremember how or when. ! had spent hours in a sort ofsemisleeping state. Don Juan and the other seers ofhis group had given me large amounts of water todrink. They had also submerged me in an ice-cold tubof water for short periods of time.

"Were those fibers I saw the Eagle's emanations?"I asked don Juan.

"Yes. But you didn't really see them," he replied."No sooner had you begun to see than the tumblerstopped you. If you had remained a moment longer itwould have blasted you."

"What exactly is the tumbler?" I asked.

"It is a force from the Eagle's emanations," he said."A ceaseless force that strikes us every instant of ourlives, it is lethal when seen, but otherwise we areoblivious to it, in our ordinary lives, because we haveprotective shields. We have consuming interests thatengage all our awareness. We are permanently wor-ried about our station, our possessions. These shields,however, do not keep the tumbler away, they simplykeep us from seeing it directly, protecting us in thisway from getting hurt by the fright of seeing the ballsof fire hitting us. Shields are a great help and a greathindrance to us. They pacify us and at the same timefool us. They give us a false sense of security."

He warned me that a moment would come in my lifewhen I would be without any shields, uninterruptedlyat the mercy of the tumbler. He said that it is an oblig-atory stage in the life of a warrior, known as losing thehuman form.

I asked him to explain to me once and for all whatthe human form is and what it means to lose it.

He replied that seers describe the human form asthe compelling force of alignment of the emanationslit by the glow of awareness on the precise spot onwhich normally man's assemblage point is fixated. Itis the force that makes us into persons. Thus, to be aperson is to be compelled to affiliate with that force ofalignment and consequently to be affiliated with theprecise spot where it originates.

By reason of their activities, at a given moment theassemblage points of warriors drift toward the left. Itis a permanent move, which results in an uncommonsense of aloofness, or control, or even abandon. Thatdrift of the assemblage point entails a new alignmentof emanations. It is the beginning of a series of greatershifts. Seers very aptly called this initial shift losingthe human form, because it marks an inexorablemovement of the assemblage point away from its orig-inal setting, resulting in the irreversible loss of ouraffiliation to the force that makes us persons.

He asked me then to describe all the details I couldremember about the balls of fire. I told him that I hadseen them so briefly I was not sure I could describethem in detail.

He pointed out that seeing is a euphemism for mov-ing the assemblage point, and that if I moved mine afraction more to the left I would have a clear pictureof the balls of fire, a picture which I could interpretthen as having remembered them.

I tried to have a clear picture, but I couldn't, so Idescribed what I remembered.

He listened attentively and then urged me to recallif they were balls or circles of fire. I told him I didn'tremember.

He explained that those balls of fire are of crucialimportance to human beings because they are theexpression of a force that pertains to all details of lifeand death, something that the new seers call the roll-ing force.

I asked him to clarify what he meant by all the de-tails of life and death.

"The rolling force is the means through which theEagle distributes life and awareness for safekeeping,"he said. "But it also is the force that, let's say, collectsthe rent. It makes all living beings die. What you sawtoday was called by the ancient seers the tumbler."

He said that seers describe it as an eternal line ofiridescent rings, or balls of fire, that roll onto livingbeings ceaselessly. Luminous organic beings meet therolling force head on, until the day when the forceproves to be too much for them and the creaturesfinally collapse. The old seers were mesmerized byseeing how the tumbler then tumbles them into thebeak of the Eagle to be devoured. That was the reasonthey called it the tumbler.

"You said that it is a mesmerizing sight. Have youyourself seen it rolling human beings?" I asked.

"Certainly I've seen it," he replied, and after apause he added, "You and I saw it only a short whileago in Mexico City."

His assertion was so farfetched that I felt obliged totell him that this time he was wrong. He laughed andreminded me that on that occasion, while both of uswere sitting on a bench in the Alameda Park in MexicoCity, we had witnessed the death of a man. He saidthat I had recorded the event in my everyday-lifememory as well as in my left-side emanations.

As don Juan spoke to me I had the sensation ofsomething inside me becoming lucid by degrees, and Icould visualize with uncanny clarity the whole scenein the park. The man was lying on the grass with threepolicemen standing by him to keep onlookers away. Idistinctly remembered don Juan hitting me on myback to make me change levels of awareness. Andthen I saw. My seeing was imperfect. I was unable toshake off the sight of the world of everyday life. WhatI ended up with was a composite of filaments of themost gorgeous colors superimposed on the buildingsand the traffic. The filaments were actually lines ofcolored light that came from above. They had innerlife; they were bright and bursting with energy.

When I looked at the dying man, I saw what donJuan was talking about; something that was at oncelike circles of fire, or iridescent tumbleweeds, wasrolling everywhere I focused my eyes. The circleswere rolling on people, on don Juan, on me. I felt themin my stomach and became ill.

Don Juan told me to focus my eyes on the dyingman. I saw him at one moment curling up, just as asowbug curls itself up upon being touched. The incan-descent circles pushed him away, as if they were cast-ing him aside, out of their majestic, inalterable path.

I had not liked the feeling. The circles of fire hadnot scared me; they were not awesome, or sinister. Idid not feel morbid or somber. The circles rather hadnauseated me. I'd felt them in the pit of my stomach.It was a revulsion that I'd felt that day.

Remembering them conjured up again the total feel-ing of discomfort I had experienced on that occasion.As I got ill, don Juan laughed until he was out ofbreath.

"You're such an exaggerated fellow." he said."The rolling force is not that bad. It's lovely, in fact.The new seers recommend that we open ourselves toit. The old seers also opened themselves to it, but forreasons and purposes guided mostly by self-impor-tance and obsession.

"The new seers, on the other hand, make friendswith it. They become familiar with that force by han-dling it without any self-importance. The result is stag-gering in its consequences."

He said that a shift of the assemblage point is allthat is needed to open oneself to the rolling force. Headded that if the force is seen in a deliberate manner,there is minimal danger. A situation that is extremelydangerous, however, is an involuntary shift of the as-semblage point owing, perhaps, to physical fatigue,emotional exhaustion, disease, or simply a minor emo-tional or physical crisis, such as being frightened orbeing drunk.

"When the assemblage point shifts involuntarily,the rolling force cracks the cocoon," he went on."I've talked many times about a gap that man hasbelow his navel. It's not really below the navel itself,but in the cocoon, at the height of the navel. The gapis more like a dent, a natural flaw in the otherwisesmooth cocoon. It is there where the tumbler hits usceaselessly and where the cocoon cracks."

He went on to explain that if it is a minor shift ofthe assemblage point, the crack is very small, the co-coon quickly repairs itself, and people experiencewhat everybody has at one time or another: blotchesof color and contorted shapes, which remain even ifthe eyes are closed.

If the shift is considerable, the crack also is exten-sive and it takes time for the cocoon to repair itself, asin the case of warriors who purposely use powerplants to elicit that shift or people who take drugs andunwittingly do the same. In these cases men feel numband cold; they have difficulty talking or even thinking;it is as if they have been frozen from inside.

Don Juan said that in cases in which the assemblagepoint shifts drastically because of the effects of traumaor of a mortal disease, the rolling force produces acrack the length of the cocoon; the cocoon collapsesand curls in on itself, and the individual dies.

"Can a voluntary shift also produce a gap of thatnature?" I asked.

"Sometimes," he replied. "We're really frail. Asthe tumbler hits us over and over, death comes to usthrough the gap. Death is the rolling force. When itfinds weakness in the gap of a luminous being it auto-matically cracks it open and makes it collapse."

"Does every living being have a gap?" I asked.

"Of course," he replied. "If it didn't have one itwouldn't die. The gaps are different, however, in sizeand configuration. Man's gap is a bowl-like depressionthe size of a fist, a very frail vulnerable configuration.The gaps of other organic creatures are very much likeman's; some are stronger than ours and others areweaker. But the gap of inorganic beings is really dif-ferent. It's more like a long thread, a hair of luminos-ity; consequently, inorganic beings are infinitely moredurable than we are.

"There is something hauntingly appealing about thelong life of those creatures, and the old seers couldnot resist being carried away by that appeal."

He said that the same force can produce two effectsthat are diametrically opposed. The old seers wereimprisoned by the rolling force, and the new seers arerewarded for their toils with the gift of freedom. Bybecoming familiar with the rolling force through themastery of intent, the new seers, at a given moment,open their own cocoons and the force floods themrather than rolling them up like a curled-up sowbug.The final result is their total and instantaneous disin-tegration.

I asked him a lot of questions about the survival ofawareness after the luminous being is consumed bythe fire from within. He did not answer. He simplychuckled, shrugged his shoulders, and went on to saythat the old seers' obsession with the tumbler blindedthem to the other side of that force. The new seers,with their usual thoroughness in refusing tradition,went to the other extreme. They were at first totallyaverse to focusing their seeing on the tumbler; theyargued that they needed to understand the force of theemanations at large in its aspect of life-giver and en-hancer of awareness.

"They realized that it is infinitely easier to destroysomething," don Juan went on, "than it is to build itand maintain it. To roll life away is nothing comparedto giving it and nourishing it. Of course, the new seerswere wrong on this count, but in due course they cor-rected their mistake."

"How were they wrong, don Juan?"

"It's an error to isolate anything for seeing. At thebeginning, the new seers did exactly the opposite fromwhat their predecessors did. They focused with equalattention on the other side of the tumbler. What hap-pened to them was as terrible as, if not worse than,what happened to the old seers. They died stupiddeaths, just as the average man does. They didn't havethe mystery or the malignancy of the ancient seers,nor had they the quest for freedom of the seers oftoday.

"Those first new seers served everybody. Becausethey were focusing their seeing on the life-giving sideof the emanations, they were filled with love and kind-ness. But that didn't keep them from being tumbled.They were vulnerable, just as were the old seers whowere filled with morbidity."

He said that for the modern-day new seers, to beleft stranded after a life of discipline and toil, just likemen who have never had a purposeful moment in theirlives, was intolerable.

Don Juan said that these new seers realized, afterthey had readopted their tradition, that the old seers'knowledge of the rolling force had been complete; atone point the old seers had concluded that there were,in effect, two different aspects of the same force. Thetumbling aspect relates exclusively to destruction anddeath. The circular aspect, on the other hand, is whatmaintains life and awareness, fulfillment and purpose.They had chosen, however, to deal exclusively withthe tumbling aspect.

"Gazing in teams, the new seers were able to seethe separation between the tumbling and the circularaspects," he explained. "They saw that both forcesare fused, but are not the same. The circular forcecomes to us just before the tumbling force; they are soclose to each other that they seem the same.

"The reason it's called the circular force is that itcomes in rings, threadlike hoops of iridescence?avery delicate affair indeed. And just like the tumblingforce, it strikes all living beings ceaselessly, but for adifferent purpose. It strikes them to give themstrength, direction, awareness; to give them life.

"What the new seers discovered is that the balanceof the two forces in every living being is a very deli-cate one," he continued, "if at any given time anindividual feels that the tumbling force strikes harderthan the circular one, that means the balance is upset;the tumbling force strikes harder and harder from thenon, until it cracks the living being's gap and makes itdie."

He added that out of what I had called balls of firecomes an iridescent hoop exactly the size of livingbeings, whether men, trees, microbes, or allies.

"Are there different-size circles?" I asked.

"Don't take me so literally," he protested. "Thereare no circles to speak of, just a circular force thatgives seers, who are dreaming it, the feeling of rings.And there are no different sizes either. It's one indi-visible force that fits all living beings, organic and in-organic."

"Why did the old seers focus on the tumbling as-pect?" I asked.

"Because they believed that their lives depended onseeing it," he replied. "They were sure that theirseeing was going to give them answers to age-oldquestions. You see, they figured that if they unraveledthe secrets of the rolling force they would be invulner-able and immortal. The sad part is that in one way oranother, they did unravel the secrets and yet theywere neither invulnerable nor immortal.

"The new seers changed it all by realizing that thereis no way to aspire to immortality as long as man hasa cocoon."

Don Juan explained that the old seers apparentlynever realized that the human cocoon is a receptacleand cannot sustain the onslaught of the rolling forceforever. In spite of all the knowledge that they hadaccumulated, they were in the end certainly no better,and perhaps much worse, off than the average man.

"In what way were they left worse off than the av-erage man?" I asked.

"Their tremendous knowledge forced them to takeit for granted that their choices were infallible," hesaid. "So they chose to live at any cost."

Don Juan looked at me and smiled. With his theat-rical pause he was telling me something I could notfathom.

"They chose to live," he repeated. "Just as theychose to become trees in order to assemble worldswith those nearly unreachable great bands."

"What do you mean by that, don Juan?"

"I mean that they used the rolling force to shift theirassemblage points to unimaginable dreaming posi-tions, instead of letting it roll them to the beak of theEagle to be devoured."

15The Death Defiers

I arrived at Genaro's house around 2: 00 p. m. Don Juanand I became involved in conversation, and then donJuan made me shift into heightened awareness.

"Here we are again, the three of us, just as we werethe day we went to that flat rock," don Juan said."And tonight we're going to make another trip to thatarea.

"You have enough knowledge now to draw veryserious conclusions about that place and its effects onawareness."

"What is it with that place, don Juan?"

"Tonight you're going to find out some gruesomefacts that the old seers collected about the rollingforce; and you're going to see what I meant when Itold you that the old seers chose to live at any cost."

Don Juan turned to Genaro, who was about to fallasleep. He nudged him.

"Wouldn't you say, Genaro, that the old seers-weredreadful men?" don Juan asked.

"Absolutely," Genaro said in a crisp tone and thenseemed to succumb to fatigue.

He began to nod noticeably. In an instant he wassound asleep, his head resting on his chest with hischin tucked in. He snored.

I wanted to laugh out loud. But then I noticed thatGenaro was staring at me, as if he were sleeping withhis eyes open.

"They were such dreadful men that they even de-fied death," Genaro added between snores.

"Aren't you curious to know how those gruesomemen defied death?" don Juan asked me.

He seemed to be urging me to ask for an example oftheir gruesomeness. He paused and looked at me withwhat I thought was a glint of expectation in his eyes.

"You're waiting for me to ask for an example,aren't you?" I said.

"This is a great moment," he said, patting me onthe back and laughing. "My benefactor had me on theedge of my seat at this point. I asked him to give mean example, and he did; now i'm going to give youone whether you ask for it or not."

"What are you going to do?" I asked, so frightenedthat my stomach was tied in knots and my voicecracked.

It took quite a while for don Juan to stop laughing.Every time he started to speak, he'd get an attack ofcoughing laughter.

"As Genaro told you, the old seers were dreadfulmen," he said, rubbing his eyes. "There was some-thing they tried to avoid at all costs: they didn't wantto die. You may say that the average man doesn'twant to die either, but the advantage that the old seershad over the average man was that they had the con-centration and the discipline to intend things away;and they actually intended death away."

He paused and looked at me with raised eyebrows.He said that I was falling behind, that I was not askingmy usual questions. I remarked that it was plain to methat he was leading me to ask if the old seers hadsucceeded in intending death away, but he himself hadalready told me that their knowledge about the tum-bled had not saved them from dying.

"They succeeded in intending death away," hesaid, pronouncing his words with extra care. "Butthey still had to die."

"How did they intend death away?" I asked.

"They observed their allies," he said, "and seeingthat they were living beings with a much greater resi-lience to the rolling force, the seers patterned them-selves on their allies."

The old seers realized, don Juan explained, thatonly organic beings have a gap that resembles a bowl.Its size and shape and its brittleness make it the idealconfiguration to hasten the cracking and collapsing ofthe luminous shell under the onslaughts of the tum-bling force. The allies, on the other hand, who haveonly a line for a gap, present such a small surface tothe rolling force as to be practically immortal. Theircocoons can sustain the onslaughts of the tumbler in-definitely. because hairline gaps offer no ideal config-uration to it.

"The old seers developed the most bizarre tech-niques for closing their gaps," don Juan continued."They were essentially correct in assuming that ahairline gap is more durable than a bowl-like one."

"Are those techniques still in existence?" I asked.

"No, they are not," he said. "But some of the seerswho practiced them are."

For reasons unknown to me, his statement caused areaction of sheer terror in me. My breathing was al-tered instantly, and I couldn't control its rapid pace.

"They're still alive to this day, isn't that so, Ge-naro?" don Juan asked.

"Absolutely," Genaro muttered from an apparentstate of deep sleep.

I asked don Juan if he knew the reason for my beingso frightened. He reminded me about a previous oc-casion in that very room when they had asked me if Ihad noticed the weird creatures that had come in themoment Genaro opened the door.

"That day your assemblage point went very deepinto the left side and assembled a frightening world,"he went on. "But I have already said that to you; whatyou don't remember is that you went directly to a veryremote world and scared yourself pissless there."

Don Juan turned to Genaro, who was snoring peace-fully with his legs stretched out in front of him.

"Wasn't he scared pissless, Genaro?" he asked.

"Absolutely pissless," Genaro muttered, and donJuan laughed.

"I want you to know that we don't blame you forbeing scared," don Juan continued. "We, ourselves,are revolted by some of the actions of the old seers.I'm sure that you have realized by now that what youcan't remember about that night is that you saw theold seers who are still alive."

I wanted to protest that I had realized nothing, butI could not voice my words. I had to clear my throatover and over before I could articulate a word. Genarohad stood up and was gently patting my upper back,by my neck, as if I were choking.

"You have a frog in your throat," he said.

I thanked him in a high squeaky voice.

"No, I think you have a chicken there," he addedand sat down to sleep.

Don Juan said that the new seers had rebelledagainst all the bizarre practices of the old seers anddeclared them not only useless but injurious to ourtotal being. They even went so far as to ban thosetechniques from whatever was taught to new warriors;and for generations there was no mention of thosepractices at all.

It was in the early part of the eighteenth centurythat the nagual Sebastian, a member of don Juan'sdirect line of naguals, rediscovered the existence ofthose techniques.

"How did he rediscover them?" I asked.

"He was a superb stalker, and because of his im-peccability he got a chance to learn marvels," donJuan replied.

He said that one day as the nagual Sebastian wasabout to start his daily routines?he was the sexton atthe cathedral in the city where he lived?he found amiddle-aged Indian man who seemed to be in a quan-dary at the door of the church.

The nagual Sebastian went to the man's side andasked him if he needed help. "I need a bit of energyto close my gap," the man said to him in a loud clearvoice. "Would you give me some of your energy?"

Don Juan said that according to the story, the na-gual Sebastian was dumbfounded. He did not knowwhat the man was talking about. He offered to takethe Indian to see the parish priest. The man lost hispatience and angrily accused the nagual Sebastian ofstalling. "I need your energy because you're a na-gual," he said. "Let's go quietly."

The nagual Sebastian succumbed to the magneticpower of the stranger and meekly went with him intothe mountains. He was gone for many days. When hecame back he not only had a new outlook about theancient seers, but detailed knowledge of their tech-niques. The stranger was an ancient Toltec. One ofthe last survivors.

"The nagual Sebastian found out marvels about theold seers," don Juan went on. "He was the one whofirst knew how grotesque and aberrant they reallywere. Before him, that knowledge was only hearsay.

"One night my benefactor and the nagual Elias gaveme a sample of those aberrations. They really showedit to Genaro and me together, so it's only proper thatwe both show you the same sample."

I wanted to talk in order to stall; I needed time tocalm down, to think things out. But before I could sayanything, don Juan and Genaro were practically drag-ging me out of the house. They headed for the sameeroded hills we had visited before.

We stopped at the bottom of a large barren hill. DonJuan pointed toward some distant mountains to thesouth, and said that between the place where we stoodand a natural cut in one of those mountains, a cut thatlooked like an open mouth, there were at least sevensites where the ancient seers had focused all the powerof their awareness.

Don Juan said that those seers had not only beenknowledgeable and daring but downright successful.He added that his benefactor had showed him andGenaro a site where the old seers, driven by their lovefor life, had buried themselves alive and actually in-tended the rolling force away.

"There is nothing that would catch the eye in thoseplaces," he went on. "The old seers were careful notto leave marks. It is just a landscape. One has to seeto know where those places are."

He said that he did not want to walk to the farawaysites, but would take me to the one that was nearest.I insisted on knowing what we were after. He said thatwe were going to see the buried seers, and that for thatwe had to stay until it got dark under the cover ofsome green bushes. He pointed them out; they wereperhaps half a mile away, up a steep slope.

We reached the patch of bushes and sat down ascomfortably as we could. He began then to explain ina very low voice that in order to get energy from theearth, ancient seers used to bury themselves for pe-riods of time, depending on what they wanted to ac-complish. The more difficult their task, the longertheir burial period.

Don Juan stood up and in a melodramatic wayshowed me a spot a few yards from where we were.

"Two old seers are buried there," he said. "Theyburied themselves about two thousand years ago toescape death, not in the spirit of running away from itbut in the spirit of defy ing it."

Don Juan asked Genaro to show me the exact spotwhere the old seers were buried. I turned to look atGenaro and realized that he was sitting by my sidesound asleep again. But to my utter amazement, hejumped up and barked like a dog and ran on all foursto the spot don Juan was pointing out. There he ranaround the place in a perfect mime of a small dog.

I found his performance hilarious. Don Juan wasnearly on the ground laughing.

"Genaro has shown you something extraordinary,"don Juan said, after Genaro had returned to where wewere and had gone back to sleep. "He has shown yousomething about the assemblage point and dreaming.He's dreaming now, but he can act as if he were fullyawake and he can hear everything you say. From thatposition he can do more than if he were awake."

He was silent for a moment as if assessing what tosay next. Genaro snored rhythmically.

Don Juan remarked how easy it was for him to findflaws with what the old seers had done, yet, in allfairness, he never tired of repeating how wonderfultheir accomplishments were. He said that they under-stood the earth to perfection. Not only did they dis-cover and use the boost from the earth, but they alsodiscovered that if they remained buried, their assem-blage points aligned emanations that were ordinarilyinaccessible, and that such an alignment engaged theearth's strange, inexplicable capacity to deflect theceaseless strikes of the rolling force. Consequently,they developed the most astounding and complextechniques for burying themselves for extremely longperiods of time without any detriment to themselves.In their fight against death, they learned how to elon-gate those periods to cover millennia.

It was a cloudy day, and night fell quickly. In notime at all, everything was in darkness. Don Juanstood up and guided me and the sleepwalker Genaroto an enormous flat oval rock that had caught my eyethe moment we got to that place. It was similar to theflat rock we had visited before, but bigger. It occurredto me that the rock, enormous as it was, had deliber-ately been placed there.

"This is another site," don Juan said. "This hugerock was placed here as a trap, to attract people. Soonyou'll know why."

I felt a shiver run through my body. I thought I wasgoing to faint. I knew that I was definitely overreactingand wanted to say something about it, but don Juankept on talking in a hoarse whisper. He said that Ge-naro, since he was dreaming, had enough control overhis assemblage point to move it until he could reachthe specific emanations that would wake up whateverwas around that rock. He recommended that I try tomove my assemblage point, and follow Genaro's. Hesaid that I could do it, first by setting up my unbendingintent to move it, and second by letting the context ofthe situation dictate where it should move.

After a moment's thought he whispered in my earnot to worry about procedures, because most of thereally unusual things that happen to seers, or to theaverage man for that matter, happen by themselves,with only the intervention of intent.

He was silent for a moment and then added that thedanger for me was going to be the buried seers' inevi-table attempt to scare me to death. He exhorted me tokeep myself calm and not to succumb to fear, butfollow Genaro's movements.

I fought desperately not to be sick. Don Juan pattedme on the back and said that I was an old pro atplaying an innocent bystander. He assured me that Iwas not consciously refusing to let my assemblagepoint move, but that every human being does it auto-matically.

"Something is going to scare the living daylights outof you," he whispered. "Don't give up, because if youdo, you'll die and the old vultures around here aregoing to feast on your energy."

"Let's get out of here," I pleaded. "I really don'tgive a damn about getting an example of the old seers'grotesqueness."

"It's too late," Genaro said, fully awake now,standing by my side. "Even if we try to get away, thetwo seers and their allies on the other spot will cut youdown. They have already made a circle around us.There are as many as sixteen awarenesses focused onyou right now."

"Who are they?" I whispered in Genaro's ear.

"The four seers and their court," he replied."They've been aware of us since we got here."

I wanted to turn tail and run for dear life, but donJuan held my arm and pointed to the sky. I noticedthat a remarkable change in visibility had taken place.Instead of the pitch-black darkness that had prevailed,there was a pleasant dawn twilight. I made a quickassessment of the cardinal points. The sky was defi-nitely lighter toward the east.

I felt a strange pressure around my head. My earswere buzzing. I felt cold and feverish at the same time.I was scared as I had never been before, but whatbothered me was a nagging sensation of defeat, ofbeing a coward. I felt nauseated and miserable.

Don Juan whispered in my ear. He said that I hadto be on the alert, that the onslaught of the old seerswould be felt by all three of us at any moment.

"You can grab on to me if you want to," Genarosaid in a fast whisper as if something were proddinghim.

I hesitated for an instant. I did not want don Juan tobelieve that I was so scared I needed to hold on toGenaro.

"Here they come!" Genaro said in a loud whisper.

The world turned upside down instantaneously forme when something gripped me by my left ankle. I feltthe coldness of death on my entire body. I knew I hadstepped on an iron clamp, maybe a bear trap. That allflashed through my mind before I let out a piercingscream, as intense as my fright.

Don Juan and Genaro laughed out loud. They wereflanking me no more than three feet away, but I wasso terrified I did not even notice them.

"Sing! Sing for dear life!" I heard don Juan orderingme under his breath.

I tried to pull my foot loose. I felt then a sting, as ifneedles were piercing my skin. Don Juan insisted overand over that I sing. He and Genaro started to sing apopular song. Genaro spoke the lyrics as he looked atme from hardly two inches away. They sang off-keyin raspy voices, getting so completely out of breathand so high out of the range of their voices that I endedup laughing.

"Sing, or you're going to perish," don Juan said tome.

"Let's make a trio," Genaro said, "We'll sing abolero."

I joined them in an off-key trio. We sang for quite awhile at the top of our voices, like drunkards. I feltthat the iron grip on my leg was gradually letting go ofme. I had not dared to look down at my ankle. At onemoment I did and I realized then that there was notrap clutching me. A dark, headlike shape was bitingme!

Only a supreme effort kept me from fainting. I felt Iwas getting sick and automatically tried to bend over,but somebody with superhuman strength grabbed mepainlessly by the elbows and the nape of my neck anddid not let me move. I got sick all over my clothes.

My revulsion was so complete that I began to fall ina faint. Don Juan sprinkled my face with some waterfrom the small gourd he always carried when we wentinto the mountains. The water slid under my collar.The coldness restored my physical balance, but it didnot affect the force that was holding me by my elbowsand neck.

"I think you are going too far with your fright," donJuan said loudly and in such a matter-of-fact tone thathe created an immediate feeling of order.

"Let's sing again," he added. "Let's sing a songwith substance?I don't want any more boleros."

I silently thanked him for his sobriety and for hisgrand style. I was so moved as I heard them singing"La Valentina" that I began to weep.

"Because of my passion, they saythat ill fortune is on my way.It doesn't matterthat it might be the devil himself.I do know how to die

Valentina, Valentina.I throw my self in your way.If I am going to die tomorrow,why not, once and for all, today?"

All of my being staggered under the impact of thatinconceivable juxtaposition of values. Never had asong meant so much to me. As I heard them sing thoselyrics, which I ordinarily considered reeking withcheap sentimentalism, I thought I understood theethos of the warrior. Don Juan had drilled into me thatwarriors live with death at their side, and from theknowledge that death is with them they draw the cour-age to face anything. Don Juan had said that the worstthat could happen to us is that we have to die, andsince that is already our unalterable fate, we are free;those who have lost everything no longer have any-thing to fear.

I walked to don Juan and Genaro and embracedthem to express my boundless gratitude and admira-tion for them.

Then I realized that nothing was holding me anylonger. Without a word don Juan took my arm andguided me to sit on the flat rock.

"The show is just about to begin now," Genaro saidin a jovial tone as he tried to find a comfortable posi-tion to sit. "You've just paid your admission ticket.It's all over your chest."

He looked at me, and both of them began to laugh.

"Don't sit too close to me," Genaro said. "I don'tappreciate pukers. But don't go too far, either. Theold seers are not yet through with their tricks."

I moved as close to them as politeness permitted. Iwas concerned about my slate for an instant, and thenall my qualms became nonsense, for I noticed thatsome people were coming toward us. I could not makeout their shapes clearly but I distinguished a mass ofhuman figures moving in the semidarkness. They didnot carry lanterns or flashlights with them, which atthat hour they would still have needed. Somehow thatdetail worried me. I did not want to focus on it and Ideliberately began to think rationally. I figured that wemust have attracted attention with our loud singingand they were coming to investigate. Don Juan tappedme on the shoulder. He pointed with a movement ofhis chin to the men in front of the group of others.

"Those four are the old seers," he said. "The restare their allies."

Before I could remark that they were just local peas-ants, I heard a swishing sound right behind me. Iquickly turned around in a state of total alarm. Mymovement was so sudden that don Juan's warningcame too late.

"Don't turn around!" I heard him yell.

His words were only background; they did not meananything to me. On turning around, I saw that threegrotesquely deformed men had climbed up on the rockright behind me; they were crawling toward me, withtheir mouths open in a nightmarish grimace and theirarms outstretched to grab me.

I intended to scream at the top of my lungs, but whatcame out was an agonizing croak, as if something wereobstructing my windpipe. I automatically rolled out oftheir reach and onto the ground.

As I stood up, don Juan jumped to my side, at thevery same moment that a horde of men, led by thosedon Juan had pointed out, descended on me like vul-tures. They were actually squeaking like bats or rats.I yelled in terror. This time I was able to let out apiercing cry.

Don Juan, as nimbly as an athlete in top form,pulled me out of their clutches onto the rock. He toldme in a stern voice not to turn around to look, nomatter how scared I was. He said that the allies cannotpush at all, but that they certainly could scare me andmake me fall to the ground. On the ground, however,the allies could hold anybody down. If I were to fallon the ground by the place where the seers were bur-ied, I would be at their mercy. They would rip meapart while their allies held me. He added that he hadnot told me all that before because he had hoped Iwould be forced to see and understand it by myself.His decision had nearly cost me my life.

The sensation that the grotesque men were just be-hind me was nearly unbearable. Don Juan forcefullyordered me to keep calm and focus my attention onfour men at the head of a crowd of perhaps ten ortwelve. The instant I focused my eyes on them, as ifon cue, they all advanced to the edge of the flat rock.They stopped there and began hissing like serpents.They walked back and forth. Their movement seemedto be synchronized. It was so consistent and orderlythat it seemed to be mechanical. It was as if they werefollowing a repetitive pattern, aimed at mesmerizingme.

"Don't gaze at them, dear," Genaro said to me asif he were talking to a child.

The laughter that followed was as hysterical as myfear. I laughed so hard that the sound reverberated onthe surrounding hills.

The men stopped at once and seemed to be per-plexed. I could distinguish the shapes of their headsbobbing up and down as if they were talking, deliber-ating among themselves. Then one of them jumpedonto the rock.

"Watch out! That one is a seer!" Genaro ex-claimed.

"What are we going to do?" I shouted.

"We could start singing again," don Juan repliedmatter-of-factly.

My fear reached its apex then. I began to jump upand down and to roar like an animal. The man jumpeddown to the ground.

"Don't pay any more attention to those clowns,"don Juan said. "Let's talk as usual."

He said that we had gone there for my enlighten-ment, and that I was failing miserably. I had to reor-ganize myself. The first thing to do was to realize thatmy assemblage point had moved and was now makingobscure emanations glow. To carry the feelings frommy usual state of awareness into the world I had as-sembled was indeed a travesty, for fear is only preva-lent among the emanations of daily life.

I told him that if my assemblage point had shifted ashe was saying it had, I had news for him. My fear wasinfinitely greater and more devastating than anythingI had ever experienced in my daily life.

"You're wrong," he said. "Your first attention isconfused and doesn't want to give up control, that'sall. I have the feeling that you could walk right up tothose creatures and face them and they wouldn't do athing to you."

I insisted that I was definitely in no condition to testsuch a preposterous thing as that.

He laughed at me. He said that sooner or later I hadto cure myself of my madness, and that to take theinitiative and face up to those four seers was infinitelyless preposterous than the idea that I was seeing themat all. He said that to him madness was to be con-fronted by men who had been buried for two thousandyears and were still alive, and not to think that thatwas the epitome of preposterousness.

I heard everything he said with clarity, but I wasnot really paying attention to him. I was terrified ofthe men around the rock. They seemed to be preparingto jump us, to jump me really. They were fixed on me.My right arm began to shake as if I were stricken bysome muscular disorder. Then I became aware thatthe light in the sky had changed. I had not noticedbefore that it was already dawn. The strange thing wasthat an uncontrollable urge made me stand up and runto the group of men.

I had at that moment two completely different feel-ings about the same event. The minor one was of sheerterror. The other, the major one, was of total indiffer-ence. I could not have cared less.

When I reached the group I realized that don Juanwas right; they were not really men. Only four of themhad any resemblance to men, but they were not meneither; they were strange creatures with huge yelloweyes. The others were just shapes that were propelledby the four that resembled men.

I felt extraordinarily sad for those creatures withyellow eyes. I tried to touch them, but I could not findthem. Some sort of wind scooped them away.

I looked for don Juan and Genaro. They were notthere. It was pitch-black again. I called out theirnames over and over again. I thrashed around in dark-ness for a few minutes. Don Juan came to my side andstartled me. I did not see Genaro.

"Let's go home," he said. "We have a long walk."

Don Juan commented on how well I had performedat the site of the buried seers, especially during thelast part of our encounter with them. He said that ashift of the assemblage point is marked by a change inlight. In the daytime, light becomes very dark; atnight, darkness becomes twilight. He added that I hadperformed two shifts by myself, aided only by animalfright. The only thing he found objectionable was myindulging in fear, especially after I had realized thatwarriors have nothing to fear.

"How do you know I had realized that?" I asked.

"Because you were free. When fear disappears allthe ties that bind us dissolve," he said. "An ally wasgripping your foot because it was attracted by youranimal terror."

I told him how sorry I was for not being able touphold my realizations.

"Don't concern yourself with that." He laughed."You know that such realizations are a dime a dozen;they don't amount to anything in the life of warriors,because they are canceled out as the assemblage pointshifts.

"What Genaro and I wanted to do was to make youshift very deeply. This time Genaro was there simplyto entice the old seers. He did it once already, and youwent so far into the left side that it will take quite awhile for you to remember it. Your fright tonight wasjust as intense as it was that first time when the seersand their allies followed you to this very room, butyour sturdy first attention wouldn't let you be awareof them."

"Explain to me what happened at the site of theseers," I asked.

"The allies came out to see you," he replied."Since they have very low energy, they always needthe help of men. The four seers have collected twelveallies.

"The countryside in Mexico and also certain citiesare dangerous. What happened to you can happen toany man or woman. If they bump into that tomb, theymay even see the seers and their allies, if they arepliable enough to let their fear make their assemblagepoints shift; but one thing is for sure: they can die offright."

"But do you honestly believe that those Toltecseers are still alive?" I asked.

He laughed and shook his head in disbelief.

"It's time for you to shift that assemblage point ofyours just a bit," he said. "I can't talk to you whenyou are in your idiot's stage."

He smacked me with the palm of his hand on threespots: right on the crest of my right hipbone, on thecenter of my back below my shoulder blades, and onthe upper part of my right pectoral muscle.

My ears immediately began to buzz. A trickle ofblood ran out of my right nostril, and something insideme became unplugged. It was as if some flow of en-ergy had been blocked and suddenly began to moveagain.

"What were those seers and their allies after?" Iasked.

"Nothing," he replied. "We were the ones whowere after them. The seers, of course, had alreadynoticed your field of energy the first time you sawthem; when you came back, they were set to feast onyou."

"You claim that they are alive, don Juan," I said."You must mean that they are alive as allies are alive,is that so?"

"That's exactly right," he said. "They cannot pos-sibly be alive as you and I are. That would be prepos-terous."

He went on to explain that the ancient seers' con-cern with death made them look into the most bizarrepossibilities. The ones who opted for the allies' pat-tern had in mind, doubtless, a desire for a haven. Andthey found it, at a fixed position in one of the sevenbands of inorganic awareness. The seers felt that theywere relatively safe there. After all, they were sepa-rated from the daily world by a nearly insurmountablebarrier, the barrier of perception set by the assem-blage point.

"When the four seers saw that you could shift yourassemblage point they took off like bats out of hell,"he said and laughed.

"Do you mean that I assembled one of the sevenworlds?" I asked.

"No, you didn't," he replied. "But you have doneit before, when the seers and their allies chased you.That day you went all the way to their world. Theproblem is that you love to act stupid, so you can'tremember it at all.

"I'm sure that it is the nagual's presence," he con-tinued, "that sometimes makes people act dumb.When the nagual Julian was still around, I was dumberthan I am now. I am convinced that when I'm nolonger here, you'll be capable of remembering every-thing."

Don Juan explained that since he needed to showme the death defiers, he and Genaro had lured them tothe outskirts of our world. What I had done at firstwas a deep lateral shift, which allowed me to see themas people, but at the end I had correctly made the shiftthat allowed me to see the death defiers and their alliesas they are.

Very early the next morning, at Silvio Manuel'shouse, don Juan called me to the big room to discussthe events of the previous night. I felt exhausted andwanted to rest, to sleep, but don Juan was pressed fortime. He immediately started his explanation. He saidthat the old seers had found out a way to utilize therolling force and be propelled by it. Instead of suc-cumbing to the onslaughts of the tumbler they rodewith it and let it move their assemblage points to theconfines of human possibilities.

Don Juan expressed unbiased admiration for suchan accomplishment. He admitted that nothing elsecould give the assemblage point the boost that thetumbler gives.

I asked him about the difference between the earth'sboost and the tumbler's boost. He explained that theearth's boost is the force of alignment of only theamber emanations, it is a boost that heightens aware-ness to unthinkable degrees. To the new seers it is ablast of unlimited consciousness, which they call totalfreedom.

He said that the tumbler's boost, on the other hand,is the force of death. Under the impact of the tumbler,the assemblage point moves to new, unpredictable po-sitions. Thus, the old seers were always alone in theirjourneys, although the enterprise they were involvedin was always communal. The company of other seerson their journeys was fortuitous and usually meantstruggle for supremacy.

I confessed to don Juan that the concerns of the oldseers, whatever they may have been, were worse thanmorbid horror tales to me. He laughed uproariously.He seemed to be enjoying himself.

"You have to admit, no matter how disgusted youfeel, that those devils were very daring," he went on."I never liked them myself, as you know, but I can'thelp admiring them. Their love for life is truly beyondme."

"How can that be love for life, don Juan? It's some-thing nauseating," I said.

"What else could push a man to those extremes if itis not love for life?" he asked. "They loved life sointensely that they were not willing to give it up.That's the way I have seen it. My benefactor sawsomething else. He believed that they were afraid todie, which is not the same as loving life. I say that theywere afraid to die because they loved life and becausethey had seen marvels, and not because they weregreedy little monsters. No. They were aberrant be-cause nobody ever challenged them and they werespoiled like rotten children, but their daring was im-peccable and so was their courage.

"Would you venture into the unknown out of greed?No way. Greed works only in the world of ordinaryaffairs. To venture into that terrifying loneliness onemust have something greater than greed. Love, oneneeds love for life, for intrigue, for mystery. Oneneeds unquenching curiosity and guts galore. So don'tgive me this nonsense about your being revolted. It'sembarrassing!"

Don Juan's eyes were shining with contained laugh-ter. He was putting me in my place, but he was laugh-ing at it.

Don Juan left me alone in the room for perhaps anhour. I wanted to organize my thoughts and feelings.I had no way to do that. I knew without any doubt thatmy assemblage point was at a position where reason-ing does not prevail, yet I was moved by reasonableconcerns. Don Juan had said that technically, as soonas the assemblage point shifts, we are asleep. I won-dered, for instance, if I was sound asleep from thestand of an onlooker, just as Genaro had been asleepto me.

I asked don Juan about it as soon as he returned.

"You are absolutely asleep without having to bestretched out," he replied. "If people in a normal stateof awareness saw you now, you would appear to themto be a bit dizzy, even drunk."

He explained that during normal sleep, the shift ofthe assemblage point runs along either edge of man'sband. Such shifts are always coupled with slumber.Shifts that are induced by practice occur along themidsection of man's band and are not coupled withslumber, yet a dreamer is asleep.

"Right at this juncture is where the new and the oldseers made their separate bids for power," he wenton. "The old seers wanted a replica of the body, butwith more physical strength, so they made their as-semblage points slide along the right edge of man'sband. The deeper they moved along the right edge themore bizarre their dreaming body became. You, your-self, witnessed last night the monstrous result of adeep shift along the right edge."

He said that the new seers were completely differ-ent, that they maintain their assemblage points alongthe midsection of man's band. If the shift is a shallowone, like the shift into heightened awareness, thedreamer is almost like anyone else in the street, ex-cept for a slight vulnerability to emotions, such as fearand doubt. But at a certain degree of depth, thedreamer who is shifting along the midsection becomesa blob of light. A blob of light is the dreaming body ofthe new seers.

He also said that such an impersonal dreaming bodyis more conducive to understanding and examination,which are the basis of all the new seers do. The in-tensely humanized dreaming body of the old seersdrove them to look for answers that were equally per-sonal, humanized.

Don Juan suddenly seemed to be groping for words.

"There is another death defier," he said curtly, "sounlike the four you've seen that he's indistinguishablefrom the average man in the street. He's accomplishedthis unique feat by being able to open and close hisgap whenever he wants."

He played with his fingers almost nervously.

"The ancient seer that the nagual Sebastian foundin 1723 is that death defier," he went on. "We countthat day as the beginning of our line, the second begin-ning. That death defier, who's been on the earth forhundreds of years, has changed the lives of every na-gual he met, some more profoundly than others. Andhe has met every single nagual of our line since thatday in 1723."

Don Juan looked fixedly at me. I got strangely em-barrassed. I thought my embarrassment was the resultof a dilemma. I had very serious doubts about thecontent of the story, and at the same time I had themost disconcerting trust that everything he had saidwas true. I expressed my quandary to him.

"The problem of rational disbelief is not yoursalone," don Juan said. "My benefactor was at firstplagued by the same question. Of course, later on heremembered everything. But it took him a long timeto do so. When I met him he had already recollectedeverything, so I never witnessed his doubts. I onlyheard about them.

"The weird part is that people who have never seteyes on the man have less difficulty accepting that he'sone of the original seers. My benefactor said that hisquandaries stemmed from the fact that the shock ofmeeting such a creature had lumped together a num-ber of emanations. It takes time for those emanationsto separate themselves."

Don Juan went on to explain that as my assemblagepoint kept on shifting, a moment would come when itwould hit the proper combination of emanations; atthat moment the proof of the existence of that manwould become overwhelmingly evident to me.

I felt compelled to talk again about my ambivalence.

"We're deviating from our subject," he said. "Itmay seem that I'm trying to convince you of the exis-tence of that man; and what I meant to talk about isthe fact that the old seer knows how to handle therolling force. Whether or not you believe that he existsis not important. Someday you'll know for a fact thathe certainly succeeded in closing his gap. The energythat he borrows from the nagual every generation heuses exclusively to close his gap."

"How did he succeed in closing it?" I asked.

"There is no way of knowing that," he replied."I've talked to two other naguals who saw that manface to face, the nagual Julian and the nagual Elias.Neither of them knew how. The man never revealedhow he closes that opening, which I suppose begins toexpand after a time. The nagual Sebastian said thatwhen he first saw the old seer, the man was veryweak, actually dying. But my benefactor found himprancing vigorously, like a young man."

Don Juan said that the nagual Sebastian nicknamedthat nameless man "the tenant," for they struck anarrangement by which the man was given energy,lodging so to speak, and he paid rent in the form offavors and knowledge.

"Did anybody ever get hurt in the exchange?" Iasked.

"None of the naguals who exchanged energy withhim was injured," he replied. "The man's commit-ment was that he'd only take a bit of superfluous en-ergy from the nagual in exchange for gifts, forextraordinary abilities. For instance, the nagual Juliangot the gait of power. With it, he could activate ormake dormant the emanations inside his cocoon inorder to look young or old at will."

Don Juan explained that the death defiers in generalwent as far as rendering dormant all the emanationsinside their cocoons, except those that matched theemanations of the allies. In this fashion they were ableto imitate the allies in some form.

Each of the death defiers we had encountered at therock, don Juan said, had been able to move his assem-blage point to a precise spot on his cocoon in order toemphasize the emanations shared with the allies andto interact with them. But they were all unable tomove it back to its usual position and interact withpeople. The tenant, on the other hand, is capable ofshifting his assemblage point to assemble the everydayworld as if nothing had ever happened.

Don Juan also said that his benefactor was con-vinced?and he fully agreed with him?that whattakes place during the borrowing of energy is that theold sorcerer moves the nagual's assemblage point toemphasize the ally's emanations inside the nagual'scocoon. He then uses the great jolt of energy producedby those emanations that suddenly become alignedafter being so deeply dormant.

He said that the energy locked within us, in thedormant emanations, has a tremendous force and anincalculable scope. We can only vaguely assess thescope of that tremendous force, if we consider that theenergy involved in perceiving and acting in the worldof everyday life is a product of the alignment of hardlyone-tenth of the emanations encased in man's cocoon.

"What happens at the moment of death is that allthat energy is released at once," he continued. "Liv-ing beings at that moment become flooded by the mostinconceivable force. It is not the rolling force that hascracked their gaps, because that force never entersinside the cocoon; it only makes it collapse. Whatfloods them is the force of all the emanations that aresuddenly aligned after being dormant for a lifetime.There is no outlet for such a giant force except toescape through the gap."

He added that the old sorcerer has found a way totap that energy. By aligning a limited and very specificspectrum of the dormant emanations inside the na-gual's cocoon, the old seer taps a limited but giganticjolt.

"How do you think he takes that energy into hisown body?" I asked.

"By cracking the nagual's gap," he replied. "Hemoves the nagual's assemblage point until the gapopens a little. When the energy of newly aligned ema-nations is released through that opening, he takes itinto his own gap."

"Why is that old seer doing what he's doing?" Iasked.

"My opinion is that he's caught in a circle he can'tbreak," he replied. "We got into an agreement withhim. He's doing his best to keep it, and so are we. Wecan't judge him, yet we have to know that his pathdoesn't lead to freedom. He knows that, and he alsoknows he can't change it; he's trapped in a situationof his own making. The only thing he can do is toprolong his ally-like existence as long as he possiblycan."

16The Mold of Man

Right after lunch, don Juan and I sat down to talk. Hestarted without any preamble. He announced that wehad come to the end of his explanation. He said thathe had discussed with me, in painstaking detail, all thetruths about awareness that the old seers had discov-ered. He stressed that I now knew the order in whichthe new seers had arranged them. In the last sessionsof his explanation, he said, he had given me a detailedaccount of the two forces that aid our assemblagepoints to move: the earth's boost and the rolling force.He had also explained the three techniques workedout by the new seers?stalking, intent, and dreaming?and their effects on the movement of the assem-blage point.

"Now, the only thing left for you to do before theexplanation of the mastery of awareness is com-pleted," he went on, "is to break the barrier of per-ception by yourself. You must move your assemblagepoint, unaided by anyone, and align another greatband of emanations.

"Not to do this will turn everything you've learnedand done with me into merely talk, just words. Andwords are fairly cheap."

He explained that when the assemblage point ismoving away from its customary position and reachesa certain depth, it breaks a barrier that momentarilydisrupts its capacity to align emanations. We experi-ence it as a moment of perceptual blankness. The oldseers called that moment the wall of fog, because abank of fog appears whenever the alignment of ema-nations falters.

He said that there were three ways of dealing withit. It could be taken abstractly as a barrier of percep-tion; it could be felt as the act of piercing a tight paperscreen with the entire body; or it could be seen as awall of fog.

In the course of my apprenticeship with don Juan,he had guided me countless times to see the barrier ofperception. At first I had liked the idea of a wall offog. Don Juan had warned me that the old seers hadalso preferred to see it that way. He had said that thereis great comfort and ease in seeing it as a wall of fog,but that there is also the grave danger of turning some-thing incomprehensible into something somber andforeboding; hence, his recommendation was to keepincomprehensible things incomprehensible rather thanmaking them part of the inventory of the first atten-tion.

After a short-lived feeling of comfort in seeing thewall of fog I had to agree with don Juan that it wasbetter to keep the transition period as an incompre-hensible abstraction, but by then it was impossible forme to break the fixation of my awareness. Every timeI was placed in a position to break the barrier of per-ception I saw the wall of fog.

On one occasion, in the past, I had complained todon Juan and Genaro that although I wanted to see itas something else, I couldn't change it. Don Juan hadcommented that that was understandable, because Iwas morbid and somber, that he and I were very dif-ferent in this respect. He was lighthearted and practi-cal and he did not worship the human inventory. I, onthe other hand, was unwilling to throw my inventoryout the window and consequently I was heavy, sinis-ter, and impractical. I had been shocked and saddenedby his harsh criticism and became very gloomy. DonJuan and Genaro had laughed until tears rolled downtheir cheeks.

Genaro had added that on top of all that I was vin-dictive and had a tendency to get fat. They hadlaughed so hard I finally felt obliged to join them.

Don Juan had told me then that exercises of assem-bling other worlds allowed the assemblage point togain experience in shifting. I had always wondered,however, how to get the initial boost to dislodge myassemblage point from its usual position. When I'dquestioned him about it in the past he'd pointed outthat since alignment is the force that is involved ineverything, intent is what makes the assemblage pointmove.

I asked him again about it.

"You're in a position now to answer that questionyourself," he replied. "The mastery of awareness iswhat gives the assemblage point its boost. After all,there is really very little to us human beings; we are,in essence, an assemblage point fixed at a certain po-sition. Our enemy and at the same time our friend isour internal dialogue, our inventory. Be a warrior;shut off your internal dialogue; make your inventoryand then throw it away. The new seers make accurateinventories and then laugh at them. Without the inven-tory the assemblage point becomes free."

Don Juan reminded me that he had talked a greatdeal about one of the most sturdy aspects of our inven-tory: our idea of God. That aspect, he said, was like apowerful glue that bound the assemblage point to itsoriginal position. If I were going to assemble anothertrue world with another great band of emanations, Ihad to take an obligatory step in order to release allties from my assemblage point.

"That step is to see the mold of man," he said."You must do that today unaided."

"What's the mold of man?" I asked.

"I've helped you see it many times," he replied."You know what I'm talking about."

I refrained from saying that I did not know what hewas talking about. If he said that I had seen the moldof man, I must have done that, although I did not havethe foggiest idea what it was like.

He knew what was going through my mind. He gaveme a knowing smile and slowly shook his head fromside to side.

"The mold of man is a huge cluster of emanationsin the great band of organic life," he said. "It is calledthe mold of man because the cluster appears only in-side the cocoon of man.

"The mold of man is the portion of the Eagle's em-anations that seers can see directly without any dangerto themselves."

There was a long pause before he spoke again.

"To break the barrier of perception is the last taskof the mastery of awareness," he said. "In order tomove your assemblage point to that position you mustgather enough energy. Make a journey of recovery.Remember what you've done!"

I tried unsuccessfully to recall what was the mold ofman. I felt an excruciating frustration that soon turnedinto real anger. I was furious with myself, with donJuan, with everybody.

Don Juan was untouched by my fury. He said mat-ter-of-factly that anger was a natural reaction to thehesitation of the assemblage point to move on com-mand.

"It will be a long time before you can apply theprinciple that your command is the Eagle's com-mand," he said. "That's the essence of the mastery ofintent. In the meantime, make a command now not tofret, not even at the worst moments of doubt. It willbe a slow process until that command is heard andobeyed as if it were the Eagle's command."

He also said that there was an unmeasurable area ofawareness in between the customary position of theassemblage point and the position where there are nomore doubts, which is almost the place where the bar-rier of perception makes its appearance. In that un-measurable area, warriors fall prey to everyconceivable misdeed. He warned me to be on thelockout and not lose confidence, for I would unavoid-ably be struck at one time or another by gripping feel-ings of defeat.

"The new seers recommend a very simple act whenimpatience, or despair, or anger, or sadness comestheir way," he continued. "They recommend thatwarriors roll their eyes. Any direction will do; I preferto roll mine clockwise.

"The movement of the eyes makes the assemblagepoint shift momentarily. In that movement, you willfind relief. This is in lieu of true mastery of intent."'

I complained that there was not enough time for himto tell me more about intent.

"It will all come back to you someday," he assuredme. "One thing will trigger another. One key wordand all of it will tumble out of you as if the door of anoverstuffed closet had given way."

He went back then to discussing the mold of man.He said that to see it on my own, unaided by anyone,was an important step, because all of us have certainideas that must be broken before we are free; the seerwho travels into the unknown to see the unknowablemust be in an impeccable state of being.

He winked at me and said that to be in an impecca-ble state of being is to be free of rational assumptionsand rational fears. He added that both my rationalassumptions and my rational fears were preventing meat that moment from realigning the emanations thatwould make me remember seeing the mold of man.He urged me to relax and move my eyes in order tomake my assemblage point shift. He repeated overand over that it was really important to rememberhaving seen the mold before I see it again. And sincehe was pressed for time there was no room for myusual slowness.

I moved my eyes as he suggested. Almost immedi-ately I forgot my discomfort and then a sudden flashof memory came to me and I remembered that I hadseen the mold of man. It had happened years earlieron an occasion that had been quite memorable to me,because from the point of view of my Catholic up-bringing, don Juan had made the most sacrilegiousstatements I had ever heard.

It had all started as a casual conversation while wehiked in the foothills of the Sonoran desert. He wasexplaining to me the implications of what he was doingto me with his teachings. We had stopped to rest andhad sat down on some large boulders. He had contin-ued explaining his teaching procedure, and this hadencouraged me to try for the hundredth time to givehim an account of how I felt about it. It was evidentthat he did not want to hear about it anymore. Hemade me change levels of awareness and told me thatif I would see the mold of man, I might understandeverything he was doing and thus save us both yearsof toil.

He gave me a detailed explanation of what the moldof man was. He did not talk about it in terms of theEagle's emanations, but in terms of a pattern of energythat serves to stamp the qualities of humanness on anamorphous blob of biological matter. At least, I under-stood it that way, especially after he further describedthe mold of man using a mechanical analogy. He saidthat it was like a gigantic die that stamps out humanbeings endlessly as if they were coming to it on amass-production conveyor belt. He vividly mimed theprocess by bringing the palms of his hands togetherwith great force, as if the die molded a human beingeach time its two halves were clapped.

He also said that every species has a mold of itsown, and every individual of every species molded bythe process shows characteristics particular to its ownkind.

He began then an extremely disturbing elucidationabout the mold of man. He said that the old seers aswell as the mystics of our world have one thing incommon?they have been able to see the mold of manbut not understand what it is. Mystics, throughout thecenturies, have given us moving accounts of their ex-periences. But these accounts, however beautiful, areflawed by the gross and despairing mistake of believ-ing the mold of man to be an omnipotent, omniscientcreator; and so is the interpretation of the old seers,who called the mold of man a friendly spirit, a protec-tor of man.

He said that the new seers are the only ones whohave the sobriety to see the mold of man and under-stand what it is. What they have come to realize is thatthe mold of man is not a creator, but the pattern ofevery human attribute we can think of and some wecannot even conceive. The mold is our God becausewe are what it stamps us with and not because it hascreated us from nothing and made us in its image andlikeness. Don Juan said that in his opinion to fall onour knees in the presence of the mold of man reeks ofarrogance and human self-centeredness.

As I heard don Juan's explanation I got terribly wor-ried. Even though I had never considered my self to bea practicing Catholic, I was shocked by his blasphe-mous implications. I had been politely listening tohim, yet I had been yearning for a pause in his barrageof sacrilegious judgments in order to change the sub-ject. But he went on drumming his point in a mercilessway. I finally interrupted him and told him that I be-lieved that God exists.

He retorted that my belief was based on faith and,as such, was a secondhand conviction that did notamount to anything; my belief in the existence of Godwas, like everyone else's, based on hearsay and noton the act of seeing, he said.

He assured me that even if I was able to see, I wasbound to make the same misjudgment that mysticshave made. Anyone who sees the mold of man auto-matically assumes that it is God.

He called the mystical experience a chance seeing,a one-shot affair that has no significance whatsoeverbecause it is the result of a random movement of theassemblage point. He asserted that the new seers areindeed the only ones who can pass a fair judgment onthis matter, because they have ruled out chanceseeings and are capable of seeing the mold of man asoften as they please.

They have seen, therefore, that what we call God isa static prototype of humanness without any power.For the mold of man cannot under any circumstanceshelp us by intervening in our behalf, or punish ourwrongdoings, or reward us in any way. We are simplythe product of its stamp; we are its impression. Themold of man is exactly what its name tells us it is, apattern, a form, a cast that groups together a particularbunch of fiberlike elements, which we call man.

What he had said put me in a state of great distress.But he seemed unconcerned with my genuine turmoil.He kept on needling me with what he called the unfor-givable crime of the chance seers, which makes usfocus our irreplaceable energy on something that hasno power whatsoever to do anything. The more hetalked, the greater my annoyance. When I became soannoyed that I was about to shout at him, he had mechange into yet a deeper state of heightened aware-ness. He hit me on my right side, between my hipboneand my rib cage. That blow sent me soaring into aradiant light, into a diaphanous source of the mostpeaceful and exquisite beatitude. That light was ahaven, an oasis in the blackness around me.

From my subjective point of view, I saw that lightfor an immeasurable length of time. The splendor ofthe sight was beyond anything I can say, and yet Icould not figure out what it was that made it so beau-tiful. Then the idea came to me that its beauty grewout of a sense of harmony, a sense of peace and rest,of having arrived, of being safe at long last. I felt my-self inhaling and exhaling in quietude and relief. Whata gorgeous sense of plenitude! I knew beyond ashadow of doubt that I had come face to face withGod, the source of everything. And I knew that Godloved me. God was love and forgiveness. The lightbathed me, and I felt clean, delivered. I wept uncon-trollably, mainly for myself. The sight of that resplen-dent light made me feel unworthy, villainous.

Suddenly, I heard don Juan's voice in my ear. Hesaid that I had to go beyond the mold, that the moldwas merely a stage, a stopover that brought temporarypeace and serenity to those who journey into the un-known, but that it was sterile, static. It was at thesame time a flat reflected image in a mirror and themirror itself. And the image was man's image.

I passionately resented what don Juan was saying; Irebelled against his blasphemous, sacrilegious words.I wanted to tell him off, but I could not break thebinding power of my seeing. I was caught in it. DonJuan seemed to know exactly how I felt and what Iwanted to tell him.

"You can't tell the nagual off," he said in my ear."It is the nagual who's enabling you to see. It is thenagual's technique, the nagual's power. The nagual isthe guide."

It was at that point that I realized something aboutthe voice in my ear. It was not don Juan's, although itsounded very much like his voice. Also, the voice wasright. The instigator of that seeing was the nagual JuanMatus. It was his technique and his power that wasmaking me see God. He said it was not God, but themold of man; I knew that he was right. Yet I could notadmit that, not out of annoyance or stubbornness, butsimply out of a sense of ultimate loyalty to and lovefor the divinity that was in front of me.

As I gazed into the light with all the passion I wascapable of, the light seemed to condense and I saw aman. A shiny man that exuded charisma, love, under-standing, sincerity, truth. A man that was the sumtotal of all that is good.

The fervor I felt on seeing that man was well beyondanything I had ever felt in my life. I did fall on myknees. I wanted to worship God personified, but donJuan intervened and whacked me on my left upperchest, close to my clavicle, and I lost sight of God.

I was left with a tantalizing feeling, a mixture ofremorse, elation, certainties, and doubts. Don Juanmade fun of me. He called me pious and careless andsaid I would make a great priest; now I could evenpass for a spiritual leader who had had a chance seeingof God. He urged me, in ajocular way, to start preach-ing and describe what I had seen to everyone.

In a very casual but seemingly interested manner hemade a statement that was part question, part asser-tion.

"And the man?" he asked. "You can't forget thatGod is a male."

The immensity of something indefinable began todawn on me as I entered into a state of great clarity.

"Very cozy, eh?" don Juan added, smiling. "Godis a male. What a relief"

After recounting to don Juan what I had remem-bered, I asked him about something that had juststruck me as being terribly odd. To see the mold ofman, I had obviously gone through a shift of my as-semblage point. The recollection of the feelings andrealizations I had had then was so vivid that it gaveme a sense of utter futility. Everything I had done andfelt at that time I was feeling now. I asked him how itwas possible that having had such a clear comprehen-sion, I could have forgotten it so completely. It was asif nothing of what had happened to me had mattered,for I always had to start from point one regardless ofhow much I might have advanced in the past.

"That's only an emotional impression," he said. "Atotal misapprehension. Whatever you did years ago issolidly enclosed in some unused emanations. That daywhen I made you see the mold of man, for instance, Ihad a true misapprehension myself. I thought that ifyou saw it, you would be able to understand it. It wasa true misunderstanding on my part."

Don Juan explained that he had always regardedhimself as being very slow to understand. He hadnever had any chance of testing his belief, because hedid not have a point of reference. When I came alongand he became a teacher, which was something totallynew to him, he realized that there is no way to speedup understanding and that to dislodge the assemblagepoint is not enough. He had thought that it would besufficient. Soon he became aware that since the as-semblage point normally shifts during dreams, some-times to extraordinarily distant positions, wheneverwe undergo an induced shift we are all experts at im-mediately compensating for it. We rebalance our-selves constantly and activity goes on as if nothing hashappened to us.

He remarked that the value of the new seers' con-clusions does not become evident until one tries tomove someone else's assemblage point. The newseers said that what counts in this respect is the effortto reinforce the stability of the assemblage point in itsnew position. They considered this to be the onlyteaching procedure worth discussing. And they knewthat it is a long process that has to be carried out littleby little at a snail's pace.

Don Juan said then that he had used power plants atthe beginning of my apprenticeship in accordance witha recommendation of the new seers. They knew byexperience and by seeing that power plants shake theassemblage point way out of its normal setting. Theeffect of power plants on the assemblage point is inprinciple very much like that of dreams: dreams makeit move; but power plants manage the shift on agreater and more engulfing scale. A teacher then usesthe disorienting effects of such a shift to reinforce thenotion that the perception of the world is never final.

I remembered then that I had seen the mold of manfive more times over the years. With each new time Ihad become less passionate about it. I could never getover the fact, however, that I always saw God as amale. At the end it stopped being God for me andbecame the mold of man, not because of what donJuan had said, but because the position of a male Godbecame untenable. I could then understand don Juan'sstatements about it. They had not been blasphemousor sacrilegious in the least; he had not made them fromwithin the context of the daily world. He was right insaying that the new seers have an edge in being capa-ble of seeing the mold of man as often as they want.But what was more important to me was that they hadsobriety in order to examine what they saw.

I asked him why it was that I always saw the moldof man as a male. He said that it was because myassemblage point did not have the stability then toremain completely glued to its new position andshifted laterally in man's band. It was the same caseas seeing the barrier of perception as a wall of fog.What made the assemblage point move laterally was anearly unavoidable desire, or necessity, to render theincomprehensible in terms of what is most familiar tous: a barrier is a wall and the mold of man cannot beanything else but a man. He thought that if I were awoman I would see the mold as a woman.

Don Juan stood up then and said that it was time forus to take a stroll in town, that I should see the moldof man among people. We walked in silence to thesquare, but before we got there I had an uncontainablesurge of energy and ran down the street to the out-skirts of town. I came to a bridge, and right there, asif it had been waiting for me, I saw the mold of manas a resplendent, warm, amber light.

I fell on my knees, not so much out of piety, but asphysical reaction to awe. The sight of the mold of manwas more astonishing than ever. I felt, without anyarrogance, that I had gone through an enormouschange since the first time I had seen it. However, allthe things I had seen and learned had only given me agreater, more profound appreciation for the miraclethat I had in front of my eyes.

The mold of man was superimposed on the bridgeat first, then I refocused my eyes and saw that themold of man extended up and down into infinity; thebridge was but a meager shell, a tiny sketch superim-posed on the eternal. And so were the minute figuresof people who moved around me, looking at me withunabashed curiosity. But I was beyond their touch,although at that moment I was as vulnerable as I couldbe. The mold of man had no power to protect me orspare me, yet I loved it with a passion that knew nolimits.

I thought that I understood then something that donJuan had told me repeatedly, that real affection cannotbe an investment. I would have gladly remained theservant of the mold of man, not for what it could giveme, for it has nothing to give, but for the sheer affec-tion I felt for it.

I had the sensation of something pulling me away,and before I disappeared from its presence I shouteda promise to the mold of man, but a great forcewhisked me away before I could finish staling what Imeant. I was suddenly kneeling at the bridge while agroup of peasants looked at me and laughed.

Don Juan got to my side and helped me up andwalked me back to the house.

"There are two ways of seeing the mold of man,"don Juan began as soon as we sat down. "You can seeit as a man or you can see it as a light. That dependson the shift of the assemblage point. If the shift islateral, the mold is a human being; if the shift is in themidsection of man's band, the mold is a light. The onlyvalue of what you've done today is that your assem-blage point shifted in the midsection."

He said that the position where one sees the moldof man is very close to that where the dreaming bodyand the barrier of perception appear. That was thereason the new seers recommend that the mold of manbe seen and understood.

"Are you sure you understand what the mold ofman really is?" he asked with a smile.

"I assure you, don Juan, that I'm perfectly awareof what the mold of man is," I said.

"I heard you shouting inanities to the mold of manwhen I got to the bridge," he said with a most mali-cious smile.

I told him that I had felt like a worthless servantworshiping a worthless master, and yet I was movedout of sheer affection to promise undying love.

He found it all hilarious and laughed until he waschoking.

"The promise of a worthless servant to a worthlessmaster is worthless," he said and choked again withlaughter.

I did not feel like defending my position. My affec-tion for the mold of man was offered freely withoutthought of recompense. It did not matter to me thatmy promise was worthless.

17The Journeyof the Dreaming Body

Don Juan told me that the two of us were going todrive to the city of Oaxaca for the last time. He madeit very clear that we would never be there togetheragain. Perhaps his feeling might return to the place, hesaid, but never again the totality of himself.

In Oaxaca, don Juan spent hours looking at mun-dane, trivial things, the faded color of walls, the shapeof distant mountains, the pattern on cracked cement,the faces of people. Then we went to the square andsat on his favorite bench, which was unoccupied, as italways was when he wanted it.

During our long walk in the city, I had tried my bestto work myself into a mood of sadness and morose-ness, but I just could not do it. There was somethingfestive about his departure. He explained it as the un-restrainable vigor of total freedom.

"Freedom is like a contagious disease," he said. "Itis transmitted; its carrier is an impeccable nagual.People might not appreciate that, and that's becausethey don't want to be free. Freedom is frightening.Remember that. But not for us. I've groomed myselfnearly all my life for this moment. And so will you."

He repeated over and over that at the stage where Iwas, no rational assumptions should interfere with myactions. He said that the dreaming body and the bar-rier of perception are positions of the assemblagepoint, and that that knowledge is as vital to seers asknowing how to read and write is to modern man.Both are accomplishments attained after years ofpractice.

"It is very important that you remember, right now,the time when your assemblage point reached that po-sition and it created your dreaming body," he saidwith tremendous urgency.

Then he smiled and remarked that time was ex-tremely short; he said that the recollection of the mainjourney of my dreaming body would put my assem-blage point in a position to break the barrier of percep-tion in order to assemble another world.

"The dreaming body is known by different names,"he said after a long pause. "The name I like the bestis, the other. That term belongs to the old seers, to-gether with the mood. I don't particularly care fortheir mood, but I have to admit that I like their termThe other. It's mysterious and forbidden. Just like theold seers, it gives me the feeling of darkness, of shad-ows. The old seers said that the other always comesshrouded in wind."

Over the years don Juan and other members of hisparty had tried to make me aware that we can be intwo places at once, that we can experience a sort ofperceptual dualism.

As don Juan spoke, I began to remember somethingso deeply forgotten that at first it was as if I had onlyheard about it. Then, step by step, I realized that Ihad lived that experience myself.

I had been in two places at once. It happened onenight in the mountains of northern Mexico. I had beencollecting plants with don Juan all day. We hadstopped for the night and I had nearly fallen asleepfrom fatigue when suddenly there was a gust of windand don Genaro sprang up from the darkness right infront of me and nearly scared me to death.

My first thought was one of suspicion. I believedthat don Genaro had been hiding in the bushes all day,waiting for darkness to set in before making his terri-fying appearance. As I looked at him prancing around,I noticed that there was something truly odd about himthat night. Something palpable, real, and yet some-thing I could not pinpoint.

He joked with me and horsed around, performingacts that defied my reason. Don Juan laughed like anidiot at my dismay. When he judged that the time wasright, he made me shift into heightened awareness andfor a moment I was able to see don Juan and donGenaro as two blobs of light. Genaro was not the flesh-and-blood don Genaro that I knew in my state of nor-mal awareness but his dreaming body. I could tell,because I saw him as a ball of fire that was above theground. He was not rooted as don Juan was. It was asif Genaro, the blob of light, were on the verge of takingoff, already up in the air, a couple of feet off theground, ready to zoom away.

Another thing I had done that night, which suddenlybecame clear to me as I recollected the event, wasthat I knew automatically that I had to move my eyesin order to make my assemblage point shift. I could,with my intent, align the emanations that made me seeGenaro as a blob of light, or I could align the emana-tions that made me see him as merely odd, unknown,strange.

When I saw Genaro as odd, his eyes had a malevo-lent glare, like the eyes of a beast in the darkness. Butthey were eyes, nonetheless. I did not see them aspoints of amber light.

That night don Juan said that Genaro was going tohelp my assemblage point shift very deeply, that Ishould imitate him and follow everything he did. Ge-naro stuck out his rear end and then thrust his pelvisforward with great force. I thought it was an obscenegesture. He repeated it over and over again, movingaround as if he were dancing.

Don Juan nudged me on the arm, urging me to imi-tate Genaro, and I did. Both of us sort of rompedaround, performing that grotesque movement. After awhile, I had the feeling that my body was executingthe movement on its own, without what seemed to bethe real me. The separation between my body and thereal me became even more pronounced, and then at agiven instant I was looking at some ludicrous scenewhere two men were making lewd gestures at eachother.

I watched in fascination and realized that I was oneof the two men. The moment I became aware of it Ifelt something pulling me and I found myself againthrusting my pelvis backward and forward in unisonwith Genaro. Almost immediately, I noticed that an-other man standing next to don Juan was watching us.The wind was blowing around him. I could see his hairbeing ruffled. He was naked and seemed embarrassed.The wind gathered around him as if protecting him, orperhaps the opposite, as if trying to blow him away.

I was slow to realize that I was the other man. WhenI did, I got the shock of my life. An imponderablephysical force pulled me apart as if I were made outof fibers, and I was again looking at a man that wasme, romping around with Genaro, gaping at me whileI looked. And at the same time, I was looking at anaked man that was me, gaping at me while I madelewd gestures with Genaro. The shock was so greatthat I broke the rhythm of my movements and felldown.

The next thing I knew, don Juan was helping me tostand up. Genaro and the other me, the naked one,had disappeared.

I had also remembered that don Juan had refused todiscuss the event. He did not explain it except to saythat Genaro was an expert in creating his double, orthe other, and that I had had long interactions withGenaro's double in states of normal awareness with-out ever detecting it.

"That night, as he has done hundreds of times be-fore, Genaro made your assemblage point shift verydeep into your left side," don Juan commented after Ihad recounted to him everything I had remembered."His power was such that he dragged your assem-blage point to the position where the dreaming bodyappears. You saw your dreaming body watching you.And his dancing did the trick."

I asked him to explain to me how Genaro's lewdmovement could have produced such a drastic effect.

"You're a prude," he said. "Genaro used your im-mediate displeasure and embarrassment at having toperform a lewd gesture. Since he was in his dreamingbody, he had the power to see the Eagle's emanations;from that advantage it was a cinch to make your as-semblage point move."

He said that whatever Genaro had helped me to dothat night was minor, that Genaro had moved my as-semblage point and made it produce a dreaming bodymany, many times, but that those events were notwhat he wanted me to remember.

"I want you to realign the proper emanations andremember the time when you really woke up in adreaming position,"' he said.

A strange surge of energy seemed to explode in-side me and I knew what he wanted me to remem-ber. I could not, however, focus my memory onthe complete event. I could only recall a fragmentof it.

I remembered that one morning, don Juan, don Ge-naro. and I had sat on that very same bench while Iwas in a state of normal awareness. Don Genaro hadsaid, all of a sudden, that he was going to make hisbody leave the bench without getting up. The state-ment was completely out of the context of what wehad been discussing. I was accustomed to don Juan'sorderly, didactic words and actions. I turned to donJuan, expecting a clue, but he remained impassive,looking straight ahead as if don Genaro and I were notthere at all.

Don Genaro nudged me to attract my attention, andthen I witnessed a most disturbing sight. I actually sawGenaro on the other side of the square. He wasbeckoning me to come. But I also saw don Genarositting next to me, looking straight ahead, just as donJuan was.

I wanted to say something, to express my awe, butI found myself dumbstruck, imprisoned by some forcearound me that did not let me talk. I again looked atGenaro across the park. He was still there, motioningto me with a gesture of his head to join him.

My emotional distress mounted by the second. Mystomach was getting upset, and finally I had tunnelvision, a tunnel that led directly to Genaro on theother side of the square. And then a great curiosity, ora great fear, which seemed to be the same thing at thatmoment, pulled me to where he was. I actually soaredthrough the air and got to where he was. He made meturn around and pointed to the three people who weresitting on a bench in a static position, as if time hadbeen suspended.

I felt a terrible discomfort, an internal itching, as ifthe soft organs in the cavity of my body were on fire,and then I was back on the bench, but Genaro wasgone. He waved goodbye to me from across thesquare and disappeared among the people going to themarket.

Don Juan became very animated. He kept on look-ing at me. He stood up and walked around me. He satdown again and could not keep a straight face as hetalked to me.

I realized why he was acting that way. I had enteredinto a state of heightened awareness without beinghelped by don Juan. Genaro had succeeded in makingmy assemblage point move by itself.

I laughed involuntarily upon seeing my writing pad,which don Juan solemnly put inside his pocket. Hesaid that he was going to use my state of heightenedawareness to show me that there is no end to the mys-tery of man and to the mystery of the world.

I focused all my concentration on his words. How-ever, don Juan said something I did not understand. Iasked him to repeat what he had said. He began talk-ing very softly. I thought he had lowered his voice soas not to be overheard by other people. I listened care-fully, but I could not understand a word of what hewas saying; he was either speaking in a language for-eign to me or it was mumbo jumbo. The strange partof it was that something had caught my undivided at-tention, either the rhythm of his voice or the fact thatI had forced myself to understand. I had the feelingthat my mind was different from usual, although Icould not figure out what the difference was. I had ahard time thinking, reasoning out what was takingplace.

Don Juan talked to me very softly in my ear. Hesaid that since I had entered into heightened aware-ness without any help from him my assemblage pointwas very loose, and that I could let it shift into the leftside by relaxing, by falling half asleep on that bench.He assured me that he was watching over me, that Ihad nothing to fear. He urged me to relax, to let myassemblage point move.

I instantly felt the heaviness of being deeply asleep.At one moment, I became aware that I was having adream. I saw a house that I had seen before. I wasapproaching it as if I were walking on the street. Therewere other houses, but I could not pay any attentionto them. Something had fixed my awareness on theparticular house I was seeing. It was a big modernstucco house with a front lawn.

When I got closer to that house, I had a feeling offamiliarity with it, as if I had dreamed of it before. Iwalked on a gravel path to the front door; it was openand I walked inside. There was a dark hall and a largeliving room to the right, furnished with a dark-redcouch and matching armchairs set in a corner. I wasdefinitely having tunnel vision; I could see only whatwas in front of my eyes.

A young woman was standing by the couch as if shehad just stood up as I came in. She was lean and tall,exquisitely dressed in a tailored green suit. She wasperhaps in her late twenties. She had dark-brown hair,burning brown eyes that seemed to smile, and apointed, finely chiseled nose. Her complexion was fairbut had been tanned to a gorgeous brown. I found herravishingly beautiful. She seemed to be an American.She nodded at me, smiling, and extended her handswith the palms down as if she were helping me up.

I clasped her hands in a most awkward movement.I scared myself and tried to back away, but she heldme firmly and yet so gently. Her hands were long andbeautiful. She spoke to me in Spanish with a fainttrace of an accent. She begged me to relax, to feel herhands, to concentrate my attention on her face and tofollow the movement of her mouth. I wanted to askher who she was, but I could not utter a word.

Then I heard don Juan's voice in my ear. He said,"Oh, there you are," as if he had just found me. I wassitting on the park bench with him. But I could alsohear the young woman's voice. She said, "Come andsit with me." I did just that and began a most incredi-ble shifting of points of view. I was alternately withdon Juan and with that young woman. I could see bothof them as clearly as anything.

Don Juan asked me if I liked her, if I found herappealing and soothing. I could not speak, but some-how I conveyed to him the feeling that I did like thatlady immensely. I thought, without any overt reason,that she was a paragon of kindness, that she was indis-pensable to what don Juan was doing with me.

Don Juan spoke in my ear again and said that if Iliked her that much I should wake up in her house,that my feeling of warmth and affection for her wouldguide me. I felt giggly and reckless. A sensation ofoverwhelming excitation rippled through my body. Ifelt as if the excitation were actually disintegrating me.I did not care what happened to me. I gladly plungedinto a blackness, black beyond words, and then Ifound myself in the young woman's house. I was sit-ting with her on the couch.

After an instant of sheer animal panic, I realizedthat somehow I was not complete. Something wasmissing in me. I did not, however, find the situationthreatening. The thought crossed my mind that I wasdreaming and that I was presently going to wake upon the park bench in Oaxaca with don Juan, where Ireally was, where I really belonged.

The young woman helped me to get up and took meto a bathroom where a large tub was filled with water.I realized then that I was stark naked. She gently mademe get into the tub and held my head up while I halffloated in it.

After a while she helped me out of the tub. I feltweak and flimsy. I lay down on the living-room couchand she came close to me. I could hear the beating ofher heart and the pressure of blood rushing throughher body. Her eyes were like two radiant sources ofsomething that was not light, or heat, but curiously inbetween the two. I knew that I was seeing the force oflife projecting out of her body through her eyes. Herwhole body was like a live furnace; it glowed.

I felt a weird tremor that agitated my whole being.It was as if my nerves were exposed and someone wasplucking them. The sensation was agonizing. Then Ieither fainted or fell asleep.

When I woke up, someone was putting face towelssoaked in cold water on my face and the back of myneck. I saw the young woman sitting by my head onthe bed where I was lying. She had a pail of water ona night table. Don Juan was standing at the foot of thebed with my clothes draped over his arm.

I was fully awake then. I sat up. They had coveredme with a blanket.

"How's the traveler?" don Juan asked, smiling."Are you in one piece now?"

That was all I could remember. I narrated this epi-sode to don Juan, and as I talked, I recalled anotherfragment. I remembered that don Juan had tauntedand teased me about finding me naked in the lady'sbed. I had gotten terribly irritated at his remarks. Ihad put on my clothes and stomped out of the housein a fury.

Don Juan had caught up with me on the front lawn.In a very serious tone he had remarked that I was myugly stupid self again, that I had put myself togetherby being embarrassed, which had proved to him thatthere was still no end to my self-importance. But hehad added in a conciliatory tone that that was notimportant at the moment; what was significant was thefact that I had moved my assemblage point verydeeply into the left side and consequently I had trav-eled an enormous distance.

He had spoken of wonders and mysteries, but I hadnot been able to listen to him, for I had been caught inthe crossfire between fear and self-importance. I wasactually fuming. I was certain that don Juan had hyp-notized me in the park and had then taken me to thatlady's house, and that the two of them had done terri-ble things to me.

My fury was interrupted. Something out there in thestreet was so horrifying, so shocking to me, that myanger stopped instantaneously. But before mythoughts became fully rearranged, don Juan hit me onmy back and nothing of what had just taken placeremained. I found myself back in my blissful every-day-life stupidity, happily listening to don Juan, wor-rying about whether or not he liked me.

As I was telling don Juan about the new fragmentthat I had just remembered I realized that one of hismethods for handling my emotional turmoil was tomake me shift into normal awareness.

"The only thing that soothes those who journey intothe unknown is oblivion," he said. "What a relief tobe in the ordinary world!

"That day, you accomplished a marvelous feat. Thesober thing for me to do was not to let you focus on itat all. Just as you began to really panic I made youshift into normal awareness; I moved your assemblagepoint beyond the position where there are no moredoubts. There are two such positions for warriors. Inone you have no more doubts because you knoweverything. In the other, which is normal awareness,you have no doubts because you don't know anything.

"It was too soon then for you to know what hadreally happened. But I think the right time to know isnow. Looking at that street, you were about to findout where your dreaming position had been. You trav-eled an enormous distance that day."

Don Juan scrutinized me with a mixture of glee andsadness. I was trying my best to keep under controlthe strange agitation I was feeling. I sensed that some-thing terribly important to me was lost inside my mem-ory, or, as don Juan would have put it, inside someunused emanations that at one time had been aligned.

My struggle to keep calm proved to be the wrongthing to do. All at once, my knees wobbled and ner-vous spasms ran through my midsection. I mumbled,unable to voice a question. I had to swallow hard andbreathe deeply before I regained my calmness.

"When we first sat down here to talk, I said that norational assumptions should interfere with the actionsof a seer," he continued in a stern tone. "I knew thatin order to reclaim what you've done, you'd have todispense with rationality, but you'd have to do it in.the level of awareness you are in now."

He explained that I had to understand that rational-ity is a condition of alignment, merely the result of theposition of the assemblage point. He emphasized thatI had to understand this when I was in a state of greatvulnerability, as I was at that moment. To understandit when my assemblage point had reached the positionwhere there are no doubts was useless, because reali-zations of that nature are commonplace in that posi-tion. It was equally useless to understand it in a stateof normal awareness; in that state, such realizationsare emotional outbursts that are valid only for as longas the emotion lasts.

"I've said that you traveled a great distance thatday," he said calmly. "And I said that because I knowit. I was there, remember?"

I was sweating profusely out of nervousness andanxiety.

"You traveled because you woke up at a distantdreaming position," he continued. "When Genaropulled you across the plaza, right here from thisbench, he paved the way for your assemblage point tomove from normal awareness all the way to the posi-tion where the dreaming body appears. Your dream-ing body actually flew over an incredible distance inthe blink of an eyelid. Yet that's not the importantpart. The mystery is in the dreaming position. If it isstrong enough to pull you, you can go to the ends ofthis world or beyond it, just as the old seers did. Theydisappeared from this world because they woke up ata dreaming position beyond the limits of the known.Your dreaming position that day was in this world,but quite a distance from the city of Oaxaca."

"How does ajourney like that take place?" I asked.

"There is no way of knowing how it is done," hesaid. "Strong emotion, or unbending intent, or greatinterest serves as a guide; then the assemblage pointgets powerfully fixed at the dreaming position, longenough to drag there all the emanations that are insidethe cocoon."

Don Juan said then that he had made me see count-less times over the years of our association, either instates of normal awareness or in states of heightenedawareness; I had seen countless things that I was nowbeginning to understand in a more coherent fashion.This coherence was not logical or rational, but it clar-ified, nonetheless, in whatever strange way, every-thing I had done, everything that was done to me, andeverything I had seen in all those years with him. Hesaid that now I needed to have one last clarification:the coherent but irrational realization that everythingin the world we have learned to perceive is inextrica-bly tied to the position where the assemblage point islocated, if the assemblage point is displaced from thatposition, the world will cease to be what it is to us.

Don Juan stated that a displacement of the assem-blage point beyond the midline of the cocoon of manmakes the entire world we know vanish from our viewin one instant, as if it had been erased?for the stabil-ity, the substantiality, that seems to belong to our per-ceivable world is just the force of alignment. Certainemanations are routinely aligned because of the fixa-tion of the assemblage point on one specific spot; thatis all there is to our world.

"The soundness of the world is not the mirage," hecontinued, "the mirage is the fixation of the assem-blage point on any spot. When seers shift their assem-blage points, they are not confronted with an illusion,they are confronted with another world; that newworld is as real as the one we are watching now, butthe new fixation of their assemblage points, which pro-duces that new world, is as much of a mirage as theold fixation.

"Take yourself, for example; you are now in a stateof heightened awareness. Whatever you are capableof doing in such a state is not an illusion; it is as realas the world you will face tomorrow in your daily life,and yet tomorrow the world you are witnessing nowwon't exist. It exists only when your assemblage pointmoves to the particular spot where you are now."

He added that the task warriors are faced with, afterthey finish their training, is one of integration. In thecourse of training, warriors, especially nagual men,are made to shift to as many individual spots as pos-sible. He said that in my case I had moved to countlesspositions that I would have to integrate someday intoa coherent whole.

"For instance, if you would shift your assemblagepoint to a specific position, you'd remember who thatlady is," he continued with a strange smile. "Yourassemblage point has been at that spot hundreds oftimes. It should be the easiest thing for you to inte-grate it."

As though my recollection depended on his sugges-tion, I began to have vague memories, feelings ofsorts. There was a feeling of boundless affection thatseemed to attract me; a most pleasant sweetness filledthe air, exactly as if someone had just come up frombehind me and poured that scent over me. I eventurned around. And then I remembered. She wasCarol, the nagual woman' I had been with her only theday before. How could I have forgotten her?

I had an indescribable moment in which I think allthe feelings of my psychological repertory ran throughmy mind. Was it possible, I asked myself, that I hadwoken up in her house in Tucson, Arizona, two thou-sand miles away? And are each of the instances ofheightened awareness so isolated that one cannot re-member them?

Don Juan came to my side and put his arm on myshoulder. He said that he knew exactly how I felt. Hisbenefactor had made him go through a similar experi-ence. And just as he himself was now trying to do withme, his benefactor had tried to do with him: soothewith words. He had appreciated his benefactor's at-tempt, but he doubted then as he doubted now thatthere is a way to soothe anyone who realizes the jour-ney of the dreaming body.

There was no doubt in my mind now. Something inme had traveled the distance between the cities ofOaxaca, Mexico, and Tucson, Arizona. I felt a strangerelief, as if I had been purged of guilt at long last.

During the years I had spent with don Juan, I hadhad lapses of continuity in my memory. My being inTucson with him on that day was one of those lapses.I remembered not being able to recall how I had gottento Tucson. I did not pay any attention to it, however.I thought the lapse was the result of my activities withdon Juan. He was always very careful not to arousemy rational suspicions in states of normal awareness,but if suspicions were unavoidable he always curtlyexplained them away by suggesting that the nature ofour activities fostered serious disparities of memory.

I told don Juan that since both of us had ended upthat day in the same place, I wondered whether it waspossible for two or more people to wake up at thesame dreaming position.

"Of course," he said. "That's the way the old Tol-tec sorcerers took off into the unknown in packs. Theyfollowed one another. There is no way of knowinghow one follows someone else. It's just done. Thedreaming body just does it. The presence of anotherdreamer spurs it to do it. That day you pulled me withyou. And I followed because I wanted to be withyou."

I had so many questions to ask him, but every oneof them seemed superfluous.

"How is it possible that I didn't remember the na-gual woman?" I muttered, and a horrible anguish andlonging gripped me. I was trying not to feel sad any-more, but suddenly sadness ripped through me likepain.

"You still don't remember her," he said. "Onlywhen your assemblage point shifts can you recollecther. She is like a phantom to you, and so are you toher. You've seen her once while you were in normalawareness, but she's never seen you in her normalawareness. To her you are as much a personage as sheis to you. With the difference that you may wake upsomeday and integrate it all. You may have enoughtime to do that, but she won't. Her time here is short."

I felt like protesting a terrible injustice. I mentallyprepared a barrage of objections, but I never voicedthem. Don Juan's smile was beaming. His eyes shonewith sheer glee and mischief. I had the sensation thathe was waiting for my statements, because he knewwhat I was going to say. And that sensation stoppedme, or rather I did not say anything because my as-semblage point had again moved by itself. And I knewthen that the nagual woman could not be pitied for nothaving time, nor could I rejoice for having it.

Don Juan was reading me like a book. He urged meto finish my realization and voice the reason for notfeeling sorry or for not rejoicing. I felt for an instantthat I knew why. But then I lost the thread.

"The excitation of having time is equal to the exci-tation of not having it," he said. "It's all the same."

"To feel sad is not the same as feeling sorry " Isaid. "And I feel terribly sad."

"Who cares about sadness?" he said. "Think onlyof the mysteries; mystery is all that matters. We areliving beings; we have to die and relinquish our aware-ness. But if we could change just a tinge of that, whatmysteries must await us! What mysteries!"

18Breaking theBarrier of Perception

In the late afternoon, still in Oaxaca, don Juan and Istrolled around the square leisurely. As we ap-proached his favorite bench the people who were sit-ting there got up and left. We hurried over to it andsat down.

"We've come to the end of my explanation ofawareness," he said. "And today, you are going toassemble another world by yourself and leave alldoubts aside forever.

"There must be no mislake about what you aregoing to do. Today, from the vantage point ofheightened awareness, you are going to make yourassemblage point move and in one instant you aregoing to align the emanations of another world.

"In a few days, when Genaro and I meet you on amountaintop, you are going to do the same from thedisadvantage of normal awareness. You will have toalign the emanations of another world on a moment'snotice; if you don't you will die the death of an aver-age man who falls from a precipice."

He was alluding to an act that he would have meperform as the last of his teachings for the right side:the act of jumping from a mountaintop into an abyss.

Don Juan stated that warriors ended their trainingwhen they were capable of breaking the barrier ofperception, unaided, starting from a normal state ofawareness. The nagual led warriors to that threshold,but success was up to the individual. The nagualmerely tested them by continually pushing them tofend for themselves.

"The only force that can temporarily cancel outalignment is alignment," he continued. "You willhave to cancel the alignment that keeps you perceivingthe world of daily affairs. By inlending a new positionfor your assemblage point and by intending to keep itfixed there long enough, you will assemble anotherworld and escape this one.

"The old seers are still defying death, to this day,by doing just that, intending their assemblage pointsto remain fixed on positions that place them in any ofthe seven worlds."

"What will happen if I succeed in aligning anotherworld?" I asked.

"You will go to it," he replied. "As Genaro did,one night in this very place when he was showing youthe mystery of alignment."

"Where will I be, don Juan?"

"In another world, of course. Where else?"

"What about the people around me, and the build-ings, and the mountains, and everything else?"

"You'll be separated from all that by the very bar-rier that you have broken: the barrier of perception.And just like the seers who have buried themselves todefy death, you won't be in this world."

There was a battle raging inside me as I heard hisstatements. Some part of me clamored that don Juan'sposition was untenable, while another part knew be-yond any question that he was right.

I asked him what would happen if I moved my as-semblage point while I was in the street, in the middleof traffic in Los Angeles.

"Los Angeles will vanish, like a puff of air," hereplied with a serious expression. "But you will re-main.

"That is the mystery I've been trying to explain toyou. You've experienced it, but you haven't under-stood it yet, and today you will."

He said that I could not as yet use the boost of theearth to shift into another great band of emanations,but that since I had an imperative need to shift, thatneed was going to serve me as a launcher.

Don Juan looked up at the sky. He stretched hisarms above his head as if he had been sitting for toolong and was pushing physical weariness out of hisbody. He commanded me to turn off my internal dia-logue and enter into inner silence. Then he stood upand began to walk away from the square; he signaledme to follow him. He took a deserted side street. Irecognized it as being the same street where Genarohad given me his demonstration of alignment. The mo-ment I recollected that, I found myself walking withdon Juan in a place that by then was very familiar tome: a deserted plain with yellow dunes of whatseemed to be sulfur.

I recalled then that don Juan had made me perceivethat world hundreds of times. I also recalled that be-yond the desolate landscape of the dunes there wasanother world shining with an exquisite, uniform, purewhite light.

When don Juan and I entered into it this time, Isensed that the light, which came from every direc-tion, was not an invigorating light, but was so soothingthat it gave me the feeling that it was sacred.

As that sacred light bathed me a rational thoughtexploded in my inner silence. I thought it was quitepossible that mystics and saints had made this journeyof the assemblage point. They had seen God in themold of man. They had seen hell in the sulfur dunes.And then they had seen the glory of heaven in thediaphanous light.

My rational thought burned out almost immediatelyunder the onslaughts of what I was perceiving. Myawareness was taken by a multitude of shapes, figuresof men, women, and children of all ages, and otherincomprehensible apparitions gleaming with a blindingwhite light.

I saw don Juan, walking by my side, staring at meand not at the apparitions, but the next instant I sawhim as a ball of luminosity, bobbing up and down afew feet away from me. The ball made an abrupt andfrightening movement and came closer to me and Isaw inside it.

Don Juan was working his glow of awareness for mybenefit. The glow suddenly shone on four or fivethreadlike filaments on his left side. It remained fixedthere. All my concentration was on it; somethingpulled me slowly as if through a tube and I saw theallies?three dark, long, rigid figures agitated by atremor, like leaves in a breeze. They were against analmost fluorescent pink background. The moment Ifocused my eyes on them, they came to where I was,not walking or gliding or flying, but by pulling them-selves along some fibers of whiteness that came out ofme. The whiteness was not a light or a glow but linesthat seemed to be drawn with heavy powder chalk.They disintegrated quickly, yet not quickly enough.The allies were on me before the lines faded away.

They crowded me. I became annoyed, and the alliesimmediately moved away as if I had chastised them. Ifelt sorry for them, and my feeling pulled them backinstantly. And they again came and rubbed themselvesagainst me. I saw then something I had seen in themirror at the stream. The allies had no inner glow.They had no inner mobility. There was no life in them.And yet they were obviously alive. They were strangegrotesque shapes that resembled zippered-up sleepingbags. The thin line in the middle of their elongatedshapes made them look as if they had been sewed up.

They were not pleasing figures. The sensation thatthey were totally alien to me made me feel uncomfort-able, impatient. I saw that the three allies were movingas if they were jumping up and down; there was a faintglow inside them. The glow grew in intensity until, inat least one of the allies, it was quite brilliant.

The instant I saw that, I was facing a black world. Ido not mean that it was dark as night is dark. It wasrather that everything around me was pitch-black. Ilooked up at the sky and I could not find light any-where. The sky was also black and literally coveredwith lines and irregular circles of various degrees ofblackness. The sky looked like a black piece of woodwhere the grain showed in relief.

I looked down at the ground. It was fluffy. It seemedto be made of flakes of agar-agar; they were not dullflakes, but they were not shiny either. It was some-thing in between, which I had never seen in my life:black agar-agar.

I heard then the voice of seeing. It said that myassemblage point had assembled a total world withother great bands of emanations: a black world.

I wanted to absorb every word I was hearing; inorder to do that I had to split my concentration. Thevoice stopped; my eyes became focused again. I wasstanding with don Juan just a few blocks away fromthe square.

I instantly felt that I had no time to rest, that itwould be useless to indulge in being shocked. I ralliedall my strength and asked don Juan if I had done whathe had expected.

"You did exactly what you were expected to do,"he said reassuringly. "Let's go back to the square andstroll around it one more time, for the last time in thisworld."

I refused to think about don Juan's leaving, so Iasked him about the black world. I had vague recollec-tions of having seen it before.

"It's the easiest world to assemble," he said. "Andof all you've experienced, only the black world isworth considering. It is the only true alignment of an-other great band you have ever made. Everything elsehas been a lateral shift along man's band, but stillwithin the same great band. The wall of fog, the plainwith yellow dunes, the world of the apparitions?allare lateral alignments that our assemblage points makeas they approach a crucial position."

He explained as we walked back to the square thatone of the strange qualities of the black world is thatit does not have the same emanations that account fortime in our world. They are different emanations thatproduce a different result. Seers that journey into theblack world feel that they have been in it for an eter-nity, but in our world that turns out to be an instant.

"The black world is a dreadful world because it agesthe body," he said emphatically.

I asked him to clarify his statements. He sloweddown his pace and looked at me. He reminded me thatGenaro, in his direct way, had tried to point that outto me once, when he told me that we had plodded inhell for an eternity while not even a minute had passedin the world we know.

Don Juan remarked that in his youth he had becomeobsessed with the black world. He had wondered, infront of his benefactor, about what would happen tohim if he went into it and stayed there for a while. Butas his benefactor was not given to explanations, hehad simply plunged don Juan into the black world tolet him find out for himself.

"The nagual Julian's power was so extraordinary,"don Juan continued, "that it took me days to comeback from that black world."

"You mean it took you days to return your assem-blage point to its normal position, don't you?" Iasked.

"Yes. I mean that," he said.

He explained that in the few days that he was lostin the black world he aged at least ten years, if notmore. The emanations inside his cocoon felt the strainof years of solitary struggle.

Silvio Manuel was a totally different case. The na-gual Julian also plunged him into the unknown, butSilvio Manuel assembled another world with anotherset of bands, a world also without the emanations oftime but one which has the opposite effect on seers.He disappeared for seven years and yet he felt he hadbeen gone only a moment.

"To assemble other worlds is not only a matter ofpractice, but a matter of intent," he continued. "Andit isn't merely an exercise of bouncing out of thoseworlds, like being pulled by a rubber band. You see, aseer has to be daring. Once you break the barrier ofperception, you don't have to come back to the sameplace in the world. See what I mean?"

It slowly dawned on me what he was saying. I hadan almost invincible desire to laugh at such a prepos-terous idea, but before the idea coalesced into a cer-tainty, don Juan spoke to me and disrupted what I wasabout to remember.

He said that for warriors the danger of assemblingother worlds is that those worlds are as possessive asour world. The force of alignment is such that oncethe assemblage point breaks away from its normal po-sition, it becomes fixed at other positions, by otheralignments. And warriors run the risk of gettingstranded in inconceivable aloneness.

The inquisitive, rational part of me commented thatI had seen him in the black world as a ball of luminos-ity. It was possible, therefore, to be in that world withpeople.

"Only if people follow you around by moving theirown assemblage points when you move yours," hereplied. "I shifted mine in order to be with you; oth-erwise you would have been there alone with the al-lies."

We stopped walking, and don Juan said that it wastime for me to go.

"I want you to bypass all lateral shifts," he said,"and go directly to the next total world: the blackworld. In a couple of days you'll have to do the samething by yourself. You won't have time to piddlearound. You'll have to do it in order to escape death."

He said that breaking the barrier of perception is theculmination of everything seers do. From the momentthat barrier is broken, man and his fate take on a dif-ferent meaning for warriors. Because of the transcen-dental importance of breaking that barrier, the newseers use the act of breaking it as a final test. The testconsists of jumping from a mountaintop into an abysswhile in a state of normal awareness. If the warriorjumping into the abyss does not erase the daily worldand assemble another one before he reaches bottom,he dies.

"What you are going to do is to make this worldvanish," he went on, "but you are going to remainsomewhat yourself. This is the ultimate bastion ofawareness, the one the new seers count on. Theyknow that after they burn with consciousness, theysomewhat retain the sense of being themselves."

He smiled and pointed to a street that we could seefrom where we were standing?the street where Ge-naro had shown me the mysteries of alignment.

"That street, like any other, leads to eternity," hesaid. "All you have to do is follow it in total silence.It's time. Go now! Go!"

He turned around and walked away from me. Ge-naro was waiting for him at the corner. Genaro wavedat me and then made a gesture of urging me to comeon. Don Juan kept on walking without turning to look.Genaro joined him. I started to follow them, but Iknew that it was wrong. Instead, I went in the oppositedirection. The street was dark, lonely, and bleak. I didnot indulge in feelings of failure or inadequacy. Iwalked in inner silence. My assemblage point wasmoving at great speed. I saw the three allies. The lineof their middle made them look as if they were smilingsideways. I felt that I was being frivolous. And then awindlike force blew the world away.

Epilogue

A couple of days later, all the nagual's party and allthe apprentices got together on the flat mountaintopdon Juan had told me about.

Don Juan said that each of the apprentices had al-ready said goodbye to everybody, and that all of uswere in a state of awareness that admitted no senti-mentalism. For us, he said, there was only action. Wewere warriors in a state of total war.

Everyone, with the exception of don Juan, Genaro,Pablito, Nestor, and me, moved a short distance awayfrom the flat mountaintop, in order to allow Pablito,Nestor, and me privacy to enter into a state of normalawareness.

But before we did, don Juan took us by the armsand walked us around the flat top.

"In a moment, you're going to infend the movementof your assemblage points," he said. "And no one willhelp you. You are now alone. You must rememberthen that intent begins with a command.

"The old seers used to say that if warriors are goingto have an internal dialogue, they should have theproper dialogue. For the old seers that meant a dia-logue about sorcery and the enhancement of their self-reflection. For the new seers, it doesn't mean dia-logue, but the detached manipulation of intent throughsober commands."

He said over and over again that the manipulationof intent begins with a command given to oneself; thecommand is then repeated until it becomes the Eagle'scommand, and then the assemblage point shifts, ac-cordingly, the moment warriors reach inner silence.

The fact that such a maneuver is possible, he said,is something of the most singular importance to seers,old and new alike, but for reasons diametrically op-posed. Knowing about it allowed the old seers tomove their assemblage point to inconceivable dream-ing positions in the incommensurable unknown; forthe new seers it means refusing to be food, it meansescaping the Eagle by moving their assemblage pointsto a particular dreaming position called total freedom.

He explained that the old seers discovered that it ispossible to move the assemblage point to the limit ofthe known and keep it fixed there in a state of primeheightened awareness. From that position, they sawthe feasibility of slowly shifting their assemblagepoints permanently to other positions beyond thatlimit?a stupendous feat fraught with daring but lack-ing sobriety, for they could never retract the move-ment of their assemblage points, or perhaps theynever wanted to.

Don Juan said that adventurous men, faced with thechoice of dying in the world of ordinary affairs ordying in unknown worlds, will unavoidably choose thelatter, and that the new seers, realizing that their pre-decessors had chosen merely to change the locale oftheir death, came to understand the futility of it all;the futility of struggling to control their fellow men,the futility of assembling other worlds, and, above all,the futility of self-importance.

One of the most fortunate decisions that the newseers made, he said, was never to allow their assem-blage points to move permanently to any positionother than heightened awareness. From that position,they actually resolved their dilemma of futility andfound out that the solution is not simply to choose analternate world in which to die, but to choose totalconsciousness, total freedom.

Don Juan commented that by choosing total free-dom, the new seers unwittingly continued in the tra-dition of their predecessors and became thequintessence of the death defiers.

He explained that the new seers discovered that ifthe assemblage point is made to shift constantly to theconfines of the unknown, but is made to return to aposition at the limit of the known, then when it issuddenly released it moves like lightning across theentire cocoon of man, aligning all the emanations in-side the cocoon at once.

"The new seers burn with the force of alignment,"don Juan went on, "with the force of will, which theyhave turned into the force of intent through a life ofimpeccability. Intent is the alignment of all the amberemanations of awareness, so it is correct to say thattotal freedom means total awareness."

"Is that what all of you are going to do, don Juan?"I asked.

"We most certainly will, if we have sufficient en-ergy," he replied. "Freedom is the Eagle's gift toman. Unfortunately, very few men understand that allwe need, in order to accept such a magnificent gift, isto have sufficient energy.

"If that's all we need, then, by all means, we mustbecome misers of energy."

After that, don Juan made us enter into a state ofnormal awareness. At dusk, Pablito, Nestor, and Ijumped into the abyss. And don Juan and the nagual'sparty burned with the fire from within. They enteredinto total awareness, for they had sufficient energy toaccept the mind-boggling gift of freedom.

Pablito, Nestor, and I didn't die at the bottom ofthat gorge?and neither did the other apprentices whohad jumped at an earlier time?because we neverreached it; all of us, under the impact of such a tre-mendous and incomprehensible act as jumping to ourdeaths, moved our assemblage points and assembledother worlds.

We know now that we were left to rememberheightened awareness and to regain the totality of our-selves. And we also know that the more we remem-ber, the more intense our elation, our wondering, butalso the greater our doubts, our turmoil.

So far, it is as if we were left only to be tantalizedby the most far-reaching questions about the natureand the fate of man, until the time when we may havesufficient energy not only to verify everything donJuan taught us, but also to accept the Eagle's gift our-selves.

